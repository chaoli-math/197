\section{算数概形}

我们最后一类概形是有限生成约态环但不包含任意域的环的谱。一般地,有限生成$\zz$-代数的谱被称为\textit{算术概形}\index{概形!算术},它们主要在数论中出现,尽管并非所有数论学家感兴趣的概形都具有这种形式。在这些例子中,我们将看到概形在算术以及几何的观点之间的惊人统一的许多线索。\nottran

\subsection{$\spec \mathbb{Z}$}

我们从最明显的例子开始,概形$\spec \zz$. 环$\zz$的素理想当然是$(p)$,其中$p$是素数,以及$(0)$,前者对应于$\spec \zz$的闭点,剩余类域是$\mathbb{F}_p$,后者是一个“一般”点,闭包是整个$\spec \zz$而剩余类域是$\mathbb{Q}$. 图像如下:

% \pic{15.png}

\noindent 这与域上的直线$\mathbb{A}_K^1$在形式上是相似的,事实上,这只是一系列类比的开始,因此在看下面的例子时应将其铭记在心。然而,这种类比也有着它的不足:尽管$\spec \zz$表现得很像$\mathbb{A}_K^1$,但比如$\mathbb{A}_K^1$是$\mathbb{P}_K^1$的稠密开子概形,而$\spec \zz$不是任何一个概形的稠密开子概形。

\subsection{数域中整数环的谱}

其次,考虑概形$\spec A$,其中$A\subset K$是数域$K$中的整数环。作为例子,我们将分析$K=\mathbb{Q}[\sqrt{3}]$以及$A=\mathbb{Z}[\sqrt{3}]$. 正如$\spec \zz$的情况,$\spec A$有两类点,一类对应于$A$的非零素理想,具有有限剩余类域,另一类是一个一般点,对应于$(0)$,剩余类域是$K$. 含入映射$\zz\hookrightarrow A$诱导的映射$\spec A\to \spec \zz$让这个例子变得有趣起来。比如说,考虑点$[(p)]\in \spec \zz$处的纤维,这就是$A$中包含理想$pA\subset A$的所有素理想的集合,他将表现为下面三种方式中的一个(关于这里没解释的材料的一个好的基础参考文献是Serre [1979]):

\begin{enumerate}[{(1)}]\setlength{\itemsep}{0pt}
	\item 如果$p$整除$K$在$\mathbb{Q}$上的判别式$12$,即对$p=2$或$3$,理想$(p)$是$A$中一个理想的平方,我们有
	\[
	2A=(1+\sqrt{3})^2
	\]
	以及,当然,
	\[
	3A=(\sqrt{3})^2.
	\]
	点$(1+\sqrt{3})$和点$(\sqrt{3})\in\spec A$的剩余类环分别为$\mathbb{F}_2$和$\mathbb{F}_3$.

	\item 否则,如果$3$是一个模$p$的二次剩余,于是素理想$(p)$能分解为两个不同素理想的乘积,比如
	\[
	11A=(4+3\sqrt{3})(4-3\sqrt{3})
	\]
	以及
	\[
	13A=(4+\sqrt{3})(4-\sqrt{3}).
	\]
	在这些点的剩余类域依然是元素个数为素数的有限域,分别是$\mathbb{F}_{11}$以及$\mathbb{F}_{13}$.

	\item 最后,如果$p>3$以及$3$不再是一个模$p$的二次剩余,比如$p=5$或者$7$时,理想$pA$依然是素理想,它对应于$\spec A$中的一个点。在这种情况下,剩余类域是$\mathbb{F}_p$的二次扩张,比如在上面的两个例子中是$\mathbb{F}_{25}$以及$\mathbb{F}_{49}$.
\end{enumerate}

一般地,,如果$K$是一个二次数域,$A$是$K$中的代数整数环,那么包含$Z\subset A$诱导了概形的映射$\psi:\spec A\to \spec \zz$,$\psi$在每一个闭点$(p)\in \spec \zz$处的纤维是下面的可能中的一个:

\begin{enumerate}[{(1)}]\setlength{\itemsep}{0pt}
	\item 一个非约态的点,其坐标环同构于$A/\pp^2$. 他在$\spec \zz$中的像是约态点$\pp$,而剩余类域是$\mathbb{F}_p$,如果$p$在$A$中\textit{分岔},即$pA$是$A$中一个素理想$\pp$的平方。

	\item 两个约态点$\pp$和$\pp'$的并,剩余类域是$A/\pp=A/\pp'=\mathbb{F}_p$,如果$pA$是$A$中两个不同素理想的乘积。

	\item 一个约态点,它的剩余类域$A/\pp$是$\mathbb{F}_p$的二次扩张,如果$p$在$A$中仍然是素的。
\end{enumerate}

在每个粒子中,纤维的坐标环作为$\mathbb{F}_p$-代数,维数都是2. 这是因为$A$是一个秩为2的自由$\zz$-模。这里感兴趣的类比是,映射$\spec A\to \spec \zz$与Riemann面的一个分支覆盖(或者更一般地,一个代数闭域,比如$\cc$上的一维概形)。 基本上,我们能将$\spec A$想象成$\spec \zz$的两层覆盖然后在“分岔”素数上分支,比如就像,$\spec \cc[z]$是$\spec \cc[z^2]$的一个双重覆盖,它在原点处分支。对于$\spec \zz$处的点,一个明显的不同是,分岔点处我们有两个不同的重数为1的点,但在不同于分岔点的点$(p)\in \spec \zz$处,我们只有一个重数为1的点,其剩余类域是$\mathbb{F}_p$的二次扩张,这样的点在下图中我们以均匀的灰色点表示:

% \pic{16.png}

一个具有更丰富内容的类比是非代数闭域上的一维概形之间的有限映射。比如考虑映射
\[
	\spec \rr[x][y]/(y^2-x)\to \mathbb{A}_\rr^1=\spec \rr[x],
\]
只看$\mathbb{A}_\rr^1$中那些剩余类域为$\rr$的点,即对实数$\lambda$形如$(x-\lambda)$的点,我们在点$(x)$处有分岔,对$\lambda\neq 0$,$(x-\lambda)$的原像是剩余类域为$\rr$的两个不同的点(若$\lambda>0$)或是一个剩余类域为$\cc$的点(若$\lambda<0$)。

我们通过观察概形$\spec B$,其中$B\subset A\subset K$是数域的一个序,即$K$中的整数环的子环也有着分式域$K$,来进一步深入上面这个类比。举个例子,$A=\zz[\sqrt 3]$以及$B=\zz[11\sqrt 3]$. 上面描述的映射$\spec A\to \spec \zz$现在可以分解为$\spec A\to \spec B\to \spec \zz$,事实上,映射$\spec A\to \spec B$除了两个点$(4+3\sqrt 3)$和$(4-3\sqrt 3)\in \spec A$同时映到$(11,11\sqrt 3)\in\spec B$之外,它就是一个同构。我们于是将$\spec B$画成一种“结点曲线”,即,$\spec A$到$\spec \zz$的双重覆盖中的两点在这里重合了。

% \pic{17.png}

或者,考虑$A=\zz[\sqrt 3]$以及$B=\zz[2\sqrt 3]$的情况。这里映射$\spec A\to\spec B$是一对一的但不是一个同构,点$[(1+\sqrt 3)]$映到了$[(2,2\sqrt{3})]$.

\begin{exe}
	证明,点$p=[(2,2\sqrt{3})]$是概形$\spec \zz[2\sqrt 3]$的一个“尖点”,即这是一个奇异点以及消除奇异性的映射$\spec A\to \spec B$在点$p$处的纤维包含一个二重点。
\end{exe}

\subsection{$\spec \mathbb{Z}$上的仿射空间}

我们下一个例子是一个二维概形$\spec \zz[x]$,或者被记作$\mathbb{A}_\zz^1$. $\zz[x]$中的素理想是
\begin{enumerate}[{(i)}]\setlength{\itemsep}{0pt}
	\item $(0)$;
	\item $(p)$,其中$p\in\zz$是一个素数;
	\item 主理想$(f)$,其中$f\in\zz[x]$是一个$\mathbb{Q}$上的不可约多项式,且所有系数的极大公因子为$1$;
	\item 极大理想$(p,f)$,其中$p\in \zz$是一个素数,而$f$是一个首一多项式,模$p$后是不可约的。
\end{enumerate}

\begin{exe}
	证明这点。
\end{exe}

这些素理想中,只有最后一类是闭点。第一类的闭包是整个$\mathbb{A}_\zz^1$,第二第三类的闭包我们下面描述。

可能图像化$\mathbb{A}_\zz^1$的最好方式是通过映射$\mathbb{A}_\zz^1\to\zz$(依然是平坦映射!)。在这个映射下, 上面的第二第四类便被映射成$(p)\in\spec \zz$,而第一第三类点将被映射成$\spec \zz$的一般点$(0)$. 实际上,这个映射在点$(p)$的纤维同构于$\mathbb{A}_{\mathbb{F}_p}^1=\spec \mathbb{F}_p$,点$(p,f)\in\mathbb{A}_\zz^1$将对应于多项式$f$在代数闭包$\bar{\mathbb{F}}_p$中的根给出的$\mathbb{A}_{\mathbb{F}_p}^1$中的点(回忆,$\mathbb{A}_{\mathbb{F}_p}^1$中的点对应于Galois群$\mathrm{Gal}\left(\bar{\mathbb{F}}_p/\mathbb{F}\right)$作用在$\mathbb{F}_p$上的轨道)。类似地,一般点$(0)$处的纤维是概形$\mathbb{A}_\mathbb{Q}^1=\spec \mathbb{Q}[x]$,$(f)\in\mathbb{A}_\zz^1$交$\mathbb{A}_\mathbb{Q}^1$于$f$在$\bar{\mathbb{Q}}$中的根对应的点。因此,图像如下:

% \pic{18.png}

点$(p)\in\mathbb{A}_\zz^1$的闭包是点$(p)\in \spec \zz$处的纤维$\mathbb{A}_{\mathbb{F}_p}^1$. 其他非闭点,即上面的第三类点,的闭包更有趣。它们将包含点$(f)$本身,它在点$(0)$处的纤维$\mathbb{A}_\mathbb{Q}^1$中,以及所有点$(p,g)\in\mathbb{A}_\zz^2$,其中$g$是$\bar{\mathbb{F}}$上$f$的一个因子,即对$\mathbb{A}_\zz^1$的每一条纤维$\mathbb{A}_{\mathbb{F}_p}^1$,$\mathbb{A}_{\mathbb{F}_p}^1$中点的并对应于$f$模$p$后的根。

\begin{exe}
	在上图中打问号的点是上面?为什么点$(4x+1)$以及$(x-2)$的闭包是相切于点$(3,x-2)$且都横截$(3)$的闭包的那两条曲线?(见\nottran)为什么点$(4x+1)$的闭包画出来在点$(2)\in\spec \zz$上有一个竖直的渐近线?
\end{exe}

% \pic{19.png}

\begin{exe}
	确定上图中三个没有标记的点是什么?
\end{exe}

\subsection{$\spec \mathbb{Z}$上的圆锥曲线}
% \subsection{$\mathbb{A}_K^1$中的双重点}