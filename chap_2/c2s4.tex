\section{算数概形}

我们最后一类概形是有限生成约态环但不包含任意域的环的谱。一般地,有限生成$\zz$-代数的谱被称为\textit{算术概形}\index{概形!算术},它们主要在数论中出现,尽管并非所有数论学家感兴趣的概形都具有这种形式。在这些例子中,我们将看到概形在算术以及几何的观点之间的惊人统一的许多线索。\nottran

\subsection{$\spec \mathbb{Z}$}

我们从最明显的例子开始,概形$\spec \zz$. 环$\zz$的素理想当然是$(p)$,其中$p$是素数,以及$(0)$,前者对应于$\spec \zz$的闭点,剩余类域是$\mathbb{F}_p$,后者是一个“一般”点,闭包是整个$\spec \zz$而剩余类域是$\mathbb{Q}$. 图像如下:\\
% \pic{15.png}\\ 
这与域上的直线$\mathbb{A}_K^1$在形式上是相似的,事实上,这只是一系列类比的开始,因此在看下面的例子时应将其铭记在心。然而,这种类比也有着它的不足:尽管$\spec \zz$表现得很像$\mathbb{A}_K^1$,比如,$\mathbb{A}_K^1$是$\mathbb{P}_K^1$的稠密开子概形,而$\spce \zz$不是任何一个概形的稠密开子概形。

\subsection{数域中整数环的谱}

其次,考虑概形$\spec A$,其中$A\subset K$是数域$K$中的整数环。作为例子,我们将分析$K=\mathbb{Q}[\sqrt{3}]$以及$A=\mathbb{Z}[\sqrt{3}]$. 正如$\spec \zz$的情况,$\spec A$有两类点,一类对应于$A$的非零素理想,具有有限剩余类域,另一类是一个一般点,对应于$(0)$,剩余类域是$K$. 含入映射$\zz\hookrightarrow A$诱导的映射$\spec A\to \spec \zz$让这个例子变得有趣起来。比如说,考虑点$[(p)]\in \spec \zz$处的纤维,这就是$A$中包含理想$pA\subset A$的所有素理想的集合,他将表现为下面三种方式中的一个(关于这里没解释的材料的一个好的基础参考文献是Serre [1979]):

% \subsection{$\spec \mathbb{Z}$上的仿射空间}
% \subsection{$\spec \mathbb{Z}$上的锥}
% \subsection{$\mathbb{A}_K^1$中的双重点}