\section{算数概形}

我们最后一类概形是有限生成约态环但不包含任意域的环的谱。一般地,有限生成$\zz$-代数的谱被称为\textit{算术概形}\index{概形!算术},它们主要在数论中出现,尽管并非所有数论学家感兴趣的概形都具有这种形式。在这些例子中,我们将看到概形在算术以及几何的观点之间的惊人统一的许多线索。\nottran

\subsection{$\spec \mathbb{Z}$}

我们从最明显的例子开始,概形$\spec \zz$. 环$\zz$的素理想当然是$(p)$,其中$p$是素数,以及$(0)$,前者对应于$\spec \zz$的闭点,剩余类域是$\mathbb{F}_p$,后者是一个“一般”点,闭包是整个$\spec \zz$而剩余类域是$\mathbb{Q}$. 图像如下:

% \pic{15.png}

\noindent 这与域上的直线$\mathbb{A}_K^1$在形式上是相似的,事实上,这只是一系列类比的开始,因此在看下面的例子时应将其铭记在心。然而,这种类比也有着它的不足:尽管$\spec \zz$表现得很像$\mathbb{A}_K^1$,但比如$\mathbb{A}_K^1$是$\mathbb{P}_K^1$的稠密开子概形,而$\spec \zz$不是任何一个概形的稠密开子概形。

\subsection{数域中整数环的谱}

其次,考虑概形$\spec A$,其中$A\subset K$是数域$K$中的整数环。作为例子,我们将分析$K=\mathbb{Q}[\sqrt{3}]$以及$A=\mathbb{Z}[\sqrt{3}]$. 正如$\spec \zz$的情况,$\spec A$有两类点,一类对应于$A$的非零素理想,具有有限剩余类域,另一类是一个一般点,对应于$(0)$,剩余类域是$K$. 含入映射$\zz\hookrightarrow A$诱导的映射$\spec A\to \spec \zz$让这个例子变得有趣起来。比如说,考虑点$[(p)]\in \spec \zz$处的纤维,这就是$A$中包含理想$pA\subset A$的所有素理想的集合,他将表现为下面三种方式中的一个(关于这里没解释的材料的一个好的基础参考文献是Serre [1979]):

\begin{enumerate}[{(1)}]\setlength{\itemsep}{0pt}
	\item 如果$p$整除$K$在$\mathbb{Q}$上的判别式$12$,即对$p=2$或$3$,理想$(p)$是$A$中一个理想的平方,我们有
	\[
	2A=(1+\sqrt{3})^2
	\]
	以及,当然,
	\[
	3A=(\sqrt{3})^2.
	\]
	点$(1+\sqrt{3})$和点$(\sqrt{3})\in\spec A$的剩余类环分别为$\mathbb{F}_2$和$\mathbb{F}_3$.

	\item 否则,如果$3$是一个模$p$的二次剩余,于是素理想$(p)$能分解为两个不同素理想的乘积,比如
	\[
	11A=(4+3\sqrt{3})(4-3\sqrt{3})
	\]
	以及
	\[
	13A=(4+\sqrt{3})(4-\sqrt{3}).
	\]
	在这些点的剩余类域依然是元素个数为素数的有限域,分别是$\mathbb{F}_{11}$以及$\mathbb{F}_{13}$.

	\item 最后,如果$p>3$以及$3$不再是一个模$p$的二次剩余,比如$p=5$或者$7$时,理想$pA$依然是素理想,它对应于$\spec A$中的一个点。在这种情况下,剩余类域是$\mathbb{F}_p$的二次扩张,比如在上面的两个例子中是$\mathbb{F}_{25}$以及$\mathbb{F}_{49}$.
\end{enumerate}

一般地,,如果$K$是一个二次数域,$A$是$K$中的代数整数环,那么包含$Z\subset A$诱导了概形的映射$\psi:\spec A\to \spec \zz$,$\psi$在每一个闭点$(p)\in \spec \zz$处的纤维是下面的可能中的一个:

\begin{enumerate}[{(1)}]\setlength{\itemsep}{0pt}
	\item 一个非约态的点,其坐标环同构于$A/\pp^2$. 他在$\spec \zz$中的像是约态点$\pp$,而剩余类域是$\mathbb{F}_p$,如果$p$在$A$中\textit{分岔},即$pA$是$A$中一个素理想$\pp$的平方。

	\item 两个约态点$\pp$和$\pp'$的并,剩余类域是$A/\pp=A/\pp'=\mathbb{F}_p$,如果$pA$是$A$中两个不同素理想的乘积。

	\item 一个约态点,它的剩余类域$A/\pp$是$\mathbb{F}_p$的二次扩张,如果$p$在$A$中仍然是素的。
\end{enumerate}

在每个粒子中,纤维的坐标环作为$\mathbb{F}_p$-代数,维数都是2. 这是因为$A$是一个秩为2的自由$\zz$-模。这里感兴趣的类比是,映射$\spec A\to \spec \zz$与Riemann面的一个分支覆盖(或者更一般地,一个代数闭域,比如$\cc$上的一维概形)。 基本上,我们能将$\spec A$想象成$\spec \zz$的两层覆盖然后在“分岔”素数上分支,比如就像,$\spec \cc[z]$是$\spec \cc[z^2]$的一个双重覆盖,它在原点处分支。对于$\spec \zz$处的点,一个明显的不同是,分岔点处我们有两个不同的重数为1的点,但在不同于分岔点的点$(p)\in \spec \zz$处,我们只有一个重数为1的点,其剩余类域是$\mathbb{F}_p$的二次扩张,这样的点在下图中我们以均匀的灰色点表示:

% \pic{16.png}

% \subsection{$\spec \mathbb{Z}$上的仿射空间}
% \subsection{$\spec \mathbb{Z}$上的锥}
% \subsection{$\mathbb{A}_K^1$中的双重点}