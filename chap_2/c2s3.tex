\section{非约态概形}

我们现在离开那些能被看作概形的对象,即那些仿射概形$\spec R$而$R$是有一个有限生成代数闭域上的代数,且没有幂零元。这里的现象将变得不再那么熟悉,我们需要在他们身上花费更多努力。

这类概形在一些足够简单的几何背景下就已经出现了:比如,下面处理的多重点已经作为两个寻常概形的相交或者作为映射“退化的”纤维出现了,正如Exercise \ref{exe.2.2}中看到的。另一类非约态概形重要的应用在一族簇的理论(theory of families of varieties):形变理论(deformation theory)以及模理论(moduli theory)。我们将解释如何对一族单参数的簇取极限,以及引入关键概念,平坦性。最后,我们将给出一些非约态概形的例子,就它们自身而言都是有趣的对象。

从最简单的情况入手,我们将关注仿射空间$\mathbb{A}_K^n$的子概形,其支撑包含原点 ------ 等价地,它由一个理想$I$给出,其零点集$V(I)$作为集合包含$(0$, $\cdots$, $0)$. (回忆一个概形的支撑是指它自有的那个拓扑空间。)

\subsection{双重点}

\begin{exa}
	此类概形最简单的例子是$\mathbb{A}_K^1$由理想$(x^2)$定义的子概形 ------ 即概形$\spec K[x]/(x^2)$,通过商映射$K[x]\to K[x]/(x^2)$诱导的映射看作$\mathbb{A}_K^1$的子概形。这个概形只有一个点,对应于理想$(x)$,
\end{exa}

\subsection{多重点}

\subsubsection*{度与重数}
\addcontentsline{toc}{subsubsection}{度与重数}

\subsection{嵌入点}

\subsubsection*{准素分解}
\addcontentsline{toc}{subsubsection}{准素分解}

\subsection{概形的平坦族}

\subsubsection*{极限}
\addcontentsline{toc}{subsubsection}{极限}

\subsubsection*{例子}
\addcontentsline{toc}{subsubsection}{例子}

\subsubsection*{平坦性}
\addcontentsline{toc}{subsubsection}{平坦性}

\subsection{多重直线}