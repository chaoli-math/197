\section{用点函子刻画空间}\label{s:6.2}

描述一个有趣概形的点函子比起给出它的一个直接构造,不管是黏出来的还是
在仿射或射影空间中由方程显示给出来的,往往更容易且有时更有启发性。
典型的例子来自于Hilbert概形和其他模(moduli)问题,那里我们想要一个
自然的空间,其点表示了一些几何对象。但是,在讨论远简单的对象时,
这个观点同样有用,比如说纤维积或者射影空间本身。

在任何情况中,基本的想法是首先确定一个从概形范畴到集合范畴的函子,
然后证明一个存在性定理表明存在一个概形使得这个函子是点函子。当然,
为使这个程序可行,一个必要的组成部分的是一个局部(且容易检验的)判据
来证明这个函子可表,下面我们给出这样一个判据。

\subsection{概形间函子的刻画}\label{s:6.2.1}
\subsection{参数空间}\label{s:6.2.2}
\subsection{用点函子表示的切空间}\label{s:6.2.3}
\subsection{模空间}\label{s:6.2.4}
