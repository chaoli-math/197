% !TeX root = ../main.tex
在本书第一章的最后,我们讨论一种将概形范畴嵌入到更大的从
概形范畴到集合范畴的反变态射函子构成的范畴的方法。

至少在以下三个方面,这种嵌入很有用的:

\begin{compactenum}[(1)]
\item 使用点函子来描述的话,许多基础的构造(比如积)
的实现要比在概形上方便很多。
\item 为构造出一个特定的概形,如果这个概形存在,构造一个
这个概形的点函子往往更加方便,而构造问题此时则约化到证明
这个函子是否可表以及使用Yoneda引理(\ref{s:6.1})。这个
过程非常类似于分析中对于分布的使用:在那里,当尝试证明
一个给定微分方程的函数解的存在性的时候,我们会首先证明分
布解的存在性,然后留下的就是\textit{正则性}问题
(可能会更便利),证明这个分布是由一个函数的积分所表示的。
\item 概形几何的许多方面都可以拓展到函子范畴中去,所以有
时忘掉函子的表示,直接在范畴里面工作是方便的。
\end{compactenum}

在这章中,我们将展示这三点,首先会引入一些基本构造,
然后是一些例子,它们来自于对参数化概形族的尝试。
我们将在第 \ref{s:6.2.4} 节中看到,很多时候它们引出的函子
并不是真的概形,尽管它们确实是闭的。

% p.252

\section{点函子}\label{s:6.1}

我们从第 \ref{chap:1} 章的最后留下的东西开始。回忆,
概形$X$的\textit{点函子}是一个函子
\[
	h_X:(\text{scheme})^\circ \to (\text{sets})
\]
其中$(\text{scheme})^\circ$和$(\text{sets})$分别代表箭头
反过来的函子范畴以及集合范畴。$h_X$将概形$Y$变成集合
\[
	h_X(Y)=\Mor(Y,X),
\]
而将态射$f:Y\to Z$变成集合间映射
\[
	h_X(Z)\to h_X(Y),
\]
它将$g\in h_X(Z)=\Mor(Z,X)$变成复合$g\circ f\in \Mor(Y,x)$.
我们称一个函子$F:(\text{scheme})^\circ \to (\text{sets})$
是\textit{可表的},如果它具有$h_X$的形式,其中$X$是一个概形。
从下面的Yoneda引理,$X$如果存在则是唯一的,此时我们会称
$X$\textit{表示}了$F$. 集合$h_X(Y)$被称为$X$的$Y$-值点的
集合(如果$Y=\spec T$是仿射的,我们经常会以$h_X(T)$来记
$h_X(\spec T)$,然后称之为$X$的$T$-值点的集合)。

同样回忆,这个构造定义了一个函子
\[
	h:(\text{scheme})\to \mathrm{Fun}
	((\text{scheme})^\circ ,(\text{sets}))
\]
(其中函子范畴的态射是自然变换),将
\[
	X\mapsto h_X
\]
以及对每一个态射$f:X\to X'$给出一个自然变换
$h_X\to h_{X'}$,使得对任意概形$Y$,它将
$g\in h_X(Y)=\Mor(Y,X)$变成复合
$f\circ g\in h_{X'}(Y)=\Mor(Y,X')$.

为使这些概念派上哪怕一点的作用,首先关键的事实就是
函子$h_X$确实决定了概形$X$. 它来自于一个基础的范畴论事实。

\begin{lem}[Yoneda引理]\label{lem:6.1}
令$\scr C$是一个范畴,而$X$, $X'$是$\scr C$的对象。
\begin{compactenum}[(\rm a)]
\item 如果$F$是从$\scr C$到集合范畴的任意反变函子,
则从$\Mor(-,X)$到$F$的自然变换自然对应于$F(X)$中的元素。
\item 如果从$\scr C$到集合范畴的函子$\Mor(-,X)$和
$\Mor(-,X')$是同构的,则$X\simeq X'$. 更一般地,
从$\Mor(-,X)$到$\Mor(-,X')$的函子间的映射与从
$X$到$X'$的映射相同,即,将$X$变为$h_X$的函子
\[
	h:\mathscr C\to \operatorname{Fun}
	(\mathscr C^\circ,(\text{sets}))
\]
是一个$\mathscr C$到函子范畴的一个满子范畴的等价。
\end{compactenum}
\end{lem}

% p.253

\begin{proof}
对(a),将映射$\alpha:\Mor(-,X)\to F$变为元素$\alpha(1_X)$,
其中$1_X:X\to X$是恒等映射。其逆将$p\in F(X)$变为
映射$\alpha$,它将$f\in \Mor(Y,X)$变为$F(f)(p)\in F(Y)$. 
对(b),我们能将(a)应用到函子$F=\Mor(-,Y)$.
\end{proof}

下一命题比 Lemma \ref{lem:6.1} 更进一步,说明了我们只需
关注限制在仿射概形范畴的点函子,或者等价地,反箭头的
交换环范畴$\text{(rings)}^\circ$,则同样的事情在相对
语境中也成立。

\begin{pro}\label{pro:6.2}
如果$R$是一个交换环,一个$R$-概形由其限制在仿射$R$-概形
上的点函子所确定,实际上,
\[
	h:(\text{$R$-schemes})\to \operatorname{Fun}
	((\text{$R$-algebras}),(\text{sets}))
\]
是一个$R$-概形范畴与函子范畴的一个满子范畴的等价。
\end{pro}

当然,一个仿射概形范畴上的反变函子与环范畴上的协变函子
相同,所以,我们可以一般地认为,我们的反变可表函子
$h_X:(\text{schemes})^\circ \to (\text{sets})$是$R$-代数
范畴上的协变函子。(如果我们需要标记这个差异,则用
$h_X^*:(\text{rings})\to (\text{sets})$来标记函子
$h_X^*(A)=h_X(\spec A)$,其中$A$是任意的$R$-代数。)

\begin{proof}
实际上,该命题不过是在说概形由仿射概形所拼成。
令$S=\spec R$. 记$h_X$为函子$\Mor_S(-,X)$在仿射$S$-概形
范畴的限制。我们只要说明任何的自然变换
$\varphi:h_X\to h_{X'}$来自于唯一的$S$-概形态射
$f:X\to X'$. 为从$\varphi$构造$f$,令$\{U_\alpha\}$
为$X$的一个仿射开覆盖,将$\varphi$应用到限制映射
$U_\alpha\subset X$得到态射$U_\alpha\to X'$. 这些态射
满足为定义所需态射$f$需要的相容性条件。唯一性来自于如下
观察,两个$X$到$X'$的态射是不同的,则它们限制在某个
$U_\alpha$上不同。
\end{proof}

\begin{exe}
假设$X$(就像我们感兴趣的所有概形那样)是
\textit{局部Noether的},即,由Noether环的谱所覆盖。
证明,$X$由$h_X$在Noether环范畴上的限制所确定。
\end{exe}

% p.254

\subsection{开与闭子函子} \label{s:6.1.1}

将概形看成函子的用处之一在于,我们可以将一些概形的几何的
基本概念推广到函子上。我们下面将考虑一些例子。

我们首先展示如何定义一个函子
\[
	F\in \operatorname{Fun}((\text{rings}),(\text{sets}))
\]
的开子函子。我们称范畴$\mathscr C$到集合范畴的函子间的
映射$\alpha:G\to F$是\textit{单}的,如果对每一个对象$X$,
诱导的集合间映射$G(X)\to F(X)$是一个单射
(这有关于一个标准的范畴论概念,不过我们这里并不需要)。
此时,我们称$\alpha:G\to F$是一个$F$的子函子。比如,
如果$U\subset X$是一个子概形,函子$h_U$就是$h_X$的一个
子函子。

我们想要将一个函子$F$的开子函子定义为一个子函子,满足当限制
到可表子函子$h_X\subset F$时,具有形式$h_U\subset h_X$,
其中$U$是$X$的一个子概形。为实现它,我们需要引入函子的
纤维积这个概念。

\begin{defi} \label{defi:6.4}
	如果$A$, $B$和$C$都是从一个范畴$\mathscr C$到集合范畴
	的函子,如果$f:A\to C$和$g:B\to C$都是函子间映射,
	则纤维积$A\times_C B$为一个从$\mathscr C$到集合范畴的
	函子,它将$\mathscr C$中的任意对象$Z$变为
	\[
		(A\times_C B)(Z)=\left\{(a,b)\in A(Z)\times B(Z)
		\;|\; f(a)=f(b)\text{ in } C(Z)\right\},
	\]
	对于态射的变换是显然的。
\end{defi}

\begin{defi} \label{defi:6.5}
一个$\operatorname{Fun}((\text{rings}),(\text{sets}))$中
的子函子$\alpha:G\to F$是一个\textit{开子函子},
如果对每一个由一个仿射概形$\spec R$表示的函子上的映射
$\psi:h_{\spec R}\to F$,函子的纤维积
\[
	\xymatrix{
		G_\psi\ar[r]\ar[d]&h_{\spec R}\ar[d]^\psi\\
		G\ar[r]^\alpha& F
	}
\]
都产生一个映射$G_\psi\to h_{\spec R}$同构于,某个
$\spec R$的开子概形所表函子的单态射。
\end{defi}

\begin{exe} \label{exe:6.6}
令$X=\spec R$为一个仿射概形。证明,$h_X$的开子概形
正是那些对某个理想$I\subset R$定义的子函子
\[
	F(T)=\{\varphi\in h_X(T)\; | \; \varphi^*(I)T=T\}.
\]
\end{exe}

\begin{exe} \label{exe:6.7}
令$X$为域$K$上的概形。定义函子
$F:(\text{schemes}/K)^\circ \to (\text{sets})$如下:
对每一个$K$-概形$Y$,令$F(Y)$为$X\times_K Y$中在$Y$上
平坦的闭子概形$Z$的集合,此外,每个$Z$还满足,在$Y$的
任意闭点上的纤维都是$X$的度为$2$的子概形。令$G$是$F$
的子函子,通过再加上要求,$Z$在$Y$的任意闭点上的纤维
是约态的。证明$G\subset F$是开的。
\end{exe}

% p.255

闭子函子的定义是类似的。一个
$\operatorname{Fun}((\text{rings}),(\text{sets}))$中的
子函子$\alpha : G\to F$是\textit{闭的},如果对每一个
映射$\psi:h_{\spec R}\to F$,$\psi$和$\alpha$的纤维积
是一个$h_{\spec R}$的子函子,其同构于$\spec R$的一个
闭子概形所表的函子。

\begin{exe} \label{exe:6.8}
	令$X=\spec R$为一个仿射概形。证明,$h_{\spec R}$的
	开与闭子函子正是那些由$\spec R$的开与闭子概形所表示
	的函子。(对任意概形,稍难一些,这也是对的。)
\end{exe}

如同往常,使用这些概念需要稍加小心。比如:

\begin{exe} \label{exe:6.9}
	假设$f$是从范畴$\mathscr C$到集合范畴的函子,$G$是$F$
	的一个子函子。证明对应$C\mapsto F(C)\setminus G(C)$
	可能不能定义一个函子。现在,假设$\mathscr C$是环范畴,
	$F$由概形$X$所表,而$G$被$X$的闭子概形$Y$所表,函子$H$
	表示了开子概形$X\setminus Y$. 你能用$G$和$F$来表示$H$吗?
\end{exe}

\begin{exe} \label{exe:6.10}
	令$X$是域$K$上的概形。定义函子$F:(\text{scheme}/K)^\circ
	\to (\text{sets})$
	同 Exercise \ref{exe:6.7},且定义$F$的一个
	子函子$H$通过对任意$K$-概形$Y$,令$H(Y)$为闭子概形$Z
	\subset X\times Y$,其在$Y$上平坦,且$Z$的任意在$Y$的
	闭点处的纤维都是$X$支于单点的次数为$2$的子概形。证明
	$H$一般不是$F$的一个闭子函子。
\end{exe}

我们下面将使用函子的\textit{开覆盖}这个概念。
这是一族开子函子,当我们拉回到一个可表函子时,其产生了
概形的一个开覆盖。更准确地,令$F:(\text{schemes})^\circ
\to (\text{sets})$
为一个函子%
\footnote{译者注:原书为$F:(\text{schemes})
\to (\text{sets})$,应该是笔误。}%
,考虑$\{G_i\to F\}$为$F$为开子函子族。对
每个从可表函子$h_X$到$F$的映射$h_X\to F$,存在开子概形
$U_i\subset X$使得纤维积$h_X\times_F G_i$为$h_{U_i}$.
我们称$\{G_i\to F\}$是一个开覆盖,如果对任意映射
$h_X\to F$,对应的开子概形们$U_i\subset X$覆盖了$X$.

注意:如果$\{G_i\to F\}$是$F$的一个开覆盖,其无需
对任意概形$T$满足$F(T)=\bigcup G_i(T)$. (比如,考虑
\[
	F=h_{\spec \zz},\quad G_1=h_{\spec \zz[1/p]},
	\quad G_2=h_{\spec \zz[1/q]},
\]
其中$p$和$q$为不同的素数。于是$\{G_i\to F\}$是一个
开覆盖,但$F(\spec \zz)$只包含一个点(恒等映射),
而对$i=1,2$都有$G_i(\spec \zz)=\varnothing$.)然而,
对足够局部平坦的概形$T$,开覆盖$\{G_i\to F\}$产生了
覆盖$F(T)=\bigcup G_i(T)$. 比如:

% p.256

\begin{exe} \label{exe:6.11}
	令$\{G_i\to F\}$为函子$F:(\text{schemes})^\circ
	\to (\text{sets})$的一族开子概形。证明,
	这是一个开覆盖当且仅当对任意域$K$都有
	$F(\spec K)=\bigcup G_i(\spec K)$.
\end{exe}

\subsection{\texorpdfstring{$K$}{K}-有理点}
\label{s:6.1.2}

如果$X$是一个域$K$上的概形,$X$在$K$上的$K$-值点是
映射$\spec K\to X$,其复合上自然映射$X\to \spec K$
将得到恒同。我们断言,这样的映射一一对应于$X$的
\textit{在$K$上有理}(或\textit{$K$-有理})的闭点,
即剩余类域$\kappa(p)$为$K$的点(通过$K$到$X$在点$p$
的局部环的含入映射)。实际上,因为$\spec K$没有
非平凡的开覆盖,一个从$\spec K$到$X$的映射将映入$X$
的某个仿射开子概形$\spec T$中,于是,这样一个映射由
一个$K$-代数映射$T\to K$确定,即由一个$T$的剩余类域
为$K$极大理想确定。反过来,我们可以把构造倒过来,
则任意$K$-有理闭点$p$给出了一个唯一的$K$-概形态射
$\spec K\to X$.

我们再次警告,以前选择在$S$-概形范畴而不在全部概形
范畴工作的理由这里也适用。譬如,当在复代数簇上工作,
可能会期待$h_{\spec \zz}(\spec \cc)$为一个单点
(恒等元),这对$\cc$-概形范畴是对的,
但对任意概形的范畴,这个集合非常大!

\begin{exe}\label{exe:6.12}
令$X=\spec \cc$,将其考虑为一个抽象概形,即一个$\zz$
上的概形。描述$\spec \cc$的所有$\cc$-值点的集合
$h_X(\spec \cc)$.
\end{exe}

\subsection{函子的切空间} \label{s:6.1.3}

有时候,比起概形的方程,其点函子,在计算它的几何信息
时会更方便!一个典型的例子出现在Zariski切空间。(我们
将看到这将应用到第 \ref{s:6.2.3} 节。)

我们将在给定域$K$上的概形上工作,所有的态射都是$K$上的
态射。

回忆,若$X$是一个概形,则对任意$K$-有理点$p\in X$,
其上的Zariski切空间为$\Hom_K(\mm/\mm^2,K)$,其中
$\mm=\mm_{X,p}$为$X$在点$p$的局部环的极大理想,而
$K=\kappa(p)=\mathscr O_{X,p}/\mm_{X,p}$为$X$在点$p$的
剩余类域。现在令$X$是$K$-概形%
% 令$X$为仿射概形$\spec K[\varepsilon]/(\varepsilon)^2$. 
\footnote{译者注:原文这里还有一句,“Let $X$ be the
affine scheme $\spec K[\varepsilon]/(\varepsilon)^2$”.
不明所以。}。%
我们断言,
\textit{$X$的一个$K[\varepsilon]/(\varepsilon)^2$-值点
等价于$X$的一个$K$-有理闭点$p$连同一个$X$在点$p$处的
Zariski切空间中的元素。}

% p.257

为验证它,首先注意到由代数映射\footnote{译者注:这里应该是模掉
$\varepsilon$.}%
$K[\varepsilon]/(\varepsilon^2)
\to K$ 诱导的含入映射$\iota :\spec K\to \spec K[\varepsilon]/
(\varepsilon^2)$,态射$\spec K[\varepsilon]/(\varepsilon^2)\to X$
确定了一个态射$\spec K\to X$,进而一个$X$的闭$K$-有理点$p$.
这样一个态射$\spec K\to X$的扩张包含一个$K$-代数同态
$\mathscr O_{X,p}\to \kappa(p)=K$到一个(局部)同态
$\mathscr O_{X,p}\to K[\varepsilon]/(\varepsilon^2)$的提升。
这样一个同态诱导了一个从极大理想$\mm_{X,p}$到$K[\varepsilon]/
(\varepsilon^2)$的极大理想$(\varepsilon)$的映射,因为$(\varepsilon^2)
=0$,这个映射可以经由一个映射$t:\mm_{X,p}/\mm_{X,p}^2\to (\varepsilon)
\cong K$分解。这个映射$t$是$(\mm_{X,p}/\mm_{X,p}^2)^*$,
$X$在$p$的Zariski切空间,的一个元素。

反过来,给定一个$K$-有理点$p\in X$以及$t:\mm_{X,p}/\mm_{X,p}^2\to
(\varepsilon)\cong K$,我们可以从映射$\pi:\mathscr O_{X,p}\to 
\kappa(p)=K$构造一个映射
$\mathscr O_{X,p}\to K[\varepsilon]/(\varepsilon^2)$如下:
$\pi$和$K$-代数结构映射$K\to \mathscr O_{X,p}$定义了一个$K$-矢量空间
$\mathscr O_{X,p}/\mm_{X,p}^2$的分解$K\oplus \mm_{X,p}/\mm_{X,p}^2$;
并且,我们用$K$的恒同映射以及$\mm_{X,p}/\mm_{X,p}^2$上的映射$t$定义
映射
\[
	\mathscr O_{X,p}/\mm_{X,p}^2\to K[\varepsilon]/(\varepsilon^2),
\]
其复合上投影给出映射$\mathscr O_{X,p}\to K[\varepsilon]/
(\varepsilon^2)$. 这个映射确定了想要的概形态射
$\spec K[\varepsilon]/(\varepsilon^2)\to K$.

在第 \ref{s:4.2.1} 节,我们将看到概形的这个切矢量的刻画是如何被用来
给出射影空间的切矢量的一个内蕴描述的,这完善了我们在第 \ref{s:4.2.4}
节中的讨论。

现在如果$F$是一个从$K$-代数范畴到集合范畴的函子,以及$p\in F(K)$,
我们可以定义$F$在点$p$的切空间$T_pF$为$F(K[\varepsilon] /
(\varepsilon^{2})) \to F(K)$在点$p$处的纤维。这定义带出了一个对象,
作为集合的切空间,而不是$K$-矢量空间。至少我们可以$K$的标量乘法:
如果$a\in K$,则映射$\varepsilon\mapsto a\varepsilon$诱导了代数
$K[\varepsilon]/(\varepsilon^2)$的一个自同构,使得如下图
\[
	\xymatrix{
	\varepsilon \ar@{|->}[rr] &  & a\varepsilon\\
	K[\varepsilon]/(\varepsilon)^2 \ar[rr]\ar[rd]& & K[\varepsilon]/(\varepsilon)^2 \ar[ld]\\
	& K &
	}
\]
交换;而这又到出了一个映射$T_pF\to T_pF$,即我们想要的乘标量$a$.
为使$T_pF$是一个矢量空间,我们现在需要定义一个加法映射
$T_pF \times T_pF \to T_pF$. 一般地,似乎好像没法做到,但对那些
保持纤维积的函子,比如可表函子,我们可以做到。

为此,考虑概形$\spec K\left[\varepsilon', \varepsilon''\right] /
\left(\varepsilon', \varepsilon''\right)^{2}$,即平面上一点的极大理想
的平反给出的闭子概形。由投影给出的交换图%
% p.258
\[
	\xymatrix{
	& K[\varepsilon']/(\varepsilon')^2 \ar[rd] &\\
	K[\varepsilon',\varepsilon'']/(\varepsilon',\varepsilon'')^2 \ar[ru]\ar[rd]& & K\\
	& K[\varepsilon'']/(\varepsilon'')^2 \ar[ru] &
	}
\]
将$\spec K[\varepsilon', \varepsilon''] /(\varepsilon', \varepsilon'')
^{2}$表为$K$上$\spec K[\varepsilon'] /(\varepsilon')^{2}$和
$\spec K[\varepsilon''] /(\varepsilon'')^{2}$的纤维积。
另一方面,这有第三个映射
$\sigma: K[\varepsilon', \varepsilon''] /(\varepsilon', \varepsilon''
)^{2} \to K[\varepsilon] /(\varepsilon^{2})$将$\varepsilon'$和
$\varepsilon''$都映到$\varepsilon$. 因此,如果$F$保持代数范畴的纤维积,
或者甚至仅上面的纤维积,我们可以做出图
\[
	\xymatrix{
	F(K[\varepsilon']/(\varepsilon')^2)\times_{F(K)} F(K[\varepsilon'']/(\varepsilon'')^2)\ar[rd]\ar@{=}[r]&F(K[\varepsilon',\varepsilon'']/(\varepsilon',\varepsilon'')) \ar[r]^-{F(\sigma)}\ar[d]& F(K[\varepsilon]/(\varepsilon)^2)\ar[ld]\\
	& F(K) &
	}
\]
然后取$p\in F(K)$处的纤维,我们就得到了想要的加法映射。

\begin{exe}
	验证这些映射将$T_pF$变成了一个$K$-矢量空间,且当$F$是一个可表函子
	时,这就是原先的矢量空间结构。
\end{exe}

\subsection{群概形} \label{s:6.1.4}

点函子在给概形上指定额外的结构时尤其方便,通过给其
指定结构。比方说,我们可以定义\textit{群概形}为概形$G$
以及函子$h_G:(\text{ring})\to (\text{sets})$的一个经由
遗忘函子$(\text{groups})\to (\text{sets})$的分解,即
一个群概形是一个概形,以及对任意$X$,还有一种自然将
$\Mor(X,G)$看作一个群的方式。

从Yoneda引理(\ref{s:6.1}),这等同于给出映射
\[
	G\times G\to G,\quad G\to G\quad \text{and}
	\quad \spec K\to G,
\]
分别代表乘法、逆和恒等元,同时满足一般的规则(结合律等),
但常常会更简单些。比如,$\operatorname{GL}_n$可被定义为
可逆整$n\times n$矩阵的仿射概形,
\[
	\spec \zz[x_{ij}][\det (x_{ij})^{-1}],
\]
但是一般大家会将其想做一个函子,对每个环$T$给出群
$\operatorname{GL}_n(T)$. 这里有趣的地方在于,这族群已经
指明了概形的结构已经附加的结构映射!
