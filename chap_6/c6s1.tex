在本书第一章的最后,我们讨论一种将概形范畴嵌入到更大的从概形范畴到集合范畴的反变态射函子构成的范畴的法子。

这种嵌入至少在三个方面是有用的:

\begin{compactenum}[(1)]
\item 使用点函子来描述的话,许多基础的构造(比如积)的实现要比在概形上方便很多。
\item 为构造出一个特定的概形,如果这个概形存在,构造一个这个概形的点函子往往更加方便,而构造问题此时则约化到证明这个函子是否可表以及使用Yoneda引理(\ref{s:6.1})。这个过程非常类似于分析中对于分布的使用:在那里,当尝试证明一个给定微分方程的函数解的存在性的时候,我们会首先证明分布解的存在性,然后留下的就是\textit{正则性}问题(可能会更便利),证明这个分布是由一个函数的积分所表示的。
\item 概形几何的许多方面都可以拓展到函子范畴中去,所以有时忘掉函子的表示,直接在范畴里面工作是方便的。
\end{compactenum}

在这章中,我们将展示这三个方面,首先会引入一些基本构造,然后是一些例子,它们来自于对参数化概形族的尝试。我们将在第 \ref{s:6.2.4} 节中看到,很多时候它们引出的函子并不是真的概形,尽管它们确实是闭的。

\section{点函子}\label{s:6.1}

我们从第 \ref{chap:1} 章的最后留下的东西开始。回忆,概形$X$的\textit{点函子}是一个函子
\[
	h_X:(\text{scheme})^\circ \to (\text{sets})
\]
其中$(\text{scheme})^\circ$和$(\text{sets})$分别代表箭头反过来的函子范畴以及集合范畴。$h_X$将概形$Y$变成集合
\[
	h_X(Y)=\Mor(Y,Z),
\]
而将态射$f:Y\to Z$变成集合间映射
\[
	h_X(Z)\to h_X(Y),
\]
它将$g\in h_X(Z)=\Mor(Z,X)$变成复合$g\circ f\in \Mor(Y,x)$. 我们称一个函子$F:(\text{scheme})^\circ \to (\text{sets})$是\textit{可表的},如果它具有$h_X$的形式,其中$X$是一个概形。从下面的Yoneda引理,$X$如果存在则是唯一的,此时我们会称$X$\textit{表示}了$F$. 集合$h_X(Y)$被称为$X$的$Y$-值点的集合(如果$Y=\spec T$是仿射的,我们经常会以$h_X(T)$来记$h_X(\spec T)$,然后称之为$X$的$T$-值点的集合)。

同样回忆,这个构造定义了一个函子
\[
	h:(\text{scheme})\to \mathrm{Fun}((\text{scheme})^\circ ,(\text{sets}))
\]
(其中函子范畴的态射是自然变换),将
\[
	X\mapsto h_X
\]
以及对每一个态射$f:X\to X'$给出一个自然变换$h_X\to h_{X'}$,使得对任意概形$Y$,它将$g\in h_X(Y)=\Mor(Y,X)$变成复合$f\circ g\in h_{X'}(Y)=\Mor(Y,X')$.

为使这些概念派上哪怕一点的作用,一个首先关键的事实就是函子$h_X$确实决定了概形$X$. 它来自于一个基础的范畴论事实。

\begin{lem}[Yoneda引理]
令$\scr C$是一个范畴,而$X$, $X'$是$\scr C$的对象。
\begin{compactenum}[(\rm a)]
\item 如果$F$是从$\scr C$到集合范畴的任意反变函子,则从$\Mor(-,X)$到$F$的自然变换自然对应于$F(X)$中的元素。
\item 如果从$\scr C$到集合范畴的函子$\Mor(-,X)$和$\Mor(-,X')$是同构的,则$X\cong X'$
\end{compactenum}
\end{lem}