\renewcommand\chapterimg{Pictures/0.png}

\chapter{例子}
\section{代数闭域上的约态概形}

概形概念的发始,即是想要推广经典的代数闭域上的仿射簇,我们将从它开始一系列关于概形的例子。在我们的上下文中,代数闭域$K$上的仿射簇是指一个仿射概形$\spec R$,其中$R$是一个簇$X$的坐标环,即$R$是一个有限生成的约态$K$-代数。(回忆一个环是约态的,是指它没有非零幂零元。)$\spec R$时常被叫做\textit{关联于簇$X$}的概形:这种概形有时就直接被称作仿射簇。在稍后的章节中,我们将考虑不同于这个基本模型的概形。

关联于代数闭域$K$上的仿射簇的$K$-概形与这个簇是等价的对象:它们都可以决定一个相同的坐标环或者被一个相同的坐标环决定。但是,在这个例子中已经有所表现的是,一些经典的概念,比如簇的相交或者映射的纤维,在概形理论中会有更精确的意思。我们将在这节以及接下来的章节中看到这个现象的例子。

\subsection{仿射空间}

我们从概形$\mathbb{A}^n_K:=\spec K[x_1$, $\cdots$, $x_n]$开始,其中$K$是一个代数闭域。这个概形被称为$K$上的$n$维仿射空间。

这里将使用一个标准的但绝不平凡的一个代数的结论------Hilbert's Nullstellensatz(零点定理)的一种形式;比如可见Eisenbud [1995].

\begin{thm}[Nullstellensatz]
	设$K$是一个域,如果$\mathfrak{m}$是多项式环$K[x_1$, $\cdots$, $x_n]$的一个极大理想(或者更一般地,$p$是$K$上的一个代数空间的子簇的一个闭点),则
	\[
		K[x_1\text{, }\cdots\text{, }x_n]/\mathfrak{m}=\kappa(p)
	\]
	是一个有限维$K$-矢量空间。
\end{thm}

在我们的例子中,$K$是代数闭的,这就意味着$\kappa(p)=K$. 因此,记$\lambda_i$是$x_i$在$\kappa(p)$中的像,可以看到
\[
	\mathfrak{m}=(x_1-\lambda_1\text{, }\cdots\text{, }x_n-\lambda_n).
\]
于是,$\mathbb{A}^n_K$中的闭点与$K$中元素的$n$元组相对应,正如期望的那般。有时,我们会用“点$(\lambda_1$, $\cdots$, $\lambda_n)$”来指代“点$[(x_1-\lambda_1$, $\cdots$, $x_n-\lambda_n)]$”. 

从一维的情况开始,仿射直线
\[
	\mathbb{A}^1_K=\spec K[x]
\]
长得就像它的经典对应,同样被叫做仿射直线的仿射簇。仿射直线对每一个$\lambda\in K$都有一个闭点,闭点构成的集合上的Zariski拓扑也与仿射簇上的经典的Zariski拓扑相同:开集是有限集的补。概形$\mathbb{A}^1_K$与簇不同的地方只在于,它多了一个点,对应于零理想$(0)$,这个点被称为\idx{一般点}。点$(0)$的闭包是整个$\mathbb{A}^1_K$,因此,$\mathbb{A}^1_K$的闭子集就是$\mathbb{A}^1_K-\{(0)\}$的有限子集。

\pic{chap_2/1.eps}

仿射平面$\mathbb{A}^2_K=\spec K[x,y]$同样与它的经典对应相似,但是现在附加的点就变多也表现得更加有趣。像之前一样,我们有闭点,来自于极大理想$(x-\lambda,y-\mu)$,对应于寻常平面中的$(\lambda,\mu)$. 但现在却有两种类型的非闭点。首先,对每一个不可约多项式$f(x,y)\in K[x,y]$,有一个点$p$对应于素理想$(f)\subset K[x,y]$,它的闭包包含$p$本身以及所有使得$f(\lambda,\mu)=0$的闭点$(\lambda,\mu)$,点$(f)$被称为这个集合的一般点。更一般地,概形中的每一个点都被称为这个点闭包的一般点。比起簇$\mathbb{A}^2_K$,我们需要对每一条不可约平面曲线加入一点,这个新的点属于这条曲线的闭包,并且这个点的闭包等于曲线的闭包。最后,正如在$\mathbb{A}^1_K$中,我们还需要加入一个对应于零理想的点,即$\mathbb{A}^2_K$的一般点,它的闭包是整个$\mathbb{A}^2_K$.

\pic{chap_2/2.eps}

因为$K[x,y]=K[x]\otimes_K K[y]$,从定义有
\[
	\mathbb{A}^2_K=\mathbb{A}^1_K\times_{\spec K}\mathbb{A}^1_K.
\]
在这里,似乎纤维积的概念是簇的积的概念的正确对应,但概形的纤维积作为集合并不等于簇的积中的点构成的集合。

仿射空间$\mathbb{A}^n_K$的情况是上一个例子的直接推广:几何上,我们可以看到概形$\mathbb{A}^n_K$是经典的$n$-维仿射空间对每一个正维数的不可约子簇$\Sigma$加上一点$p_\Sigma$. 和上面的例子一样,$p_\Sigma$属于$\Sigma$的闭包中,且$p_\Sigma$的闭包等于$\Sigma$的闭包,正如$\Sigma$的一般点。

更一般的,设$X\subset \mathbb{A}^n_K$是任意的仿射簇,对应于理想$I\subset \mathbb{A}^n_K$以及坐标环$R=K[x_1$, $\cdots$, $x_n]/I$,我们能将其关联到仿射概形$\spec R$;通过商映射$K[x_1$, $\cdots$, $x_n]\to R$可以将其看作$\mathbb{A}^n_K$的一个子概形。这个概形,与$\mathbb{A}^n_K$类似,长得就像仿射簇$X$除了我们需要对每一个正维数的不可约子簇$\sigma\subset X$加入一个新的一般点$p_\Sigma$.

\nottran % 纤维,或者更一般的原像,即使在经典代数几何中也可能会出现。

\begin{exe}
设$\varphi:K[x]\to K[x]$是一个环同态,他将$x$变成$x^2$,考虑$\varphi$诱导的$\spec K[x]$到自身的映射。证明,在概形意义上,点$0$的纤维是由$(x^2)$定义的子概形。

\pic{chap_2/3.eps}
\end{exe}

在所有的概形中,这些关联于代数闭域上的仿射簇的概形被如下性质的环$R$的谱所刻划,

\begin{itemize}
\item[-] 有限生成
\item[-] 约态代数
\item[-] 在一个域上
\item[-] 这个域是代数闭域
\end{itemize}

为了对更一般的概形长成哪样有所感觉,以及对它们有什么益处有所把握,我们在接下来的几节中考虑当撤去上面的几条四条限制后将会发生什么。我们会主要考虑那些上面四条限制中只有一条不满足的情况,因为理解它们将使得我们可以理解更一般的情况;偶尔在习题中会提到更复杂的例子。

\section{局部概形}

我们的第一类不同于簇的概形的例子是局部环的谱,称之为\idx{局部概形}。我们这里将考虑的例子是代数闭域上的约态代数的谱,但是,不一定是有限生成的。局部概形常常作为技术性工具来研究其他更几何的概形;它们经常被用来将注意力集中在一个仿射概形的局部结构。我们在概形中比起簇加进去的点在下面的例子中将变得更加醒目,在这里例子里,只有一个闭点。当然,用一个点来表示这些概形是错误的,取而代之的,它们应当被看成簇的芽。

% \pic{chap_2/4.eps}
% 
% \pic{chap_2/5.eps}
% 
% \pic{chap_2/6.eps}