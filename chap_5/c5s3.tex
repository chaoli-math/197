\section{奇异概形和判别式}\label{s:5.3}

\subsection{定义}\label{s:5.3.1}

% p.230
在这节中,我们将考虑概形间态射的光滑以及奇异点。为启发我们的定义,
考虑微分流形间的态射$f:M\to N$,并假设$\dim M\geq \dim N$. 这样的态射
表现得最简单的时候是在$x\in M$其上微分$Df_x$是满的:限制到这样的点$x$
的适当领域上,态射$f$看上去就像乘积到其中一个因子的投影一般,此时我们称
$x$是$f$的一个光滑点。

在概形范畴中,Zariski开集往往太大了以至于不能容纳乘积结构,而
(如我们将看到的)可能还有来自于这些点具有不同的剩余类域的困难。但,
依然可能通过局部乘积结构来定义概形间态射的光滑性,如见 Altman, Kleiman [1970],
但在目前的情况下,通过推广流形上的微分刻画来刻画光滑点和奇异点会更有意义。

为此,我们引入环同态的K\"ahler微分模,以及其整体版本,概形间态射的相对余切层。
若$\psi:A\to B$为一个环同态,我们定义\textit{$A$-线性K\"ahler微分模}
$\Omega_{B/A}$,这是一个$B$-模,为由符号$db$所生成的自由$B$-模,其中$b\in B$,
模掉了关系
\[
    d(b_1 b_2)=b_1 d b_2+b_2 d b_1 \quad \text { 对所有 }\,b_1, b_2 \in B
\]
以及 
\[
    d \psi_a=0 \quad \text { 对所有 }\,a \in A.
\]

这些关系保证了同态
\[
    \begin{aligned}
        B & \longrightarrow \Omega_{B / A}, \\
        b & \longmapsto d b,
    \end{aligned}
\]
是一个$A$-线性导子;实际上,这是适当意义下的万有$A$-线性导子。(细节见
Eisenbud [1995].) 于是,容易见到,比如若$B=A[x_1,\dots,x_n]/(f_1,\dots,f_n)$,
则$\Omega_{B/A}$在Jacobi矩阵
\[
    \left(\frac{\partial f_i}{\partial x_j}\right)
\]
的余核中。

为整体化,令$\varphi:X\to Y$为概形间态射。我们如下定义$\varphi:X\to Y$的
\textit{相对余切层},记作$\Omega_{X/Y}$,该层在仿射开集$U\subset X$上
(映到仿射开集$V\subset Y$)取为$\mathscr O_X(U)$的
$\mathscr O_Y(V)$-线性K\"ahler微分模。上面说的仿射开集们在$X$中构成了一个
拓扑基$\mathscr B$. 则$\mathscr B$-层(见第 \ref{s:1.1.3} 节)的公理是
容易检查的,相当于说微分模的构造是局部相容的,于是这些数据确实定义了一个层。

% p.231

概形间态射$\varphi:X\to Y$在点$x\in X$的\textit{相对维数}被定义为差
$\dim (X,x)-\dim(Y,\varphi(x))$.

\begin{defi}\label{defi:5.21}
    令$\varphi:X\to Y$为一个Noether概形间的态射,且假设$\varphi$是平坦的、
    有限型的,且有着常相对余维数$d$. 我们定义$\varphi$的\textit{奇异概形}
    $\sing\varphi\subset X$为$\varphi$的相对余切层的
    第$d$个Fitting理想,即
    \[
        \sing\varphi=V(\fitt_d \Omega_{X/Y})\subset X.
    \]
    当态射$\varphi|_{\sing \varphi}:\sing \varphi\to Y$是有限的,我们定义
    \textit{判别式概形}$\Delta(\varphi)\subset Y$为$\sing \varphi$在$Y$
    中的Fitting像,即
    \[
        \Delta(\varphi)=V(\fitt_0(\varphi_* \mathscr{O}_{\sing \varphi})) 
        \subset Y.
    \]
\end{defi}

\[
    X=\spec A[x_1, \ldots, x_n] /(f_1, \ldots, f_m)
\]

\[
    \begin{pmatrix}
        \partial f_1 / \partial x_1 & \ldots & \partial f_m / \partial x_1 \\
        \vdots & & \vdots \\
        \partial f_1 / \partial x_n & \ldots & \partial f_m / \partial x_n
    \end{pmatrix}
\]

\[
    \varphi: X=\spec K^{\prime} \longmapsto Y=\spec K
\]

\subsection{判别式}

\[
    D_m(f)=\frac{(-1)^{m(m-1) / 2}}{a_0} R_{m, m-1}(f, f^{\prime}).
\]

\subsection{例子}

\[
    \varphi: X=\spec K[t, x] /(f) \rightarrow Y=\spec K[t]
\]

\[
    f(x)=x^k+a_{k-1}(t) x^{k-1}+\ldots+a_1(t) x+a_0(t)
\]

\[
    \varphi_{k, m}: X=X_{k, m}=\spec K[t, x] /\left(x^k-t^m\right) \rightarrow Y=\spec K[t]
\]

\[
    \mu_\alpha: Y=\spec K[t] \longrightarrow Y=\spec K[t]
\]

\[
    \xymatrix{
        X_{k,\alpha m}\ar[r]\ar[d]_{\varphi_{k,\alpha m}}& X_{k,m}\ar[d]^{\varphi_{k, m}}\\
        Y\ar[r]_{\mu_\alpha}& Y
    }
\]

\[
    \delta_{k, \alpha m}=\alpha \cdot \delta_{k, m}
\]

\[
    \delta_{2, m}=\begin{vmatrix}
        1 & 0 & t^m \\
        2 & 0 & 0 \\
        0 & 2 & 0
    \end{vmatrix}=4 t^m
\]

\[
    \begin{aligned}
        \delta_{k, m} & =\left|\begin{array}{ccccccccccc}
        1 & 0 & \ldots & 0 & 0 & 0 & -t^m & 0 & \ldots & 0 & 0 \\
        0 & 1 & \ldots & 0 & 0 & 0 & 0 & -t^m & \ldots & 0 & 0 \\
        \vdots & \vdots & \ddots & \vdots & \vdots & \vdots & \vdots & \vdots & \ddots & \vdots & \vdots \\
        0 & 0 & \ldots & 1 & 0 & 0 & 0 & 0 & \ldots & -t^m & 0 \\
        0 & 0 & \ldots & 0 & 1 & 0 & 0 & 0 & \ldots & 0 & -t^m \\
        k & 0 & \ldots & 0 & 0 & 0 & 0 & 0 & \ldots & 0 & 0 \\
        0 & k & \ldots & 0 & 0 & 0 & 0 & 0 & \ldots & 0 & 0 \\
        \vdots & \vdots & \ddots & \vdots & \vdots & \vdots & \vdots & \vdots & \ddots & \vdots & \vdots \\
        0 & 0 & \ldots & k & 0 & 0 & 0 & 0 & \ldots & 0 & 0 \\
        0 & 0 & \ldots & 0 & k & 0 & 0 & 0 & \ldots & 0 & 0 \\
        0 & 0 & \ldots & 0 & 0 & k & 0 & 0 & \ldots & 0 & 0
        \end{array}\right| \\
        & =(-1)^{k-1} k^k \cdot t^{m(k-1)}
        \end{aligned}
\]

\[
    \delta_{k, m}=m(k-1)
\]

\[
    f(x)=x^k-t^{k l}=\prod_{i=0}^{k-1}(x-\zeta^i t^l)
\]

\[
    \Delta(f)=\prod_{0 \leq i<j \leq k-1}(\zeta^j t^l-\zeta^i t^l)^2=\prod_{0 \leq i<j \leq k-1}(\zeta^j-\zeta^i) t^{2 l}
\]

\begin{defi}\label{defi:5.31}
    令$C\to \spec K$为一个代数闭域$K$上的曲线,而$p\in C$为一个闭点。
    我们称$p$为一个\textit{结点}、\textit{尖点}或\textit{互自切点}若
    局部环$\mathscr O_{C,p}$关于其极大理想的形式完备化$\hat{\mathscr O}$
    同构于相应的$K[\![x,y]\!]/(y^2-x^2)$, $K[\![x,y]\!]/(y^2-x^3)$或
    $K[\![x,y]\!]/(y^2-x^4)$.
\end{defi}