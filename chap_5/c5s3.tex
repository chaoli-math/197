\section{奇异概形和判别式}\label{s:5.3}

\subsection{定义}\label{s:5.3.1}

% p.230
在这节中,我们将考虑概形间态射的光滑以及奇异点。为启发我们的定义,
考虑微分流形间的态射$f:M\to N$,并假设$\dim M\geq \dim N$. 这样的态射
表现得最简单的时候是在$x\in M$其上微分$Df_x$是满的:限制到这样的点$x$
的适当领域上,态射$f$看上去就像乘积到其中一个因子的投影一般,此时我们称
$x$是$f$的一个光滑点。

在概形范畴中,Zariski开集往往太大了以至于不能容纳乘积结构,而
(如我们将看到的)可能还有来自于这些点具有不同的剩余类域的困难。但,
依然可能通过局部乘积结构来定义概形间态射的光滑性,如见 Altman, Kleiman [1970],
但在目前的情况下,通过推广流形上的微分刻画来刻画光滑点和奇异点会更有意义。

为此,我们引入环同态的K\"ahler微分模,以及其整体版本,概形间态射的相对余切层。
若$\psi:A\to B$为一个环同态,我们定义\textit{$A$-线性K\"ahler微分模}
$\Omega_{B/A}$,这是一个$B$-模,为由符号$db$所生成的自由$B$-模,其中$b\in B$,
模掉了关系
\[
    d(b_1 b_2)=b_1 d b_2+b_2 d b_1 \quad \text { 对所有 }\,b_1, b_2 \in B
\]
以及 
\[
    d \psi_a=0 \quad \text { 对所有 }\,a \in A.
\]

这些关系保证了同态
\[
    \begin{aligned}
        B & \longrightarrow \Omega_{B / A}, \\
        b & \longmapsto d b,
    \end{aligned}
\]
是一个$A$-线性导子;实际上,这是适当意义下的万有$A$-线性导子。(细节见
Eisenbud [1995].) 于是,容易见到,比如若$B=A[x_1,\dots,x_n]/(f_1,\dots,f_n)$,
则$\Omega_{B/A}$在Jacobi矩阵
\[
    \left(\frac{\partial f_i}{\partial x_j}\right)
\]
的余核中。

为整体化,令$\varphi:X\to Y$为概形间态射。我们如下定义$\varphi:X\to Y$的
\textit{相对余切层},记作$\Omega_{X/Y}$,该层在仿射开集$U\subset X$上
(映到仿射开集$V\subset Y$)取为$\mathscr O_X(U)$的
$\mathscr O_Y(V)$-线性K\"ahler微分模。上面说的仿射开集们在$X$中构成了一个
拓扑基$\mathscr B$. 则$\mathscr B$-层(见第 \ref{s:1.1.3} 节)的公理是
容易检查的,相当于说微分模的构造是局部相容的,于是这些数据确实定义了一个层。

% p.231

概形间态射$\varphi:X\to Y$在点$x\in X$的\textit{相对维数}被定义为差
$\dim (X,x)-\dim(Y,\varphi(x))$.

\begin{defi}\label{defi:5.21}
    令$\varphi:X\to Y$为一个Noether概形间的态射,且假设$\varphi$是平坦的、
    有限型的,且有着常相对余维数$d$. 我们定义$\varphi$的\textit{奇异概形}
    $\sing\varphi\subset X$为$\varphi$的相对余切层的
    第$d$个Fitting理想,即
    \[
        \sing\varphi=V(\fitt_d \Omega_{X/Y})\subset X.
    \]
    当态射$\varphi|_{\sing \varphi}:\sing \varphi\to Y$是有限的,我们定义
    \textit{判别式概形}$\Delta(\varphi)\subset Y$为$\sing \varphi$在$Y$
    中的Fitting像,即
    \[
        \Delta(\varphi)=V(\fitt_0(\varphi_* \mathscr{O}_{\sing \varphi})) 
        \subset Y.
    \]
\end{defi}

为理解这个定义,我们首先转到最经典的情况,即代数闭域上的不可约非奇异簇之间的
$\varphi:X\to Y$. 此时,常相对维数的条件是自动满足的,平坦性则变成了纤维
$\varphi^{-1}(y)$的维数是常数,且等于相对维数。

奇异概形的概念为局部性的,我们可以转入仿射的情况并假设$Y=\spec A$,以及记
\[
    X=\spec A[x_1, \ldots, x_n] /(f_1, \ldots, f_m).
\]
如前所说,此时$A$-线性 K\"ahler 微分层$\Omega_{X/Y}$为$n\times m$ Jacobian
矩阵
\[
    \begin{pmatrix}
        \partial f_1 / \partial x_1 & \ldots & \partial f_m / \partial x_1 \\
        \vdots & & \vdots \\
        \partial f_1 / \partial x_n & \ldots & \partial f_m / \partial x_n
    \end{pmatrix}
\]
的余核。因此,概形$\sing \varphi$的底空间恰是$X$中的那些$x$其上Jacobian矩阵的
秩小于$n-d$构成的,换句话说,即使得微分$D\varphi$不是满射的点们,恰如经典的定义。

然而,概形学的语境下新的现象涌现了。比如,考虑一个有限域扩张
$K\hookrightarrow K'$,且令
\[
    \varphi: X=\spec K^{\prime} \longmapsto Y=\spec K
\]
为相应的一点概形间的态射。这里相对维数为$0$,于是$\varphi$是奇异的当且仅当
$\Omega_{X/Y}\neq 0$. 从一个域论的经典结论,这当且仅当扩张$K\hookrightarrow K'$
并不是可分的。

% p.232

\subsection{判别式}

我们现在想要更仔细地考虑态射$\varphi:X\to Y$的奇异概形$\sing \varphi$在$Y$上是
有限的情况,以及描述关联于这个态射的判别式概形$\Delta(\varphi)\subset Y$. 
我们将主要(但不完全)关注于当$X$是$\mathbb A_Y^1$的闭子概形的情况,其为系数
为$Y$上的正则函数的一个多项式$f(x)$的零点集,这将领我们到一元多项式的
\textit{判别式}的定义。此时,和前面章节里关于结式的讨论类似,我们将这个问题
看作如何给出那些点$y=[\mathfrak p]\in Y=\spec A$的定义方程
的公式,这些点上$f$具有重根,即使得$f\in A[x]$模$\mathfrak p$后
$\bar f\in A/\mathfrak p[x]$在$\kappa (y)$的代数闭包中有重根。如结式的讨论,
我们将有两种一般的定义,一种用Fitting理想,以及另一种
(在更特殊的情况$X=\spec A[x]/(f)$)更经典的判别式的定义,作为
一个适当的多项式“万有族”的(既约)分支概形的拉回。我们将最终证明这两种定义在后者
可定义的情况中相同。

为经典的构造,我们首先需要定义我们的万有分支覆盖。令
\[
    A=\zz[a_0,\dots,a_m],
\]
以及令
\[
    f=a_0x^m+\cdots+a_m\in A[x]
\]
为一般的$m$-次一元多项式。扩张我们的多项式环,定义
\[
    B_0=\zz [\alpha_1,\dots,\alpha_m]\quad \text{和}\quad 
    B=B_0[a_0]
\]
以及映射$A\to B$,其将$a_i$变到$f=a_0\prod_i (x-\alpha_i)$的第$i$个系数,
其为$\pm a_0\sigma_i(\alpha)$. 正如 Theorem \ref{thm:5.15} 的证明中所示,
这些系数是代数独立的,我们将$A$看成$B$的一个子环。

在$B$中,我们可以构造多项式
\[
    D_1=\prod_{i<j}(\alpha_i-\alpha_j).
\]
当两个$f$的根相同时,这个多项式为零,但这并不是我们问题的解答,因为这不是
$A$中的多项式。一方面,他在根的置换下并不是不变的。但是,容易看到,如果$\pi$
是一个根的置换,$D_1$在置换作用后等于$\operatorname{sgn}(\pi)D_1$,其中
$\operatorname{sgn}(\pi)$是置换的符号。因此,$D_1^2$在根的置换下不变,
于是可以表为对称函数$\sigma_i(\alpha)=a_i/a_0$的多项式。每个$\alpha_i$在
$D_1$中出现的次数小于等于$m-1$,于是在$D_1^2$中至多为$2m-2$,从此可见%
% p.233
\[
    D_m(f)=a_0^{2 m-2} D_1^2=a_0^{2 m-2} \prod_{i<j}(\alpha_i-\alpha_j)^2
\]
定义了$A$中的一个元素。多项式$D_m(f)$被称为$f$的\textit{判别式}。

\begin{pro}\label{pro:5.22}
    记号同上,我们有
    \[
    D_m(f)=\frac{(-1)^{m(m-1) / 2}}{a_0} R_{m, m-1}(f, f^{\prime}).
    \]
    若$f_0\in L[x]$为域$L$上的一个$m$次的一元首一多项式,则$D_m(f_0)=0$
    当且仅当$f_0$在$L$的代数闭包中有重根。
\end{pro}

\begin{proof}

将Corollary \ref{coro:5.17}应用到上面所定义的环$B$,可知
\[
    R_{m, m-1}(f, f^{\prime})=a_0^{m-1} \prod_{i=1}^m f^{\prime}(\alpha_i).
\]
但是,
\[
    f^{\prime}(x)=\sum_{j=1}^m f(x) /(x-\alpha_j),
\]
故
\[
    f^{\prime}(\alpha_i)=a_0 \prod_{j \neq i}(\alpha_i-\alpha_j)
\]
以及
\[
    \begin{aligned}
        R_{m, m-1}(f, f^{\prime}) & =a_0^{m-1} \prod_{i=1}^m f^{\prime}(\alpha_i)=a_0^{2 m-1} \prod_{j \neq i}(\alpha_i-\alpha_j) \\
        & =(-1)^{m(m-1) / 2} a_0^{2 m-1} \prod_{i<j}(\alpha_i-\alpha_j)^2 \\
        & =(-1)^{m(m-1) / 2} a_0 D_m(f),
        \end{aligned}
\]
从此可知$D_m(f)$的公式。推论的第二点来自于,一个首一多项式与另一个多项式在域
上的结式为零当且仅当他们有共同的根,而从常规的计算可知,多项式$f(x)$和
他的导数有共同的根当且仅当$f$有重根。
\end{proof}

作为推论,在特殊情况$X=\spec A[x]/(f)$中,我们得到了允诺的经典和现代定义的等同。

% p.234

\begin{coro}\label{coro:5.23}
    令$A$是一个环,$f\in A[x]$是一个$m$次首一多项式,而$\varphi:X=\spec A[x]/(f)
    \to Y=\spec A$为相应的态射。$\varphi$的判别式概形为$f$的判别式的零点,即
    \[
    \Delta(\varphi)=V(D_m(f)) \subset Y.\]
\end{coro}

比如,若$B=\mathbb Z[\alpha]$为某个数域的阶(order)\footnote{
    译者注:我不确定order是不是翻译成阶。代数数域$K$的一个order $O$是$K$
    的一个子环,他作为$\mathbb Z$-模有限生成,且$O\otimes_{\mathbb Z}\mathbb Q=K$.
}%
以及
\[
    f(x)=x^n+a_{n-1} x^{n-1}+\cdots+a_0 \in \mathbb{Z}[x]
\]
为$\alpha$满足的不可约首一多项式,则$f$为态射
\[
    \spec B\to \spec \zz
\]
的判别式概形的定义方程。

\begin{exa}\label{exe:5.24}
    证明这个态射的奇异概形为$f$的\textit{离差}(different)%
    \footnote{译者注:离差源于超理用户imsane. }%
    ,其比如定义于Lang [1994].
\end{exa}

\begin{exa}\label{exe:5.25}
    考虑在第 \ref{s:2.4.2} 节和Exercise \ref{exe:4.51} 中所讨论的
    阶$A=\mathbb Z[\sqrt{3}]$, $B=\mathbb Z[11\sqrt{3}]$, 
    $C=\mathbb Z[2\sqrt{3}]$和$D=\mathbb Z[121\sqrt{3}]$. 
    态射$\spec A\to \spec \mathbb Z$, $\spec B\to \spec \mathbb Z$, 
    $\spec C\to \spec \mathbb Z$, $\spec D\to \spec \mathbb Z$的
    判别式概形是什么?这和之前所画的概形的图有什么联系?
\end{exa}

\subsection{例子}

为演示我们的态射的判别式概形的定义,我们将计算一些具体的例子。
对下面所有的态射$\varphi:X\to Y$,目标概形将是仿射直线$Y=\spec K[t]$,
其中$K$是一个特征为零的域。对所有但除了最后一个例子,态射$\varphi$
将是有限的,对除了最后两个例子,$X$将是仿射直线$\mathbb A_Y^1=\spec K[t][x]$
的一个子概形,在$Y$上有限且平坦。具体地,我们取
\[
    \varphi: X=\spec K[t, x] /(f) \rightarrow Y=\spec K[t],
\]
其中
\[
    f(x)=x^k+a_{k-1}(t) x^{k-1}+\ldots+a_1(t) x+a_0(t)
\]
是一个$x$的系数在$K[t]$中的首一多项式。在所有的例子中,我们将考虑
判别式概形,作为$\mathbb A_K^1$的支于零点的子概形,由其次数所确定。

\begin{exa}\label{exa:5.26}
我们从例子$f(x)=x^k-t^m$开始,即态射
\[
    \varphi_{k, m}: X=X_{k, m}=\spec K[t, x] /\left(x^k-t^m\right) \rightarrow Y=\spec K[t].
\]
我们将记这个态射的判别式概形的次数为$\delta_{k,m}$,并将用三种方法来计算他。
% p.235

在我们开始前,注意到,若
\[
    \mu_\alpha: Y=\spec K[t] \longrightarrow Y=\spec K[t]
\]
是由$t\mapsto t^\alpha$给出的态射,我们有纤维积图
\[
    \xymatrix{
        X_{k,\alpha m}\ar[r]\ar[d]_{\varphi_{k,\alpha m}}& X_{k,m}\ar[d]^{\varphi_{k, m}}\\
        Y\ar[r]_{\mu_\alpha}& Y
    }
\]
于是,由于判别式概形在拉回下的不变性,对任意的$\alpha$和$m$,我们有
\[
    \delta_{k, \alpha m}=\alpha \cdot \delta_{k, m}.
\]

第一个计算,我们用多项式$f(x)$的判别式作为$f$和$f'$的结式的定义,然后应用
Sylvester行列式。因此,比如在$k=2$时,判别式为
\[
    \delta_{2, m}=\left|\begin{array}{ccc}
        1 & 0 & t^m \\
        2 & 0 & 0 \\
        0 & 2 & 0
    \end{array}\right|=4 t^m,
\]
因此($K$的特征为零)$\delta_{2,m}=m$. 更一般地,对任意的$k$,
我们有$a_{k-1}=\cdots=a_1=0$以及$a_0=-t^m$. 因此,
\[
    \begin{aligned}
        \delta_{k, m} & =\left|\begin{array}{ccccccccccc}
        1 & 0 & \ldots & 0 & 0 & 0 & -t^m & 0 & \ldots & 0 & 0 \\
        0 & 1 & \ldots & 0 & 0 & 0 & 0 & -t^m & \ldots & 0 & 0 \\
        \vdots & \vdots & \ddots & \vdots & \vdots & \vdots & \vdots & \vdots & \ddots & \vdots & \vdots \\
        0 & 0 & \ldots & 1 & 0 & 0 & 0 & 0 & \ldots & -t^m & 0 \\
        0 & 0 & \ldots & 0 & 1 & 0 & 0 & 0 & \ldots & 0 & -t^m \\
        k & 0 & \ldots & 0 & 0 & 0 & 0 & 0 & \ldots & 0 & 0 \\
        0 & k & \ldots & 0 & 0 & 0 & 0 & 0 & \ldots & 0 & 0 \\
        \vdots & \vdots & \ddots & \vdots & \vdots & \vdots & \vdots & \vdots & \ddots & \vdots & \vdots \\
        0 & 0 & \ldots & k & 0 & 0 & 0 & 0 & \ldots & 0 & 0 \\
        0 & 0 & \ldots & 0 & k & 0 & 0 & 0 & \ldots & 0 & 0 \\
        0 & 0 & \ldots & 0 & 0 & k & 0 & 0 & \ldots & 0 & 0
        \end{array}\right| \\
        & =(-1)^{k-1} k^k \cdot t^{m(k-1)},
        \end{aligned}
\]
故
\[
    \delta_{k, m}=m(k-1).
\]

\nottran

% p.236

\[
    f(x)=x^k-t^{k l}=\prod_{i=0}^{k-1}(x-\zeta^i t^l)
\]

\[
    \Delta(f)=\prod_{0 \leq i<j \leq k-1}(\zeta^j t^l-\zeta^i t^l)^2=\prod_{0 \leq i<j \leq k-1}(\zeta^j-\zeta^i) t^{2 l}
\]


\end{exa}

\[
    \Omega_{X / Y}=\mathscr{O}_X\{d x\} /\left(k x^{k-1} d x\right)
\]

\[
    0 \longrightarrow \mathscr{O}_X \stackrel{\nu}{\longrightarrow} \mathscr{O}_X \longrightarrow \Omega_{X / Y} \longrightarrow 0
\]

\[
    \sing\varphi=V(\fitt_0 \Omega_{X/Y})=V(x^{k-1})\subset X
\]

\[
    0 \longrightarrow \mathscr{O}_Y^{\oplus k} \stackrel{\varphi_* \nu}{\longrightarrow} \mathscr{O}_Y^{\oplus k} \longrightarrow \varphi_* \mathscr{O}_{\sing \varphi} \longrightarrow 0
\]


% p.237
\[
    \varphi_* \nu=\begin{pmatrix}
        0 & t^m & 0 & \ldots & 0 & 0 \\
        0 & 0 & t^m & \ldots & 0 & 0 \\
        0 & 0 & 0 & \ldots & 0 & 0 \\
        \vdots & \vdots & \vdots & \ddots & \vdots & \vdots \\
        0 & 0 & 0 & \ldots & 0 & t^m \\
        1 & 0 & 0 & \ldots & 0 & 0
    \end{pmatrix}
\]


\begin{defi}\label{defi:5.31}
    令$C\to \spec K$为一个代数闭域$K$上的曲线,而$p\in C$为一个闭点。
    我们称$p$为一个\textit{结点}、\textit{尖点}或\textit{互自切点}若
    局部环$\mathscr O_{C,p}$关于其极大理想的形式完备化$\hat{\mathscr O}$
    同构于相应的$K[\![x,y]\!]/(y^2-x^2)$, $K[\![x,y]\!]/(y^2-x^3)$或
    $K[\![x,y]\!]/(y^2-x^4)$.
\end{defi}