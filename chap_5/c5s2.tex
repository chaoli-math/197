% p.222
\section{结式}\label{s:5.2}
\subsection{结式的定义}\label{s:5.2.1}
我们前两个小节中所引入的思想的最古老、但依然是最重要的应用之一为
\textit{两个一元多项式的结式}。给定两个多项式
\[
\begin{aligned}
    &f(x)=a_0x^m+\cdots+a_m,
    &g(x)=b_0x^n+\cdots+b_n,
\end{aligned}
\]
其中$a_i$, $b_i$为属于域$K$的系数。我们将描述系数$a_i$, $b_i$满足何种条件时
$f(x)$和$g(x)$有共同的因子,即在$K$的代数闭包中有相同的根。在应用中,很自然地也
会去研究多项式族的情况。即,我们将$a_i$, $b_i$取做一个基概形$B$上的正则函数,
于是,我们可以将$f$和$g$想作“由$B$参数化的一元多项式族”,以及我们将试图描述$B$
中使得$f$和$g$有共同的因子的点们。更精确地,我们想要描述(某种意义下的!)概形
$V(f,g)\subset \mathbb A_B^1$在$B$中的像。

粗略地说,有四种方式来描述这个问题:我们可以去问如下理想以何种系数$a_i$ $b_i$
的方程生成的公式:
\begin{compactenum}
\item $V(f,g)$的既约像,即与其概形学式像关联的既约概形;
\item $V(f,g)$的概形学式像;
\item $V(f,g)$的 Fitting 像;
\item 像关于一个合适的万有族的拉回。
\end{compactenum}

如前面的例子 \ref{exa:5.6} 和 \ref{exa:5.7} 所示,前两个意义上的像的公式可能
并不存在。我们会从经典的方法开始,然后证明(在适当选取的“万有族”)其给出了
第三和第四个意义上的像的公式,已经其与基变换可交换。

对经典的进路,我们从选择万有族开始。我们将在不仅仅一个域,而是任意的环$S$上
工作。令$A$为多项式环,
\[
    A=S[a_0,\dots,a_m,b_0,\dots,b_n].
\]
对上面定义的$f$和$g$,我们记
\[
    X:=V(f,g)\subset \mathbb A_A^1,
\]
再令
\[
    \varphi:X\subset \mathbb A_A^1=\spec A\times_S \mathbb A_S^1
    \to \spec A=:Y.
\]%
% p.223
因为$X$由两个多项式所定义,我们可能会期待其在$\mathbb A_A^1$中的余维数为$2$.
在这个例子中,假设映射“一般地有限”(即在某些开集上是有限的),其像可能在
$\spec A$中的余维数为$1$. 在经典的情况中,$S$是一个域,$A$于是是唯一分解整环,
像的闭包可以由一个方程所描述,叫做$f$和$g$的\textit{结式}。

事实证明,这个启发性论证的结论是正确的。我们会在适当的时候看到,
$\varphi$的概形学式像$\bar\varphi(X)$是既约的,并且与Fitting像一致;
因此,在这个万有族的情况下,$X$像的含义是明确的。

可能写下方程最直接的方法如下。考虑环$\tilde A\subset A[a_0^{-1},b_0^{-1}]$,
定义为
\[
    \tilde A=A\biggl[
        \frac{a_1}{a_0},\dots,\frac{a_m}{a_0},
        \frac{b_1}{b_0},\dots,\frac{b_n}{b_0}
    \biggr].
\]
再令
\[
    B=\tilde A[\alpha_1,\dots,\alpha_m,\beta_1,\dots,\beta_n]/I,
\]
其中理想$I$由如下$m+n$个元素所生成
\[
    (-1)^i\sigma_i(\alpha)-\frac{a_i}{a_0}
    \quad \text{和}\quad 
    (-1)^i\sigma_i(\beta)-\frac{b_i}{b_0},
\]
其中$\sigma_i$为第$i$个基础对称函数。我们可以直观地将 $B$ 描述为从 $A$ 通过
加上多项式$f$和$g$的根而得到的环。注意到$\tilde A\subset B$,因为基础对称函数
是代数独立的。

表达式
\[
    \mathscr R=\prod_{i,j}(\alpha_i-\beta_j)
\]
是一个对$\alpha_i$以及$\beta_j$各自的对称函数。因为$(-1)^ia_i/a_0$为
$\alpha_i$的第$i$次基本对称函数,而$(-1)^jb_j/b_0$为$\beta_j$的第$j$次基本
对称函数,因此$\mathscr R$是一个$\tilde A$中的元素。每个$\alpha_i$都在
$\mathscr R$中出现$n$次,于是$\mathscr R$是$a_i/a_0$和$b_j/b_0$的双齐次多项式,
其次数分别为$n$和$m$. 因此
\[
    R_{m,n}(f,g):=a_0^nb_0^m \mathscr R=\prod_{i,j}(\alpha_i-\beta_j),
\]
在$A$中;这是一个双齐次多项式,对$a_i$是$n$次的而对$b_j$是$m$次的,叫做
$f$和$g$的\textit{结式}。

% p.224

一般地,如果$f_0$和$g_0$为$S$-代数$S_0$上的$m$和$n$次多项式,我们记
$R_{m,n}(f_0,g_0)\in S_0$为代入$f_0$和$g_0$的系数到$R_{m,n}(f,g)$得到的结果,
即$R_{m,n}(f,g)$在如下同态下的像,
\[
    A=S[a_0, \ldots, a_m, b_0, \ldots, b_n] \to S_0,
\]
其将$a_i$和$b_j$映作$f_0$和$g_0$的系数。我们称$R_{m,n}(f_0,g_0)$为
$f_0$和$g_0$的\textit{结式}。当我们将$S_0$看成不同的环$S$上的代数时,
结式并不依赖于$S$;特别地,通过取$S=\mathbb Z$,我们将对任意的$S_0$得到相同的
结果。因此,对每个$m$和$n$,我们可以说“结式”$R_{m,n}$为环$\mathbb Z
[a_0,\dots,a_m,b_0,\dots,b_n]$的一个元素,而“两个多项式$f$和$g\in S[x]$的结式”
为$R_{m,n}$在相应的同态下的像
\[
    \mathbb{Z}[a_0, \ldots, a_m, b_0, \ldots, b_n] \to S.
\]

如果$f_0$和$g_0$的领头系数并不为零,即多项式的次数确实如说的那样,则
$R_{m,n}(f_0,g_0)=0$当且仅当$f_0$和$g_0$有共同的因子,原始经典的目标就已达成了。
我们也可以看到,这个对至多一个领头系数为零的情况也是正确的;但这就是说
$R_{m,n}(f,g)$包含于理想$(a_0,b_0)$中,于是其恒为零若两个多项式的领头系数都为零。
实际上,映射$\varphi$在这个情况中并不有限,因为在$\mathbb A_S^{m+n+2}$的原点处的
纤维为一条仿射直线,其集合论式的像并不是闭的。

\subsection{Sylvester行列式}\label{s:5.2.2}

我们下面将两个多项式的结式的零点集和他们共同零点集的Fitting像联系起来。
我们从计算定义的Fitting像$\varphi_{\text{Fitt}}(X)$的方程开始。为此,令$S$
为任何的环,而$f,g\in S[x]$为两个多项式,其次数分别为$m$和$n$. 为定义
$X=V(f,g)\subset \mathbb A_S^1$在投影$\pi:\mathbb A_S^1\to \spec S$下的
Fitting像,我们必须假设这个映射是有限的,或者等价地,理想$(f,g)\subset S[x]$
包含了一个首一多项式,见Eisenbud [1995, Proposition 4.1]. 方便起见,我们可以
假设$f$是首一的。

为计算定义了$X$的Fitting像的理想$\operatorname{Fitt}_0 \pi_*(\mathscr{O}_X)$,
我们首先将$\pi_*(\mathscr O_X)$为$S$-模$S[x]/(f,g)$. 这个模是同态
\[
    S[x] \oplus S[x] \stackrel{(f, g)}{\longrightarrow} S[x]
\]
的余核,而同态的两侧都被认作(无限生成的)自由$S$-模。但是,为计算Fitting理想,
我们需要一个有限表示。因为,我们已经假设了$f$是一个次数为$m$的首一多项式,
而$S$-子模
\[
    S[x]_{<m}=S \oplus S x \oplus S x^2 \oplus \cdots \oplus S x^{m-1} 
    \subset S[x]
\]
满映到$S[x]/(f,g)$上;而因为$S[x]_{<m}$作为$S$-模同构于$S$-代数$S[x]/(f)$,
我们可以将其实现为$S[x]/(f,g)$为乘$g$诱导的同态的余核:
\[
    G: S^m \cong S[x]_{<m} \cong S[x] /(f) \stackrel{\times g}{\longrightarrow} 
    S[x] /(f) \cong S[x]_{<m} \cong S^m.
\]
因此Fitting理想$\operatorname{Fitt}_0 \pi_*(\mathscr{O}_X)$由一个
表示了映射$G:S^m\to S^m$的$m\times m$矩阵的行列式所生成。
% p.225

现在,并不难直接写出$G$的矩阵表示,而这给出了Fitting像的公式。但是,如果
如果我们用从\textit{任意}自由表示出发计算Fitting理想的自由度,我们可以得到一个
更保持$f$, $g$对称性的图像。

为此,考虑自由模
\[
    S[x]_{<m+n}=S \oplus S x \oplus \cdots \oplus S x^{m+n-1} \cong S^{m+n} .
\]
其显然映满到$S[x]/(f,g)$,这个映射的核为在$(f,g)\subset S[x]$中的
至多为$m+n-1$次的多项式$h\in S[x]$. 我们断言,对任意这样的
$h \in S[x]_{<m+n} \cap(f, g)$可以表为线性组合
\[
    h=a \cdot f+b \cdot g
\]
其中$a$为至多$n-1$次的多项式而$b$为至多$m-1$次多项式。为看到这点,首先我们必须有
\[
    h=a^{\prime} \cdot f+b^{\prime} \cdot g
\]
其中$a',b'\in S[x]$. 现在,因为$f$为次数为$m$的首一多项式,
我们可以将$b'$除以$f$,写作
\[
    b^{\prime}=q f+b
\]
其中$b$是一个至多为$m-1$次的多项式。将表达式
\[
    0=(q g) \cdot f-(q f) \cdot g
\]
加上去,我们可以得到$h$的另一表示:
\[
    h=\left(a^{\prime}+q g\right) \cdot f+b \cdot g.
\]
再记$a=a'+pg$即可。因为$h$和$bg$都至多为$m+n-1$次,$af$的次数也至多为$m+n-1$,
再因为$f$是$m$次首一多项式,于是$a$的次数至多为$n-1$,如前所断言。

% p.226

于是,模$S[x]/(f,g)$为映射
\[
    \begin{aligned}
        S[x]_{<n} \oplus S[x]_{<m} & \longrightarrow S[x]_{<m+n} \\
        (a, b) & \longmapsto a f+b g 
    \end{aligned}
\]
的余核。很容易写下这个映射的矩阵表示。在显然的基下,他就是为
\textit{Sylvester矩阵}
\[
    \operatorname{Syl}_{(m, n)}(f, g)=
    \begin{pmatrix}
        a_0 & a_1 & \ldots & \ldots & a_{m-1} & a_m & 0 & \ldots & 0 \\
        0 & a_0 & a_1 & \ldots & \ldots & a_{m-1} & a_m & \ldots & 0 \\
        \vdots & \ddots & \ddots & \ddots & & & \ddots & \ddots & \\
        0 & \ldots & 0 & a_0 & a_1 & \ldots & \ldots & a_{m-1} & a_m \\
        b_0 & b_1 & \ldots & b_{n-1} & b_n & 0 & \ldots & \ldots & 0 \\
        0 & b_0 & b_1 & \ldots & b_{n-1} & b_n & \ldots & \ldots & 0 \\
        \vdots & \ddots & \ddots & \ddots & & \ddots & \ddots & & \\
        \vdots & & \ddots & \ddots & \ddots & & \ddots & \ddots & \\
        0 & \ldots & & 0 & b_0 & b_1 & \ldots & b_{n-1} & b_n
    \end{pmatrix},
\]
其中有$n$行$a$和$m$行$b$.

下个结果表明,结式的零点集实际上恰好对上上面的多项式偶对的万有族的Fitting像。

\begin{thm}\label{thm:5.15}
    令
    \[
        A=\mathbb{Z}\left[a_0, \ldots, a_m, b_0, \ldots, b_n\right],
    \]
    再令
    \[
        f=a_0 x^m+\cdots+a_m
    \]
    以及
    \[
        g=b_0 x^n+\cdots+b_n \in A[x]
    \]
    为两个次数为$m$和$n$的一元多项式。令$\varphi:\mathbb A_A^1\to \spec A$
    为投影态射。则$X=V(f,g)\subset \mathbb A_A^1$在$\varphi$的概形学式像
    的定义理想由$R_{m,n}(f,g)$生成,其也等于Sylvester行列式
    $\operatorname{det}\left(\operatorname{Syl}_{m, n}(f, g)\right)$.
\end{thm}

\begin{proof}
为简化记号,令$R'=\operatorname{det}\left(\operatorname{Syl}_{m, n}(f, g)\right)
\in A$, 再令$R=R_{m,n}(f,g)$. 在代数的语言中,我们需要证明$R$等于$R'$以及
生成了理想$A\cap (f,g)A[x]$.

首先,我们证明$R'\in (f,g)A[x]$. 如果我们对$\operatorname{Syl}_{m, n}(f, g)$的
列进行操作,对$t=1,\dots,m+n-1$,将第$t$列乘以$x^{m+n-t}$后加到最后一列,
因为总共有$m+n$列,我们得到了一个新的矩阵,有着相同的行列式$R'$. 而新矩阵的
最后一列为
\[
    \begin{pmatrix}
        x^{n-1} f \\
        \vdots \\
        f \\
        x^{m-1} g \\
        \vdots \\
        g
    \end{pmatrix},
\]
于是行列式如所需的在$(f,g)A[x]$中。

% p.227

下面假设$P\in A$处于$(f,g)A[x]$中,我们下面证明$P$在$A$中可以被$R$整除。
注意到$f$和$g$是双齐次的:他们分别对$a_i$和$b_j$齐次。 于是,任何$(f,g)$
中的多项式的双齐次分量依然在$(f,g)$中,于是我们可以假设$P$也是双齐次的,
具有次数$(d,e)$. (注意到$R'$自己也是双齐次的,次数为$(n,m)$.)

为分析情况,我们将$A$嵌入到更大的多项式环中。
令$B_0=\mathbb Z[\alpha_1,\dots,\alpha_m,\beta_1,\dots,\beta_n]$和
$B=B_0[a_0,b_0]$. 我们将$A$映到$B$, 通过将$a_i$映到多项式
\[
    f^{\prime}=a_0 \prod_{j=1}^m(x-\alpha_j)
\]
的系数,而将$b_i$映到多项式$g^{\prime}=b_0 \prod_{j=1}^n(x-\beta_j)$的系数。
$f'$的系数也即$\pm a_0\sigma_i(\alpha)$, 其中$\sigma_i$为第$i$个基本对称函数,
而$g'$也类似。因为$a_0$, $b_0$以及$\alpha_i$和$\beta_j$的基本对称函数都是代数
独立的,于是
\[
    a_0, a_0 \sigma_1(\alpha), \ldots, a_0 \sigma_m(\alpha), b_0, 
    b_0 \sigma_1(\beta), \ldots, b_0 \sigma_m(\beta),
\]
也代数独立,于是映射$A\to B$确实是一个嵌入。回忆,$R$被定义为元素
$a_0^n b_0^m \prod_{i, j}\left(\alpha_i-\beta_j\right) \in B$, 其确实处于
子环$A$中。

我们现在回到多项式$P \in A \cap(f, g) A[x]$. 因为$P$是对
$a_i= \pm a_0 \sigma_i(\alpha)$和$b_j= \pm b_0 \sigma_j(\beta)$是双齐次的,
次数为$(d,e)$, 我们可以记$P=a_0^d b_0^e h$,其中$h\in B_0$. 因为基本对称函数
$\sigma_i(\alpha)$对每个$\alpha_i$是一个线性多项式,$\beta_j$也类似,我们见到
$h$对每个$\alpha_i$的次数$\leq d$,而对每个$\beta_j$的次数$\leq e$.

\nottran

% p.228
\end{proof}

我们并不能直接应用Proposition \ref{pro:5.8} 到Theorem \ref{thm:5.15} 的情况,
因为正如我们说过的,态射$V(f,g)\to \spec A$并不有限。然而,若我们首先限制到
$A$的一个开子集上,而此时态射是有限的,则:

\begin{coro}\label{coro:5.16}
    \nottran
\end{coro}

\[
    f=a_0 x^m+\cdots+a_m \quad \text { and } \quad g=b_0 x^n+\cdots+b_n \in B[x]
\]

\begin{coro}\label{coro:5.17}
令$S$是一个环,而$f,g\in S[x]$为$S$上的一元多项式。若
$f(x)=a_0 \prod_{i=1}^m(x-\alpha_i)$在$S$上完全分解成线性因子,则
\[
    R_{m, n}(f, g)=a_0^n \prod_{i=1}^m g(\alpha_i).
\]
\end{coro}

\begin{proof}
结式即如下式子的特化(自多项式 $f,g$ 定义所处的环和因子):
\[
    a_0^n b_0^m \prod_{i, j}(\alpha_i-\beta_j)=a_0^n \prod_i g(\alpha_i).
    \qedhere
\]
\end{proof}

% p.229

\begin{exa}\label{exa:5.18}
    \nottran
\end{exa}

\begin{exe}\label{exe:5.19}
    \nottran
\[
    0 \to(R_{m, n}(f, g)) \to A \to A /(R_{m, n}(f, g)) \to 0
\]
\end{exe}

\begin{exe}\label{exe:5.20}
    \nottran
\end{exe}
