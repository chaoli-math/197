% p.222
\section{结式}\label{s:5.2}
\subsection{结式的定义}\label{s:5.2.1}
我们前两个小节中所引入的思想的最古老,但依然是最重要的,应用为\textit{两个一元多项式的结式}。给定两个多项式
\[
\begin{aligned}
    &f(x)=a_0x^m+\cdots+a_m,
    &g(x)=b_0x^n+\cdots+b_n,
\end{aligned}
\]
其中$a_i$, $b_i$为属于域$K$的系数。我们将描述系数$a_i$, $b_i$满足何种条件时$f(x)$和$g(x)$有共同的因子,即
在$K$的代数闭包中有相同的根。在应用中,很自然去看多项式族的情况。即,我们将$a_i$, $b_i$取做一个基概形$B$
上的正则函数,于是,我们可以将$f$和$g$想作“由$B$参数化的一元多项式族”,以及我们将试图描述$B$中轨迹使得$f$和$g$
有共同的因子。更精确地,我们想要描述 --- 某种意义下的!--- 概形$V(f,g)\subset \mathbb A_B^1$在$B$中的像。

\subsection{Sylvester行列式}\label{s:5.2.2}