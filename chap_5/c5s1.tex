\section{像}\label{s:5.1}

% p.209

这节中我们关心一个基本的概念:概形范畴中态射的\textit{像}。如将所见,我们希求像有如下两种基本的性质,
\textit{推-拉性}和\textit{在基变换下不变},而此二者却不兼容。我们将相应给出两个定义,一个直接些,另一个
稍不,但他们都在相应的情况下非常有用。

\subsection{概形的态射的像}\label{s:5.1.1}

假设$\varphi:X\to Y$是概形间的态射。$\varphi$的\textit{集合学式像}当然地如下定义:他是$Y$中的点$y\in Y$
使得存在点$x\in X$满足$\varphi(x)=y$. 此像可以是也可以不是闭子集:比如,如果$X$是在
$\mathbb{A}_K^2=\spec K[x, y]$中由理想$(xy-1)\subset K[x,y]$定义的仿射概形,那么$X$到
$\mathbb{A}_K^1=\spec K[x]$的投影的像就是原点的补集,是一个开子集。某种意义上来说,这是因为我们“忘掉”了
源概形的一些点:如果我们将此态射从$X$延拓到$X$在$\mathbb{A}_K^1 \times_K \mathbb{P}_K^1$的闭包上,则
他的像将是整个射影直线。

这是两个典型的例子。这种情况可以总结为下面的定理。为描述之,首先会议拓扑空间的子集$V$是\textit{可构造}的,
如果他是一族局部闭\footnote{译者注:局部闭子集是指他可以写成一个开子集与一个闭子集的交。}%
子集$V_i$的有限并。

% p.210

\begin{thm}\label{thm:5.1}
    如果Noether概形间态射$\varphi: X\to Y$是有限型的,则$\varphi$在$Y$中的集合学式像是可构造的。如果$\varphi$
    是射影的,则$\varphi$的集合学式像是闭的。
\end{thm}

前半个命题来自于Chevalley. 注意,$\varphi$是有限型的假设是必要的:多项式环到其局部化的含入同态诱导的态射
\[
    \spec K\left[x_1, \ldots, x_n\right]_{\left(x_1, \ldots, x_n\right)} \longrightarrow 
    \spec K\left[x_1, \ldots, x_n\right]
\]
对任意的$n$都没有可构造的像。后半个命题有点古老,一般被叫做消元理论主定理。
紧合性,其强化了定理的结论,本质上就是为了表达这种性质而做出的(见第 \ref{s:3.1} 节)。定理 \ref{thm:5.1} 的
证明和簇的经典情况完全相同,比如可见Harris [1995] 或 Hartshorne [1977]. 我们这里就不重复了。

稍新一点的是,像的闭包具有一个自然的概形结构,这是我们在这节的后面要探讨的。

现在,若我们想要将态射$\varphi:X\to Y$的像定义为一个概形。最理想地,尽管如下所见是不可能的,就是将$\varphi(X)$
取做$Y$中最小的原像是整个$X$的子概形。换句话说,我们想要用\textit{推-拉性}刻画$\varphi(X)$:对任意的子概形
$Z\subset Y$,
\[
    Z \supset \varphi(X) \quad \Longleftrightarrow \quad \varphi^{-1}(Z)=X.
\]
但事实上,$Y$并不需要存在这样的子概形,如下例:由将$x$映为$s$, $y$映为$st$的环同态
$\varphi^{\#}: K[x, y] \rightarrow K[s, t]$诱导的态射
\[
    \varphi: X=\mathbb{A}_K^2 \longrightarrow Y=\mathbb{A}_K^2.
\]

\inclugra{1.png}

% p.211

对任意的$\lambda\in K$,令$Y_\lambda$是$Y$中点$(0,\lambda)$的补。对$\lambda \neq 0$,$Y_\lambda$的原像是整个$X$.
因此,如果存在一个子概形$\varphi(X)\subset Y$满足$Z=Y$的推-拉性质,则$\varphi(X)$的底空间应该包含于集合
$I=\bigcap_{\lambda\neq 0}Y_\lambda$,其是点$(0,0)$与$x$-轴的补的并。此外,从$\varphi^{-1}\varphi(X)=X$,
$\varphi(X)$的底空间应该包含$I$,故$\operatorname{supp}\varphi(X)=I$. 但是$I$并不是一个$Y$的局部闭子集,
因此也不能是任何$Y$的子概形的底空间。

然而,我们离我们想要的也并不远:确实存在一个$Y$的极小\textit{闭}子概形,其原像就是整个$X$,且这个子概形也对
闭子概形$Z\subset Y$满足推-拉性质。

\begin{defi}\label{defi:5.2}
若$\varphi:X\to Y$是一个有限型态射,则\textit{概形学式的像},写作$\bar\varphi(X)$,是$Y$中的闭子概形,其理想层
为$Y$的开子集上的由$\varphi^\#$拉回到$0$的正则函数的层,即,
\[
    \bar{\varphi}(X)=V\left(\ker
    \left(\varphi^{\#}: \mathscr{O}_Y \rightarrow \varphi_* \mathscr{O}_X\right)\right) \subset Y.
\]
我们称$\varphi$是\textit{支配}的,如果$\bar\varphi(X)=Y$,或等价地,如果拉回映射$\varphi^\#$是单的。
\end{defi}

态射要支配的条件并\textit{不}仅是个拓扑空间间的条件:比如,含入$\varphi:\spec K\hookrightarrow 
\spec K[\epsilon]/(\epsilon^2)$是底空间间的一个满射,但其像是一个真闭子概形,拉回$\varphi^\#$并不是单的,
于是态射并不是支配的。

\begin{pro}\label{pro:5.3}
若$\varphi:X\to Y$是一个概形间的态射,则集合学式像的闭包是$\bar\varphi(x)_{\text{red}}$.
\end{pro}

\begin{proof}
我们可以立即约化到仿射的情况,然后证明,如果$\varphi^\#:B\to A$是一个环同态,则$B$中所有
可以写作$Q=(\varphi^\#)^{-1}(P)$的素理想(其中$P$是某个$A$的素理想)的交$J$是$\ker(\varphi^\#)$的根
(见第 \ref{s:1.2.1} 节)。令$I\subset A$为$A$的幂零根,我们有 
\[
    J=\bigcap_{\text { 素理想 } P \subset A}\left(\varphi^{\#}\right)^{-1}(P)=\left(\varphi^{\#}\right)^{-1}(I) .
\]
因此,如果$f\in J$,则$\varphi^\#(f)\in I$是幂零的,于是$f\in \operatorname{rad}(\ker \varphi^\#)$. 反方向的包含
是立即的。
\end{proof}

有时用明显更一般的形式会很方便:若$\varphi: X\to Y$是任何态射,$X'\subset X$是一个闭子概形,则我们可以定义$X'$的
\textit{概形学式的像}$\bar\varphi(X')$为由理想层$I$所以定义的子概形,其中
\[
    I(U)=\left\{f \in \mathscr{O}_Y(U) \mid \varphi^{\#}(f) \in I_{X^{\prime}}\left(\varphi^{-1}(U)\right)\right\}.
\]%
% p.212
但这个“推广”实际上仅描述了一种特殊情况,因为$\bar\varphi(X')$为复合态射$X'\to X\to Y$的概形学式的像。

如果两个概形$X$和$Y$是仿射的,且态射$\varphi$是支配的,则$X'\subset X$的像的理想可以很简单得到:
我们可以将$A(Y)$看成$A(X)$的一个子环,而$\bar\varphi(X')$是由理想$I_{X'}\cap A(Y)$定义的$Y$的子概形。
我们将主要关注这种情况,因为他已经包含了所有的新现象。

考虑往第一个坐标的线性投影
\[
    \varphi: X=\mathbb{A}_K^2=\spec K[x, y] \longrightarrow Y=\mathbb{A}_K^1=\spec K[x].
\]
令$X'$和$X''$为由理想$I'=(x,y^2)$和$I''=(x^2,y)$给出的(抽象来说是同构的)零维子概形。$X'$的像为
由$(x,y^2)\cap K[x]=(x)$定义的既约概形,而$X''$的像却由$(x^2,y)\cap K[x]=(x^2)$给出。更一般地,
“丰满点”$V(x^2,xy,y^2)\subset \mathbb A_K^2$的像为双重点$V(x^2)\subset \mathbb A_K^1$,而双重点
$V(x^2,xy,y^2,\alpha x+\beta y)\subset \mathbb A_K^2$的像在$\beta\neq 0$时为双重点$V(x^2)\subset \mathbb A_K^1$,
在$[\alpha,\beta]=[1,0]$时候为既约点$V(x)$. 可从此例看出,像的概形结构依赖于子概形与态射$\varphi$的纤维的关系。

\inclugra{2.png}

\begin{exe}\label{exe:5.4}
    现在考虑一“族”这样的双重点的投影:取$B=\spec K[t]=\mathbb A_K^1$,然后考虑态射
    \[
        \varphi: X=\mathbb{A}_B^2=\spec K[x, y, t] \longrightarrow Y=\mathbb{A}_B^1=\spec K[x, t].
    \]
    令$X'\subset X$为在第 \ref{s:2.3.5} 节中描述过的双重直线
    \[
        X^{\prime}=V\left(x^2, x y, y^2, x+t y\right) \subset \mathbb{A}_B^2.
    \]
    证明,$X'$的概形学式的像$\bar\varphi(X')$为双重直线$V(x^2)\subset Y$,尽管
    $X'$在原点$(t)\in B$处的纤维的像为$Y$在$(t)$处的纤维中的既约点$V(x)$.
    实际上,如我们将在下面章节中所解释的,这是许多有趣并发现象的起源。
\end{exe}

% p.213

\begin{exe}\label{exe:5.5}
    证明,如果$\varphi:X\to Y$是一个态射,$X'\subset X$是一个既约子概形,则$\bar\varphi(X')$是既约的。
    (提示:将问题约化到证明,一个根式理想在环同态下的原像依然是一个根式理想。)
\end{exe}

\subsection{万有族}\label{s:5.1.2}

% p.214
\inclugra{3.png}

\[
    I_a=\ker\left(K[x] \rightarrow K[x, y] /\left(y^2-y, x+a y\right)\right)= 
    \begin{cases}
        (x^2+a x) & \text{ 若 } a \neq 0, \\
        (x) & \text{ 若 } a=0.
    \end{cases}
\]

\[
    \begin{aligned}
        K[t, x, y] /\left(y^2-y, y x+\right. & \left.y t, y x-x, x^2+t x\right) \\
        & \rightarrow K\left[t, t^{-1}, x, y\right] /\left(y^2-y, y x+y t, y x-x, x^2+t x\right) \\
        & =K\left[t, t^{-1}, x\right] /\left(x^2+t x\right)
    \end{aligned}
\]

% p.215

\[
    \varphi: X=\mathbb{A}_B^2=\spec K[x, y, t] \longrightarrow Y=\mathbb{A}_B^1=\spec K[x, t].
\]

\[
    X=V\left(x^2, x y, y^2\right) \cap V(t y+x)=V\left(t y+x, y^2\right).
\]

\inclugra{4.png}

% p.216

\[
    K[x, y] /\left(a y+x, y^2\right)= 
    \begin{cases}K[x] /\left(x^2\right) & \text { 若 } a \neq 0 \\ 
        K[x, y] /\left(x, y^2\right) & \text { 若 } a=0\end{cases}
\]

\[
    I_a=\ker\left(K[x] \rightarrow K[x, y] /\left(a y+x, y^2\right)\right)= 
    \begin{cases}\left(x^2\right) & \text { 若 } a \neq 0 \\ 
        (x) & \text { 若 } a=0\end{cases}
\]

\[
    K[t, x, y] /\left(t y+x, y^2\right) \rightarrow K\left[t, t^{-1}, x, y\right] /
    \left(t y+x, y^2\right)=K\left[t, t^{-1}, x\right] /\left(x^2\right)
\]

\[
    \xymatrix{
        X\ar[rr]^\varphi\ar[rd]&&Y\ar[ld]\\
        &B&
    }
\]

\[
    \bar{\varphi}\left(X_b\right) \subset \bar{\varphi}(X)_b.
\]

% p.217

\subsection{Fitting理想和Fitting像}\label{s:5.1.3}