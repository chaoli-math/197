\section{相对概形}
\subsection{纤维积} \label{s.1.3.1}

这里有一个用概形的纤维积描述的,集合关于函数的原像思想的重要拓展。为了准备这个定义,我们首先复习集合范畴的情况。

两个集合$X$和$Y$在第三个集合$S$上的纤维积,在已知一个映射的图
\[
	\xymatrix{
	&X\ar[d]^\varphi\\
	Y\ar[r]_\psi &S
	}
\]
下,被定义为
\[
	X\times_S Y=\{(x,y)\in X\times Y\,:\,\varphi x=\psi y\}.
\]
纤维积有时候被叫做$X$(或者关于$X\to S$)到$Y$的\textit{拉回}。这个构造以一种非常有用的方式拓展了许多基础构造:

如果$S$是一个点,则它给出了一般的直积。

如果$X$和$Y$都是$S$的子集,而$\varphi$和$\psi$都是含入映射,则它给出交。

如果$Y\subset S$以及$\psi$是含入,他给出了$Y$在$X$中的原像。

如果$X=Y$,他给出了使得$\varphi$, $\psi$相等的集合,即映射的\textit{等值子}。

\begin{exe}
	验证这些断言!
\end{exe}

注意到,$X\times_S Y$与它到$X$和$Y$自然的投射给出了交换图
\[
	\xymatrix{
	X\times_S Y\ar[r]\ar[d]&X\ar[d]^\varphi\\
	Y\ar[r]_\psi &S
	}
\]
实际上,集合$X\times_S Y$能被下面的泛性质所确定:在所有使得下图交换的$Z$与给定的映射下
\[
	\xymatrix{
	Z\ar[r]\ar[d]&X\ar[d]^\varphi\\
	Y\ar[r]_\psi &S
	}
\]
$X\times_S Y$与投射是唯一的“最有效的”选择,意即,给定上图中的$Z$,存在唯一的映射$Z\to X\times_S Y$使得图
\[
	\xymatrix{
	Z\ar[dr]\ar[ddr]\ar[drr] & &\\
	& X\times_S Y \ar[r]\ar[d] &  X \ar[d]^\varphi\\
	& Y\ar[r]^\psi &  S
	}
\]
交换。

在概形范畴中,我们就\textit{定义}纤维积是一个满足上面泛性质的概形,特别地,泛性质保证了这个概形连同他到$X$和$Y$的投射是唯一的。于是我们能用纤维积来定义直积、交、原像以及等值子!但这就有一个问题,是否作为纤维积的这些对象在概形范畴是存在的?答案是是的,我们下面将描述它的构造。

首先,我们处理仿射情形。回忆仿射概形范畴与交换环范畴对偶,见Corollary \ref{coro.1.41}. 因此,如果我们有概形
\[
	X=\spec A,\quad Y=\spec B,\quad S=\spec R,
\]
其中$X$, $Y$映到$S$(因此$A$和$B$是$R$\hyp 代数),我们必须定义纤维积$X\times_S Y$为
\[
	X\times_S Y=\spec(A\otimes_R B).
\]
这是因为自然的图
\[
	\xymatrix{
	A\otimes_R B&A\ar[l]\\
	B\ar[u]&R\ar[l]\ar[u]_\varphi
	}
\]
有着与纤维积相反的泛性质。时髦一点说法即,张量积即一个交换环范畴的\textit{纤维余积}或者\textit{纤维直和}。

为了检查这个定义是合理的,可以注意到当$Y$是$S$的由$I$定义的闭子概形,即$B=R/I$时,我们有$A\otimes_R B=A/IA$. 于是$X\times_S Y=\spec A/IA$,这就是以前定义过的$Y$在$X$中的原像。

\begin{exe}
一些简单的特殊例子将在计算纤维积的时候给出巨大的帮助。直接通过代数的张量积的泛性质证明下面的事实:

\begin{compactenum}[(a)]
\item 对任意的$R$\hyp 代数$S$,我们有$R\otimes_R S=S$.
\item 若$S$, $T$是$R$\hyp 代数,$I\subset S$是一个理想,于是
\[
	(S/I)\otimes_R T=(S\otimes_R T)/(I\otimes 1)(S\otimes_R T).
\]
\item 如果$x_1$, $\dots$, $x_n$, $y_1$, $\dots$, $y_m$是不定元,于是
\[
	R[\text{$x_1$, $\dots$, $x_n$}]\otimes R[\text{$y_1$, $\dots$, $y_m$}]=R[\text{$x_1$, $\dots$, $x_n$, $y_1$, $\dots$, $y_m$}].
\]
\end{compactenum}
用这些原理来解决习题的剩余部分。
\begin{compactenum}[(a)] \setcounter{enumi}{3}
\item 令$m$, $n$是整数。计算纤维积
\[
	\spec \zz/(m)\times_{\spec \zz} \spec \zz/(n).
\]
\item 计算纤维积$\spec \cc \times_{\spec \rr} \spec \cc$.
\item 证明,对$R$上的任意多项式环$R[x]$和$R[y]$,我们有
\[
	\spec R[x]\times_{\spec R}\spec R[y]=\spec R[x,y].
\]
\end{compactenum}
注意在例子(d)中,纤维积的支集是两个对应支集的纤维积,但是这在(e)和(f)中并不正确。
\begin{compactenum}[(a)] \setcounter{enumi}{6}
\item 考虑环同态
\[
	R[x]\to R;\quad x\mapsto 0
\]
以及
\[
	R[x]\to R[y];\quad x\mapsto y^2.
\]
证明关于这些映射,我们有
\[
	\spec R[y]\times_{\spec R[x]}\spec R=\spec R[y]/(y^2).
\]
\end{compactenum}

在一般的情况中,我们用仿射概形$\spec R_\rho$来覆盖$S$,
\end{exe}