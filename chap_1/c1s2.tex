\section{一般的概形}

在大幅描述了仿射概形后,定义一般的概形也变得简单了。一个\textit{概形}$X$是一个拓扑空间$X$附上一个环层$\oo_X$,其中$X$被称为概形的\textit{支集},记作$|X|$或者$\mathrm{supp}\,X$,使得对$(|X|,\oo_X)$是\textit{局部仿射的}\index{仿射!局部}。所谓的局部仿射,就是说,$|X|$被一族开集$U_i$所覆盖,对每一个开集$U_i$存在$R_i$以及同胚$U_i\cong |\spec R_i|$使得$\oo_X|_{U_i}\cong \oo_{\spec R_i}$.

为了更好地理解这个定义,我们必须确定仿射概形的结构层的关键属性。\nottran 如果$X$是任意的拓扑空间,而$\oo$是它上面的一个环层,则我们称偶对$(X,\oo)$是一个\text{附环空间},继而我们要问,什么时候它同构于一个仿射概形$(|\spec R|,\oo_{\spec R})$. 注意到,如果$(X,\oo)$是一个仿射概形,则他将一定是概形$\spec R$.

现在令$(X,\oo)$是任意的附环空间,以及$R=\oo(X)$. 对任意的$f\in R$,我们能定义一个集合$U_f$,其中的点$x\in X$在$f$的作用下变成了茎$\oo_x$中的可逆元。如果$(X,\oo)$是一个仿射概形,我们必须有:
\begin{compactenum}[(i)]
\item $\oo(U_f)=R[f^{-1}]$.

然而,这个条件还不够,它甚至都没有要求$X$和$|\spec R|$之间有一个映射。为了给出这样一个映射,我们需要假设$\oo$有更多的仿射概形有的条件:

\item $\oo$的茎$\oo_x$是一个局部环。

一个

\item 映射$X\to |\spec \oo(X)|$是一个同胚。

\end{compactenum}

\subsection{子概形}\label{s.1.2.1}

令$U$是概形$X$的一个开子集。偶对$(U,\oo_X|_U)$依然是一个层,尽管这不是完全显然的。