\section{一般的概形}

在大幅描述了仿射概形后,定义一般的概形也变得简单了。一个\textit{概形}$X$是一个拓扑空间$X$附上一个环层$\oo_X$,其中$X$被称为概形的\textit{支集},记作$|X|$或者$\mathrm{supp}\,X$,使得对$(|X|,\oo_X)$是\textit{局部仿射的}\index{仿射!局部}。所谓的局部仿射,就是说,$|X|$被一族开集$U_i$所覆盖,对每一个开集$U_i$存在$R_i$以及同胚$U_i\cong |\spec R_i|$使得$\oo_X|_{U_i}\cong \oo_{\spec R_i}$.

为了更好地理解这个定义,我们必须确定仿射概形的结构层的关键属性。设$X$是任意的拓扑空间,而$\oo$是它上面的一个环层,我们称偶对$(X,\oo)$为\text{附环空间},继而我们会问,什么时候它同构于一个仿射概形$(|\spec R|,\oo_{\spec R})$. 注意到,如果$(X,\oo)$是一个仿射概形,则他将一定是概形$\spec R$.

现在令$(X,\oo)$是任意的附环空间,以及$R=\oo(X)$. 对任意的$f\in R$,我们能定义一个集合$U_f$,其中的点$x\in X$在$f$的作用下变成了茎$\oo_x$中的可逆元。如果$(X,\oo)$是一个仿射概形,我们必须有:
\begin{compactitem}
\item[(i)] $\oo(U_f)=R[f^{-1}]$.
\end{compactitem}
然而,这个条件还不够,它甚至都没有要求$X$和$|\spec R|$之间有一个映射。为了给出这样一个映射,我们需要假设$\oo$有更多的仿射概形有的条件:
\begin{compactitem}
\item[(ii)] $\oo$的茎$\oo_x$是一个局部环。
\end{compactitem}
满足(ii)的一个赋环空间$(X,\oo)$经常被称为一个\textit{局部赋环空间}。

如果$(X,\oo)$满足(ii),那么存在一个自然映射$X\to |\spec \oo(X)|$,他将$x\in X$变成$\oo_x$的极大理想在$\oo(X)$中的原像。$(X,\oo)$要成为仿射概形的第三个要求是:
\begin{compactitem}
\item[(iii)] 映射$X\to |\spec \oo(X)|$是一个同胚。
\end{compactitem}

经过这些考虑,我们称一个偶对$(X,\oo)$是\textit{仿射的},如果它满足条件(i)--(iii). 前面给出的概形的定义现在变成了:一个偶对$(X,\oo)$是一个概形,如果他是局部仿射的。

同样,如果不产生歧义,我们将用同样的符号$X$来记概形以及其拓扑空间$|X|$,比如在构造“令$p\in X$是一个点”中。

\begin{exe}
\begin{compactenum}[(a)]
\item 取$Z=\spec \cc[x]$,令$X$是在$|Z|$中等同两个闭点$(x)$和$(x-1)$得到的拓扑空间,以及令$\varphi:Z\to X$是自然的投射。令$\oo$是$\varphi_* \oo_Z$,$X$上的一个环层。证明,$(X,\oo)$对每一个$f\in\oo(X)=\cc[x]$都满足前面的条件(i),但不满足条件(ii). 注意到,不存在自然映射$X\to |\spec\cc[x]|$.
\item 取$Z=\spec \cc[x,y]$,即对应于仿射平面的概形,令$X$是挖去原点的得到开子集,即$X=|Z|-\{(x,y)\}$. 令$\oo$是层$\oo_Z|_X$(即,对任意开子集$V\subset X\subset |Z|$成立$\oo(V)=\oo_Z(V)$.)证明,$\oo(X)=\cc[x,y]$,$X$, $\oo$满足条件(i)和(ii),以及自然映射$X\to |\spec\oo(X)|$是含入$X\subset |Z|$.
\end{compactenum}
\end{exe}

这里,有一些符号和术语约定如下。

一个开集$U\subset X$上的\textit{正则函数}是层$\oo_X$的在$U$上的截面。一个\textit{整体正则函数}是一个$X$上的正则函数。

结构层$\oo_X$在点$x\in X$处的茎$\oo_{X,x}$被称为$\oo_X$的\textit{局部环}。而$\oo_{X,x}$的剩余类域被记作$\kappa(x)$. 如同第 {\ref{s.1.1.1}} 节,$\oo_X$的一个截面能被想成一个取值于这些域$\kappa(x)$的“函数”:如果$f\in\oo_X(U)$以及$x\in U$,则$f$在点$x$处的值即$f$在复合映射
\[
	\oo_X(U)\to \oo_{X,x}\to \kappa(x)
\]
下的像。

\begin{exe}[最小的非仿射概形]
令$X$是一个拓扑开集,它只有三个点$p$, $q_1$和$q_2$. 通过令$X_1:=\{p$, $q_i\}$以及$X_2:=\{p$, $p_2\}$是开集,我们赋予了$X$一个拓扑结构(即,除了$X_1$和$X_2$外,$\varnothing$, $\{p\}$以及$X$本身是开集)。定义$X$上的一个环预层$\oo$通过置
\[
	\oo(X)=\oo(X_1)=\oo(X_2)=K[x]_{(x)},\quad \oo(\{p\})=K(x),
\]
以及限制映射$\oo(X)\to \oo(X_i)$为恒等映射,而限制映射$\oo(X_i)\to \oo(\{p\})$是显然的含入。验证这个预层是一个层以及$(X,\oo)$是一个概形。证明这不是一个仿射概形。(几何来说,这个概形$(X,\oo)$是在Exercise \ref{exe.1.44} 里的概形$X_1$中的“双重点的芽”。)
\end{exe}

\subsection{子概形}\label{s.1.2.1}

令$U$是概形$X$的一个开子集。偶对$(U,\oo_X|_U)$依然是一个层,虽然这不完全是显然的。为了检查这点,注意到,至少仿射概形的一个\textit{基本}开集依然是一个仿射概形:如果$X=\spec R$以及$U=X_f$,于是$(U,\oo_X|_U)=\spec R_f$. 因为$X$中包含在$U$中的基础开集覆盖了$U$,这就证明了$(U,\oo_X|_U)$由仿射概形所覆盖,此即所需。在理解了概形结构的情况下,概形的一个开子集意恉概形的一个\textit{开子概形}。

闭子概形的定义更加复杂,指定$X$的一个闭子集是不够的,因为上面的层结构不能顺道定义出来。

首先考虑一个仿射概形$X=\spec R$. 对环$R$的任意理想$I$,我们将等同闭子集$V(I)\subset X$与仿射概形$Y=\spec R/I$. 这个等同的成立是因为$R/I$的素理想正好就是$R$中那些素理想模去$I$得到的,因此拓扑空间$|\spec R/I|$是典范同构于闭子集$V(I)\subset X$. 我们定义$X$的一个\textit{闭子概形}是概形$Y$,它是$R$的一个商环的谱(于是按定义,$X$的闭子概形一一对应于环$R$中的一个理想)。

我们于是能定义给定概形$X=\spec R$的闭子概形上的寻常的那些操作以及闭子概形间的关系。因此,我们称$X$闭子概形$Y=\spec R/I$\textit{包含}闭子概形$Z=\spec R/J$,如果$Z$是一个$Y$的闭子概形,即$J\supset I$. 这将推出$V(J)\subset V(I)$,但是反过来并不对。

\begin{exe}
	Exercise \ref{exe.1.20}中的概形$X_1$, $X_2$, $X_3$都能被看作$\spec \cc[x]$的闭子概形。证明
	\[
	X_1\subset X_3\quad \text{以及}\quad X_2\subset X_3,
	\]
	但是没有其他包含$X_i\subset X_j$成立,尽管$X_2$和$X_3$的支撑相同,且$X_1$的支撑包含其中。
\end{exe}

两个闭子概形$\spec R/I$和$\spec R/J$的\textit{并}被定义为$\spec R/(I\cap J)$,它们的\textit{交}被定义为$\spec R/(I+J)$. 注意,这些包含、相交与并的概念\textit{并不}满足它们在集合中的对应概念的那些寻常性质:比如,我们将在第\pageref{p.69}页上发现这样一个例子,一个概形的闭子概形们$X$, $Y$, $Z$满足$X\cup Y=X\cup Z$以及$X\cap Y=X\cap Z$但$Y\neq Z$.

现在我们要把闭子概形的概念拓展到一般的概形$X$上。为此,第一步需是用一个层代替仿射概形$X=\spec R$中伴随于闭子概形$Y$的理想$I\subset R$,现表述如下。定义$Y$\textit{在}$X$\textit{中的理想层} $\scr{I}=\scr{I}_{Y/X}$为$\oo_X$的理想层,它在每一个$X$的基本开集$V=X_f$上给出$\scr{I}(X_f)=I R_f$. 现在可以将$Y=\spec R/I$的结构层$\oo_Y$(更准确地说,是在含入映射$j:|Y|\hookrightarrow |X|$下的前推$j_{*}\oo_Y$)与商层$\oo_X/\scr{I}$等同起来。(请自行检验这个等同。)理想层$\scr{I}$将作为限制映射$\oo_X\to j_*\oo_Y$的核被还原出来。

这儿有一个不易察觉的点需要指出:并非所有$\oo_X$的理想层都可以从$R$的理想得到。比如,在Exercise \ref{exe.1.22} 中考虑的$R=K[x]_{(x)}$,我们可以如下定义一个理想层
\[
	\scr{I}(X)=0,\quad \text{以及对$U=\{(0)\}$, }\scr{I}(U)=\oo_X(U).
\]
但是如果$\scr{I}$来自于$R$的一个理想,我们应该有
\[
	\scr{I}(U)=\scr I (X)_x = \scr I (X) K(x),
\]
于是$\scr I$并不是来自于$R$的某个理想。在前面的闭子概形的定义中,我们将只关注来自于$R$的理想的理想层。这样一个理论中的层显然需要一个名字:它们被称为\textit{拟凝聚}理想层。(对这样一个基础而简单的对象,这看上去是一个缺乏启发性的名字,但它却坚实扎根于各个文献中。它来自于 \nottran )

\begin{exe}
	一个概形是不可约的当且仅当每一个开子集都是稠密的。
\end{exe}

\begin{exe}
	一个仿射概形$X=\spec R$是约态且不可约的,当且仅当$R$是一个整环。$X$是不可约的当且仅当$R$只有一个极小素理想,或者等价地,$R$的幂零根是一个素理想。
\end{exe}

\begin{exe}
	一个概形$X$是约态的,当且仅当每一个$X$的仿射开子概形是约态的,当且仅当,对每一个闭点$p\in X$,局部环$\oo_{X,p}$是约态的。(一个环被称为约态的,如果他唯一的幂零元是$0$.)
\end{exe}

\begin{exe}
	如何定义两个概形的不交并?证明两个仿射概形$\spec R$和$\spec S$的不交并应当等同于概形$\spec R\times S$.
\end{exe}

\begin{exe}
	一个任意的概形$X$是不可约的,当且仅当它每一个仿射开子集是不可约的。如果它还是连通的(即指其拓扑空间$|X|$是连通的),于是它是不可约的当且仅当每一个$\oo_X$的局部环都只有一个唯一的极小素理想。
\end{exe}

我们现在已经引入了概形$X$的开子概形与闭子概形的概念。更进一步,一个$X$的\textit{局部闭子概形}的定义是直接的:它是一个$X$的开子概形的闭子概形。这是本书中我们可能考虑的最一般概念,于是,当我们只说$X$的一个子概形,没有修饰语的时候,我们就是在说局部闭子概形。

\begin{exe}
令$X$是一个任意的概形,而$Y$, $Z$是它的闭子概形。解释$Y$包含于$Z$中的意思是什么。同样,如果$Y$, $Z$只是局部闭子概形,则$Y$包含于$Z$中的意思是什么。
\end{exe}

\subsection{一点处的局部环}

\[
	\oo_{X,x}:={\varinjlim}_{x\in U}\oo_X(U).
\]

\[
	\oo_{X,x}:={\varinjlim}_{f\not\in \pp}R_f=R_{\pp}.
\]

\[
	\mm_{X,x}:={\varinjlim}_{f\not\in \pp}\pp R_f=\pp R_{\pp}.
\]

\begin{exe}
	零维仿射概形的支集是有限的。
\end{exe}

\begin{exe}
如果$K$是一个域,则概形$\spec K[x_1$, $\dots$, $x_n]$在点$[(x_1$, $\dots$, $x_n)]$处的Zariski切空间是$n$维的。
\end{exe}

\begin{exe}
一个零维Noether概形是非奇异的,当且仅当他是约态点的并。
\end{exe}

\subsection{态射}

下面定义概形的态射。在经典理论中,仿射簇之间的正则函数由

\begin{thm}
对任意的概形$X$以及任意的环$R$,态射
\[
	(\varphi,\varphi^\#):X\to \spec R
\]
一一对应于环同态
\[
	\varphi:R\to \oo_X(X)
\]
通过
\[
	\varphi=\psi^\#(\spec R):R=\oo_{\spec R}(\spec R)\to \psi_*(\oo_X)(\spec R)=\oo_X(X).
\]
\end{thm}

\begin{coro}\label{coro.1.41}
	仿射概形范畴等价于箭头倒过来的含幺交换环范畴,即所谓的对偶范畴。
\end{coro}

\begin{exe}\label{exe.1.42}
	\begin{compactenum}[(a)]
	\item 使用这个证明,从任意概形到$\spec \zz$存在且只存在一个映射。以范畴论的语言,这就是在说$\spec \zz$是概形范畴的\idx{终对象}。
	\item 证明,单点集是集合范畴的终对象。
	\end{compactenum}
\end{exe}

比如,$X=\spec R$的每个点$[\pp]$对应有个概形$\spec \kappa(\pp)$,以及一个由如下复合环映射
\[
	R\to R_\pp\to R_\pp/\pp_\pp=\kappa(\pp)
\]
定义的到$X$的自然映射。当然,含入使得$[p]$是一个闭子概形当且仅当$\pp$是$R$的一个极大理想(一般地,$[\pp]$是一个闭子概形的无限开子概形的交)。

\subsection{黏合构造}