\section{一般的概形}

在大幅描述了仿射概形后,定义一般的概形也变得简单了。一个\textit{概形}$X$是一个拓扑空间$X$附上一个环层$\oo_X$,其中$X$被称为概形的\textit{支集},记作$|X|$或者$\mathrm{supp}\,X$,使得对$(|X|,\oo_X)$是\textit{局部仿射的}\index{仿射!局部}。所谓的局部仿射,就是说,$|X|$被一族开集$U_i$所覆盖,对每一个开集$U_i$存在$R_i$以及同胚$U_i\cong |\spec R_i|$使得$\oo_X|_{U_i}\cong \oo_{\spec R_i}$.

为了更好地理解这个定义,我们必须确定仿射概形的结构层的关键属性。\nottran 如果$X$是任意的拓扑空间,而$\oo$是它上面的一个环层,则我们称偶对$(X,\oo)$是一个\text{附环空间},继而我们要问,什么时候它同构于一个仿射概形$(|\spec R|,\oo_{\spec R})$. 注意到,如果$(X,\oo)$是一个仿射概形,则他将一定是概形$\spec R$.

现在令$(X,\oo)$是任意的附环空间,以及$R=\oo(X)$. 对任意的$f\in R$,我们能定义一个集合$U_f$,其中的点$x\in X$在$f$的作用下变成了茎$\oo_x$中的可逆元。如果$(X,\oo)$是一个仿射概形,我们必须有:
\begin{compactitem}
\item[(i)] $\oo(U_f)=R[f^{-1}]$.
\end{compactitem}
然而,这个条件还不够,它甚至都没有要求$X$和$|\spec R|$之间有一个映射。为了给出这样一个映射,我们需要假设$\oo$有更多的仿射概形有的条件:
\begin{compactitem}
\item[(ii)] $\oo$的茎$\oo_x$是一个局部环。
\end{compactitem}
满足(ii)的一个赋环空间$(X,\oo)$经常被称为一个\textit{局部赋环空间}。

如果$(X,\oo)$满足(ii),那么存在一个自然映射$X\to |\spec \oo(X)|$,他将$x\in X$变成$\oo_x$的极大理想在$\oo(X)$中的原像。$(X,\oo)$要成为仿射概形的第三个要求是:
\begin{compactitem}
\item[(iii)] 映射$X\to |\spec \oo(X)|$是一个同胚。
\end{compactitem}

经过这些考虑,我们称一个偶对$(X,\oo)$是\textit{仿射的},如果它满足条件(i)--(iii). 前面给出的概形的定义现在变成了:一个偶对$(X,\oo)$是一个概形,如果他是局部仿射的。

同样,如果不产生歧义,我们将用同样的符号$X$来记概形以及其拓扑空间$|X|$,比如在构造“令$p\in X$是一个点”中。

\begin{exe}
\begin{compactenum}[(a)]
\item 取$Z=\spec \cc[x]$,令$X$是在$|Z|$中等同两个闭点$(x)$和$(x-1)$得到的拓扑空间,以及令$\varphi:Z\to X$是自然的投射。令$\oo$是$\varphi_* \oo_Z$,$X$上的一个环层。证明,$(X,\oo)$对每一个$f\in\oo(X)=\cc[x]$都满足前面的条件(i),但不满足条件(ii). 注意到,不存在自然映射$X\to |\spec\cc[x]|$.
\item 取$Z=\spec \cc[x,y]$,即对应于仿射平面的概形,令$X$是挖去原点的得到开子集,即$X=|Z|-\{(x,y)\}$. 令$\oo$是层$\oo_Z|_X$(即,对任意开子集$V\subset X\subset |Z|$成立$\oo(V)=\oo_Z(V)$.)证明,$\oo(X)=\cc[x,y]$,$X$, $\oo$满足条件(i)和(ii),以及自然映射$X\to |\spec\oo(X)|$是含入$X\subset |Z|$.
\end{compactenum}
\end{exe}

\subsection{子概形}\label{s.1.2.1}

令$U$是概形$X$的一个开子集。偶对$(U,\oo_X|_U)$依然是一个层,虽然这不完全是显然的。为了检查这点,注意到,至少仿射概形的一个\textit{基本}开集依然是一个仿射概形:如果$X=\spec R$以及$U=X_f$,于是$(U,\oo_X|_U)=\spec R_f$. 因为$X$中包含在$U$中的基础开集覆盖了$U$,这就证明了$(U,\oo_X|_U)$由仿射概形所覆盖,此即所需。在理解了概形结构的情况下,概形的一个开子集意恉概形的一个\textit{开子概形}。

闭子概形的定义更加复杂,指定$X$的一个闭子集是不够的,因为上面的层结构不能顺道定义出来。

首先考虑一个仿射概形$X=\spec R$. 对环$R$的任意理想$I$,我们将等同闭子集$V(I)\subset X$与仿射概形$Y=\spec R/I$. 这个等同的成立是因为$R/I$的素理想正好就是$R$中那些素理想模去$I$得到的,因此拓扑空间$|\spec R/I|$是典范同构于闭子集$V(I)\subset X$. 我们定义$X$的一个\textit{闭子概形}是概形$Y$,它是$R$的一个商环的谱(于是按定义,$X$的闭子概形一一对应于环$R$中的一个理想)。

我们

\begin{exe}
	概形
\end{exe}

\begin{thm}
对任意的概形$X$以及任意的环$R$,态射
\[
	(\varphi,\varphi^\#):X\to \spec R
\]
一一对应于环同态
\[
	\varphi:R\to \oo_X(X)
\]
通过
\[
	\varphi=\psi^\#(\spec R):R=\oo_{\spec R}(\spec R)\to \psi_*(\oo_X)(\spec R)=\oo_X(X).
\]
\end{thm}

\begin{coro}\label{coro.1.41}
	仿射概形范畴等价于箭头倒过来的含幺交换环范畴,即所谓的对偶范畴。
\end{coro}