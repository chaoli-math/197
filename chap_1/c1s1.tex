就像拓扑或者微分流形是由Euclid空间的开球粘起来的一样,概形是由一类被称为\textit{仿射概形}的开集粘合起来的。但这有一个主要的不同,在流形上,一点局部来看和另一点是相似的,开球也是整个构造需要的唯一开集,它们都是一样的并且非常简单。不同的是,概形容许更多的局部多样性。即使是概形中最小的开集也大到许多有趣与不平凡的几何将在这些开集中发生。实际上,在许多概形中,没有两个点有同构的开邻域(除了整个概形)。因此,我们将花费一大部分时间来描述仿射概形。

在这章中,我们将列出一些基本的定义,同时将通过一系列简单的习题来体现与运用这些定义。这章给出的例子大部分是最简单的那种,进而也不一定是那种具有有趣几何的例子。一类更有代表性的例子将在下一章给出,我们意欲指出概形与簇概念的不同,同时给出概形观点所具有的统一的力量的感觉。

\section{仿射概形}

一个\textit{仿射概形}\index{概形!仿射}是一个从交换环构造出的对象。仿射概形与环的联系是仿射簇与其坐标环之间联系的模仿与推广。实际上,我们能以如下方式得到概形的定义。经典代数几何的基本对应是双射
\[
	\{\text{仿射簇}\}\quad \leftrightarrow\quad \{\text{代数闭域$K$上的有限生成无非平凡零因子环}\}
\]

上面对应的左侧是我们感兴趣的朴素几何对象:多项式族的零点集。如果我们说这些是感兴趣的对象,我们将得到右侧受限制的环范畴。当我们改用相反的观点,即我们不再满足于限制“有限生成”、“无非平凡零因子”或者“$K$-代数”,希望右侧包含所有的交换环,此时怎样的几何对象将出现在左侧呢?于是,概形理论出现了,问题的答案是“仿射概形”,在这节中,我们将展示如何拓展上面对应到下图:
\[
\begin{xy}
	\xymatrix{
		\{\text{仿射簇}\}\ar[d]&\ar@{<->}[r]&&\left\{\text{代数闭域$K$上的有限生成无非平凡零因子环}\right\}\ar[d]\\
		\{\text{仿射概形}\}&\ar@{<->}[r]&&\{\text{含幺交换环}\}
	}
\end{xy}
\]

我们会看到,实际上,环与对应的仿射概形是等价对象。概形是许多几何命题更自然的背景,概形的语言将允许我们在下面的章节中一般化我们的构造。

\nottran

令$R$是一个交换环。$R$所定义的仿射概形将被记作$\spec R$,称为环$R$的\idx{谱}. \nottran

\subsection{作为集合的概形}

我们定义$\spec R$中的一个点是$R$的一个素理想。为了避免符号混用,我们通常以$[\pp]$来记$\spec R$中对应于素理想$\pp$的点。按照一般习惯,我们将认为$R$本身不是一个素理想。当然,如果$R$是一个整环,则$(0)$理想是素理想。

如果$R$是一个寻常的代数闭域上的仿射概形$V$的坐标环,$\spec R$中的对应于$R$的极大理想的点将对应于仿射概形中的点,同时对$V$的每个不可约子簇,$\spec R$中还将对应有一点。这些新添的点,对应于正维数的子概形,虽然一开始会让人心神不安但后来就变得方便起来,它们将担当经典代数几何中“一般点”的角色。

\begin{exe}
确定$\spec R$,当$R$是~~~~(a) $\zz$;~~~~(b) $\zz/(3)$;~~~~(c) $\zz/(6)$;~~~~(d) $\zz_{(3)}$;~~~~(e) $\cc[x]$;~~~~(f) $\cc[x]/(x^2)$.
\end{exe}

每一个$R$中的元素$f$都将在$X=\spec R$上定义一个“函数”,我们依然记作$f$:对$x=[\pp]\in \spec R$,我们记$\kappa(x)$或者$\kappa(\pp)$为整环$R/\pp$的商域,称为$X$在点$x$处的剩余类域,同时典范映射
\[
	R\to R/\pp\to \kappa(x)
\]
下$f$在$\kappa(x)$中的像,我们定义为$f(x)$.

\begin{exe}
	“函数”$15$在点$(7)\in \spec \zz$的值是什么?在点$(5)$处呢?
\end{exe}

\begin{exe}
	\begin{enumerate}[{(a)}]\setlength{\itemsep}{0pt}
		\item 考虑多项式环$\cc[x]$,令$p(x)$是一个多项式。证明,若$\alpha\in\cc$是一个数,则$(x-\alpha)$是一个$\cc[x]$的素理想,并且,存在$\kappa((x-a))$与$\cc$的自然等同,使得点$(x-\alpha)\in\spec\cc[x]$处的值$p(x)$就是数$p(\alpha)$.
		\item 更一般地,如果$R$是一个代数闭域上的仿射簇$V$的坐标环,$\pp$是一般意义上对应于点$x\in V$的极大理想,则$\kappa(x)=K$以及$f(x)$就是一般意义上函数$f$在点$x$处的值。
	\end{enumerate}
\end{exe}

一般地,“函数”$f$取值的域逐点不同。此外,$f$也没必要由在其“函数”每一点的值决定。比如,如果$K$是一个域,环$R=K[x]/(x^2)$只有一个素理想$(x)$,于是对应元素$x\in R$,虽然是非零的,但诱导了一个“函数”在$\spec R$上处处为零。

我们直接将$\spec R$上的“正则函数”定义为$R$的一个元素。于是一个正则函数给出了$\spec R$上的一个“函数”,但他并不由这个“函数”的值决定。

\subsection{作为拓扑空间的概形}

通过正则函数,我们可以将$\spec$修饰成一个拓扑空间,这个拓扑被称为\textit{Zariski拓扑}. 闭集定义如下,对$R$的任意子集$S$,令
\[
	V(S)=\{x\in \spec R\,|\, f(x)=0 \text{对每一个} f\in S\text{都成立}\}=\{[\pp]\in\spec R\,|\, \pp\supset S\}.
\]

这个定义的动机是使得每个$f\in R$都表现得尽可能像一个连续函数。当然,域$\kappa(x)$没有拓扑,因为它逐点变化让一般意义上的连续无从谈起。但至少,它们都包含了一个零元,于是我们就可以谈$f\in R$在$\spec R$上的零点集,如果$f$像一个连续函数,则零点集应该是一个闭集。因为闭集族相交必须是闭集,我们立即得到了上述定义:$V(S)$就是$S$中所有元素的零点集的交。