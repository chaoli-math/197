\section{点态函子}
fasdfas\footnote{译者注:没找到靠谱的翻译,先这样翻着。}

更形式地,概形$X$的\textit{点态函子}是由$X$决定的“可表”函子,即,函子
\[
	h_X:(\text{schemes})^\circ\to (\text{sets}),
\]
其中$(\text{schemes})^\circ$和$(\text{sets})$分别表示箭头反过来的概形范畴以及集合范畴。$h_X$将每一个概形$Y$变成集合
\[
	h_X(Y)=\mathrm{Mor}(Y,X)
\]
以及每一个态射$f:Y\to Z$变成集合的映射
\[
	h_X(Z)\to h_X(Y)
\]
通过将$g\in h_X(Z)=\mathrm{Mor}(Z,X)$变成复合$g\circ f\in \mathrm{Mor}(Y,X)$.