\section{点函子}

概形的一个有趣地方是它们有着许多并不能由拓扑空间所表达的结构,所以那些集合上的熟悉操作,比如取直积,此时就需要警惕地审查,以免它们在概形上没有意义。 但值得注意的是,许多集合论式的想法可以通过点函子
\footnote{译者注:没找到靠谱的翻译,先这样翻着。}
(the functor of points)这个简单的技术来应用到概形上。虽然这个观点一开始为问题增加了一层复杂性,但它是富于启发性的,因此,点函子以及相关术语的应用相当普遍。我们将在这里简要介绍必要的定义,后面的章节会偶尔使用它们,直至第 \ref{chap:6} 章会详细介绍。

从一个观察开始,概形的点一般来说并不相似:我们有非闭点和闭点;如果我们在非代数闭域上工作,那么即使是闭点也可以通过它们不同的剩余类域来区分。类似地,如果我们在$\mathbb{Z}$上工作,不同点可能具有不同特征的剩余类域;如果我们把点的概念扩展到“底空间只有一点的闭子概形”,我们甚至会有更大的多样性。 而且当然地,概形之间的态射根本不由其底空间之间的映射决定。

然而,存在一种观察概形的方法,它将观察约化到集合上,它就是概形的点函子。更准确地说,我们可能将一个概形看作一族集合,一个概形范畴上的函子,在上面集合中那些熟悉的操作依然成立%
%\footnote{译者吐槽:这样啥都不说,然后藏着掩着说一些特性,读着真是让人摸不着头脑。个人建议,读这本书先把定义看了,然后再反过头看前面的引入,作者爷爷把定义的许多解释先摆出来了。嗯,再或者,似乎可以假设大家都精通这些定义。}%
。在本节中,我们将研究如何用函子来描述概形。这样做的一个巨大回报是,我们能将概形范畴嵌入到一个更大的函子范畴中,在那里许多构造将变得更简单。这种好处很类似于在分析中,我们考虑分布而不是普通的函数%
%\footnote{译者注:说的 是广义函数那一套,大家都懂的。}
。它把在概形范畴的如何进行构造的问题转移到了理解哪些函子来自于概形。此外,许多概形范畴中的几何构造能以一种有用的方式拓展到更大的函子范畴中。

为了引入点函子的概念,我们从一般的范畴理论开始。首先,在那些对象是具有附加结构的集合的范畴,对象$X$的集合$|X|$能用从一个泛对象到$X$的态射集描述;比如:
\begin{compactenum}[(a)]
\item 在可微流形范畴,如果$Z$是只包含一个点的流形,则对于任意的流形$X$,我们有$|X|=\Hom(Z,X)$.
\item 在群范畴,对任意的群$X$我们有$|X|=\Hom(\zz,X)$.
\item 在同态都将单位元映作单位元的含幺环范畴,如果我们令$Z=\zz[x]$,则对任意的含幺环$X$,我们有$|X|=\Hom(Z,X)$.
\end{compactenum}

一般地,对范畴$\mathscr{X}$中的任意对象$Z$,对应
\[
	X\mapsto \Hom_{\mathscr{X}}(Z,X)
\]
定义了一个从范畴$\mathscr{X}$到集合范畴的函子。正如前面第一段所说的,然而,把集合$\varepsilon(X)= \Hom_{\mathscr{X}}(Z,X)$叫做对象$X$的点的集合并不是令人满意的,除非这个函子是\textit{忠实的},即除非对$\mathscr X$的任意两个对象$X_1$和$X_2$,态射
\[
	f:X_1\to X_2
\]
由集合之间的映射
\[
	f':\Hom_{\mathscr{X}}(Z,X_1)\to \Hom_{\mathscr{X}}(Z,X_1)
\]
所决定。

这个条件不是总能满足的。比方说,令$\text{(Hot)}$是$CW$-复形的范畴,其中$\Hom_\text{(Hot)}(X,Z)$是$X$到$Z$之间连续函数的同伦类集合。如果$Z$是一个单点复形,则
\[
	\Hom_\text{(Hot)}(X,Z)=\pi_0(X)
\]
是$X$的连通分支的集合,所以这并不能给出一个忠实函子。而且,我们也不可能选一个更好的对象$Z$. 同样,在概形范畴,没有对象$Z$能干这种事。

为修正这点,Grothendieck机智地选择同时考虑全部的集合$\Mor(Z,X)$而不仅是其中一个!即,我们对每个概形$X$,给定一个“结构化集合”,它包含了所有的集合$\Mor(Z,X)$,同时,对每个态射$f:Z\to Z'$,通过复合$f$给出了映射$\Mor(Z',X)\to \Mor(Z,X)$.

用更形式化地语言来说,概形$X$的\textit{点函子}是由$X$决定的“可表”函子,即,函子
\[
	h_X:(\text{schemes})^\circ\to (\text{sets}),
\]
其中$(\text{schemes})^\circ$和$(\text{sets})$分别表示箭头反过来的概形范畴以及集合范畴。$h_X$将每个概形$Y$变成集合
\[
	h_X(Y)=\Mor(Y,X)
\]
以及每个态射$f:Y\to Z$变成集合的映射
\[
	h_X(Z)\to h_X(Y)
\]
通过将$g\in h_X(Z)=\Mor(Z,X)$变成复合$g\circ f\in \Mor(Y,X)$. “可表函子”的名字来自于这个函子被一个概形$X$所\textit{表示}。集合$h_X(Y)$被称为$X$的\textit{$Y$-值点}的集合(如果$Y=\spec T$是仿射的,我们经常用$h_X(T)$来代替$h_X(\spec T)$,并且称其为$X$的$T$-值点的集合)。

再引入一层抽象,注意到这定义了一个函子
\[
	h:(\text{scheme})\to \text{Fun}((\text{scheme})^\circ,(\text{sets}))
\]
(其中函子范畴的态射是自然变换),满足
\[
	X\mapsto h_X
\]
以及对每个态射$f:X\to X'$给出了一个自然变换$h_X\to h_{X'}$,即对任意的概形$Y$,将$g\in h_X(Y)=\Mor(Y,X)$变成了复合$f\circ g\in h_{X'}(Y)=\Mor(Y,X')$.

当然,当我们操作$S$-概形的时候,我们也应只考虑$S$-概形态射。这种情况完全类似上面的情况:我们这样描述了一个函子
\[
	X\mapsto h_X,
\]
它从$S$-概形范畴到范畴
\[
	 \text{Fun}((\text{$S$-scheme})^\circ,(\text{sets})).
\]

这种看着抽象的想法根植于对方程组解的研究。令$X=\spec R$是一个仿射概形,其中$R=\zz[x_1$, $x_2$, $\dots]/(f_1$, $f_2$, $\dots)$. 如果$T$是任意其他环(可以将$T$想做$\zz$, $\zz/(p)$, $\zz_{(p)}$, $\hat \zz_{(p)}$, $\mathbb{Q}_p$, $\rr$, $\cc$等),一个从$\spec T$到$\spec R$的态射等价于一个从$R$到$T$的环同态,它将由$x_i$的像$a_i$所确定。当然,一族元素$a_i\in T$在这种方式下确定了一个态射,当且仅当,它们是方程组$f_i=0$的解。综上,我们已展示了
\[
	h_X(T)=\left\{
		\parbox{11em}{
			那些是方程组$f_i=0$的解的$T$中的点列$a_1$, $\dots$
		}
	\right\}.
\]

类似地,如果$X$是任意概形,于是$X$是一族仿射概形$X_a$沿着开集黏合起来得到的,于是从一个仿射概形$Y$到$X$的态射可以通过给出$Y$的一个基本仿射开覆盖$Y_{f_a}$以及从$Y_{f_a}$到$X_a$的映射来描述,这些态射在相交的开集上相同(一些$Y_{f_a}$当然可能是空的)。于是,$h_X(Y)$中的一个元素依然可以在一个更广义的上下文中理解为一族方程组的解,对应于某些$X_a$,在特定的多项式的非零点集上相容。

即使有了上面的解释,点函子的概念看上去还是略显乏味:尽管可以用这种新的语言来表述问题,但是否可以用它来解决问题还不甚明晰。我们可以在这套语言下工作的关键在于,许多概形范畴的显然的几何概念能被自然地拓展到更大的函子范畴中。比如,我们可以讨论一个函子的开子函子、闭子函子,光滑函子,一个函子的切空间等等。这些概念将在第 \ref{chap:6} 章展开,在那里我们同样会给出如何使用它们的一个更好理解。

在这章中,我们以两种方式用过“点”这个词:一是概形$X$的点,二是对任意概形$Y$,$X$的$Y$-值点的集合。千万不要混淆这两种用法,因为它们是非常不同的概念:比如,如果$Y=\spec L$,其中$L$是$\mathbb{Q}$的一个有限扩张,于是我们有映射
\[
	\{\textit{$X$的$Y$-值点}\}\to |X|,
\]
但这个一般来说既不是单的也不是满的:它的像是$X$中剩余类域$\kappa(p)$是$L$的子域的那些$p$的集合,以及,在这样的一个点处的纤维是一族从$\kappa(p)$到$L$的环同态。另一个差别是,$|X|$中的点是绝对的,但是$X$的$Y$-值点的集合是相对的,它可能依赖于基概形$S$以及结构态射$X\to S$. 最后,当$S=\spec K$时,$X$的$K$-值点的集合,即那些满足$\kappa(p)=K$的$p\in X$的集合,经常被叫做$X$的\textit{$K$-有理点}。

这两种“点”的概形都具有一些(但不是全部)我们期望的集合范畴中点的性质。比方说,积$X_1\times X_2$的$Y$-值点的集合就是$X_1$和$X_2$的$Y$-值点集合的积。然而,对并$X=U\cup V$的$Y$-值点的集合,此时并不是$U$, $V$的$Y$-值点集合的并(比方说,恒等映射$X\to X$是$X$的一个$X$-值点,但一般来说它并不包含于$U$或者$V$中)。相较之下,对概形$X$的点的集合$|X|$,上面的论断成立。

现在,我们已经列出了概形理论的基本定义。下一章中,我们将给出许多例子,从这些例子身上,读者将得到概形的一些“直观与感受”。