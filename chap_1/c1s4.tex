\section{点函子}

关于概形的一个有趣的事情正是他们拥有许多并不由它们的拓扑空间所表达的结构,所以集合上的熟悉操作,比如取直积此时就需要警惕的审查,以免他们没有意义。 因而,值得注意的就是,许多集合论式的想法可以通过点函子\footnote{译者注:没找到靠谱的翻译,先这样翻着。}这个简单的技术来应用到概形上。虽然,这个观点最初为问题增加了一层复杂性,但它们往往是非常明显的。因此,点函子以及伴随的术语应用相当普遍。我们将在这里简要介绍必要的定义,并在以下章节中偶尔使用它们,然后在第\ref{chap:6}章中详细介绍。

我们从一个观察开始,即概形的点一般来说并不相似:我们有非闭点和闭点;如果我们在非代数闭域上工作,那么即使是闭点也可以通过它们不同的剩余类域来区分。 类似地,如果我们在$\mathbb{Z}$上工作,不同点可能具有不同特征的剩余类域;如果我们把点的概念扩展到“支撑于一点的闭子概形”,我们甚至会有更大的多样性。 而且当然地,概形之间的态射根本不由其在支撑之间的映射决定。

% 然而,有一种方法是通过其方法来观察一个概形 - 这样可以将其减少到一个集合。更准确地说,我们可能将一个计划看作是一套有组织的集合集合,即计划类别的函子,熟悉的集合行为就像往常一样。 在本节中,我们将研究这个说明。 很大的回报是,我们将看到嵌入更大类型的函子中的概形类别,其中许多结构更容易。 这样做的好处就是分析工作中的优势,而不仅仅是普通的功能; 它将将计划类别中的建设转变为理解哪些函子来自计划的问题。 此外,在概形类别中出现的许多几何构造可以以有用的方式扩展到更大类的函子。

为了引入点函子的概形,我们从一般的范畴理论开始。首先,许多范畴的对象是具有附加结构的集合,对象$X$的集合$|X|$能被从一个泛对象到$X$的态射集所描述;比如:
\begin{compactenum}[(a)]
\item 在可微流形范畴,如果$Z$是只包含一个点的流形,则对于任意的流形$X$,我们有$|X|=\Hom(Z,X)$.
\item 在群范畴,对任意的群$X$我们有$|X|=\Hom(\zz,X)$.
\item 在态射将单位元映作单位元的含幺环范畴,如果我们令$Z=\zz[x]$,则对任意的含幺环$X$,我们有$|X|=\Hom(Z,X)$.
\end{compactenum}

一般地,对范畴$\mathscr{X}$中的任意对象$Z$,对应
\[
	X\mapsto \Hom_{\mathscr{X}}(Z,X)
\]
定义了一个从范畴$\mathscr{X}$到集合范畴的函子。

更形式地,概形$X$的\textit{点函子}是由$X$决定的“可表”函子,即,函子
\[
	h_X:(\text{schemes})^\circ\to (\text{sets}),
\]
其中$(\text{schemes})^\circ$和$(\text{sets})$分别表示箭头反过来的概形范畴以及集合范畴。$h_X$将每一个概形$Y$变成集合
\[
	h_X(Y)=\mathrm{Mor}(Y,X)
\]
以及每一个态射$f:Y\to Z$变成集合的映射
\[
	h_X(Z)\to h_X(Y)
\]
通过将$g\in h_X(Z)=\mathrm{Mor}(Z,X)$变成复合$g\circ f\in \mathrm{Mor}(Y,X)$.