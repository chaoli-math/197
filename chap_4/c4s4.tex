\section{型}\label{s:4.4}

% p.204

令$S$为任意概形,而$X$为任意$S$-概形。我们称$S$-概形$Y$是$X$上的一个\textit{型},若对每一点$p\in S$,
$S$中存在一个$p$的开邻域$U$以及一个平坦满同态$T\to U$使得$Y\times_S T\cong X\times_S T$作为$T$-概形同构。

从一个经典的例子开始,考虑一个实射影平面$\bbp_\rr^2$上由方程$X^2+Y^2+Z^2=0$给出的二次曲线,即曲线
$C=\proj \rr[X,Y,Z]/(X^2+Y^2+Z^2)$. 曲线没有定义在$\rr$上的点,即剩余类域为$\rr$的点,于是并不能同构于
射影直线$\bbp_\rr^1$. 然而,将基域推广到$\rr$的代数闭域$\cc$后,
$C\times_{\spec \rr}\spec \cc\cong \proj \cc[X,Y,Z]/(X^2+Y^2+Z^2)\cong \bbp_\cc^1$. 因此,$C$是一个
$\bbp_\rr^1$在$\spec \rr$上的型,或者更简明地,一个射影直线的$\rr$-型。

另一个例子,读者可以检查,域扩张$\spec \mathbb Q[x]/(x^2+1)$在$\spec \mathbb Q$上是一个包含两个不同的
点的概形的型,然而$\spec \zz/(2)[x]/(x^2+1)$在$\spec \zz/(2)$上是一个二次点的型。

在数论中,观察在给定曲线的型的族中有理点集的变化是有意思的。我们给出一个例子,对任意的$t\in \mathbb Q$,
\textit{Pell方程}
\[
    ty^2=x^2-1
\]
的有理点$(x,y)$的集合是曲线$C_t=\spec \mathbb Q[x,y]/(ty^2-x^2+1)$上的$\mathbb Q$-有理点集。这些曲线$C_t$
是$\spec \mathbb Q$上的$\mathbb P^1$的型。类似地,曲线$E_t=\spec \mathbb Q[x,y]/(ty^2-x^3+1)$的$j$-不变量
都为零,因此曲线$E_1\subset \mathbb P_{\mathbb Q}^2$的型都由$y^2=x^3-1$给出(间上面的第 \ref{s:4.2.3} 节和
下面的第 \ref{s:6.2.4} 节),但具有变化的算数性质。

在上面的各个例子中,容易看到,给出的曲线是互相的型,曲线$E_t\times_{\spec \mathbb Q}\spec \mathbb Q[\sqrt{t}]$
和$E_1\times_{\spec \mathbb Q}\spec \mathbb Q[\sqrt{t}]$是明显同构的,没那么显然但也不难可以看到,他们并不是
全部同构的。(实际上,最\naive 的猜想是$E_t\cong E_1$当且仅当$t\in (\mathbb Q^*)^2$,即,$t$是一个非零有理数的
平凡,这个猜想是对的,但证明他是个并不平凡习题。)

% p.205


\nottran