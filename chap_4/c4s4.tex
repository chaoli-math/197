\section{型}\label{s:4.4}

% p.204

令$S$为任意概形,而$X$为任意$S$-概形。我们称$S$-概形$Y$是$X$上的一个\textit{型},若对每一点$p\in S$,$S$中存在一个$p$的开邻域$U$以及一个平坦满同态$T\to U$使得$Y\times_S T\cong X\times_S T$作为$T$-概形同构。

从一个经典的例子开始,考虑一个实射影平面$\bbp_\rr^2$上由方程$X^2+Y^2+Z^2=0$给出的二次曲线,即曲线$C=\proj \rr[X,Y,Z]/(X^2+Y^2+Z^2)$. 曲线没有定义在$\rr$上的点,即剩余类域为$\rr$的点,于是并不能同构于射影直线$\bbp_\rr^1$. 然而,将基域推广到$\rr$的代数闭域$\cc$后,$C\times_{\spec \rr}\spec \cc\cong \proj \cc[X,Y,Z]/(X^2+Y^2+Z^2)\cong \bbp_\cc^1$. 因此,$C$是一个$\bbp_\rr^1$在$\spec \rr$上的型,或者更简明地,一个射影直线的$\rr$-型。

另一个例子,读者可以检查,域扩张$\spec \mathbb Q[x]/(x^2+1)$在$\spec \mathbb Q$上是一个包含两个不同的点的概形的型,然而$\spec \zz/(2)[x]/(x^2+1)$在$\spec \zz/(2)$上是一个二次点的型。

\nottran

% p.205