\section{型}\label{s:4.4}

% p.204

令$S$为任意概形,而$X$为任意$S$-概形。我们称$S$-概形$Y$是$X$上的一个\textit{型},若对每一点$p\in S$,
$S$中存在一个$p$的开邻域$U$以及一个平坦满同态$T\to U$使得$Y\times_S T\cong X\times_S T$作为$T$-概形同构。

从一个经典的例子开始,考虑一个实射影平面$\bbp_\rr^2$上由方程$X^2+Y^2+Z^2=0$给出的二次曲线,即曲线
$C=\proj \rr[X,Y,Z]/(X^2+Y^2+Z^2)$. 曲线没有定义在$\rr$上的点,即剩余类域为$\rr$的点,于是并不能同构于
射影直线$\bbp_\rr^1$. 然而,将基域推广到$\rr$的代数闭域$\cc$后,
$C\times_{\spec \rr}\spec \cc\cong \proj \cc[X,Y,Z]/(X^2+Y^2+Z^2)\cong \bbp_\cc^1$. 因此,$C$是一个
$\bbp_\rr^1$在$\spec \rr$上的型,或者更简明地,一个射影直线的$\rr$-型。

另一个例子,读者可以检查,域扩张$\spec \mathbb Q[x]/(x^2+1)$在$\spec \mathbb Q$上是一个包含两个不同的
点的概形的型,然而$\spec \zz/(2)[x]/(x^2+1)$在$\spec \zz/(2)$上是一个二次点的型。

在数论中,观察在给定曲线的型的族中有理点集的变化是有意思的。我们给出一个例子,对任意的$t\in \mathbb Q$,
\textit{Pell方程}
\[
    ty^2=x^2-1
\]
的有理点$(x,y)$的集合是曲线$C_t=\spec \mathbb Q[x,y]/(ty^2-x^2+1)$上的$\mathbb Q$-有理点集。这些曲线$C_t$
是$\spec \mathbb Q$上的$\mathbb P^1$的型。类似地,曲线$E_t=\spec \mathbb Q[x,y]/(ty^2-x^3+1)$的$j$-不变量
都为零,因此曲线$E_1\subset \mathbb P_{\mathbb Q}^2$的型都由$y^2=x^3-1$给出(间上面的第 \ref{s:4.2.3} 节和
下面的第 \ref{s:6.2.4} 节),但具有变化的算数性质。

在上面的各个例子中,容易看到,给出的曲线是互相的型,曲线$E_t\times_{\spec \mathbb Q}\spec \mathbb Q[\sqrt{t}]$
和$E_1\times_{\spec \mathbb Q}\spec \mathbb Q[\sqrt{t}]$是明显同构的,没那么显然但也不难可以看到,他们并不是
全部同构的。(实际上,最\naive 的猜想是$E_t\cong E_1$当且仅当$t\in (\mathbb Q^*)^2$,即,$t$是一个非零有理数的
平凡,这个猜想是对的,但证明他是个并不平凡的习题。)

% p.205

射影空间$\mathbb P^n$(对任何$n$)的$S$-型的同构类以自然的方式构成了一个群,
叫做$S$的\textit{Brauer群}(或,当$S=\spec K$为一个域的谱时,$K$的Brauer群),
其中射影空间$\mathbb P_S^n$为恒等元。这个群可以由Galois上同调计算,见
Serre [1975]. 下面是Brauer群的构造,其与数论有关。

令$K$为一个域,而$A$为一个$K$上的$n$-维 Azumaya 代数,即$A$是一个代数,
作为一个向量空间是$n$-维的,有一个非平凡的双边理想,而其中心恰好就是$K$.
比如,所有$K$上的$d\times d$矩阵构成的代数$M_d(K)$,这是个$d^2$-维 Azumaya 代数。
从Wedderburn结构定理,如果$A$是一个$K$上的$n$-维 Azumaya 代数,且$n$是一个平凡,
记$n=d^2$,则$\overline{K} \otimes A \cong M_d(\overline{K}) \cong 
\overline{K} \otimes M_d(K)$ (在这意义下,$A$是$M_d(K)$的一个型)。

将$M_d(K)$等同到$K$上的一个$d$-维向量空间$V$的自同态代数,容易看到$M_d(K)$
的左理想都具有形式
\[
    \left\{a \in M_d(K) \mid \operatorname{Im}(a) \subset W\right\},
\]
其中$W\subset V$是一个子空间。此左理想的向量空间维数为$\dim(V)\dim(W)$.
特别地,维数为$d(d-1)$的左理想对应到$V$中的超平面,即$\mathbb P(V)$中的点。
于是,$M_d(K)$中的$d(d-1)$-平面的Grassmannian的子概形在$M_d(K)$的乘法下封闭,
即是理想,这个子概形同构于$\mathbb P^d(K)$.

\[
    S \otimes A_G \subset A_G \otimes A_G \xrightarrow{\text { multiplication }}
     A \longrightarrow A / S
\]


% p.206

\begin{pro}\label{pro:4.84}
    $C_U$是$U$上的$\mathbb P_U^1$的一个非平凡型,$C_L$是
    $L$上的$\mathbb P_L^1$的一个非平凡型。
\end{pro}

这里的关键在于,尽管每个$\mathbb P_K^2$中的光滑二次曲线是有理的,
但并没有一种方式可以在一个$U$的Zariski开子集一致地对每个光滑二次曲线
选取一个有理的参数化。

\[
    V=C_U \backslash\left(C_U \cap(U \times M)\right) \subset U \times \mathbb{P}_K^2 .
\]

\[
    C_V=V \times_U C_U \subset U \times \mathbb{P}_K^2 \times \mathbb{P}_K^2
\]


\[
    \varphi:(C, p ; q) \longmapsto(C, p ; \overline{p, q} \cap M)
\]

% p.207

\[
    1+a F^2+b G^2+c F+d G+e F G=0 .
\]

\[
    1+a \cdot\left(\frac{f(a, b)}{h(a, b)}\right)^2+b \cdot\left(\frac{g(a, b)}{j(a, b)}\right)^2=0
\]

\[
    h(a, b)^2 j(a, b)^2+a \cdot f(a, b)^2 j(a, b)^2+b \cdot g(a, b)^2 h(a, b)^2=0 .
\]

\nottran

\begin{exe}\label{exe:4.86}
    $K$是一个域,证明对$\spec K$上的$\mathbb P_K^1$的任意型$X$同构于一个平面
    圆锥曲线$C\subset \mathbb P_K^2$. 特别地,推出一个$\spec K$上的
    $\mathbb P_K^1$的型同构于$\mathbb P_K^1$当且仅当它有一个点的剩余类域为$K$.
\end{exe}

\begin{exe}\label{exe:4.87}
    用上个习题,证明$\spec K$上的$\mathbb P_K^1$的型$X$同构于$\mathbb P_K^1$
    当且仅当他有一个零维子概形$\Gamma\subset X$的次数为奇数,即,其$\Gamma$
    的坐标环作为$K$-向量空间的维数为奇数。
\end{exe}

\begin{exe}\label{exe:4.88}
    证明,$\spec K$上的$\mathbb A_K^1$不存在非平凡的型,即,所有
    $\spec K$上的$\mathbb A_K^1$的型都同构于$\mathbb A_K^1$.
\end{exe}

\begin{exe}\label{exe:4.89}
    用例子证明,上面三个习题都不再正确若我们不再明确$S=\spec K$,即,如果
    我们考虑一个一般的概形$S$上的$\mathbb P_S^1$和$\mathbb A_S^1$的型。
    (为找到反例,取$S=\mathbb P_K^1$为一个域上的射影直线以及考虑
    仿射平面$\mathbb A_K^2$关于原点的爆破。)
\end{exe}

另一种万有圆锥曲线的无理度的推广将在 Exercise \ref{exe:6.38} 中讨论,
这次断言$d$-次万有有理正规曲线是有理的当且仅当$d$为奇数。

另一个射影空间的型的例子出现在几何语境中,见上面的 Exerecise \ref{exe:4.71}.