\section{爆破}\label{s:4.2}

爆破是经典代数几何的一个基本工具。它用以消解奇点、消解有理映射的不定性,以及相互联系双有理簇。
称一个簇是另一个簇沿着一个给定子簇的爆破,表出了此二者间的一种关系,一方面其同时足够接近去密切联系两个簇的结构,另一方面其足够灵活,在簇之间的映射的表达式中非常常见。
在这节中,我们将推广爆破的定义到概形范畴,定义任意的(Noether)概形沿着任意闭子概形的爆破。

以这种方式概括爆破的定义实际上有两个目的。首先是期望的好处:概形和簇的爆破的作用相同,即消解奇点或者联系两个双有理概形(比如说,我们将在第 \ref{s:4.2.4} 节中爆破算术概形)。

此外,我们还可以看到,甚至在簇之间的映射的语境中,概形的语言意味着相关概念非常有用之推广,特别地,可以去讨论一个概形$X$沿着一个可能非既约的子概形$Y\subset X$的爆破。比方说,我们将在下面的第 \ref{s:4.2.3} 节展示这点,在那里用这个推广的爆破的定义,我们推广了对非奇异二次曲面作为平面沿着一个二次曲线的爆破的传统的描述。类似地,在第 \ref{4.2.3} 节,我们将看到一个自然出现的簇之间的映射是沿着一个子概形的爆破。这些例子实际上并不特殊:当我们以这种方式推广“爆破”的定义,则任意簇之间的射影双有理态射都是爆破!这在Hartshorne [1977, Theorem II.7.17]中有证明。

\subsection{定义与构造}\label{s:4.2.1}

% p.163

下面,我们将假设读者熟悉经典语境中的经典的爆破,即簇沿着一个非奇异子簇的爆破。(这个材料在许多文本中被详细地讨论了,比如Harria [1995], Hartshorn [1977, Chapter 1]和Shafarevich [1974].)在最简单的情况中,比如对一个代数闭域$K$上的仿射平面中的一个既约点爆破,爆破映射将与经典的情形一模一样。我们从复习一个例子开始,对平面上原点的爆破,来看如何将通过粘起来定义的经典的爆破同样带到域上的概形范畴。将此推广到任意概形$X$沿着任意闭子概形$Y\subset X$的爆破只是以足够自然的方式表达这种标准构造的问题。在下面的小节中,我们将给出几个一般的爆破的刻画:一个定义、两个构造以及一个在一些特殊例子中的描述,比如一个概形沿着一个正则子概形的爆破(Definition \ref{defi:4.15})。

\paragraph*{例子:平面的爆破}\addcontentsline{toc}{subsubsection}{例子:平面的爆破}

\begin{exa}\label{exa:4.14}
	我们从域$K$上的仿射平面$\mathbb A_K^2=\spec K[x,y]$对原点的爆破$Z$开始。它可以由最多两个开集的并所描述,其中每个都同构于$\mathbb A_K^2$:我们令$U'=\spec K[x',y']$和$U''=\spec K[x'',y'']$,然后考虑与映射$\varphi':U'\to \mathbb A_K^2$和$\varphi'':U''\to \mathbb A_K^2$对偶的环同态
	\[
	\begin{aligned}
		(\varphi')^\#:K[x,y]&\longrightarrow K[x',y']&\quad \text{和}\qquad (\varphi'')^\#:K[x,y]&\longrightarrow K[x'',y'']\\
		x&\longmapsto x'& x&\longmapsto x''y''\\
		y&\longmapsto x'y'& y&\longmapsto y''.
	\end{aligned}
	\]
	映射$\varphi'$给出了开集
	\[
	U'_x=\spec K[x',y',{x'}^{-1}]\quad \text{和}\quad 
	U_x=\spec K[x,y,x^{-1}]
	\]
	之间的同构,类似地,$\varphi''$给出了开子集$U''_y=\spec K[x',y',1/y']$和$U_y=\spec K[x,y,y^{-1}]$之间的同构。特别地,它们给出了
	交集$U_{xy}=U_x\cap U_y=\spec K[x,y,x^{-1},y^{-1}]$的原像
	\[
	U'_{xy}=\spec K[x',y',{x'}^{-1},{y'}^{-1}]\quad \text{和}\quad 
	U''_{xy}=\spec K[x'',y'',{x''}^{-1},{y''}^{-1}]
	\]
	之间的同构。我们于是能等同开集$U'_{xy}\subset U'$和$U''_{xy}\subset U''$,然后将$U'$和$U''$粘起来得到一个概形
	\[
		Z=U'\cup U''=\spec K[x',y']\bigcup_{U'_{xy}\cong U''_{xy}}\spec K[x'',y''],
	\]
	其中同构$U'_{xy}\cong U''_{xy}$由环同态
	\[
	\begin{aligned}
		K[x',y',{x'}^{-1},{y'}^{-1}]&\longrightarrow 
		K[x'',y'',{x''}^{-1},{y''}^{-1}]\\
		x'&\longmapsto x''y''\\
		y'&\longmapsto {x''}^{-1}
	\end{aligned}
	\]
	给出。我们称并$Z$,连同结构态射$\varphi:Z\to \mathbb A_K^2$,为$\mathbb A_K^2$在原点的\textit{爆破}。原点的原像$E=\varphi^{-1}(0,0)\subset Z$同构于$\mathbb P_K^1$(这被称为这个爆破的\textit{例外除子}),而$\varphi$除此之外是一个同构,即$Z\setminus E\cong \mathbb A_K^2\setminus \{(0,0)\}$.
\end{exa}

% p.164

一种理解这个构造的方式是去观察到,爆破的开集的坐标环被扩大了以分别加入$y'=y/x$和$x''=x/y$. 这导致了一系列结果。首先,$\mathbb A_K^2$上的函数对$x$, $y$在原点的补集上定义了一个映射$f:\mathbb A_K^2\setminus \{(0,0)\}\to \mathbb P_K^1$:在经典语言中,这是映射$(a,b)\mapsto [a,b]$,或者用更现代的语言,这是对应于由$(f,g)\mapsto xf+yg$给出的满射$\mathscr O\oplus \mathscr O\to \mathscr O$. 这个映射并不能延拓为$\mathbb A_K^2$上的一个正则映射,但是如果我们复合$f$与同构$Z\setminus E\cong \mathbb A_K^2\setminus \{(0,0)\}$,我们看到其确实延拓为了整个$Z$上的一个正则函数。这是因为由函数$x$和$y$(的拉回)生成的理想,在$Z$上是局部主理想(有一个非零因子生成),于是在$x$和$y$有公共零点的地方,我们可以简单地将齐次矢量$[x,y]$除以它们的公共因子来延拓这个映射。在爆破中扩大坐标环的另一好处是其分离了穿过原点的曲线。即,如果$L$和$L'$是穿过$\mathbb A_K^2$原点的不同曲线,则$L\setminus \{(0,0)\}$和$L'\setminus \{(0,0)\}$的原像有着不同的闭包,如图所示(这些都是映射$f$的纤维)。
\inclugra{1.png}
同样地,如果我们有一条曲线$C\subset \mathbb A_K^2$在原点有一个结点,则$C$中原点的补集的原像在$Z$中非奇异,且交例外除子于两点。

\paragraph*{一般的爆破的定义}\addcontentsline{toc}{subsubsection}{一般的爆破的定义}
我们将用这些观察作为出发点来推广一个任意概形沿着任意子概形的爆破的定义。基本的事实如下,在一个$X$沿着子概形$Y\subset X$的爆破$\varphi:\Bl_Y(X)\to X$中,$Y$的原像是局部主的。为形式化这点,我们从一个定义开始:

% p.165

\begin{defi}\label{defi:4.15}
	令$X$为任意概形,$Y\subset X$是一个子概形。我们称$Y$是一个$X$中的\textit{Cartier子概形},如果其局部是单个非零因子的零点集,即,对任意的$p\in X$,$X$中存在一个$p$的仿射邻域$U=\spec A$和存在某个非零因子$f\in A$使得$Y\cap U=V(f)\subset U$. 更一般地,我们称$Y$是一个\textit{正则子概形},如果其局部是$X$上函数的一个正则列\footnote{
		译者注:设$R$是一个交换环,则一族元素$f_1$, $\dots$, $f_r$被称为一个正则列,如果对每个$i$,$f_i$都是$R/(f_1,\dots,f_{i-1})$中的非零因子,且$R/(f_1,\dots,f_{r})\neq 0$.
	}%
	(regular sequence)的零点集。
\end{defi}

\begin{defi}\label{defi:4.16}
	令$X$为任意概形,$Y\subset X$是一个子概形。\textit{$X$沿着$Y$的爆破},记作$\varphi:\Bl_Y(X)\to X$,是一个由下面几个性质刻画的态射:
	\begin{compactenum}[(1)]
		\item $Y$的原像$\varphi^{-1}(Y)$是$\Bl_Y(X)$中的一个Cartier子概形。
		\item $\varphi:\Bl_Y(X)\to X$具有万有性质;即,如果$f:W\to X$是任意的态射使得$f^{-1}(Y)$是一个$Z$中的Cartier子概形,则存在一个态射$g:W\to \Bl_Y(X)$使得$f=\varphi\circ g$.
	\end{compactenum}
	在爆破$\Bl_YX$中$Y$原像$E=\varphi^{-1}(Y)$被称为爆破的\textit{例外除子},而$Y$是爆破的\textit{中心}。
\end{defi}

很清楚,这些性质唯一刻画了概形沿着一个子概形的爆破$\varphi:\Bl_Y(X)\to X$. 但并不清楚的是爆破的存在性,但我们下面会看到这确实存在。

仿射情形的爆破可以以态射的图像的闭包来简单地实现,下面将首先描述这个构造。我们从推广Example \ref{exa:4.14} 的构造到任意环上的仿射空间对原点的爆破开始。

\begin{exa}\label{exa:4.17}
	令$A$为任意环,再令$\mathbb A_A^n=\spec A[x_1,\dots,x_n]$. 考虑概形
	\[
	U_i=\spec T_i\cong \mathbb A_A^n,
	\]
	其中
	\[
		T_i=A\left[\frac{x_1}{x_i},\dots,\frac{x_n}{x_i},x_i\right]
	\]
	是$T=A[x_1,x_1^{-1},\dots,x_n,x_n^{-1}]$在$A$上由函数$x_j/x_i$和$x_i$生成的子代数。环$(T_{i})_{x_j}$和$(T_j)_{x_i}$作为$T$的子环是相同的,于是我们有交换同构
	\[
		(U_i)_{x_j}\cong (U_j)_{x_i}.
	\]
	因此我们可以构造一个概形$Z$,其是$U_i$的并且上面的开集被等同了。注意到,对应含入$A[x_1,\dots,x_n]\hookrightarrow T_i$的态射$U_i\to \mathbb A_A^n$在交叠部分相容给出了一个自然的结构态射$\varphi:Z\to \mathbb A_A^n$.
\end{exa}

这个例子展现了诸多 Example \ref{exa:4.14} 中描述的经典爆破的特性:
% p.166
\begin{compactenum}[(1)]
	\item 令$U=\mathbb A_A^n\setminus V(x_1,\dots,x_n)$为$V(x_1,\dots,x_n)$(“原点”)在$\mathbb A_A^n$中的补。我们有一个由函数$(x_1,\dots,x_n)$给出的态射
	\[
		\alpha_{(x_1,\dots,x_n)}:U\to \mathbb P_A^{n-1},
	\]
	或者,更形式地,由满射
	\[
		\begin{aligned}
			\mathscr O_U^n & \longrightarrow \mathscr O_U,\\
			(a_1,\dots,a_n)&\longmapsto \sum a_ix_i.
		\end{aligned}
	\]
	给出的,则$Z$是 \textit{$\alpha$的图像在$\mathbb A_A^n\times_A \mathbb P_A^{n-1}=\mathbb P_A^{n-1}$中的闭包}。为看到这点,我们我们观察到$(U_i)_{x_i}\subset Z$为映射
	\[
		\alpha_{(x_1,\dots,x_n)}|_{(U_i)_{x_i}}:(U_i)_{x_i}\to (\mathbb P_A^{n-1})_{x_i}=\spec \left[\frac{x_1}{x_i},\dots,\frac{x_n}{x_i}\right]
	\]
	的图像,而开集$(U_i)_{x_i}$在$Z$中稠密。
	\item $V(x_1,\dots,x_n)\subset \mathbb A_A^n$的原像$E=\varphi^{-1}V(x_1,\dots,x_n)\subset Z$在结构映射$\varphi:Z\to \mathbb A_A^n$同构于$\mathbb P_A^{n-1}$,而
	\[
		\varphi:Z\setminus E\xrightarrow{\sim}\mathbb A_A^n\setminus V(x_1,\dots,x_n)
	\]
	是一个同构。
	\item 因子$(x_1,\dots,x_n)T_i=(x_i)T_i$,$V(x_1,\dots,x_n)\subset \mathbb A_A^n$的原像$E\subset Z$是局部由单一方程定义的。
\end{compactenum}

\begin{pro}\label{pro:4.18}
	态射$\varphi:Z\to \mathbb A_A^n$是$\mathbb A_A^n$沿着子概形$V(x_1,\dots,x_n)$的爆破。
\end{pro}

\begin{proof}
	我们已经看到$Z\to \mathbb A_A^n$满足Definition \ref{defi:4.16} 的条件(1),还需证明,如果$\psi:W\to \mathbb A_A^n$是任意态射使得$\psi^{-1}V(x_1,\dots,x_n)$为Cartier的,则$\psi$可经由$\varphi$分解,即,存在一个映射$\alpha:W\to Z$满足$\psi=\varphi\circ\alpha$.

	我们首先对$W=\spec R$且$R$是一个局部环的情况证明。通过映射$\varphi^\#:A[x_1,\dots,x_n]\to R$将$R$看作$A[x_1,\dots,x_n]$上的代数。因为理想$(x_1,\dots,x_n)R$是主理想,Nakayama引理(Eisenbud [1995, Corollary 4.8])告诉我们它是由某个$x_i$生成的。更准确地说,如果
	\[
		(x_1,\dots,x_n)R=(\gamma),
	\]
	则我们可以记
	\[
		\gamma=\alpha_1 x_1+\cdots +\alpha_n x_n
	\]
	对某些$\alpha_i\in R$,类似地,$x_i=\beta_i\gamma$. 于是
	\[
		\gamma=\sum_i \alpha_i x_i=\sum_i \alpha_i \beta_i \gamma,
	\]
	从这里,我们看到至少有一个$\beta_i$是$R$中的单位元,即$(x_1,\dots,x_n)R=(\gamma)=(x_i)$对某个$i$成立。

% p.167
	我们现在可以对每个$j$记$x_j=\nu_j x_i$(其中$\nu_j=\beta_j\beta_i^{-1}$),进而定义想要的映射
	\[
		\alpha:W\to U_i\hookrightarrow Z
	\]
	作为环同态
	\[
	\begin{aligned}
		A\left[\frac{x_1}{x_i},\dots,\frac{x_n}{x_i},x_i\right] & \longrightarrow R,\\
		\frac{x_i}{x_j}&\longmapsto \nu_j
	\end{aligned}
	\]
	的对偶。现在假设$W$为任意概形,$\psi:W\to \mathbb A_A^n$是一个态射且$\psi^{-1}V(x_1,\dots,x_n)$是Cartier的。对每点$w\in W$,上面的结论产生了一个映射$\alpha:\spec \mathscr O_{W,w}\to Z$,其像包含于某个覆盖$Z$的仿射开集$U_i\cong \mathbb A_A^n\subset Z$. 这样一个映射可以延拓到$w\in W$的Zariski开领域上,在上面像$a^\#(x_j/x_i)$是正则的,于是我们可以找到$W$的一个开覆盖$W_k$以及态射$\alpha_k:W_k\to Z$使得$\varphi\circ \alpha_k=\psi|_{W_k}$.

	我们将证明映射$\alpha_k$在交叠$W_i\cap W_j$上相容,进而定义了一个整个$W$上的态射,这就完成了我们的证明。

	因为$\varphi$的限制$Z\setminus E\to \mathbb A_A^n\setminus V(x_1,\dots,x_n)$是一个同构,所以证明原像$\psi^{-1}(\mathbb A_A^n\setminus V(x_1,\dots,x_n))$在$W$中稠密足矣。但是从假设,$\psi^{-1}V(x_1,\dots,x_n)$在$W$中是一个Cartier除子。因此下面的引理就完成了证明。

	\begin{lem}\label{lem:4.19}
		如果$X\subset Y$是一个概形的Cartier子概形,则$Y\setminus X$在$Y$中稠密(作为概形而不仅仅是拓扑空间)。
	\end{lem}

	\begin{proof}
		我们可以假设$Y$是仿射的,记$Y=\spec A$,于是$X=V(f)$,其中$f\in A$是一个非零因子。说存在一个逆紧闭子概形$Y'$包含$Y\setminus X$,就是说局部化映射$A\to A_f$经由$A/I(Y')$分解。但因为$f$是一个非零因子,局部化映射是一个单射。
	\end{proof}
	\let\qed\relax
\end{proof}

\begin{exe}\label{exe:4.20}
	\begin{compactenum}[(a)]
		\item 证明Lemma \ref{lem:4.19} 的结论对
		\[
			X=V(x)\subset Y=\spec K[x,y]/(xy,y^2)
		\]
		失效。
		\item 更一般地证明,其在$Y$的所有局部主子概形中刻画了Cartier子概形。
		\item 证明$\Bl_Y=\varnothing$ 当且仅当$\operatorname{supp} Y=\operatorname{supp} X$.
	\end{compactenum}
\end{exe}

Proposition \ref{pro:4.18} 的构造将产生所有仿射概形的爆破,只要我们理解了爆破是如何作用在子概形上,或更一般地,在拉回下。这直接来自于定义:

\begin{pro}\label{pro:4.21}
	令$X$为任意概形,$Y\subset X$是一个子概形,而$\varphi:\Bl_Y(X)\to X$为$X$沿着$Y$的爆破。令$\nu:X'\to X$为任意态射,$Y'=\nu^{-1}(Y)\subset X'$. 如果$W$是原像$\pi_1^{-1}(X'\setminus Y')$在纤维积$X'\times_X \Bl_Y X$中的闭包,则$\pi_1:W\to X'$是$X'$沿着$Y'$的爆破。
\end{pro}

% p.168

这个引理在$X'=X$的情况下已然足够有趣了,它断言了 \textit{$X\setminus Y$在$\Bl_Y X$中的原像是稠密的}。

Proposition \ref{pro:4.21} 最经常应用于$X'\subset X$是一个闭子概形的情况。此时,$W$就是$\Bl_Y X$中原像$\varphi^{-1}(X'\setminus (X'\cap Y))$的闭包。这被称作$X'$在$\Bl_Y X$中的\textit{严格变换}或者\textit{真变换}. (完整的原像$\varphi^{-1}(X')\subset \Bl_Y X$被称作\textit{全变换}.)因此我们可以说,在爆破$\Bl_p \mathbb A_K^2$中,穿过原点$p\in \mathbb A_K^2$的直线们的真变换是不交的(注意到,直线的真变换同构于线本身,它们本该如此,因为原点是每条线上的一个Cartier子概形),以及,一个有结的曲线在结点的爆破在结点拉开的点上都是非奇异的。

当$X'\subset X$是一个开子概形,Proposition \ref{pro:4.21} 就说明了爆破与基变换可交换,即
\[
	\varphi^{-1}(X')\cong \Bl_{X'\cap Y}X'\to X'.
\]
但正确的不止于此:因为$\varphi^{-1}(X'\setminus Y)$是稠密的,在$X$上\textit{只有一个}这样的同构。作为结论,如果$\pi:Z\to X$是一个态射,且我们假设有一个$X$的开覆盖,其中开集$U$在$X$上满足$\pi^{-1}U\cong \Bl_{U\cap Y}U$,则$Z\cong \Bl_Y X$. 简而言之:爆破是由局部确定的。

\begin{proof}[Proposition \ref{pro:4.21} 的证明]
	我们首先检查$Y'$的原像
	\[
		E'=\pi_1^{-1}(Y')\subset W
	\]
	是$W$的一个Cartier子概形。他当然是主的:其原像$E=\varphi^{-1}(Y)\subset \Bl_Y X$在$\Bl_Y X$中是局部主的,同时$E'\subset W$就是它对投影$\pi_2:W\to \Bl_Y X$的原像。此外,因为$W$相伴素理想们正是$X'$并不包含$Y'$的理想的相伴素理想们,则$E$在$\Bl_Y X$中的局部方程并不能拉回到$W$上的一个零因子。

	接着,我们必须验证$W$有万有性质。假设$T$为任意概形,$f:T\to X'$是任意态射使得$Y'$在$T$中的原像$f^{-1}(Y')$是一个Cartier子概形。特别地,因为$f^{-1}(Y')\subset T$是Cartier的,没有$T$的分支或嵌入分支映射到$Y'$,因此$f^{-1}(X'\setminus Y')$在$T$的闭包是整个$T$.

	我们须证明,$f$可以提升为一个态射$g:T\to W$(即,存在一个态射$g:T\to W$使得满足复合$\pi_1\circ g=f$)。分三步来。首先,令
	\[
		h=\nu\circ f:T\to X
	\]
	为$f$与态射$\nu:X'\to X$的复合,因为原像$h^{-1}(Y)=f^{-1}(Y')$是Cartier的,从爆破$\Bl_Y X\to X$的万有性质,$h$可以提升为一个态射$\tilde h:T\to \Bl_Y X$.
% p.169
	接着,映射$f:T\to X'$和$\tilde h:T\to \Bl_Y X$给出了映射
	\[
		\tilde g:T\to X'\times_X \operatorname{Bl}_Y X,
	\]
	其与投影$\pi_1:X'\times_X \Bl_Y X\to X'$的复合为$f$. 最后,因为$\tilde g$将原像$f^{-1}(X'\setminus Y')$映射到$W$,$f^{-1}(X'\setminus Y')$在$T$中的闭包就是$T$,于是映射$\tilde g:T\to X'\times_X \operatorname{Bl}_Y X$经由$W$在$X'\times_X \Bl_Y X$的含入映射分解,其给出了想要的映射$g:T\to W$.
\end{proof}

我们现在已经到了沿着任意闭子概形爆破一个仿射概形的地方了。若$X=\spec A$以及$f_1,\dots,f_n\in A$,则$(f_1,\dots,f_n)$定义了一个态射
\[
	\alpha_{(f_1,\dots,f_n)}:U=X\setminus V(f_1,\dots,f_n)\longrightarrow \mathbb P_A^{n-1};
\]
更准确地,$(f_1,\dots,f_n)$定义了一个映射$\mathscr O_X^n\to \mathscr O_X$,其将$(a_1,\dots,a_n)$映射为$\sum a_if_i$,正为一个$U$上的满射。

\begin{pro}\label{pro:4.22}
令$X=\spec A$是一个仿射概形,再令
\[
	Y=V(f_1,\dots,f_n)\subset X
\]
是一个闭子概形。$Y$在$X$中的爆破是态射
\[
	\alpha_{(f_1,\dots,f_n)}:X\setminus Y\longrightarrow \mathbb P_A^{n-1}
\]
的图像在$X\times_A \mathbb P_A^{n-1}=\mathbb P_A^{n-1}$中的闭包。
\end{pro}

\begin{proof}
	考虑由环同态
	\[
		\begin{aligned}
			A\left[x_1,\dots,x_n\right] & \longrightarrow A,\\
			x_i &\longmapsto f_i
		\end{aligned}
	\]
	给出的嵌入$X\hookrightarrow \mathbb A_A^n=\spec A[x_1,\dots,x_n]$.
	注意到在这个嵌入下我们有$X\cap V(x_1,\dots,x_n)=Y$. 从Proposition \ref{pro:4.21}, $X$沿着$Y$的爆破是一个$X$在$\mathbb A_A^n$沿着$V(x_1,\dots,x_n)$的爆破$Z$中的真变换。从Proposition \ref{pro:4.18},另一方面,$\mathbb A_A^n$沿着$V(x_1,\dots,x_n)$的爆破$Z$是映射
	\[
		\alpha_{(x_1,\dots,x_n)}:\mathbb A_A^n\setminus V(x_1,\dots,x_n)\longrightarrow \mathbb P_A^{n-1}
	\]
	的图像$\Gamma$的闭包。因为$\alpha_{(f_1,\dots,f_n)}$的图像就是$\Gamma$与$X\subset \mathbb A_A^n$的原像的交,其闭包为$X\subset \mathbb A_A^n$在$Z$中的真变换,于是得到了结论。
\end{proof}

在这个命题中,我们限制了子概形$Y\subset X$是由有限多函数$f_i$所定义的,但这并不必要。读者可以检查所有的事情对无限集也成立(尽管态射映到了无限维射影空间)。

% p.170

\paragraph*{作为$\proj$的爆破}\addcontentsline{toc}{subsubsection}{作为Proj的爆破} 我们已经证明了一个仿射概形沿着一个闭子概形的爆破的存在性。我们已然可以用黏合去推出一般情况下爆破的存在性。然而,通过全局$\proj$,存在一个更优雅的方式来定义爆破,一步到位。

\begin{thm}\label{thm:4.23}
	令$X$为一个概形,$Y\subset X$是一个闭子概形。令$\mathscr I=\mathscr I_{Y,X}\subset \mathscr O_X$为$Y$在$X$中的理想层。若$\mathscr A$是分次$\mathscr O_X$-代数层
	\[
	\mathscr A=\bigoplus_{n=0}^\infty \mathscr I^n=\mathscr O_X\oplus \mathscr I\oplus \mathscr I^2\oplus \cdots
	\]
	(其中第$k$个直和项取作$\mathscr A$的$k$-次分次部分),则概形$\proj(\mathscr A)\to X$是$X$沿着$Y$的爆破。
\end{thm}

\paragraph*{注记:}这个构造经常导致记号上的混淆:若$f\in \mathscr O_X(U)$是在$Y$上为零的正则函数,记号``$f$''既可以被用来标记$\mathscr A_0=\mathscr O_X$的截面或$\mathscr A_1=\mathscr I$的截面,它们是$\mathscr A$的两个不同的截面。为避免这点,我们将经常把$\mathscr A$实现为层
\[
	\mathscr O_X[t]=\bigoplus_{n=0}^\infty t^n\mathscr O_X,
\]
的一个子层,记
\[
	\mathscr A=\mathscr O_X\oplus t\mathscr I\oplus t^2\mathscr I^2\oplus \cdots.
\]
我们将在下面的证明中用这个记号。

\begin{proof}
	我们必须证明态射
\[
	\varphi:B=\proj(\mathscr A)\to X
\]
满足刻画爆破的两个性质:$Y$在$B$中的原像$\varphi^{-1}Y$是Cartier的,以及任意态射$f:Z\to X$使得$f^{-1}Y$是Cartier的都可以唯一地经由$B$分解。我们将记$\mathscr I$为$Y$在$X$中的理想层$\mathscr I_Y$.

为看到$Y$在$B$中的原像是Cartier的,回忆第 \ref{s:1.3.1} 节,$\varphi^{-1}Y$是$B$的由理想层$\mathscr I\mathscr O_B$定义的子概形。因为结构层$\mathscr O_B$是关联于分次$\mathscr A$-模层$\mathscr A$的层,我们看到$\mathscr I\mathscr O_B$是关联于分次$\mathscr A$-模
\[
	\begin{aligned}
	\mathscr I\mathscr A&=\mathscr I\cdot \mathscr O_B\oplus \mathscr I\cdot \mathscr I\oplus \mathscr I\cdot \mathscr I^2 \oplus \cdots\\
	&=\mathscr I\oplus \mathscr I^2 \oplus \mathscr I^3\oplus \cdots
	\end{aligned}
\]
的层,其中$d$-次分量为$\mathscr I\cdot \mathscr I^d=\mathscr I^{d+1}$. 这是分次模层
\[
	\mathscr A(1)=\mathscr O\oplus \mathscr I \oplus \mathscr I^2\oplus \cdots
\]
的截断(同样,$d$-次分量为$\mathscr I^{d+1}$)因此,从 Exercise \ref{exe:3.46},$\mathscr I\mathscr O_B=\mathscr O_B(1)$是可逆层。

% p.171

剩下还需证明,若$f:Z\to X$是一个映射使得$f^{-1}Y$是Cartier的,则$f$可以经由$B$唯一分解。简单起见,我们将假设$\mathscr I$是凝聚层。我们下面将$B$实现为$\mathbb P(\mathscr I)=\proj \operatorname{Sym}(\mathscr I)$,然后通过给出一个$Z$到$\mathbb P(\mathscr I)$的映射,其像包含于$B$中,来产生希望的$Z$到$B$的映射。

映射$\operatorname{Sym}_d(\mathscr I)\to \mathscr I^d$给出了一个满射$\operatorname{Sym}(\mathscr I)\to \mathscr A$. 它的核是$\mathscr A$的一个分次理想层,于是就像第 \ref{s:3.2.2} 节中那样,可将$B=\proj \mathscr A$与$\mathbb P(\mathscr I)$的一个闭子概形等同。

因为$f^{-1}Y$是Cartier的,其理想层$\mathscr I\cdot \mathscr O_Z$是可逆的。因此,自然的满射
\[
	f^*\mathscr I=\mathscr I\otimes_{\mathscr O_X}\mathscr O_Z\to \mathscr I\cdot \mathscr O_Z
\]
如同Therorem \ref{thm:3.44} 中那样对应一个映射$\alpha:Z\to \mathbb P(\mathscr I)$. 再者,从 Lemma \ref{lem:4.19},$f^{-1}Y$的补集在$Z$中稠密。因为$\varphi$是$\varphi^{-1}Y$上的一个同构,于是$\alpha(Z\setminus f^{-1}Y)$被包含于$B$中,进而因此$\alpha(Z)$整个就被包含在$B$中。因为$\alpha$因此就是希望的从$Z$到$B$的映射。

$f=\varphi\alpha$以及$\alpha$的唯一性都来自于$Z\setminus f^{-1}Y$在$Z$中稠密,以及Exercise \ref{exe:3.24} 的最后一句话。
\end{proof}

简单起见,我们假设了理想层$\mathscr I$是凝聚的(而不仅仅拟凝聚),拟凝聚的情况可以用 \ref{thm:3.44} 的一个直接推广来处理。

爆破给了我们另一种方式来将射影化切锥解释为一个概形,我们将在本节后面讨论。

\begin{exe}\label{exe:4.24}
	证明一个概形$X$在一个点$p\in X$的爆破$\Bl_p(X)$中的例外除子就是$X$在点$p$的射影化切锥$\mathbb PTC_p(X)$.
\end{exe}

\paragraph*{沿着正则子概形的爆破}\addcontentsline{toc}{subsubsection}{沿着正则子概形的爆破} 正如我们在Theorem \ref{thm:4.23} 前面提到过的,在实践中,爆破的构造可能不像它看起来那么明确。理由是,即使给定了概形$X$和子概形$Y$的显示的方程,用明确的生成元和关系来表达\textit{Rees代数}
\[
	\mathscr A=\bigoplus_{n=0}^\infty t^n\mathscr I_{Y,X}^n\subset \mathscr O_X[t]
\]
可能也并不显然。(生成元是清楚的,若我们局部地知道了理想层$\mathscr I_{Y,X}$的生成元;当我们已经知道所有关系时,它们可能是麻烦的,这是心照不宣的。)但是,当子概形$Y\subset X$是一个正则子概形,Rees代数有一个很好的描述。我们将首先陈述$Y$的余维数为$2$的情况。

% p.172

\begin{pro}\label{pro:4.25}
令$A$是一个Noether环,$x$, $y\in A$,令$B$是Rees代数
	\[
		B=A[xt,yt]\subset A[t].
	\]
若$x$, $y\in A$是一个正则列,则
	\[
		B\cong A[X,Y]/(yX-xY)
	\]
通过映射$X\mapsto xt$, $Y\mapsto yt$.
\end{pro}

\begin{proof}
首先我们除掉$x$,置$X'=x^{-1}X\in A[x^{-1}][X,Y]$. 元素$yX'-Y\in A[x^{-1}][X,Y]=A[x^{-1}][X',Y]$生成了映射
	\[
		\begin{aligned}
			A[x^{-1}][X',Y] & \longrightarrow A[x^{-1}][t],\\
			X' &\longmapsto t,\\
			Y  &\longmapsto yt
		\end{aligned}
	\]
的核。因为在环$A[x^{-1}][X,Y]$中$(yX-xY)=(yX'-Y)$,所以只需证明$x$在模掉$yX-xY$意义下在$A[X,Y]$中是一个非零因子。注意到,在其他阶,$yX-xY$在模掉$x$意义下显然是一个非零因子,其可以等同于$yX$,是两个非零因子的乘积!一般地,一个正则列的置换往往不是一个正则列,但在这里和其他的一些情况中,这是成立的,可见Eisenbud [1995, Section 17.1].

在其他的情况中,我们可以这样论证:为证明$x$在模掉$yX-xY$意义下是一个非零因子,我们必须证明
	\[
		M:=\frac{(yX-xY):(x)}{(yX-xY)}=0,
	\]
其中$(yX-xY):(x)$为理想$\{f\in A[X,Y]\,|\, fx\in (yX-xY)\}$. 注意到在模掉$x$意义下,$yX-xY\equiv yX$,于是$(x,yX-xY)$是一个$A[X,Y]$中的正则列。再进一步,$yX-xY$显然是一个非零因子(若多项式$f(X,Y)$乘它为零,则$f(X,Y)$的$X$的最高次项必然乘以$x$为零,但这由假设是一个非零因子)。于是,商$M$同构于Koszul复习的第一阶同调群
	\[
		0\xrightarrow{\qquad\quad\quad} A \xrightarrow{
		\begin{pmatrix}
			-x\\ yX-xY
		\end{pmatrix}}A^2
		\xrightarrow{\begin{pmatrix}
			yX-xY& x
		\end{pmatrix}}A.
	\]
同样地,这个群同构于
	\[
		\frac{(x):(yX-xY)}{(x)},
	\]	
他是零因为$x$, $yX-xY$是一个正则列。(一个最后一点更充分的处理可以参见Eisenbud [1995, Section 17.1])
\end{proof}

% p.173

上面的证明的核心是,若$I$由一个长度为$2$的正则列所生成,则Rees代数
\[
	A\oplus I \oplus I^2\oplus \cdots
\]
同构于对称代数
\[
	\operatorname{Sym}_A(I)
\]
而这又由$2\times 2$矩阵
\[
	\begin{pmatrix}
		x&y\\
		X&Y
	\end{pmatrix}
\]
的行列式所定义。类似的命题在对更大的正则列的时候也对:

\begin{exe}\label{exe:4.26}
	若$I=(x_1,\dots,x_n)\subset A$由一个正则列所生成,则
\[
	A\oplus I \oplus I^2\oplus \cdots\cong A[X_1,\dots,X_n]/J
\]
其中$J$有矩阵
\[
	\begin{pmatrix}
		x_1&\dots&x_n\\
		X_1&\dots&X_n
	\end{pmatrix}
\]
的$2\times 2$子式所生成。
\end{exe}

\subsection{一些经典爆破构造}\label{s:4.2.2}

\begin{exa}\label{exa:4.27}
	令$K$是一个域,考虑二次锥
	\[
		Q=\spec K[x,y,z]/(xy-z^2)\subset \spec K[x,y,z]=\mathbb A_K^3.
	\]
	令$p=(0,0,0)\in Q$为锥$Q$的顶点,$L$为$Q$上穿过$p$的直线,比方说$L=V(x,z)$. 我们想要描述$Q$同时沿着$p$和$L$的爆破。
\end{exa}

我们可以直接用 Theorem \ref{thm:4.23} 或者 Proposition \ref{pro:4.22} 得到它。但也许最简单的方式是用 Proposition \ref{pro:4.21}. 首先,我们可以用 Theorem \ref{thm:4.23} 或者 Proposition \ref{pro:4.22} 检查$\mathbb A_K^3$在原点$p$的爆破是态射
\[
	\varphi:\tilde{\mathbb A}_K^3=\proj K[x,y,z][A,B,C]/(xB-yA,xC-zA,yC-zB)\longrightarrow \spec K[x,y,z]=\mathbb A_K^3.
\]
例外除子$E=\varphi^{-1}(p)\subset \tilde{\mathbb A}_K^3$确实是Cartier的:比方说,我们可以记开集$U_A=\tilde{\mathbb A}_K^3\setminus V(A)$为
\[
	U_A=\spec K[x,y,z][b,c]/(xb-y,xc-z)=\spec K[x,b,c],
\]
则在$U_A$中,例外除子$E$是函数$x$(的拉回)的零点集。就像平面关于原点的爆破的情况中一样,穿过$p$的直线$L\subset \mathbb A_K^3$的真变换$\tilde L$在$\tilde{\mathbb A}_K^3$中都不相交,所以实际上例外除子$E$就是一个$\mathbb P_K^2$,其$K$-有理点一一对应于那些直线,通过$L\mapsto \tilde L\cap E$.

现在,当拉回$Q$的定义方程$xy-z^2$到$\mathbb A_K^3$,我们发现它可以因式分解:它可以被$E$的定义方程除两次。比方说,在$U_A$中,
\[
	\varphi^\# (xy-z^2)=x^2b-x^2c^2=x^2(b-c^2)
\]
我们额可以将这点整体地表为
\[
	\varphi^{-1}(Q)=V((x,y,z)^2)\cup V(AB-C^2),
\]
则从 Proposition \ref{pro:4.21} 我们额可以得到$Q$在点$p$的爆破$\Bl_p Q$是$\varphi$在零点集$V(AB-C^2)\subset \tilde{\mathbb A}_K^3$上的限制,即
\[
	\varphi:\tilde Q=\proj K[x,y,z][A,B,C]/(xB-yA,xC-zA,yC-zB,AB-C^2)\longrightarrow \spec K[x,y,z]/(xy-z^2)=Q.
\]

我们可以用$Q$中穿过$p$的直线的(真变换的)不交并来给出$\tilde Q$的图像:

\inclugra{2.png}

现在,那$Q$沿着$L$的爆破$\Bl_L Q\to Q$是什么呢?首先,注意到$L$除了在点$p$,处处都是$Q$的Cartier子概形($p$是$Q$的一个奇异点,但不是$L$的)。于是,爆破$\Bl_L Q\to Q$在$Q\setminus \{p\}$将是个同构,但本身不是同构。再者,因为$L$在关于点$p$的爆破$\tilde Q=\Bl_pQ\to B$中的原像$\psi^{-1}(L)\subset \tilde Q$是Cartier的,映射$\psi:\tilde Q\to Q$必须经由爆破$\Bl_L Q\to Q$分解。
到目前为止,读者还不会惊讶地发现,事实上,这两个爆破是相同的!我们将验证留作以下习题。

\begin{exe}\label{exe:4.28}
	证明,$\bba^3_K$沿着$L$的爆破$\Bl_L\bba_K^3$可以实现为映射
	\[
		\varphi:\Bl_L \mathbb A_K^3=\proj K[x,y,z][A,B]/(xB-zA)\longrightarrow \spec K[x,y,z]=\mathbb A_K^3
	\]
	(我们可以将其图像为,$\bba_K^3$中包含$L$的平面的不交并。)使用这个来描述爆破$\Bl_L Q\to Q$,然后证明作为$Q$-概形,其与$\Bl_p Q\to Q$同构。
\end{exe}

另一个令人惊喜的内涵丰富的例子是三维二次锥的爆破。

\begin{exa}\label{exa:4.29}
	现在考虑二次超平面
	\[
	X=V(xw-yz)\subset \spec K[x,y,z,w]=\mathbb A_K^4.
	\]
	$X$是一个非奇异二次曲面$Q=V(xw-yz)\subset \proj K[x,y,z,w]=\bbp_K^3$上的锥。我们想要考虑$X$沿着三个子簇的爆破:点$p=(0,0,0,0)$;平面$\Lambda_1=V(x=y=0)\subset X$;平面$\Lambda_2=V(x=z=0)\subset X$. 有趣的是,尽管三个爆破都是$X\setminus \{p\}$上的同构,但他们是三个不同的$X$-概形,此外爆破$\Bl_{\Lambda_1}X$和$\Bl_{\Lambda_1}X$虽然是同构的概形,但是作为$X$-概形并不同构。
\end{exa}

首先,令$\varphi:\tilde X\to X$为$X$在点$p$的爆破。这可以沿着与前一例子中的二次曲面在一点处的爆破大致相同的路线来描述:所有$X$上穿过点$p$的直线都变得不交了;$\tilde X$是非奇异的;例外除子是一个非奇异二次曲面,可以自然地等同于$Q\subset \bbp_K^3$.

$X$沿着平面$\Lambda_i$的爆破$X_i$由下面的习题描述:

\begin{exe}\label{exe:4.30}
	令$\varphi:X_1=\Bl_{\Lambda_1}X\to X$为$X$沿着平面$\Lambda_1$的爆破。证明下面的断言。
	\begin{compactenum}[(a)]
		\item 概形$X_1$非奇异。
		\item 映射$\varphi_1$在$X\setminus \{p\}$上是一个同构。
		\item $X_1$在点$p$的纤维$C=\varphi_1^{-1}(p)$同构于$\bba_K^1$.
		\item 例外除子$E=\varphi_1^{-1}(\Lambda_1)$,其同样也是$\Lambda_1$在$X_1$中的真变换,同构于$\Lambda_1\cong \bba_K^2$在点$p$的爆破。
		\item 更一般地,平面
			\[
				\Lambda_{1,\mu}=V(x-\mu z,y-\mu w)
			\]
			的真变换$\tilde \Lambda_{1,\mu}$,由$X$的顶点$p$和与$Q$的一条直纹相交的直线们张成,与他们的全变换一致;他们同构于$\Lambda_{1,\mu}$在点$p$的爆破,并沿着曲线$C$两两相交。\nottran
		\item 相反地,平面
			\[
				\Lambda_{2,\mu}=V(x-\mu y,z-\mu w)
			\]
			的原像$\varphi^{-1}(\Lambda_{2,\mu})$,由$X$的顶点$p$和与$Q$的另一条直纹相交的直线们张成,它有两个不可约分支:真变换$\tilde \Lambda_{2,\mu}$和曲线$C$.(特别地,它们都不是$X$的Cartier子概形。)真变换$\tilde \Lambda_{2,\mu}$同构地映到平面$\Lambda_{2,\mu}$,它与$X_1$不交;因此我们可以尝试通过使得平面们$\Lambda_{2,\mu}$不交来图像化$X_1$. \nottran

			\inclugra{3.png}
	\end{compactenum}
\end{exe}

因为平面$\Lambda_1$和$\Lambda_2$的原像都是$\tilde X$的Cartier子概形(它们都在非奇异概形$\tilde X$中纯余维数$1$),爆破$\tilde X=\Bl_pX\to X$经由每个爆破$X_i=\Bl_{\Lambda_i}X\to X$分解。实际上:

\begin{exe}\label{exe:4.31} 
	\begin{compactenum}[(a)]
		\item 证明作为$X$概形,$\tilde X=X_1\times_X X_2$.
		\item 证明诱导的映射$\psi_i:\tilde X\to X_i$就是$X_i$沿着曲线$C$的爆破。
	\end{compactenum}
\end{exe}

概形$X_i$们作为$X$-概形当然并不同构,因为$\Lambda_2$在$X_1$的原像并不是Cartier的,反之亦然,虽然它们作为$K$-概形同构($X$有一个自同构交换平面$\Lambda_1$和$\Lambda_2$)。类似地,它们并不与$\tilde X$作为$X$-概形同构,因为$\Lambda_1$和$\Lambda_2$的原像都在$\tilde X$中是Cartier的。

\begin{exe}\label{exe:4.32} 
	证明实际上$X_1$和$X_2$作为$K$-概形也不同构于$\tilde X$.(提示:一种法子是去证明$X_i$并不包含在$K$上逆紧的二维子概形。)
\end{exe}

\begin{exe}\label{exe:4.33} 
	这里有一个有趣的方式来实现上面提到的所有三种爆破。将$\mathbb A_K^4$等同于$2\times 2$矩阵的矢量空间的对应的仿射空间,这个矢量空间也是由在两个二维$K$-矢量空间间的线性映射$A:V\to W$们所构成的:
	\[
	M=\Hom(V,W)=\left\{\begin{pmatrix}x&y\\z&w\end{pmatrix}\right\}.
	\]
	令$\bbp V^*$为$V^*$的一维商空间的射影空间,即$V$的一维子空间们,类似地,令$\bbp W^*$为$W$的一维子空间构成的射影空间。证明$X$以及爆破$X_1$, $X_2$和$\tilde X$分别是下面的簇对应的概形
	\[
	\begin{aligned}
		X&=\{A:V\to W\,|\,\operatorname{rank}A\leq 1\}\subset \mathbb A_K^4,\\
		X_1&=\{(A,L)\,|\,L\subset \operatorname{Ker}A\}\subset \mathbb A_K^4\times \mathbb PV^*,\\
		X_2&=\{(A,L')\,|\,\operatorname{Im}A\subset L'\}\subset \mathbb A_K^4\times \mathbb PW^*,\\
		\tilde X&=\{(A,L,M)\,|\, L\subset \operatorname{Ker}A\text{ 且 }\operatorname{Im}A\subset M\}\subset \mathbb A_K^4\times \mathbb PV^*\times \mathbb PW^*.
	\end{aligned}
	\]
\end{exe}

实际上,Example \ref{exa:4.27} 以及 Example \ref{exa:4.29} 的结果不仅仅可以用到二次锥上,也可以用到局部长得像它们的概形上。这是后续习题的内容,而这需要一个进一步的定义:

\begin{defi}\label{defi:4.34}
	令$K$为一个特征不为$2$的代数闭域,而$X$是一个$K$上的概形。我们称点$p\in X$是一个\textit{寻常二重点},若局部环$\mathscr O_{X,p}$的完备化是
	\[
		\hat{\mathscr O}_{X,p}\cong K[\![x_1,\dots,x_n]\!]/(x_1^2+x_2^2+\cdots+x_n^2).
	\]
\end{defi}

% p.178

比方说,曲线的寻常二重点常被我们称为结点。更一般地,一个$n$-维概形$X$的寻常二重点$p$可以被刻画为,$X$在点$p$的射影化切锥是一个在$\bbp T_pX\cong \bbp_K^n$中的非奇异二次超曲面。

\begin{exe}\label{exe:4.35}
	假若$X$为二维的而$p\in X$是一个寻常二重点。令$\tilde X=\Bl_p X\to X$为$X$在点$p$的爆破。证明Example \ref{exa:4.29} 的结论同样可以用在$X$上:$X$是非奇异的;例外除子$E\subset \tilde X$是$\bbp T_pX\cong \bbp_K^2$中的一个二次曲线,以及若$C\subset X$是在点$p$非奇异的任意曲线,则作为$X$-概形$\Bl_C X\cong \tilde X$.
\end{exe}

\begin{exe}\label{exe:4.36}
	遵循Exercise \ref{exe:4.35},现在假设$X$的维数为$3$而$x\in X$是一个普通的二次点。令$\tilde X=\Bl_p X\to X$为$X$在$p$处的爆破。证明,Example \ref{exa:4.29}的结论对$X$同样使用:$\tilde X$非奇异,例外除子$E\subset \tilde X$是一个非奇异二次曲面$Q\subset \mathbb PT_pX\cong \mathbb P_K^3$,以及,如果$S\subset X$是任意在点$p$处非奇异的曲面,则爆破$\Bl_SX$在点$p$处的纤维同构于$\mathbb P_K^1$(特别地,其不同构于$\tilde X$)。此外,证明,如果$S$和$S'\subset X$是两个这样的曲面,则爆破$\Bl_SX$与$\Bl_{S'}X$作为$X$-概形同构当且仅当射影化切平面$\mathbb PT_pS$和$\mathbb PT_pS'\subset \mathbb PT_pX$同属于二次曲面$Q$相同的直纹中。
\end{exe}

用人话来讲,对有一个普通的二次点$p$的三维概形$X$,通过(在$p$局部)沿着一个在$p$处非奇异的曲面的爆破得到的概形$X'\to P$被称为$X$在点$p$处的\textit{小消解}。一般地,一个奇点消解$\pi:X'\to X$,即一个双有理态射是的$X'$非奇异,被称为\textit{小的},如果对任意的子簇$\Gamma\subset X$,原像$\pi^{-1}(\Gamma)$的维数至多
\[
	\dim (\pi^{-1}(\Gamma))\leq \frac{\dim(\Gamma)+\dim(X)-1}2 .
\]

有一个普通的二次点$p$的三维概形$X$的两个小消解的双有理同构被称为一个\textit{翻转},见Clemens等 [1988].

\begin{exe}\label{exe:4.37}
	令$K$是一个域,$\mathbb A_K^3=\spec K[x,y,z]$. 令$L$和$M\subset \mathbb A_K^3$为直线$V(x,y)$和$V(z,x)$,$N=L\cup M=V(x,yz)$为他们的并。描述爆破$x=\Bl_N \mathbb A_K^3\to \mathbb A_K^3$,特别地,证明$X$在$N$上的每个点的纤维都同构于$\mathbb P_K^1$,但并却是奇异的:它在$\mathbb A_K^3$的原点处有着通常的二次点$p$.
\end{exe}

% p.179

\begin{exe}\label{exe:4.38}
	记号同上题,令$Y\to \mathbb A_K^3$为$\mathbb A_K^3$沿着直线$L$的爆破,$\tilde M\subset Y$为$M$在$Y$中的真变换,$X'\to Y$为$Y$沿着$\tilde M$的爆破。证明,复合$X'\to Y\to \mathbb A_K^3$经由爆破$X=\Bl_N \mathbb A_K^3\to \mathbb A_K^3$分解,诱导的映射$X'\to X$是一个寻常二重点$p\in X$的小消解。
\end{exe}

\begin{exe}\label{exe:4.39}
	现在令$Z\to \mathbb A_K^3$为$\mathbb A_K^3$沿着直线$M$的爆破,$\tilde{L}\subset Y$是$L$在$Z$中的真变换,$X''\to Z$是$Z$沿着$\tilde L$的爆破。证明,复合$X''\to Y\to X$也经由爆破$X\to \mathbb A_K^3$分解,且诱导的映射$X'\to X$是来自于$X'\to X$的寻常二重点$p\in X$\textit{反}小消解。为直接看到$X'\to X$和$X''\to X$不是同构的$X$-概形,令$N'$和$N''$是$L\setminus \{0\}$在$X'$和$X''$中的原像,然后直接比较$N'$和$N''$在$0\in \mathbb A_K^3$上的纤维。
\end{exe}

\subsection{沿着非既约概形的爆破}\label{s:4.2.3}

至今,我们还仅处理了一些爆破$\Bl_Y X\to X$的例子,例子中涉及的三个对象,概形$X$、子概形$Y$和爆破$\Bl_YX$都是簇。在这节的剩下两个部分中,我们将考虑更一般概形的爆破,首先给出沿着非既约概形的爆破的例子,然后是算术概形的爆破。我们这里将从给出一些簇沿着非既约子概形的爆破的例子开始。

\paragraph*{一个二次点的爆破}\addcontentsline{toc}{subsubsection}{一个二次点的爆破} 
令$X=\bba_K^2=\spec K[x,y]$,以及令$\Gamma\subset \bba_K^2$为由理想$I=(x^2,y)$给出的子概形。爆破$Z=\Bl_\Gamma(\bba_K^2)$将是$\proj A$,其中$A$为环
\[
	A=K[x,y]\oplus I\oplus I^2\oplus \cdots
\]
由Proposition \ref{pro:4.25},我们可以同样将$Z$写作
\[
	Z=\proj K[x,y][A,B]/(yA-x^2B)
\]
其被开集
\[
	U_A=\spec K[x,y][b]/(y-x^2b)
\]
和
\[
	U_B=\spec K[x,y][a]/(ya-x^2)
\]
所覆盖,其中$a=A/B$而$b=B/A$.

我们立刻看到这个概形与寻常的$\mathbb A_K^2$在原点的爆破之间的不同。其中,尽管两者在原点处的纤维都同构于$\mathbb A_K^1$,概形$Z=\Bl_\Gamma(\bba_K^2)$在其中一点$P$($U_B$中的点$a=x=y=0$)处奇异,而寻常的爆破是非奇异的。

% p.180 

若我们以具有约化中心的爆破来表$Z$,我们可以看到更多。粗略地说,在经典语言中,$Z$的“recipe”\nottran 为(见下图):首先,令$Z_1$为$\mathbb A_K^2$在原点处的爆破,令$E\subset Z_1$为例外除子,即原点的原像。令$P$为$E$在$x$-轴的真变换中的点,即$x$-轴在$Z_1\setminus E$中原像的闭包。令$Z_2$为$Z_1$在点$P$的爆破,令$F\subset Z_2$为这个爆破的例外除子,以及(稍微混用一下记号)$E\subset Z_2$为$E$在$Z_2$中的真变换。于是,在经典语言中,$Z=\operatorname{Bl}_\Gamma (\mathbb A_K^2)$是由$Z_2$吹倒$E$ 得到的,换句话说:

\begin{pro}\label{pro:4.40}
	$Z=\Bl_\Gamma(\mathbb A_K^2)$在其奇异点$P$的爆破$Z'$是$Z_2$.
\end{pro}

\inclugra{4.png}

% p.181

我们从这个描述看到,在平面上穿过原点的直线并没有变得不交,就像他们在
$\mathbb A_K^2$关于既约原点的爆破的情况中那样:它们在第一个爆破里分开了,
但当我们吹倒$E\subset Z_2$后,它们又接着交在了一起。另一方面,
与$x$-轴相切着穿过原点的具有不同曲率的非奇异曲线被分开了:在序列中的
第一个爆破后,它们横截相交于点$P$,然后它们又被第二个爆破分开了,且不会
被吹到影响。

\begin{proof}[Proposition \ref{pro:4.40} 的证明]
由Exercise \ref{exe:4.35},$Z$在其奇异点处的爆破与$Z$在
对应于$Z=\Bl_\Gamma (\mathbb A_K^2)\to \mathbb A_K^2$的例外除子的
既约概形$F$处的爆破相同。这个概形$F$是$Z$中$\mathbb A_K^2$中(既约)原点
的真变换,就像我们直接从方程看到的那样。

另一方面,我们断言,$Z_2$可以从下面的操作得到,首先爆破$\mathbb A_K^2$
的既约原点来得到$Z_1$,然后爆破$\Gamma$在$Z_1$中的真变换 ------
前面过程的逆。\nottran 为看到这点,首先观察到$\Gamma$在$Z_1$中的理想的方程是
$E\subset Z_1$的理想和点$P$的理想的乘积;因为$E$是Cartier的,于是
$\Bl_{\Gamma'}Z_1=\Bl_P Z_1$.

有了这些评注,现在我们只要应用下面的这个引理:

\begin{lem}\label{lem:4.41}
	令$X$为一个概形,$Y_1$和$Y_2\subset X$为其闭子概形。若$f_i:Z_i=\Bl_{Y_i}X\to X$为$X$沿着$Y_1$和$Y_2$的爆破,则作为$X$-概形,
	\[
		\Bl_{f_1^{-1}(Y_2)}Z_1 \cong \Bl_{f_2^{-1}(Y_1)}Z_2.
	\]
\end{lem}

\begin{proof}
令$W_1=\Bl_{f_1^{-1}(Y_2)}Z_1$,再令$g_1:W_1\to Z_1$为爆破映射,类似
定义$W_2$和$g_2$. 置$h_i=f_i\circ g_i:W_i\to X$. 因为
$h_1^{-1}(Y_2)=g_1^{-1}(f_1^{-1}(Y_2))\subset W_1$是Cartier的,结构
映射$h_1:W_1\to X$经由$Z_2$分解,即,存在一个映射$g_1:W_1\to Z_2$使得
$h_1=f_2\circ j_1$. 类似地,因为$j_1^{-1}(f_2^{-1}(Y_1))=
h_1^{-1}(Y_1)=g_1^{-1}(f_1^{-1}(Y_1))\subset W_1$同样也是Cartier的,
映射$j_1:W_1\to Z_2$经由$W_2=\Bl_{f_2^{-1}(Y_1)}Z_2$分解,诱导了一个
映射$k_1:W_1\to W_2$使得$h_1=h_2\circ k_1$. 另一方向,我们可以类似
获得一个映射$k_2:W_2\to W_1$. 因为$W_1$作为一个$X$-概形没有自同态,
$k_2\circ k_1$为恒同,特别地,$k_1$为同构。
\renewcommand{\qedsymbol}{}
\end{proof}

\renewcommand{\qedsymbol}{$\square\hspace{-1.9ex}\square$}
\end{proof}\renewcommand{\qedsymbol}{$\square$}

比较这个引理与Exercise \ref{exe:4.37} 到 \ref{exe:4.39},其中我们看到,
若我们用“全变换”替换“真变换”,顺序确实是重要的。

\paragraph*{多次点的爆破}\addcontentsline{toc}{subsubsection}{多次点的爆破}
我们这里将考虑一些平面沿着支于一点的子概形的爆破的例子。
% p.182

\begin{exe}\label{exe:4.42}
	另一个例子,令$\Omega_1\subset \A_K^2$为由理想$(y,x^3)\subset K[x,y]$
	定义的子概形,而$\Omega_2$为理想$(y^2,x^3)\subset K[x,y]$定义的子概形。
	考虑平面在这两个子概形处的爆破$\varphi_i:Z_i=\Bl_{\Omega_i}(\A_K^2)
	\to \A_K^2$. 证明,在两种情况中,概形$Z_i$都是奇异的,原点
	$P=(x,y)\in \A_K^2$处的纤维$\varphi_i^{-1}(P)$同构于$\P_K^1$. 同样
	证明,在两种情况中,爆破映射都可以经由三个爆破的序列再接着两个收缩
	分解,即,存在一个概形$W_i$,其由$\A_K^2$连续爆破三个既约点得到,以及
	一个映射$W_i\to Z_i$,其在前两个爆破的例外除子上常值,且在它们的补
	上是同构。两个例子中被爆破的点的序列的区别是什么?
\end{exe}

为不至于给出一个错误的印象,我们需要指出,爆破的纤维,即使是非奇异簇,
也不必是射影空间。(当然,若已断言任意逆紧双有理态射都是一个爆破,
则可以看到很难如此。)下面习题的主题即一个其他行为的简单例子。

\begin{exe}\label{exe:4.43}
令$\A_K^2=\spec K[x,y]$为在代数闭域$K$上的仿射平面,$\Gamma\subset \A_K^2$
为由
\[
	\Gamma=V(x^3,xy,y^2)
\]
给出的子概形。令$X$为爆破$X=\Bl_\Gamma(\A_K^2)$. 证明$X$由
\[
	X=\proj K[x,y][A,B,C]/I
\]
给出,其中理想$I$为
\[
	I=(yA-x^2B,yB-xC,AC-xB^2).
\]
\emph{提示}:沿着正则子概形$V(z-xy,x^3,y^2)$爆破$\A_K^3=\spec [x,y,z]$,
然后考虑平面$V(z)$的真变换。
\end{exe}

\begin{exe}\label{exe:4.44}
	证明,上面习题中的概形$X$是非奇异的,点$(x,y)\in\A_K^2$处的纤维
	是两个交于一点的$\P_K^1$的并。(实际上,$X$为Proposition 
	\ref{pro:4.40} 中的概形$Z_2$.)
\end{exe}

我们并没有理想与爆破之间的一一对应,不同的理想可能产生相同的爆破。
这当然存在很多平凡的例子,比如任意的主理想都产生平凡的爆破。稍不平凡点,
令$X$为任意Noether概形,$\Gamma\subset X$为任意的闭子概形,而$\mathscr I
\subset \mathscr O_X$为其理想层。令$\Gamma_n$为由理想$\mathscr I^n$
所定义的$X$的子概形。则从定义,万有性质告诉我们爆破$Z_n=\Bl_{\Gamma_n}(X)$
都是同构的。下面是一些更有意思的例子:

% p.183

\begin{exe}\label{exe:4.45}
令$\A_K^2=\spec K[x,y]$为代数闭域$K$上的仿射平面。考虑由下面理想给出的
子概形$\Gamma_n\subset \A_K^2$:
	\[
	I_n=(x^{n+1},x^{n-1}y,x^{n-2}y^2,\dots,xy^{n-1},y^n)=
	(x,y)^n\cap (x^{n+1},y).
\]
(换句话说,$I_n$为原点是$n$-次零点,且在$x$-轴方向有更高一次零点的
多项式构成的理想。)注意到,$\Gamma_1$就是Proposition \ref{pro:4.40}
中的概形$\Gamma$.

证明对$n\geq 2$,概形$X_n$同构于另一个概形,通过同构$\varphi_n:X\to X_n$,
其中$X=\Bl_{\Gamma}(\A_K^2)$为在Proposition \ref{pro:4.40} 中所描述的
爆破。
\end{exe}

\paragraph*{$j$-函数}\addcontentsline{toc}{subsubsection}{$j$-函数}
下面是一个类似于我们已描述过的例子的、但会很自然出现的例子。它涉及了
平面三次曲线的$j$-函数,这是一个直到这本书非常后面前并不会正式提及的
主题,但读者可能对其非常熟悉。不管怎么样,在下文中,我们将假设对
$j$-函数有一定的知晓。

我们考虑由下面的方程给出的平面三次曲线的(平坦)族$\mathscr E\to 
\A_K^2=\spec K[a,b]$,
\[
	y^2=x^3+ax+b.
\]
现在,当曲线$C_{a,b}$,其在$\A_K^2=\spec K[x,y]$中由方程$y^2=x^3+ax+b$
给出,是非奇异的,我们对其给出一个标量
\[
	j(C_{a,b})=1728\frac{4a^3}{4a^3+27b^2}.
\]
正如读者可能知道,两个这样的曲线$C_{a,b}$和$C_{a',b'}$是同构的当且仅当
它们的$j$-函数相同。

\nottran 

\[
	\begin{aligned}
		j:\mathbb A_K^2\setminus \{Q\}&\longrightarrow \mathbb P_K^1 \\
		(a,b)&\longmapsto j(C_{a,b})
	\end{aligned}
\]

\[
	\varphi_2:Z_2=\Bl_{\Omega_2}(\mathbb A_K^2)\to \mathbb A_K^2
\]

\[
	y^2=x^3+\alpha tx+\beta t.
\]

\subsection{算术概形的爆破}\label{s:4.2.4}

因为我们已经如此一般地定义了爆破,我们可以用这个构造去联系大量算术概形,如下面的例子与习题所示。

我们从爆破$\mathbb P_{\mathbb Z}^1$中的一个既约点开始:我们令$P$为既约点$P=(3,X)\in P_{\mathbb Z}^1$,然后考虑$\mathbb P_{\mathbb Z}^1$在点$P$的爆破$Z=\Bl_P(\mathbb P_{\mathbb Z}^1)$. 这是直接的,像之前一样,唯一的问题就是记号。因为概形$\mathbb P_{\mathbb Z}^1=\proj \zz[X,Y]$并不仿射,我们用仿射开集$U_X=\spec \zz[y]\cong \mathbb A_{\mathbb Z}^1$和$U_Y=\spec \zz[x]\cong \mathbb A_{\mathbb Z}^1$覆盖之,其中$y=Y/X$, $x=X/Y$. 因为需爆破的点在$U_X$的补中,$U_X$在$Z$中的原像即$U_X$.

接着,我们描述$U_Y$的爆破。明晰起见,我们记$A$和$B$为$P\in U_Y=\spec \zz[x]$的理想$I=(3,x)$的两个生成元$3$和$x$,于是我们可将环$\zz[x]\oplus I\oplus I^2\oplus \dots$表为
\[
	\bigoplus_{n=0}^\infty I^n=(\zz[x])[A,B]/(xA-3B).
\]%
% p. 185
我们可描述这个环的$\proj$为两个开集$W_A$和$W_B$的并。前者更简单些:置$b=B/A$,我们有
\[
	W_A=\spec \zz[x][b]/(x-3b)=\spec \zz[b]=\mathbb A_\zz^1,
\]
因此开集$W_A\cong \mathbb A_{\mathbb Z}^1$,但映射$W_A\to U_Y\cong \mathbb A_\zz^1\subset \mathbb P_\zz^1$并不是同构,而是对偶于将$x$变为$3b$的环同态的映射$\spec \zz[b]\to \spec \zz[x]$.

对另一个开集,我们有
\[
	W_B=\spec \zz[x][a]/(ax-3),
\]

\nottran

\[
	Z'=U_X\cup W_A=\spec \zz[y]\bigcup_{\spec \zz[y,\frac 1y,\frac 13]=\spec \zz[b,\frac 1b,\frac 13]}\spec \zz[b]
\]

\[
	C=\proj \zz[S,T,U]/(ST-3U^2)\subset \mathbb P_\zz^2=\proj \zz[S,T,U].
\]

\inclugra{5.png}

\[
	U_X=\spec \zz[y]\longrightarrow U_T=\spec \zz \left[\frac ST,\frac UT\right]\bigg / \left(\frac ST-3\left(\frac UT\right)^2\right)=\spec \zz \left[\frac UT\right]
\]

\[
	W_A=\spec \zz[b]\longrightarrow U_S=\spec \zz \left[\frac TS,\frac US\right]\bigg / \left(\frac TS-3\left(\frac US\right)^2\right)=\spec \zz \left[\frac US\right]
\]

\[
	W_B=\spec \zz[a,x]/(ax-3)\longrightarrow U_U=\spec \zz[\frac SU,\frac TU]\bigg / \left(\frac SU \frac TU-3\right)
\]

\[
	\Gamma = V(9,X)\subset \mathbb P_\zz^1=\proj \zz[X,Y]
\]

\[
	\Omega=V(3,X^2)\subset \mathbb P_\zz^1=\proj \zz[X,Y].
\]

\[
	Z=\{[a,b,c,d]\,:\,\operatorname{rank}\begin{pmatrix}
		a&c&3d\\ b&d&a
	\end{pmatrix}\leq 1\}\subset \mathbb P_\zz^3;
\]

\[
	A\longrightarrow B\longrightarrow C.
\]

\[
	d\pi_P:T_P(A)\longrightarrow T_{(11,11\sqrt 3)}(B)
\]

\[
	d\pi_Q:T_Q(A)\longrightarrow T_{(11,11\sqrt 3)}(B)
\]

\[
	d\eta_P:T_P(A)\longrightarrow T_{(11,121\sqrt 3)}(B)
\]

% p.189

\[
	d\eta_Q:T_Q(A)\longrightarrow T_{(11,121\sqrt 3)}(B)
\]

\[
	\begin{aligned}
		&C_1=\proj \zz[X,Y,Z]/(XY-3Z^2),&&C_2=\proj \zz[X,Y,Z]/(XY-9Z^2)\\
		&C_3=\proj \zz[X,Y,Z]/(XY-27Z^2),&&C_4=\proj \zz[X,Y,Z]/(XY-81Z^2)
	\end{aligned}
\]

\[
	S=\spec \zz[\frac 13]=\spec \zz\setminus\{(3)\}\subset \spec \zz
\]

% p.190

\[
	C_n=\proj \zz[X,Y,Z]/(XY-3^nZ^2).
\]

\subsection{企划:作为爆破的三次和四次曲面}\label{s:4.2.5}

\[
	Q_1=V(X^2+Y^2-Z^2-W^2)\quad \text{和}\quad Q_2=V(X^2+Y^2+Z^2-W^2)
\]

% p.191

\[
	\bbp_K^2\setminus \Gamma\to \bbp_K^3,
\]