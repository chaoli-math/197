\section{爆破}\label{s:4.2}

爆破是经典代数几何的一个基本工具。它用以消解奇点、消解有理映射的不定性,以及相互联系双有理簇。
称一个簇是另一个簇沿着一个给定子簇的爆破,表出了此二者间的一种关系,一方面其同时足够接近去密切联系两个簇的结构,另一方面其足够灵活,在簇之间的映射的表达式中非常常见。
在这节中,我们将推广爆破的定义到概形范畴,定义任意的(Noether)概形沿着任意闭子概形的爆破。

以这种方式概括爆破的定义实际上有两个目的。首先是期望的好处:概形和簇的爆破的作用相同,即消解奇点或者联系两个双有理概形(比如说,我们将在第 \ref{s:4.2.4} 节中爆破算术概形)。

此外,我们还可以看到,甚至在簇之间的映射的语境中,概形的语言意味着相关概念非常有用之推广,特别地,可以去讨论一个概形$X$沿着一个可能非约态的子概形$Y\subset X$的爆破。比方说,我们将在下面的第 \ref{s:4.2.3} 节展示这点,在那里用这个推广的爆破的定义,我们推广了对非奇异二次曲面作为平面沿着一个二次曲线的爆破的传统的描述。类似地,在第 \ref{4.2.3} 节,我们将看到一个自然出现的簇之间的映射是沿着一个子概形的爆破。这些例子实际上并不特殊:当我们以这种方式推广“爆破”的定义,则任意簇之间的射影双有理态射都是爆破!这在Hartshorne [1977, Theorem II.7.17]中有证明。

\subsection{定义与构造}\label{s:4.2.1}

% p.163

下面,我们将假设读者熟悉经典语境中的经典的爆破,即簇沿着一个非奇异子簇的爆破。(这个材料在许多文本中被详细地讨论了,比如Harria [1995], Hartshorn [1977, Chapter 1]和Shafarevich [1974].)在最简单的情况中,比如对一个代数闭域$K$上的仿射平面中的一个约态点爆破,爆破映射将与经典的情形一模一样。我们从复习一个例子开始,对平面上原点的爆破,来看如何将通过粘起来定义的经典的爆破同样带到域上的概形范畴。推广这个到任意概形$X$沿着任意闭子概形$Y\subset X$的爆破只是以足够自然的方式表达这种标准构造的问题。在下面的小节中,我们将给出几个一般的爆破的刻画:一个定义、两个构造以及一个在一些特殊例子中的描述,比如一个概形沿着一个正则子概形的爆破(Definition \ref{defi:4.15})。

\paragraph*{例子:平面的爆破}\addcontentsline{toc}{subsubsection}{例子:平面的爆破}

\begin{exa}\label{exa:4.14}
	我们从域$K$上的仿射平面$\mathbb A_K^2=\spec K[x,y]$的原点的爆破$Z$开始。它可以由最多两个开集的并所描述,其中每个都同构于$\mathbb A_K^2$:我们令$U'=\spec K[x',y']$和$U''=\spec K[x'',y'']$,然后考虑与映射$\varphi':U'\to \mathbb A_K^2$和$\varphi'':U''\to \mathbb A_K^2$对偶的环同态
	\[
	\begin{aligned}
		(\varphi')^\#:K[x,y]&\longrightarrow K[x',y']&\quad \text{and}\qquad (\varphi'')^\#:K[x,y]&\longrightarrow K[x'',y'']\\
		x&\longmapsto x'& x&\longmapsto x''y''\\
		x&\longmapsto x'y'& y&\longmapsto y''.
	\end{aligned}
	\]
	映射$\varphi'$给出了开集
	\[
	U'_x=\spec K[x',y',{x'}^{-1}]\quad \text{and}\quad 
	U_x=\spec K[x,y,x^{-1}]
	\]
	之间的同构,类似地,$\varphi''$给出了开子集$U''_y=\spec K[x',y',1/y']$和$U_y=\spec K[x,y,y^{-1}]$之间的同构。特别地,它们给出了
	交集$U_{xy}=U_x\cap U_y=\spec K[x,y,x^{-1},y^{-1}]$的原像
	\[
	U'_{xy}=\spec K[x',y',{x'}^{-1},{y'}^{-1}]\quad \text{and}\quad 
	U''_{xy}=\spec K[x'',y'',{x''}^{-1},{y''}^{-1}]
	\]
	之间的同构。我们于是能等同开集$U'_{xy}\subset U'$和$U''_{xy}\subset U''$,然后将$U'$和$U''$粘起来得到一个概形
	\[
		Z=U'\cup U''=\spec K[x',y']\bigcup_{U'_{xy}\cong U''_{xy}}\spec K[x'',y''],
	\]
	其中同构$U'_{xy}\cong U''_{xy}$由环同态
	\[
	\begin{aligned}
		K[x',y',{x'}^{-1},{y'}^{-1}]&\longrightarrow 
		K[x'',y'',{x''}^{-1},{y''}^{-1}]\\
		x'&\longmapsto x''y''\\
		y'&\longmapsto {x''}^{-1}
	\end{aligned}
	\]
	给出。我们称并$Z$,连同结构态射$\varphi:Z\to \mathbb A_K^2$,为$\mathbb A_K^2$在原点的\textit{爆破}。原点的原像$E=\varphi^{-1}(0,0)\subset Z$同构于$\mathbb P_K^1$(这被称为这个爆破的\textit{例外除子}),而$\varphi$除此之外是一个同构,即$Z\setminus E\cong \mathbb A_K^2\setminus \{(0,0)\}$.
\end{exa}

% p.164

一种理解这个构造的方式是去观察到,爆破的开集的坐标环被扩大了以分别加入$y'=y/x$和$x''=x/y$. 这导致了一系列结果。首先,$\mathbb A_K^2$上的函数对$x$, $y$在原点的补集上定义了一个映射$f:\mathbb A_K^2\setminus \{(0,0)\}\to \mathbb P_K^1$:在经典语言中,这是映射$(a,b)\mapsto [a,b]$,或者用更现代的语言,这是对应于由$(f,g)\mapsto xf+yg$给出的满射$\mathscr O\oplus \mathscr O\to \mathscr O$. 这个映射并不能延拓为$\mathbb A_K^2$上的一个正则映射,但是如果我们复合$f$与同构$Z\setminus E\cong \mathbb A_K^2\setminus \{(0,0)\}$,我们看到其确实延拓为了整个$Z$上的一个正则函数。这是因为由函数$x$和$y$(的拉回)生成的理想,在$Z$上是局部主理想(有一个非零因子生成),于是$x$和$y$有公共零点,进而我们可以将齐次矢量$[x,y]$出意它们的公共因子来延拓这个映射。在爆破中扩大坐标环的另一好处是其分离了穿过原点的曲线。即,如果$L$和$L'$是穿过$\mathbb A_K^2$原点的不同曲线,则$L\setminus \{(0,0)\}$和$L'\setminus \{(0,0)\}$的原像有着不同的闭包,如图所示(这些都是映射$f$的纤维)。
\inclugra{1.png}
同样地,如果我们有一条曲线$C\subset \mathbb A_K^2$在原点有一个结点,则$C$中原点的补集的原像在$Z$中非奇异,且交例外除子于两点。

\paragraph*{一般的爆破的定义}\addcontentsline{toc}{subsubsection}{一般的爆破的定义}
我们将用这些观察作为出发点来推广一个任意概形沿着任意子概形的爆破的定义。基本的事实如下,在一个$X$沿着子概形$Y\subset X$的爆破$\varphi:\Bl_Y(X)\to X$中,$Y$的原像是局部主的。为形式化这点,我们从一个定义开始:

% p.165

\begin{defi}\label{defi:4.15}
	令$X$为任意概形,$Y\subset X$是一个子概形。我们成$Y$是一个$X$中的\textit{Cartier子概形},如果其局部是单个非零因子的零点集,即,对任意的$p\in X$,$X$中存在一个$p$的仿射邻域$U=\spec A$和存在某个非零因子$f\in A$使得$Y\cap U=V(f)\subset U$. 更一般地,我们称$Y$是一个\textit{正则子概形},如果其局部是$X$上函数的一个正则列\footnote{
		译者注:设$R$是一个交换环,则一族元素$f_1$, $\dots$, $f_r$被称为一个正则列,如果对每个$i$,$f_i$都是$R/(f_1,\dots,f_{i-1})$中的非零因子,且$R/(f_1,\dots,f_{r})\neq 0$.
	}%
	(regular sequence)的零点集。
\end{defi}

\begin{defi}\label{defi:4.16}
	令$X$为任意概形,$Y\subset X$是一个子概形。\textit{$X$沿着$Y$的爆破},记作$\varphi:\Bl_Y(X)\to X$,是一个由下面几个性质刻画的态射:
	\begin{compactenum}[(1)]
		\item $Y$的原像$\varphi^{-1}(Y)$是$\Bl_Y(X)$中的一个Cartier子概形。
		\item $\varphi:\Bl_Y(X)\to X$具有万有性质;即,如果$f:W\to X$是任意的态射使得$f^{-1}(Y)$是一个$Z$中的Cartier子概形,则存在一个态射$g:W\to \Bl_Y(X)$使得$f=\varphi\circ g$.
	\end{compactenum}
	在爆破$\Bl_YX$中$Y$原像$E=\varphi^{-1}(Y)$被称为爆破的\textit{例外除子},而$Y$是爆破的\textit{中心}。
\end{defi}

很清楚,这些性质唯一刻画了概形沿着一个子概形的爆破$\varphi:\Bl_Y(X)\to X$. 但并不清楚的是爆破的存在性,但我们下面会看到这确实存在。

仿射情形的爆破可以以态射的图的闭包这种简单的方式来实现,我们将首先描述这个构造。我们从推广Example \ref{exa:4.14} 的构造到任意环上的仿射空间对原点的爆破开始。

\begin{exa}\label{exa:4.17}
	令$A$为任意环,再令$\mathbb A_A^n=\spec A[x_1,\dots,x_n]$. 考虑概形
	\[
	U_i=\spec T_i\cong \mathbb A_A^n,
	\]
	其中
	\[
		T_i=A\left[\frac{x_1}{x_i},\dots,\frac{x_n}{x_i},x_i\right]
	\]
	是$T=A[x_1,x_1^{-1},\dots,x_n,x_n^{-1}]$在$A$上由函数$x_j/x_i$和$x_i$生成的子代数。环$(T_{i})_{x_j}$和$(T_j)_{x_i}$作为$T$的子环是相同的,于是我们有交换同构
	\[
		(U_i)_{x_j}\cong (U_j)_{x_i}.
	\]
	因此我们可以构造一个概形$Z$,其是$U_i$的并且上面的开集被等同了。注意到,对应于含入$A[x_1,\dots,x_n]\hookrightarrow T_i$的态射$U_i\to \mathbb A_A^n$在交叠部分相容给出了一个自然的结构态射$\varphi:Z\to \mathbb A_A^n$.
\end{exa}

这个例子展现了诸多 Example \ref{exa:4.14} 中描述的经典爆破的特性:
% p.166
\begin{compactenum}[(1)]
	\item 令$U=\mathbb A_A^n\setminus V(x_1,\dots,x_n)$为$V(x_1,\dots,x_n)$(“原点”)在$\mathbb A_A^n$中的补。我们有一个由函数$(x_1,\dots,x_n)$给出的态射
	\[
		\alpha_{(x_1,\dots,x_n)}:U\to \mathbb P_A^{n-1},
	\]
	或者,更形式地,由满射
	\[
		\begin{aligned}
			\mathscr O_U^n & \longrightarrow \mathscr O_U,\\
			(a_1,\dots,a_n)&\longmapsto \sum a_ix_i.
		\end{aligned}
	\]
	给出的,则$Z$是\textit{$\alpha$的图像在$\mathbb A_A^n\times_A \mathbb P_A^{n-1}=\mathbb P_A^{n-1}$中的闭包}。为看到这点,我们我们观察到$(U_i)_{x_i}\subset Z$为映射
	\[
		\alpha_{(x_1,\dots,x_n)}|_{(U_i)_{x_i}}:(U_i)_{x_i}\to (\mathbb P_A^{n-1})_{x_i}=\spec \left[\frac{x_1}{x_i},\dots,\frac{x_n}{x_i}\right]
	\]
	的图像,而开集$(U_i)_{x_i}$在$Z$中稠密。
	\item $V(x_1,\dots,x_n)\subset \mathbb A_A^n$的原像$E=\varphi^{-1}V(x_1,\dots,x_n)\subset Z$在结构映射$\varphi:Z\to \mathbb A_A^n$同构于$\mathbb P_A^{n-1}$,而
	\[
		\varphi:Z\setminus E\xrightarrow{\sim}\mathbb A_A^n\setminus V(x_1,\dots,x_n)
	\]
	是一个同构。
	\item 因子$(x_1,\dots,x_n)T_i=(x_i)T_i$,$V(x_1,\dots,x_n)\subset \mathbb A_A^n$的原像$E\subset Z$是局部由单一方程定义的。
\end{compactenum}

\begin{pro}\label{pro:4.18}
	态射$\varphi:Z\to \mathbb A_A^n$是$\mathbb A_A^n$沿着子概形$V(x_1,\dots,x_n)$的爆破。
\end{pro}

\begin{proof}
	我们已经看到$Z\to \mathbb A_A^n$满足Definition \ref{defi:4.16} 的条件(1),还需证明,如果$\psi:W\to \mathbb A_A^n$是任意态射使得$\psi^{-1}V(x_1,\dots,x_n)$为Cartier的,则$\psi$可经由$\varphi$分解,即,存在一个映射$\alpha:W\to Z$满足$\psi=\varphi\circ\alpha$.

	我们首先对$W=\spec R$且$R$是一个局部环的情况证明。通过映射$\varphi^\#:A[x_1,\dots,x_n]\to R$将$R$看作$A[x_1,\dots,x_n]$上的代数。因为理想$(x_1,\dots,x_n)R$是主理想,Nakayama引理(Eisenbud [1995, Corollary 4.8])告诉我们它是由某个$x_i$生成的。更准确地说,如果
	\[
		(x_1,\dots,x_n)R=(\gamma),
	\]
	则我们可以记
	\[
		\gamma=\alpha_1 x_1+\cdots +\alpha_n x_n
	\]
	对某些$\alpha_i\in R$,类似地,$x_i=\beta_i\gamma$. 于是
	\[
		\gamma=\sum_i \alpha_i x_i=\sum_i \alpha_i \beta_i \gamma,
	\]
	从这里,我们看到至少有一个$\beta_i$是$R$中的单位元,即$(x_1,\dots,x_n)R=(\gamma)=(x_i)$对某个$i$成立。

% p.167
	我们现在可以对每个$j$记$x_j=\nu_j x_i$(其中$\nu_j=\beta_j\beta_i^{-1}$),进而定义想要的映射
	\[
		\alpha:W\to U_i\hookrightarrow Z
	\]
	作为环同态
	\[
	\begin{aligned}
		A\left[\frac{x_1}{x_i},\dots,\frac{x_n}{x_i},x_i\right] & \longrightarrow R,\\
		\frac{x_i}{x_j}&\longmapsto \nu_j
	\end{aligned}
	\]
	的对偶。现在假设$W$为任意概形,$\psi:W\to \mathbb A_A^n$是一个态射且$\psi^{-1}V(x_1,\dots,x_n)$是Cartier的。对每点$w\in W$,上面的结论产生了一个映射$\alpha:\spec \mathscr O_{W,w}\to Z$,其像包含于某个覆盖$Z$的仿射开集$U_i\cong \mathbb A_A^n\subset Z$. 这样一个映射可以延拓到$w\in W$的Zariski开领域上,在上面像$a^\#(x_j/x_i)$是正则的,于是我们可以找到$W$的一个开覆盖$W_k$以及态射$\alpha_k:W_k\to Z$使得$\varphi\circ \alpha_k=\psi|_{W_k}$.

	我们将证明映射$\alpha_k$在交叠$W_i\cap W_j$上相容,进而定义了一个整个$W$上的态射,这就完成了我们的证明。

	因为$\varphi$的限制$Z\setminus E\to \mathbb A_A^n\setminus V(x_1,\dots,x_n)$是一个同构,所以证明原像$\psi^{-1}(\mathbb A_A^n\setminus V(x_1,\dots,x_n))$在$W$中稠密就够了。但是从假设,$\psi^{-1}V(x_1,\dots,x_n)$在$W$中是一个Cartier除子。因此下面的引理就完成了证明。

	\begin{lem}\label{lem:4.19}
		如果$X\subset Y$是一个概形的Cartier子概形,则$Y\setminus X$在$Y$中稠密(作为概形而不仅仅是拓扑空间)。
	\end{lem}

	\begin{proof}
		我们可以假设$Y$是仿射的,记$Y=\spec A$,于是$X=V(f)$,其中$f\in A$是一个非零因子。说存在一个逆紧闭子概形$Y'$包含$Y\setminus X$,就是说局部化映射$A\to A_f$经由$A/I(Y')$分解。但因为$f$是一个非零因子,局部化映射是一个单射。
	\end{proof}
	\let\qed\relax
\end{proof}

\begin{exe}\label{exe:4.20}
	\begin{compactenum}[(a)]
		\item 证明Lemma \ref{lem:4.19} 的结论对
		\[
			X=V(x)\subset Y=\spec K[x,y]/(xy,y^2)
		\]
		失效。
		\item 更一般地证明,其在$Y$的所有局部主子概形中刻画了Cartier字概形。
		\item 证明$\Bl_Y=\varnothing$ 当且仅当$\operatorname{supp} Y=\operatorname{supp} X$.
	\end{compactenum}
\end{exe}

Proposition \ref{pro:4.18} 的构造将产生所有仿射概形的爆破,只要我们理解了爆破是如何作用在子概形上,或更一般地,在拉回下。这直接来自于定义:

\begin{pro}\label{pro:4.21}
	令$X$为任意概形,$Y\subset X$是一个子概形,而$\varphi:\Bl_Y(X)\to X$为$X$沿着$Y$的爆破。令$\nu:X'\to X$为任意态射,$Y'=\nu^{-1}(Y)\subset X'$. 如果$W$是原像$\pi_1^{-1}(X'\setminus Y')$在纤维积$X'\times_X \Bl_Y X$中的闭包,则$\pi_1:W\to X'$是$X'$沿着$Y'$的爆破。
\end{pro}

% p.168

这个引理在$X'=X$的情况下已然足够有趣了,它断言了 \textit{$X\setminus Y$在$\Bl_Y X$中的原像是稠密的}。

Proposition \ref{pro:4.21} 最经常应用于$X'\subset X$是一个闭子概形的情况。此时,$W$就是$\Bl_Y X$中原像$\varphi^{-1}(X'\setminus (X'\cap Y))$的闭包。这被称作$X'$在$\Bl_Y X$中的\textit{严格变换}或者\textit{真变换}. (完整的原像$\varphi^{-1}(X')\subset \Bl_Y X$被称作\textit{全变换}.)因此我们可以说,在爆破$\Bl_p \mathbb A_K^2$中,穿过原点$p\in \mathbb A_K^2$的直线们的真变换是不交的(注意到,直线的真变换同构于线本身,它们本该如此,因为原点是每条线上的一个Cartier子概形),以及,一个有结的曲线在结点的爆破在结点拉开的点上都是非奇异的。

当$X'\subset X$是一个开子概形,Proposition \ref{pro:4.21} 就说明了爆破与基变换可交换,即
\[
	\varphi^{-1}(X')\cong \Bl_{X'\cap Y}X'\to X'.
\]
但正确的不止于此:因为$\varphi^{-1}(X'\setminus Y)$是稠密的,在$X$上\textit{只有一个}这样的同构。作为结论,如果$\pi:Z\to X$是一个态射,且我们假设有一个$X$的开覆盖,其中开集$U$在$X$上满足$\pi^{-1}U\cong \Bl_{U\cap Y}U$,则$Z\cong \Bl_Y X$. 简而言之:爆破是由局部确定的。

\begin{proof}[Proposition \ref{pro:4.21} 的证明]
	我们首先检查$Y'$的原像
	\[
		E'=\pi_1^{-1}(Y')\subset W
	\]
	是$W$的一个Cartier子概形。他当然是主的:其原像$E=\varphi^{-1}(Y)\subset \Bl_Y X$在$\Bl_Y X$中是局部主的,同时$E'\subset W$就是它对投影$\pi_2:W\to \Bl_Y X$的原像。此外,因为$W$相伴素理想们正是$X'$并不包含$Y'$的理想的相伴素理想们,则$E$在$\Bl_Y X$中的局部方程并不能拉回到$W$上的一个零因子。

	接着,我们必须验证$W$有万有性质。假设$T$为任意概形,$f:T\to X'$是任意态射使得$Y'$在$T$中的原像$f^{-1}(Y')$是一个Cartier子概形。特别地,因为$f^{-1}(Y')\subset T$是Cartier的,没有$T$的分支或嵌入分支映射到$Y'$,因此$f^{-1}(X'\setminus Y')$在$T$的闭包是整个$T$.

	我们须证明,$f$可以提升为一个态射$g:T\to W$(即,存在一个态射$g:T\to W$使得满足复合$\pi_1\circ g=f$)。分三步来。首先,令
	\[
		h=\nu\circ f:T\to X
	\]
	为$f$与态射$\nu:X'\to X$的复合,因为原像$h^{-1}(Y)=f^{-1}(Y')$是Cartier的,从爆破$\Bl_Y X\to X$的万有性质,$h$可以提升为一个态射$\tilde h:T\to \Bl_Y X$.
% p.169
	接着,映射$f:T\to X'$和$\tilde h:T\to \Bl_Y X$给出了映射
	\[
		\tilde g:T\to X'\times_X \operatorname{Bl}_Y X,
	\]
	其与投影$\pi_1:X'\times_X \Bl_Y X\to X'$的复合为$f$. 最后,因为$\tilde g$将原像$f^{-1}(X'\setminus Y')$映射到$W$,$f^{-1}(X'\setminus Y')$在$T$中的闭包就是$T$,于是映射$\tilde g:T\to X'\times_X \operatorname{Bl}_Y X$经由$W$在$X'\times_X \Bl_Y X$的含入映射分解,其给出了想要的映射$g:T\to W$.
\end{proof}

我们现在已经到了沿着任意闭子概形爆破一个仿射概形的地方了。若$X=\spec A$以及$f_1,\dots,f_n\in A$,则$(f_1,\dots,f_n)$定义了一个态射
\[
	\alpha_{(f_1,\dots,f_n)}:U=X\setminus V(f_1,\dots,f_n)\longrightarrow \mathbb P_A^{n-1};
\]
更准确地,$(f_1,\dots,f_n)$定义了一个映射$\mathscr O_X^n\to \mathscr O_X$,其将$(a_1,\dots,a_n)$映射为$\sum a_if_i$,正为一个$U$上的满射。

\begin{pro}\label{pro:4.22}
令$X=\spec A$是一个仿射概形,再令
\[
	Y=V(f_1,\dots,f_n)\subset X
\]
是一个闭子概形。$Y$在$X$中的爆破是态射
\[
	\alpha_{(f_1,\dots,f_n)}:X\setminus Y\longrightarrow \mathbb P_A^{n-1}
\]
的图像在$X\times_A \mathbb P_A^{n-1}=\mathbb P_A^{n-1}$中的闭包。
\end{pro}

\begin{proof}
	考虑由环同态
	\[
		\begin{aligned}
			A\left[x_1,\dots,x_n\right] & \longrightarrow A,\\
			x_i &\longmapsto f_i
		\end{aligned}
	\]
	给出的嵌入$X\hookrightarrow \mathbb A_A^n=\spec A[x_1,\dots,x_n]$.
	注意到在这个嵌入下我们有$X\cap V(x_1,\dots,x_n)=Y$. 从Proposition \ref{pro:4.21}, $X$沿着$Y$的爆破是一个$X$在$\mathbb A_A^n$沿着$V(x_1,\dots,x_n)$的爆破$Z$中的真变换。从Proposition \ref{pro:4.18},另一方面,$Z$$\mathbb A_A^n$沿着$V(x_1,\dots,x_n)$的爆破$Z$是映射
	\[
		\alpha_{(x_1,\dots,x_n)}:\mathbb A_A^n\setminus V(x_1,\dots,x_n)\longrightarrow \mathbb P_A^{n-1}
	\]
	的图像$\Gamma$的闭包。因为$\alpha_{(f_1,\dots,f_n)}$的图像就是$\Gamma$与$X\subset \mathbb A_A^n$的原像的交,其闭包为$X\subset \mathbb A_A^n$在$Z$中的真变换,于是得到了结论。
\end{proof}

在这个命题中,我们限制了子概形$Y\subset X$是由有限多函数$f_i$所定义的,但这并不必要。读者可以检查所有的事情对无限集也成立(尽管态射映到了无限维射影空间)。

% p.170

\paragraph*{作为$\proj$的爆破}\addcontentsline{toc}{subsubsection}{作为Proj的爆破} 我们已经证明了一个仿射概形沿着一个闭子概形的爆破的存在性。我们已然可以用黏合去推出一般情况下爆破的存在性。然而,通过全局$\proj$,存在一个更优雅的方式来定义爆破,一步到位。

\begin{thm}\label{thm:4.23}
	令$X$为一个概形,$Y\subset X$是一个闭子概形。令$\mathscr I=\mathscr I_{Y,X}\subset \mathscr O_X$为$Y$在$X$中的理想层。若$\mathscr A$是分次$\mathscr O_X$-代数层
	\[
	\mathscr A=\bigoplus_{n=0}^\infty \mathscr I^n=\mathscr O_X\oplus \mathscr I\oplus \mathscr I^2\oplus \cdots
	\]
	(其中第$k$个直和项去做$\mathscr A$的$k$-次分次部分),则概形$\proj(\mathscr A)\to X$是$X$沿着$Y$的爆破。
\end{thm}

\paragraph*{注记:}这个构造时常导致记号上的混淆:若$f\in \mathbb O_X(U)$实在$Y$上为零的正则函数,记号``$f$''既可以被用来标记$\mathscr A_0=\mathscr O_X$的截面或$\mathscr A_1=\mathscr I$的截面,它们是$\mathscr A$两个不同的截面。为避免这点,我们将经常将$\mathscr A$实现为层
\[
	\mathscr O_X[t]=\bigoplus_{n=0}^\infty t^n\mathscr O_X,
\]
的一个子层,记
\[
	\mathscr A=\mathscr O_X\oplus t\mathscr I\oplus t^2\mathscr I^2\oplus \cdots.
\]
我们将在下面的证明中用这个记号。

\begin{proof}
	我们必须证明态射
\[
	\varphi:B=\proj(\mathscr A)\to X
\]
满足刻画爆破的两个性质:$Y$在$B$中的原像$\varphi^{-1}Y$是Cartier的,以及任意态射$f:Z\to X$使得$f^{-1}Y$是Cartier的都可以唯一地经由$B$分解。我们将记$\mathscr I$为$Y$在$X$中的理想层$\mathscr I_Y$.

\nottran
\[
	\begin{aligned}
	\mathscr I\mathscr A&=\mathscr I\cdot \mathscr O_B\oplus \mathscr I\cdot \mathscr I\oplus \mathscr I\cdot \mathscr I^2 \oplus \cdots\\
	&=\mathscr I\oplus \mathscr I^2 \oplus \mathscr I^3\oplus \cdots
	\end{aligned}
\]

\[
	\mathscr A(1)=\mathscr O\oplus \mathscr I \oplus \mathscr I^2\oplus \cdots
\]

% p.171

\[
	f^*\mathscr I=\mathscr I\otimes_{\mathscr O_X}\mathscr O_Z\to \mathscr I\cdot \mathscr O_Z
\]

\end{proof}

\begin{exe}\label{exe:4.24}
	证明一个概形$X$在一个点$p\in X$的爆破$\Bl_p(X)$中的例外除子就是$X$在点$p$的射影化切锥$\mathbb PTC_p(X)$.
\end{exe}

\paragraph*{沿着正则子概形的爆破}\addcontentsline{toc}{subsubsection}{沿着正则子概形的爆破}

\[
	\mathscr A=\bigoplus_{n=0}^\infty t^n\mathscr I_{Y,X}^n\subset \mathscr O_X[t]
\]

% p.172

\begin{pro}\label{pro:4.25}
令$A$是一个Noether环,$x$, $y\in A$,令$B$是Rees代数
	\[
		B=A[xt,yt]\subset A[t].
	\]
若$x$, $y\in A$是一个正则列,则
	\[
		B\cong A[X,Y]/(yX-xY)
	\]
通过映射$X\mapsto xt$, $Y\mapsto yt$.
\end{pro}

\begin{proof}

	\[
		\begin{aligned}
			A[x^{-1}][X',Y] & \longrightarrow A[x^{-1}][t],\\
			X' &\longmapsto t,\\
			Y  &\longmapsto yt.
		\end{aligned}
	\]
	
	\[
		M:=\frac{(yX-xY):(x)}{(yX-xY)}=0,
	\]
	
	\[
		0\xrightarrow{\qquad\quad\quad} A \xrightarrow{
		\begin{pmatrix}
			-x\\ yX-xY
		\end{pmatrix}}A^2
		\xrightarrow{\begin{pmatrix}
			yX-xY& x
		\end{pmatrix}}A.
	\]
	
	\[
		\frac{(x):(yX-xY)}{(x)},
	\]	

\end{proof}

% p.173

\[
	A\oplus I \oplus I^2\oplus \cdots
\]

\[
	\operatorname{Sym}_A(I)
\]

\[
	\begin{pmatrix}
		x&y\\
		X&Y
	\end{pmatrix}.
\]

\[
	A\oplus I \oplus I^2\oplus \cdots\cong A[X_1,\dots,X_n]/J
\]

\[
	\begin{pmatrix}
		x_1&\dots&x_n\\
		X_1&\dots&X_n
	\end{pmatrix}.
\]

\subsection{一些经典爆破构造}\label{s:4.2.2}

\[
	Q=\spec K[x,y,z]/(xy-z^2)\subset \spec K[x,y,z]=\mathbb A_K^3.
\]

\[
	\varphi:\tilde A_K^3=\proj K[x,y,z][A,B,C]/(xB-yA,xC-zA,yC-zB)\longrightarrow \spec K[x,y,z]=\mathbb A_K^3.
\]

\[
	U_A=\spec K[x,y,z][b,c]/(xb-y,xc-z)=\spec K[x,b,c]
\]

\[
	\varphi^\# (xy-z^2)=x^2b-x^2c^2=x^2(b-c^2)
\]

\[
	\varphi^{-1}(Q)=V((x,y,z)^2)\cup V(AB-C^2)
\]

\[
	\varphi:\tilde Q=\proj K[x,y,z][A,B,C]/(xB-yA,xC-zA,yC-zB,AB-C^2)\longrightarrow \spec K[x,y,z]/(xy-z^2)=Q.
\]

\inclugra{2.png}

\[
	\varphi:\Bl_L \mathbb A_K^3=\proj K[x,y,z][A,B]/(xB-zA)\longrightarrow \spec K[x,y,z]=\mathbb A_K^3
\]

\[
	X=V(xw-yz)\subset \spec K[x,y,z,w]=\mathbb A_K^4.
\]

\[
\Lambda_{1,\mu}=V(x-\mu z,y-\mu w)
\]

\[
\Lambda_{2,\mu}=V(x-\mu y,z-\mu w)
\]

\inclugra{3.png}

\[
	M=\Hom(V,W)=\left\{\begin{pmatrix}x&y\\z&w\end{pmatrix}\right\}.
\]

\[
	\begin{aligned}
		X&=\{A:V\to W\,|\,\operatorname{rank}A\leq 1\}\subset \mathbb A_K^4,\\
		X_1&=\{(A,L)\,|\,L\subset \operatorname{Ker}A\}\subset \mathbb A_K^4\times \mathbb PV^*,\\
		X_2&=\{(A,L')\,|\,\operatorname{Im}A\subset L'\}\subset \mathbb A_K^4\times \mathbb PW^*,\\
		\tilde X&=\{(A,L,M)\,|\, L\subset \operatorname{Ker}A\text{ and }\operatorname{Im}A\subset M\}\subset \mathbb A_K^4\times \mathbb PV^*\times \mathbb PW^*.
	\end{aligned}
\]

\[
	\hat{\mathscr O}_{X,p}\cong K[\![x_1,\dots,x_n]\!]/(x_1^2+x_2^2+\cdots+x_n^2).
\]

\[
	\dim (\pi^{-1}(\Gamma))\leq \frac{\dim(\Gamma)+\dim(X)-1}2
\]

\subsection{沿着非约态概形的爆破}\label{s:4.2.3}

\paragraph*{一个双重点的爆破}\addcontentsline{toc}{subsubsection}{一个双重点的爆破}