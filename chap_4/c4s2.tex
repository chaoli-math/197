\section{爆破}\label{s:4.2}

爆破是经典代数几何的一个基本工具。它用以消解奇点、消解有理映射的不定性,以及相互联系双有理簇。
称一个簇是另一个簇沿着一个给定子簇的爆破,表出了此二者间的一种关系,一方面其同时足够接近去密切联系两个簇的结构,另一方面其足够灵活,在簇之间的映射的表达式中非常常见。
在这节中,我们将推广爆破的定义到概形范畴,定义任意的(Noether)概形沿着任意闭子概形的爆破。

以这种方式概括爆破的定义实际上有两个目的。首先是期望的好处:概形和簇的爆破的作用相同,即消解奇点或者联系两个双有理概形(比如说,我们将在第 \ref{s:4.2.4} 节中爆破算术概形)。

此外,我们还可以看到,甚至在簇之间的映射的语境中,概形的语言意味着相关概念非常有用之推广,特别地,可以去讨论一个概形$X$沿着一个可能非约态的子概形$Y\subset X$的爆破。比方说,我们将在下面的第 \ref{s:4.2.3} 节展示这点,在那里用这个推广的爆破的定义,我们推广了对非奇异二次曲面作为平面沿着一个二次曲线的爆破的传统的描述。类似地,在第 \ref{4.2.3} 节,我们将看到一个自然出现的簇之间的映射是沿着一个子概形的爆破。这些例子实际上并不特殊:当我们以这种方式推广“爆破”的定义,则任意簇之间的射影双有理态射都是爆破!这在Hartshorne [1977, Theorem II.7.17]中有证明。

\subsection{定义与构造}\label{s:4.2.1}

% p.163

\[
	\begin{aligned}
		(\varphi')^\#:K[x,y]&\longrightarrow K[x',y']&\quad \text{and}\qquad (\varphi'')^\#:K[x,y]&\longrightarrow K[x'',y'']\\
		x&\longmapsto x'& x&\longmapsto x''y''\\
		x&\longmapsto x'y'& y&\longmapsto y''.
	\end{aligned}
\]

\[
	U'_x=\spec K[x',y',{x'}^{-1}]\quad \text{and}\quad 
	U_x=\spec K[x,y,x^{-1}],
\]

\[
	U'_{xy}=\spec K[x',y',{x'}^{-1},{y'}^{-1}]\quad \text{and}\quad 
	U''_{xy}=\spec K[x'',y'',{x''}^{-1},{y''}^{-1}]
\]

\[
	Z=U'\cup U''=\spec K[x',y']\bigcup_{U'_{xy}\cong U''_{xy}}\spec K[x'',y''],
\]

\[
	\begin{aligned}
		K[x',y',{x'}^{-1},{y'}^{-1}]&\longrightarrow 
		K[x'',y'',{x''}^{-1},{y''}^{-1}]\\
		x'&\longmapsto x''y''\\
		y'&\longmapsto {x''}^{-1}.
	\end{aligned}
\]

% p.164

\inclugra{1.png}

% p.165

\[
	U_i=\spec T_i\cong \mathbb A_A^n,
\]
其中
\[
	T_i=A\left[\frac{x_1}{x_i},\dots,\frac{x_n}{x_i},x_i\right]
\]

\[
	(U_i)_{x_j}\cong (U_j)_{x_i}.
\]

% p.166

\[
	\alpha_{(x_1,\dots,x_n)}:U\to \mathbb P_A^{n-1}
\]

\[
	\begin{aligned}
		\mathscr O_U^n & \longrightarrow \mathscr O_U,\\
		(a_1,\dots,a_n)&\longmapsto \sum a_ix_i.
	\end{aligned}
\]

\[
	\alpha_{(x_1,\dots,x_n)}|_{(U_i)_{x_i}}:(U_i)_{x_i}\to (\mathbb P_A^{n-1})_{x_i}=\spec \left[\frac{x_1}{x_i},\dots,\frac{x_n}{x_i}\right]
\]

\[
	\varphi:Z\setminus E\xrightarrow{\sim}\mathbb A_A^n\setminus V(x_1,\dots,x_n)
\]

\begin{pro}\label{pro:4.18}
	态射$\varphi:Z\to \mathbb A_A^n$是$\mathbb A_A^n$沿着子概形$V(x_1,\dots,x_n)$的爆破。
\end{pro}

% p.167

\[
	\alpha:W\to U_i\hookrightarrow Z
\]

\[
	\begin{aligned}
		A\left[\frac{x_1}{x_i},\dots,\frac{x_n}{x_i},x_i\right] & \longrightarrow R,\\
		\frac{x_i}{x_j}&\longmapsto \nu_j.
	\end{aligned}
\]

\[
	X=V(x)\subset Y=\spec K[x,y]/(xy,y^2).
\]

% p.168

% p.169

\[
	\tilde g:T\to X'\times_X \operatorname{Bl}_Y X
\]

\[
	\alpha_{(f_1,\dots,f_n)}:U=X\setminus V(f_1,\dots,f_n)\longrightarrow \mathbb P_A^{n-1};
\]

\[
	Y=V(f_1,\dots,f_n)\subset X
\]

\[
	\alpha_{(f_1,\dots,f_n)}:X\setminus Y\longrightarrow \mathbb P_A^{n-1}.
\]

\[
	\begin{aligned}
		A\left[x_1,\dots,x_n\right] & \longrightarrow A,\\
		x_i &\longmapsto f_i.
	\end{aligned}
\]

\[
	\alpha_{(x_1,\dots,x_n)}:\mathbb A_A^n\setminus V(x_1,\dots,x_n)\longrightarrow \mathbb P_A^{n-1}.
\]

% p.170

\[
	\mathscr A=\bigoplus_{n=0}^\infty \mathscr I^n=\mathscr O_X\oplus \mathscr I\oplus \mathscr I^2\oplus \cdots
\]

\[
	\mathscr O_X[t]=\bigoplus_{n=0}^\infty t^n\mathscr O_X,
\]
记
\[
	\mathscr A=\mathscr O_X\oplus t\mathscr I\oplus t^2\mathscr I^2\oplus \cdots.
\]

\[
	\varphi:B=\proj(\mathscr A)\to X
\]

\[
	\begin{aligned}
	\mathscr I\mathscr A&=\mathscr I\cdot \mathscr O_B\oplus \mathscr I\cdot \mathscr I\oplus \mathscr I\cdot \mathscr I^2 \oplus \cdots\\
	&=\mathscr I\oplus \mathscr I^2 \oplus \mathscr I^3\oplus \cdots
	\end{aligned}
\]

\[
	\mathscr A(1)=\mathscr O\oplus \mathscr I \oplus \mathscr I^2\oplus \cdots
\]

% p.171

\[
	f^*\mathscr I=\mathscr I\otimes_{\mathscr O_X}\mathscr O_Z\to \mathscr I\cdot \mathscr O_Z
\]

\[
	\mathscr A=\bigoplus_{n=0}^\infty t^n\mathscr I_{Y,X}^n\subset \mathscr O_X[t]
\]

% p.172

\[
	B=A[xt,yt]\subset A[t].
\]

\[
	B\cong A[X,Y]/(yX-xY)
\]

\[
	\begin{aligned}
		A[x^{-1}][X',Y] & \longrightarrow A[x^{-1}][t],\\
		X' &\longmapsto t,\\
		Y  &\longmapsto yt.
	\end{aligned}
\]

\[
	M:=\frac{(yX-xY):(x)}{(yX-xY)}=0,
\]

\[
	0\xrightarrow{\qquad\quad\quad} A \xrightarrow{
	\begin{pmatrix}
		-x\\ yX-xY
	\end{pmatrix}}A^2
	\xrightarrow{\begin{pmatrix}
		yX-xY& x
	\end{pmatrix}}A.
\]

\[
	\frac{(x):(yX-xY)}{(x)},
\]

% p.173

\[
	A\oplus I \oplus I^2\oplus \cdots
\]

\[
	\operatorname{Sym}_A(I)
\]

\[
	\begin{pmatrix}
		x&y\\
		X&Y
	\end{pmatrix}.
\]

\[
	A\oplus I \oplus I^2\oplus \cdots\cong A[X_1,\dots,X_n]/J
\]

\[
	\begin{pmatrix}
		x_1&\dots&x_n\\
		X_1&\dots&X_n
	\end{pmatrix}.
\]

\subsection{一些经典爆破构造}\label{s:4.2.2}