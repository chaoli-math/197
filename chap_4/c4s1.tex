在这章中,我们将展示如何在概形理论中给出一些经典代数几何中的几何构造。
我们将在所有例子中看到,这种崭新的语言是如何允许我们扩大定义的范围
(以及我们对该对象可能问的问题),如何让我们能够给出经典问题的精确表述,
以及在一些情况中,如何帮助我们解决他们。

\section{平面曲线的拐点}\label{s:4.1}

在这节中,我们将描述代数闭域$K$上的非奇异平面曲线
$C\subset \mathbb P_K^2$的\textit{拐点}的经典定义。然后指出
这个定义是如何推广到概形语境中的,并展示这个推广是如何阐明拐点
的集合的,即使是在经典情形中。

\subsection{定义}\label{s:4.1.1}

我们需要一个预备定义。令$K$是任意域,再令$C$, $D\subset C\subset \mathbb P_K^2$是没有公共分支的两条平面曲线,$p\in C\cap D$是一个交点。
我们定义$C$和$D$在点$p$的\textit{相交重数}为$p$所处的概形$C\cap D$
的分支$\Gamma$的重数,记作$\mu_p(C\cdot D)$. 因为平面曲线是
Cohen-Macaulay的,这和在第 \ref{s:3.3.5} 节中引入的相交重数的概念
是相合的。%
% p.152
同样注意次数的概念的联系:$\Gamma$作为$\mathbb P_K^2$的子概形的次数
是相交重数$\mu_p(C\cdot D)$乘以剩余类域作为$K$的扩张的
次数$(\kappa(p):K)$. 因此,比如平面曲线的
B\'ezout定理(\ref{thm:3.71})告诉我们
\[
	\deg C\deg D=\sum_{p\in C\cap D}
	(\kappa(p):K)\mu_p(C\cdot D).
\]

话说回来,经典代数几何中平面曲线的拐点是直接且几何的:如果
$C\subset \mathbb P_{\mathbb C}^2$是一个复数域上的
次数为$d$的非奇异平面曲线,点$p\in C$被称为一个\textit{拐点},
如果射影切线$\mathbb T_pC\subset \mathbb P_{\mathbb C}^2$
在点$p$处交$C$三次或以上,或者用现代的语言,如果
$\mathbb T_pC$和$C$在点$p$的相交重数$\mu_p(C\cdot \mathbb T_pC)$
至少为三。(这里,因为我们在代数闭域上工作,相交重数等于
$p$所处的$\mathbb T_pC\cap C$的分支的次数,
即$\dim_{\mathbb C}(\mathscr O_{\mathbb T_pC\cap C,p})$.)
有一个经典的定理(我们将在后面建立之),如果$C$不是一条直线,
则$C$的拐点有限,并且,如果辅以合适的重数来数,则拐点数应为
$3d(d-2)$.

\nottran 

% p.152

\[\CenteredArraystretch{1.7}
	H(X,Y,Z)=
	\begin{vmatrix}
		\pfrac{^2F}{X^2}&\ppfrac FXY &\ppfrac FXZ\\
		\ppfrac FYX&\pfrac{^2F}{Y^2}&\ppfrac FYZ\\
		\ppfrac FZX&\ppfrac FZY&\pfrac{^2F}{Z^2}
	\end{vmatrix}
\]

\[
	x=\frac XZ,\quad y=\frac YZ
\]

\[\CenteredArraystretch{1.7}
	h(x,y)=
	\begin{vmatrix}
		f&\pfrac fx &\pfrac fy\\
		\pfrac fx&\pfrac{^2f}{x^2}&\ppfrac fxy\\
		\pfrac fy&\ppfrac fxy&\pfrac{^2f}{y^2}
	\end{vmatrix}
\]