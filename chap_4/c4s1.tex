在这章中,我们将展示如何在概形理论中给出一些经典代数几何中的几何构造。
我们将在所有例子中看到,这种崭新的语言是如何允许我们扩大定义的范围
(以及我们对该对象可能问的问题),如何让我们能够给出经典问题的精确表述,
以及在一些情况中,如何帮助我们解决他们。

\section{平面曲线的拐点}\label{s:4.1}

在这节中,我们将描述代数闭域$K$上的非奇异平面曲线
$C\subset \mathbb P_K^2$的\textit{拐点}的经典定义。然后指出
这个定义是如何推广到概形语境中的,并展示这个推广是如何阐明拐点
的集合的,即使是在经典情形中。

\subsection{定义}\label{s:4.1.1}

我们需要一个预备定义。令$K$是任意域,再令$C$, $D\subset C\subset \mathbb P_K^2$是没有公共分支的两条平面曲线,$p\in C\cap D$是一个交点。
我们定义$C$和$D$在点$p$的\textit{相交重数}为$p$所处的概形$C\cap D$
的分支$\Gamma$的重数,记作$\mu_p(C\cdot D)$. 因为平面曲线是
Cohen-Macaulay的,这和在第 \ref{s:3.3.5} 节中引入的相交重数的概念
是相合的。%
% p.152
同样注意其与次数这个概念的联系:$\Gamma$作为$\mathbb P_K^2$的子概形
的次数是相交重数$\mu_p(C\cdot D)$乘以剩余类域作为$K$的扩张的
次数$(\kappa(p):K)$. 因此,比如平面曲线的
B\'ezout定理(\ref{thm:3.71})告诉我们
\[
	\deg C\deg D=\sum_{p\in C\cap D}
	(\kappa(p):K)\mu_p(C\cdot D).
\]

话说回来,经典代数几何中平面曲线的拐点是直接且几何的:如果
$C\subset \mathbb P_{\mathbb C}^2$是一个复数域上的
次数为$d$的非奇异平面曲线,点$p\in C$被称为一个\textit{拐点},
如果射影切线$\mathbb T_pC\subset \mathbb P_{\mathbb C}^2$
(见Section \ref{s:3.2.4})在点$p$处交$C$三次或以上,或者用现代的语言,如果
$\mathbb T_pC$和$C$在点$p$的相交重数$\mu_p(C\cdot \mathbb T_pC)$
至少为三。(这里,因为我们在代数闭域上工作,相交重数等于
$p$所处的$\mathbb T_pC\cap C$的分支的次数,
即$\dim_{\mathbb C}(\mathscr O_{\mathbb T_pC\cap C,p})$.)
有一个经典的定理(我们将在后面建立之),如果$C$不是一条直线,
则$C$的拐点有限,并且,如果辅以合适的重数来数,则拐点数应为
$3d(d-2)$.

这个简单的定义可被推广到奇异曲线,比如可见Coolidge [1931],
尽管这个定义和现代标准定义比并不总是精确的。这个定义在
我们考虑非代数闭域或有限特征的域$K$
上的曲线$C\subset \mathbb P_K^2$时会出问题,此外还有考虑
包含直线或者一个多次分支的曲线时。

我们将在这里给任意概形$S$上的平面曲线$C\subset \mathbb P_S^2$的拐点
一个统一的定义。首先回忆第 \ref{s:3.2.8} 节,概形$S$上的
$d$次\textit{平面曲线}是指子概形$C\subset \mathbb P_S^2$,其局部地
在$S$上,是一个$d$-次齐次多项式
\[
	F(X,Y,Z)=\sum_{i+j+k=d}a_{ijk}X^iY^jZ^k
\]
的零点集$V(F)$,其中系数$a_{ijk}$是$S$上的正则函数,它们并不同时为零。
再回忆,如果$S$是一个仿射概形,则我们可以扔掉“局部”,即,如果$S=\spec A$,
一个$S$上的平面曲线$C$具有形式
\[
	C=\proj A[X,Y,Z]/(F)
\]
其中$F$是一个多项式。

现在,给定$S$上的一条平面曲线$C\subset \mathbb P_S^2$,我们将定义一个
闭子概形$\mathscr F=\mathscr F_C\subset C$,我们将称之为$C$上的
\textit{拐点概形}。他将与基变换$S'\to S$可交换(即,如果
$C'=S'\times_S C\subset \mathbb P_{S'}^2$,则$\mathscr F_{C'}=(\pi_2)^{-1}(\mathscr F_C)$)以及$\mathscr F$将至少在$S$的使得$\mathscr F$在其上的相对维数为$0$的开集上有限且平坦,次数为$3d(d-2)$.
这个闭子概形的重要性在于,如果我们有一族平面曲线,则一般纤维的拐点的极限
是一个特殊纤维的拐点(即,$\mathscr F$是闭的),% p.153
且当$\mathscr F$相对维数为$0$($\mathscr F$是平坦的)时候反过来也成立。
此外,在经典语境中,即如果$C$是一个特征为零的代数闭域的谱$S=\spec K$
上的非奇异曲面曲线,$\mathscr F$的底空间将是$C$经典定义的该店的集合
(因此,在一般的情况中,如果$s\in S$是任意的剩余类域$\kappa(s)$为代数闭
且特征零的点,则$\mathscr F$在点$s$的纤维$\mathscr F_s$的底空间将是$C_s$
的拐点的集合)。

为启发我们一般的定义,回忆经典代数几何中某个最早的结论:对一个
代数闭域$K$上非奇异平面曲线$C=V(F)\subset \mathbb P_K^2$,
其由多项式$F(X,Y,Z)$的零点集给出,则$C$的拐点为其与它的\textit{Hessian}
的交点,Hessian是由多项式
\[\CenteredArraystretch{1.7}
	H(X,Y,Z)=
	\begin{vmatrix}
		\pfrac{^2F}{X^2}&\ppfrac FXY &\ppfrac FXZ\\
		\ppfrac FYX&\pfrac{^2F}{Y^2}&\ppfrac FYZ\\
		\ppfrac FZX&\ppfrac FZY&\pfrac{^2F}{Z^2}
	\end{vmatrix}
\]
的零点集定义的曲线。我们将这个事实的证明留作一个习题:
\begin{exe}\label{exe:4.1}
	令$K$是一个特征为零的代数闭域,$C\subset \mathbb P_K^2$是一个平面曲线,
	$p\in C$是$C$的一个非奇异点。证明,射影切线$\mathbb T_pC$
	与$C$在点$p$交于三次或以上当且仅当$H(p)=0$.

	\textit{提示}:在$\mathbb P_K^2$的相应子集上引入仿射坐标
	\[
	x=\frac XZ,\quad y=\frac YZ,
	\]
	然后利用Euler关系来看到Hessian行列式的非齐次化$h(x,y)=H(x,y,1)$
	(在相差一个相乘因子下)为
	\[\CenteredArraystretch{1.7}
	h(x,y)=
	\begin{vmatrix}
		f&\pfrac fx &\pfrac fy\\
		\pfrac fx&\pfrac{^2f}{x^2}&\ppfrac fxy\\
		\pfrac fy&\ppfrac fxy&\pfrac{^2f}{y^2}
	\end{vmatrix},
	\]
	其中$f(x,y)=F(x,y,1)$为$F$的非齐次化。
\end{exe}

为定义一般情况下任意平面曲线$C\subset \mathbb P_S^2$的拐点概形,
我们可以简单地推广Hessian和这个刻画:假设在一些仿射开集
$U=\spec R\subset S$中,曲线
\[
	C\cap \mathbb P_U^2=\proj R[X,Y,Z]
\]%
% p.154
是多项式$F\in R[X,Y,Z]$零点集。我们定义\textit{Hessian行列式}为多项式
\[\CenteredArraystretch{1.7}
	H(X,Y,Z)=
	\begin{vmatrix}
		\pfrac{^2F}{X^2}&\ppfrac FXY &\ppfrac FXZ\\
		\ppfrac FYX&\pfrac{^2F}{Y^2}&\ppfrac FYZ\\
		\ppfrac FZX&\ppfrac FZY&\pfrac{^2F}{Z^2}
	\end{vmatrix}.
\]
因为$F$以及$H$在$R=\mathscr O(U)$中是由$C$决定的,至多差一个相乘可逆元,
我们可以定义$C$的\textit{Hessian} $C'$为$\mathbb P_S^2$中由在每一个
仿射开集$U\subset S$上的Hessian行列式确定的那个子概形,进而我们定义$C$
的拐点概形$\mathscr F$为交
\[
	\mathscr F=C\cap C'.
\]
立即看到,这是$C$的一个闭子概形以及它与基变换可交换。特别地,对任意点
$s\in S$,$\mathscr F$在点$s$的纤维$\mathscr F_s$将就是$C$在点$s$的纤维
$C_s\subset \mathbb P_{\kappa(s)}^2$的拐点概形。
作为两个次数为$d$和$3(d-2)$的平面曲线的交,它至少在$S$的其上纤维维数为零的开子集上是有限、平坦的,且次数为$3d(d-2)$(从Proposition \ref{pro:2.32} 完全交的族是平坦的)。
同时,从 Exercise \ref{exe:4.1},特征为零的代数闭域上的曲线$C$的非奇异点处于$\mathscr F$中当且仅当它在经典意义下是一个拐点。

但是这里要提醒一下,我们的定义对奇异曲线$C\subset \mathbb P_K^2$与经典情况并不一致:在我们的定义中,$C$的奇异点永远在$\mathscr F$的底空间中。(如我们将看到的,若一族曲线的拐点在整个空间中是闭的,则必须如此。)至于经典公式,我们将在下面看到如何从我们的定义中推导出它们。

我们可以再进一步,将$\mathscr F$在点$p$处的概形结构与$C$在点$p$处的几何联系起来。

\begin{exe}\label{exe:4.2}
	令$C\subset \mathbb P_K^2$如Exercise \ref{exe:4.1}中所述,以及$p\in C$为$C$的一个非奇异点。证明$C$在点$p$的射影切线$\mathbb T_pC\subset \mathbb P_K^2$与$C$在点$p$有相交重数$m\geq 3$,当且仅当交$C\cap C'$在点$p$处的分支$\Gamma_p$同构于
	\[
		\Gamma_p\cong \spec K[x]/(x^{m-2}).
	\]
\end{exe}

如这个习题所建议的,我们定义一个拐点$p\in C_{\text{smooth}}$的\textit{重数}为$\mathbb T_p C$与$C$在点$p$相交的次数减二。我们可以用B\'ezout定理来推出,一个代数闭域$K$上的次数$d>1$的非奇异平面曲线(记上重数)正好有$3d(d-2)$个拐点,但还有一个问题:我们要知道$\mathscr F$是$C$的一个逆紧子概形,即,不是$C$的每一点都是拐点!虽然这看似显然,但实际上它在正特征时是错误的。

% p.155

\begin{exe}\label{exe:4.3}
	令$K$是一个特征为$p$的域,$C\subset \mathbb P_K^2$是一个由多项式$X^p Y+XY^p-Z^{p+1}$给出的平面曲线。证明$C$是非奇异的,但$C$的每一点都是拐点。
\end{exe}

但在特征为零时,我们的直观是正确的:

\begin{thm}\label{thm:4.4}
	令$C\subset \mathbb P_K^2$是任意特征为零的代数闭域$K$上的$d>1$次非奇异曲线,则不是$C$的每一点都是拐点(于是,特别地,记上重数$C$正好有$3d(d-2)$个拐点)。
\end{thm}

\begin{proof}
	比如见 Hartshorne [1977, Chapter IV, Exercise 2.3e] 或者 Griffiths \& Harris [1978, Chapter 2,Section 4].
\end{proof}

重数$m>1$的拐点当然可以出现在非奇异曲线上。这自然导出了一个问题,在任意曲线上,什么时候所有拐点都是单的(即重数为一)。实际上,情况如下:

\begin{exe}\label{exe:4.5}
	令$K$为一个代数闭域。固定一个整数$d>2$,$B=\mathbb P_K^N$为参数化$d$-次平面曲线$C\subset \mathbb P_K^2$的射影空间。证明,对一个一般的点$[C]\in B$,即对$B$中一个稠密开集中的所有点$[C]$,对应曲线$C\subset \mathbb P_K^2$的拐点都是单的。

	\textit{提示:}考虑万有曲线$\mathscr C\subset \mathbb P_B^2$(在第 \ref{s:3.2.8} 中定义)的拐点概形$\mathscr F$. 证明$\mathscr C$是不可约的,然后推出,我们只需对只有一个单拐点的平面曲线$C\subset \mathbb P_K^2$展示就足够了。\nottran
\end{exe}

\begin{exe}\label{exe:4.6}
	假设我们想要移去 Theorem \ref{thm:4.4} 中关于$K$是代数闭的假设。我们要怎么去定义一个拐点$p\in C$的重数,这里,$p$处的剩余类域$L$是$K$的有限扩张,以便保证我们还能得到$X$有$3d(d-2)$个拐点?
\end{exe}

\subsection{奇异曲线上的拐点}\label{s:4.1.2}

当我们考虑奇异曲线时,有趣的新问题出现了。首先,每个奇异点都是一个拐点:
\begin{exe}\label{exe:4.7}
	令$C\subset \mathbb P_K^2$为一个平面曲线。证明$C$的每个奇异点都是拐点。

	\textit{提示:}可以找一条直线穿过$C$的奇异点$p$使得相交重数大于等于$3$,这可以通过看$C$在点$p$的切锥(即,在点$p$附近展开$f$,然后取二次项的零点集的一个分支);或者用 Exerciese \ref{exe:4.2} 证明Hessian在点$p$为零,注意到$X$乘以Hessian行列式的首列加上$Y$乘以第二列加上$Z$乘以第三列,在点$p$处为零。
\end{exe}

% p.156

关于奇异曲线的拐点,有两类问题。其一,我们可以考虑具有孤立奇点且没有直线分支的曲线$C$,这样Hessian $C'$ 仍将交$C$于一个零维概形$\Gamma$中,故此$C$将具有有限拐点;我们求支于$C$的非奇异点的拐点数。为找到这个数,我们仅需找到概形$\Gamma$其底空间包含于奇异分支$C_{\text{sing}}$的那部分的次数,然后用$3d(d-2)$减去它。在特定情况下,它有一个很好的答案,其中两种情况表于下面的练习中。

\begin{exe}\label{exe:4.8}
	令$C\subset \mathbb P_K^2$为约态不可约曲线,Hessian为$C'$. 往书后翻见Definition \ref{defi:5.31},令$p\in C$为$C$的任意寻常结点(这里“寻常”指没有$C$的分支在点$p$交于其射影切线三次或以上)。
	证明,交$C\cap C'$支于点$p$的分支$\Gamma_p$在点$p$的剩余类域$\kappa(p)$上次数为$6$. 同样,证明,在$C$的尖点处的分支的次数为$8$. 那$C$的一个寻常切点处的次数呢?(结点、尖点、切点的严格定义见Definition \ref{defi:5.31})。
\end{exe}

因此,在代数闭域上,一条不含任意直线的$d$-次平面曲线,若奇异点为$\delta$寻常结点与$\kappa$个尖点,则非奇异拐点数为
\[
	3d(d-2)-6\delta -8\kappa.
\]
这是经典的平面曲线的\textit{Pl\"ucker公式}的一个例子。

\begin{exe}\label{exe:4.9}
	验证,如果$C$是可约的(再次假设$C$没有分支是直线),我们可以通过单独地考虑$C$的分支得到相同的答案。
\end{exe}

\subsection{具有多重分支的曲线}\label{s:4.1.3}

当我们考虑具有多重分支的曲线,比如由一个多项式$G(X,Y,Z)$的幂次$F=G^m$定义的曲线时,一类非常不同的问题出现了。当然,对这样的曲线$C$,拐点概形$\mathscr F_C$维数为正,故往往并不有趣。相反,当我们考虑特化\footnote{译者注:原文是``Rather, the interesting questions arise when we consider families of curves specializing to such a multiple curve.'',这里specializing我们译作特化,其大概可以理解为包含于族中,或是极限为这个曲线,或其他类似含义。具体情况见下面例子。}这样一个多重曲线的曲线族时,有趣的问题出现了。我们问:在这样一个曲线族中,拐点会跑到哪里去?

为给出这样的问题的一个例子,我们下面考虑一个非奇异四次平面曲线,它在一个线性族中退化到一个双重的二次曲线。令$K$是一个特征为零的代数闭域,然后考虑概形$B=\mathbb A_K^1=\spec K[t]$上的曲线$\mathscr C$. 假设$U=U(X,Y,Z)$是一个不可约四次多项式,而$G=G(X,Y,Z)$是任意使得曲线$V(U)$和$V(G)\subset \mathbb P_K^2$横截相交的多项式。
% p.157
考虑四次平面曲线族$\pi:\mathscr C\to \mathbb A_K^1$,其由方程$F=U^2+tG=0$给出,故$\pi$即概形
\[
	\begin{aligned}
	\pi:\mathscr C&=\proj K[t][X,Y,Z]/\bigl(U(X,Y,Z)^2+tG(X,Y,Z)\bigr)\\
	&\subset \proj K[t][X,Y,Z]=\mathbb P_B^2\longrightarrow B.
	\end{aligned}
\]

令$\mathscr F$为曲线$\mathscr C\subset \mathbb P_B^2$的拐点概形。令$\mathscr F^*\subset \mathbb P_{B^*}^2$为挖洞直线$B^*=\spec K[t,t^{-1}]\subset B$在$\mathscr F$中的原像,$\mathscr F'$为$\mathscr F^*$在$\mathbb P_B^2$中的闭包。概形$\mathscr F^*$在$B^*$上平坦且有限,且易刻画如下:若$C_\mu\subset \mathbb P_K^2$为$\mathscr C$在点$(t-\mu)\in B=\mathbb A_K^1$处的纤维,则对$\mu\neq 0$,$\mathscr F$在点$(t-\mu)$处的纤维$F_\mu$将是$C_\mu$的$3d(d-2)=24$个拐点。换言之,原点$(t)\in B=\mathbb A_K^1$外,曲线$C_\mu$的拐点概形构成了一个平坦族。

令$\mathscr F'$为$\mathscr F^*$在$\mathbb P_B^2$中的闭包,$F'_0$为$\mathscr F'$在原点处的纤维。因为$B$是一维非奇异的,$\mathscr F'$在$B$上处处平坦,于是特别地,$F'_0\subset C_0\subset \mathbb P_K^2$在$K$上维数为零,次数为$24$. 我们可以将$F'_0$想做附近非奇异曲线$C_\mu$的$24$个拐点随着$\mu$趋于零的“极限位置”。因此,\naive 问题“当一个四次曲线退化到一个双重二次曲线时,拐点如何变化?”就转为下面这个精确的问题:确定平坦极限$F'_0$,特别地,其底空间。

这个问题的难点在于概形$F'_0$并不是$\mathscr F$在原点的纤维。反之,$\mathscr F$有两个分支:其一,$\mathscr F^*$的闭包$\mathscr F'$,其包含“真正的”拐点和它们的极限,另一个,其底空间为二次曲线$V(t,U)$,在$\mathbb P_B^2$的特殊纤维$\pi^{-1}((t))=\mathbb P_K^2$中。于是,我们并不能希望通过仅观察曲线$C_0$来得到简单地回答这个问题的任意线索(实际上,因为$\mathscr P_K^2$将$C_0$变到本身的自同构群在二次曲线$(C_0)_{\text{red}}$的闭点上的作用是可迁的,可见答案必然依赖于族$\mathscr C$)。

为回答这个问题,我们首先写下概形$\mathscr F$的理想$I$(在一个放射开集$\spec K[t][x,y]\cong \mathbb A_B^2\subset \mathbb P_B^2$),然后$\mathscr F^*$的理想$I^*=I\cdot K[t,t^{-1}][x,y]$,然后闭包$\mathscr F'$的理想$I'=I^*\cap K[t][x,y]$,最后是$\mathscr F'$在原点$(t)\in B$的纤维$\mathscr F'_0$理想$I'_0=(I',t)$. 为了说明如何完成这些计算,我们将详细执行这些步骤。(可能直到你有类似的问题要解决的时候,你可能才希望看到这些细节!)

首先,如果$u(x,y)=U(X,Y,1)$以及$g(x,y)=G(X,Y,1)$分别为$U$和$G$在仿射开集$\spec K[t][x,y]\cong \mathbb A_B^2\subset \mathbb P_B^2$的非齐次型,理想$I$从定义由两个元素所生成,方程$u^2+tg$以及仿射Hessian
\[
	\begin{vmatrix}
		u^2+tg & 2uu_x+tg_x & 2uu_y+tg_y\\
		2uu_x+tg_x&2uu_{xx}+2u_x^2+tg_{xx}&2uu_{xy}+2u_xu_y+tg_{xy}\\
		2uu_y+tg_y&2uu_{xy}+2u_xu_y+tg_{xy}&2uu_{yy}+2u_y^2+tg_{yy}
	\end{vmatrix}
\]
% p.158
因此,$I=(u^2+tg,H)$,其中
\[
	H=
	\begin{vmatrix}
		0 & 2uu_x+tg_x & 2uu_y+tg_y\\
		2uu_x+tg_x&2uu_{xx}+2u_x^2+tg_{xx}&2uu_{xy}+2u_xu_y+tg_{xy}\\
		2uu_y+tg_y&2uu_{xy}+2u_xu_y+tg_{xy}&2uu_{yy}+2u_y^2+tg_{yy}
	\end{vmatrix}.
\]

我们可以展开$H$,以$t$的幂次整理得到:
\[
\begin{aligned}
	H=\ &8u^2\bigl(-u_x^2(uu_{yy}+u_y^2)+2u_xu_y(uu_{xy}+u_xu_y)-u_y^2(uu_x+u_x^2))\\
	&+8tu\bigl(
		-g_xu_x(uu_{yy}+u_y^2)+g_xu_y(uu_{xy}+u_xu_y)\\
	&\qquad\quad +g_yu_x(uu_{xy}+u_xu_y)-g_yu_y(uu_{xx}+u_x^2)\bigr)\\
	&+4tu^2\bigl(-u_x^2g_{yy}+u_xu_yg_{xy}-u_y^2g_{xx}\bigr)\\
	&+2t^2\bigl(-g_x^2(uu_{yy}+u_y^2)+2g_xg_y(uu_{xy}+u_xu_y)-g_y^2(uu_{xx}+u_x^2)\bigr)\\
	&+4t^2u(-g_xu_xg_{yy}+g_xu_yg_{xy}+g_yu_xg_{xy}-g_yu_yg_{xx})\\
	&+t^3\bigl(-g_x^2g_{yy}+g_xg_yg_{xy}-g_y^2g_{xx}\bigr).
\end{aligned}
\]

右侧的前两项可以化简,得到
\[
\begin{aligned}
	H=\ &8u^3\bigl(-u_x^2 u_{yy}+u_xu_y u_{xy}-u_y^2 u_x)\\
	&+8tu^2 (
		-g_xu_x u_{yy}+g_xu_y u_{xy}
		+g_yu_x u_{xy}-g_yu_y u_{xx})\\
	&+4tu^2\bigl(-u_x^2g_{yy}+u_xu_yg_{xy}-u_y^2g_{xx}\bigr)\\
	&+2t^2\bigl(-g_x^2(uu_{yy}+u_y^2)+2g_xg_y(uu_{xy}+u_xu_y)-g_y^2(uu_{xx}+u_x^2)\bigr)\\
	&+4t^2u(-g_xu_xg_{yy}+g_xu_yg_{xy}+g_yu_xg_{xy}-g_yu_yg_{xx})\\
	&+t^3\bigl(-g_x^2g_{yy}+g_xg_yg_{xy}-g_y^2g_{xx}\bigr).
\end{aligned}
\]

现在,模掉$I$的另一个生成元$u^2+tg$,我们可以用$-tg$替换表达式中的$u^2$得到一个可被$t$整除的多项式。因此,$I^*=I(\mathscr F^*)=I\cdot K[t,t^{-1}][x,y]\subset K[t,t^{-1}][x,y]$,于是$I'=I(\mathscr F')=I^*\cap K[t][x,y]\subset K[t][x,y]$,也包含元素
\[
	\begin{aligned}
		H'=\ &-8ug\bigl(-u_x^2 u_{yy}+u_xu_y u_{xy}-u_y^2 u_x)\\
		&-8tg (
			-g_xu_x u_{yy}+g_xu_y u_{xy}
			+g_yu_x u_{xy}-g_yu_y u_{xx})\\
		&-4tg\bigl(-u_x^2g_{yy}+u_xu_yg_{xy}-u_y^2g_{xx}\bigr)\\
		&+2t\bigl(-g_x^2(uu_{yy}+u_y^2)+2g_xg_y(uu_{xy}+u_xu_y)-g_y^2(uu_{xx}+u_x^2)\bigr)\\
		&+4tu(-g_xu_xg_{yy}+g_xu_yg_{xy}+g_yu_xg_{xy}-g_yu_yg_{xx})\\
		&+t^2\bigl(-g_x^2g_{yy}+g_xg_yg_{xy}-g_y^2g_{xx}\bigr).
	\end{aligned}
\]

此外,如果我们将这个$I^*$的生成元乘以$u$,然后再将$u^2$换成$-tg$,我们再一次得到了一个可以由$t$整除的多项式,于是理想$I^*$和$I'$也包含
% p.159
\[
	\begin{aligned}
		J=\ &8g^2\bigl(-u_x^2 u_{yy}+u_xu_y u_{xy}-u_y^2 u_x)\\
		&-8gu (
			-g_xu_x u_{yy}+g_xu_y u_{xy}
			+g_yu_x u_{xy}-g_yu_y u_{xx})\\
		&-4gu\bigl(-u_x^2g_{yy}+u_xu_yg_{xy}-u_y^2g_{xx}\bigr)\\
		&+2u\bigl(-g_x^2(uu_{yy}+u_y^2)+2g_xg_y(uu_{xy}+u_xu_y)-g_y^2(uu_{xx}+u_x^2)\bigr)\\
		&-4tg(-g_xu_xg_{yy}+g_xu_yg_{xy}+g_yu_xg_{xy}-g_yu_yg_{xx})\\
		&+tu\bigl(-g_x^2g_{yy}+g_xg_yg_{xy}-g_y^2g_{xx}\bigr).
	\end{aligned}
\]

为继续这个分析,我们需用如下事实,对任意齐次四次多项式$U(X,Y,Z)$,Hessian
\[\CenteredArraystretch{1.7}
	\begin{vmatrix}
		\pfrac{^2U}{X^2}&\ppfrac UXY &\ppfrac UXZ\\
		\ppfrac UYX&\pfrac{^2U}{Y^2}&\ppfrac UYZ\\
		\ppfrac UZX&\ppfrac UZY&\pfrac{^2U}{Z^2}
	\end{vmatrix}.
\]
是一个标量$\mu=\mu(U)$,当$U$是不可约时非零(即,当$V(U)\subset \mathbb P^2$非奇异),此外为零。于是
\[\CenteredArraystretch{1.7}
	\begin{vmatrix}
		u&\pfrac ux &\pfrac uy\\
		\pfrac ux&\pfrac{^2u}{x^2}&\ppfrac uxy\\
		\pfrac uy&\ppfrac uxy&\pfrac{^2u}{y^2}
	\end{vmatrix}
	=-u_x^2u_{yy}+u_xu_yu_{xy}-u_y^2u_{xx}=\lambda u+\mu,
\]
其中$\lambda$是标量。在$J$中减去这项,我们有
\[
	\begin{aligned}
		J=\ &8\mu g^2\\
		&+8\lambda g^2 u\\
		&-8gu (
			-g_xu_x u_{yy}+g_xu_y u_{xy}
			+g_yu_x u_{xy}-g_yu_y u_{xx})\\
		&-4gu\bigl(-u_x^2g_{yy}+u_xu_yg_{xy}-u_y^2g_{xx}\bigr)\\
		&+2u\bigl(-g_x^2(uu_{yy}+u_y^2)+2g_xg_y(uu_{xy}+u_xu_y)-g_y^2(uu_{xx}+u_x^2)\bigr)\\
		&-4tg(-g_xu_xg_{yy}+g_xu_yg_{xy}+g_yu_xg_{xy}-g_yu_yg_{xx})\\
		&+tu\bigl(-g_x^2g_{yy}+g_xg_yg_{xy}-g_y^2g_{xx}\bigr).
	\end{aligned}
\]

现在,我们已经看到理想$I'\supset (u^2+tg,H',J)$. 限制到$B$的原点处的纤维,即置$t=0$,我们看到$\mathscr F'$的纤维$\mathscr F'_0$理想$I'_0=(I',t)$包含
\[
	\begin{aligned}
		u^2+tg \equiv u^2 &\quad \text{ mod } (t),\\
		H' \equiv ug &\quad \text{ mod } (t,u^2),
	\end{aligned}
\]
% p.160
以及
\[
	J\equiv 8\mu g^2-2\nu u\quad \text{ mod } (t,u^2,ug),
\]
其中
\[
	\nu=g_x^2(uu_{yy}+u_y^2)-2g_xg_y(uu_{xy}+u_xu_y)+g_y^2(uu_{xx}+u_x^2).
\]

从这里,我们看到$I'_0\supset (t,u^2,ug,8\mu g^2+2\nu u)$. 现在,我们可以记
\[
	\begin{aligned}
		\nu &=g_x^2(uu_{yy}+u_y^2)-2g_xg_y(uu_{xy}+u_xu_y)+g_y^2(uu_{xx}+u_x^2)\\
		&\equiv (g_xu_y-g_yu_x)^2\quad \text{ mod } (u)\\
		&\equiv \begin{vmatrix}
			g_x&g_y\\
			u_x&u_y
		\end{vmatrix}^2.
	\end{aligned}
\]

特别地,由于$V(U)$和$V(G)$横截相交,$\nu$不能在$u=g=0$的点处为零。我们可因此识出理想$(t,u^2,ug,8\mu g^2+2\nu u)$是特殊纤维$C_0$的一个子概形的理想,底空间为二次曲线$U=0$和四次曲线$G=0$在平面$t=0$处的八个点,且在每一点处次数为$3$. 因为$8\times 3=24$,纤维$F'_0$不能比这还要小,所以必然取到等号,即
\[
	I'_0=(t,u^2,ug,8\mu g^2+2\nu u).
\]
换而言之:

\begin{pro}\label{pro:4.10}
	概形$F'_0$的底空间是二次曲线$V(U)$和四次曲线$V(G)$在平面$V(t)$上由$t=U=G=0$给出的八个点。在每个点处,它包含一个次数为$3$的曲线概形,与二次曲线$V(U)$相切,但不包含于$V(U)$中。
\end{pro}

此答案之一端为,约态曲线$(C_0)_{\text{red}}$的任何闭点可以是一族合适的非奇异曲线$C_\mu$的拐点随着趋于$C_0$的极限。这是一个一般的现象,实际上,曲线的多重分支的每个点都是附近的非奇异曲线的拐点的极限。

该示例中描述的现象相当普遍。下面的习题给出了两个推广。

\begin{exe}\label{exe:4.11}
	\nottran
\end{exe}

\begin{exe}\label{exe:4.12}
	\nottran
\end{exe}

\nottran