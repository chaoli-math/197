在这章中,我们将展示如何在概形理论中给出一些经典代数几何中的几何构造。
我们将在所有例子中看到,这种崭新的语言是如何允许我们扩大定义的范围
(以及我们对该对象可能问的问题),如何让我们能够给出经典问题的精确表述,
以及在一些情况中,如何帮助我们解决他们。

\section{平面曲线的拐点}\label{s:4.1}

在这节中,我们将描述代数闭域$K$上的非奇异平面曲线
$C\subset \mathbb P_K^2$的\textit{拐点}的经典定义。然后指出
这个定义是如何推广到概形语境中的,并展示这个推广是如何阐明拐点
的集合的,即使是在经典情形中。

\subsection{定义}\label{s:4.1.1}

我们需要一个预备定义。令$K$是任意域,再令$C$, $D\subset C\subset \mathbb P_K^2$是没有公共分支的两条平面曲线,$p\in C\cap D$是一个交点。
我们定义$C$和$D$在点$p$的\textit{相交重数}为$p$所处的概形$C\cap D$
的分支$\Gamma$的重数,记作$\mu_p(C\cdot D)$. 因为平面曲线是
Cohen-Macaulay的,这和在第 \ref{s:3.3.5} 节中引入的相交重数的概念
是相合的。%
% p.152
同样注意次数的概念的联系:$\Gamma$作为$\mathbb P_K^2$的子概形的次数
是相交重数$\mu_p(C\cdot D)$乘以剩余类域作为$K$的扩张的
次数$(\kappa(p):K)$. 因此,比如平面曲线的
B\'ezout定理(\ref{thm:3.71})告诉我们
\[
	\deg C\deg D=\sum_{p\in C\cap D}
	(\kappa(p):K)\mu_p(C\cdot D).
\]

话说回来,经典代数几何中平面曲线的拐点是直接且几何的:如果
$C\subset \mathbb P_{\mathbb C}^2$是一个复数域上的
次数为$d$的非奇异平面曲线,点$p\in C$被称为一个\textit{拐点},
如果射影切线$\mathbb T_pC\subset \mathbb P_{\mathbb C}^2$
在点$p$处交$C$三次或以上,或者用现代的语言,如果
$\mathbb T_pC$和$C$在点$p$的相交重数$\mu_p(C\cdot \mathbb T_pC)$
至少为三。(这里,因为我们在代数闭域上工作,相交重数等于
$p$所处的$\mathbb T_pC\cap C$的分支的次数,
即$\dim_{\mathbb C}(\mathscr O_{\mathbb T_pC\cap C,p})$.)
有一个经典的定理(我们将在后面建立之),如果$C$不是一条直线,
则$C$的拐点有限,并且,如果辅以合适的重数来数,则拐点数应为
$3d(d-2)$.

这个简单的定义可被推广到奇异曲线,比如可见Coolidge [1931],
尽管这个定义和现代标准定义比并不总是精确的。这个定义在
我们考虑非代数闭域或有限特征的域$K$
上的曲线$C\subset \mathbb P_K^2$时会出问题,此外还有考虑
包含直线或者一个多次分支的曲线时。

我们将在这里做的是给任意概形$S$上的平面曲线$C\subset \mathbb P_S^2$
一个统一的拐点的定义。首先回忆第 \ref{s:3.2.8} 节,概形$S$上的
$d$次\textit{平面曲线}是指子概形$C\subset \mathbb P_S^2$,其局部地
在$S$上,是一个$d$-次齐次多项式
\[
	F(X,Y,Z)=\sum_{i+j+k=d}a_{ijk}X^iY^jZ^k
\]
的零点集$V(F)$,其中系数$a_{ijk}$是$S$上的正则函数,它们并不同时为零。
再回忆,如果$S$是一个仿射概形,则我们可以扔掉“局部”,即,如果$S=\spec A$,
一个$S$上的平面曲线$C$具有形式
\[
	C=\proj A[X,Y,Z]/(F)
\]
其中$F$是一个多项式。

现在,给定$S$上的一条平面曲线$C\subset \mathbb P_S^2$,我们将定义一个
闭子概形$\mathscr F=\mathscr F_C\subset C$,我们将称之为$C$上的
\textit{拐点概形}。他将与基变换$S'\to S$可交换(即,如果
$C'=S'\times_S C\subset \mathbb P_{S'}^2$,则$\mathscr F_{C'}=(\pi_2)^{-1}(\mathscr F_C)$)以及$\mathscr F$将在至少一个$S$的开集上是有限且平坦的,次数为$3d(d-2)$,其中$\mathscr F$在这个开集上的相对维数为$0$. 
这个闭子概形的重要性在于,如果我们有一族平面曲线,则一般纤维的拐点的极限
是一个特殊纤维的拐点(即,$\mathscr F$是闭的),% p.153
且当$\mathscr F$相对维数为$0$($\mathscr F$是平坦的)时候反过来也成立。
此外,在经典语境中,即如果$C$是一个特征为零的代数闭域的谱$S=\spec K$
上的非奇异曲面曲线,$\mathscr F$的底空间将是$C$经典定义的该店的集合
(因此,在一般的情况中,如果$s\in S$是任意的剩余类域$\kappa(s)$为代数闭
且特征零的点,则$\mathscr F$在点$s$的纤维$\mathscr F_s$的底空间将是$C_s$
的拐点的集合)。

为启发我们一般的定义,回忆经典代数几何中的一个最早的结论:对一个
代数闭域$K$上非奇异平面曲线$C=V(F)\subset \mathbb P_K^2$,
由多项式$F(X,Y,Z)$的零点集给出,则$C$的拐点为其与它的\textit{Hessian}
的交点,Hessian由多项式
\[\CenteredArraystretch{1.7}
	H(X,Y,Z)=
	\begin{vmatrix}
		\pfrac{^2F}{X^2}&\ppfrac FXY &\ppfrac FXZ\\
		\ppfrac FYX&\pfrac{^2F}{Y^2}&\ppfrac FYZ\\
		\ppfrac FZX&\ppfrac FZY&\pfrac{^2F}{Z^2}
	\end{vmatrix}
\]
的零点集定义。我们将这个事实的证明留作一个习题:
\begin{exe}\label{exe:5.1}
	令$K$是一个特征为零的代数闭域,$C\subset \mathbb P_K^2$是一个平面曲线,
	$p\in C$是$C$的一个非奇异点。证明,射影切线$\mathbb T_pC$
	与$C$在点$p$交于三次或以上当且仅当$H(p)=0$.

	\textit{提示}:在$\mathbb P_K^2$的相应子集上引入仿射坐标
	\[
	x=\frac XZ,\quad y=\frac YZ,
	\]
	然后利用Euler关系来看到Hessian行列式的非齐次化$h(x,y)=H(x,y,1)$
	(在相差一个相乘因子下)为
	\[\CenteredArraystretch{1.7}
	h(x,y)=
	\begin{vmatrix}
		f&\pfrac fx &\pfrac fy\\
		\pfrac fx&\pfrac{^2f}{x^2}&\ppfrac fxy\\
		\pfrac fy&\ppfrac fxy&\pfrac{^2f}{y^2}
	\end{vmatrix},
	\]
	其中$f(x,y)=F(x,y,1)$为$F$的非齐次化。
\end{exe}

为定义一般情况下任意平面曲线$C\subset \mathbb P_S^2$的拐点概形,
我们可以简单地推广Hessian \nottran:假设在一些仿射开集
$U=\spec R\subset S$中,曲线
\[
	C\cap \mathbb P_U^2=\proj R[X,Y,Z]
\]%
% p.154
是多项式$F\in R[X,Y,Z]$零点集。我们定义\textit{Hessian行列式}为多项式
\[\CenteredArraystretch{1.7}
	H(X,Y,Z)=
	\begin{vmatrix}
		\pfrac{^2F}{X^2}&\ppfrac FXY &\ppfrac FXZ\\
		\ppfrac FYX&\pfrac{^2F}{Y^2}&\ppfrac FYZ\\
		\ppfrac FZX&\ppfrac FZY&\pfrac{^2F}{Z^2}
	\end{vmatrix}.
\]

\nottran