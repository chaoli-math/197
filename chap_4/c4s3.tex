\section{Fano概形}\label{s:4.3}
\subsection{定义}\label{s:4.3.1}

在经典几何中,一种研究射影簇$X\subset \bbp_K^n$的方式是通过其与$\bbp_K^n$的线性子空间之间的关系。于是,对这样一个簇,就给出了一系列Grassmannian $\mathbb G_K(k,n)$的子簇。比如说,我们可以对$X\subset \bbp_K^n$给出$\mathbb G_K(k,n)$中的一族与$X$相交的线性空间;或者是$X$的切空间们;或者是$X$的割线们;或者$X$包含的线性子空间。所有这些子簇现在可以重新定义为对应$X\subset \bbp_S^n$的一个子概形的$\mathbb G_S(k,n)$的子概形,同样地,它们有着更丰富的结构,反应了$X$的几何。但即使当我们从一个代数闭域$K$上的簇$X\subset \bbp_K^n$出发,此时对应于此的概形也可能是非约态的。

在这节中,我们将定义和研究概形$F_k(X)\subset \mathbb G_S(k,n)$,其参数化了包含在一个概形$X\subset \mathbb P_S^n$中的$k$-维线性空间,它被称为$X$的第$k$-个\textit{Fano概形}。我们将特别地尝试展示一个非约态概形结构如何和何时出现,以及其如何允许我们推广一些关于Fano簇的经典定理。比方说,我们将看到,如果$K$是任意域而$X\subset \mathbb P_K^3$是任意的并不为直线们扫过的三次曲面,则$X$上的直线的Fano概形在$K$上的次数则正好为$27$,
尽管,包含在$X$中的直线的集合仅当$X$非奇异时的大小是$27$,甚至当$K$
不是代数闭的时候,非奇异也无法保证。更一般地,我们将看到,在许多情况中,
簇的平坦族$\mathscr X\subset \P_B^n\to B$对应的Fano概形族本身在$B$上是
平坦的,因此我们可以更一般地去给出一些关于数量和次数的论断。

在这届中,我们将通过给出定义理想来定义Fano概形,相同的理想也给出了
Fano簇的经典定义,这里唯一的不同之处在于我们不用再取根式了,这会
扔掉一些信息。然而,我们在第 \ref{chap:6} 章中将看到,通过点函子,可以
更内蕴地定义Fano概形$F_k(X)$,且这个定义将反过来给我们一个它们的几何的
诸多方面的刻画(如切空间),这将更加直接地联系到概形$X$的几何。这些
描述即使在$X$和$F_k(X)$是簇的情况下依然很有用。

% p.193

令$S$为任意概形,令$X\subset \P_S^n$为$S$上的射影空间的任意子概形,
令$k<n$为任意的正整数。$X$的Fano概形$F_k(X)\subset \mathbb G=
\mathbb G_S(k,n)$是参数化了$\P_S^n$中位于$X$上的$k$-维线性子空间的概形。
(一如既往,词“参数化”有一个很明确的意思,我们将在下面的
第 \ref{s:6.2.2} 节中进一步讨论它。)我们首先在$S=\spec R$为仿射概形
的时定义$F_k(X)$. 我们将用第 \ref{s:3.2.7} 节中将$\mathbb G$看成
仿射空间$W_I\cong \mathbb A_S^{(k+1)(n-k)}$的并的刻画来描述它。

回忆,在这个构造中,我命令
\[
	W=\spec R[\dots,x_{i,j},\dots]\cong \bba_S^{(k+1)(n+1)}
\]
(我们将其看作关联于$(k+1)\times (n+1)$矩阵的矢量空间的仿射空间),
对每个多重指标$I=(i_0,\dots,i_k)\subset \{0,1,\dots,n\}$,令
$W_I\cong \mathbb A_S^{(k+1)(n-k)}\subset W$为由理想
$(\dots,x_{i_\alpha,j_\beta}-\delta_{\alpha,\beta},\dots)$给出的
闭子概形(我们可以将其想做关联于第$I$个子矩阵为恒等矩阵的矩阵的
子空间的仿射空间)。现在,假设$G(Z_0,\dots,Z_n)\in I(X)$为任意$X$
理想中的齐次多项式。将其应用到一个$(k+1)\times (n+1)$矩阵的行的
一个一般的线性组合上,我们得到了一个多项式
\[
	H_G(u,x)=G\bigl(\sum u_ix_{0,i},\sum u_ix_{1,i},\dots,\sum u_ix_{k,i}\bigr),
\]
其我们可以将其写作一个变量为$u_0$, $\dots$, $u_k$的
单项式$u^J=u_0^{j_0}u_1^{j_1}\cdots u_k^{j_k}$的线性组合:
\[
	H_G(u,x)=\sum H_{G,J}(x)\cdot u^J.
\]
系数多项式$H_{G,J}(x)$是变量$x_{i,j}$的多项式,限制到子概形
$W_I\cong \A_S^{(k+1)(n-k)}\subset W$上,它们也是那里的正则函数。
我们定义\emph{Fano概形}$F_k(X)$为$\mathbb G$的子概形,其在每个开集
$W_I$上,由多项式$H_{G,J}(x)$生成的理想给出,其中$G$跑遍所有理想
$I(X)\subset R[Z_0,\dots,Z_n]$的元素,而$J$标记了变量为$u_0$, $\dots$, 
$u_k$的、次数为$d$的多项式。

或者,对$R$中元素的任意$(k+1)$-元组$c=(c_0,\dots,c_k)$,我们可以定义一个
多项式
\[
	H_{G,c}(x)=G\bigl(\sum c_ix_{0,i},\sum c_ix_{1,i},\dots,\sum c_ix_{k,i}\bigr)
\]
然后取Fano概形$F_k(X)$为$\mathbb G$的子概形,其在$W_I$中由
多项式$H_{G,c}(x)$生成的理想给出,其中$G$跑遍$I(X)\subset R[Z_0,\dots,Z_n]$,
而$c$跑遍$R^{n+1}$.

% p.194

\subsection{二次曲面上的直线}\label{s:4.3.2}

\paragraph*{代数闭域上的光滑二次曲面上的直线}\addcontentsline{toc}{subsubsection}{代数闭域上的光滑二次曲面上的直线}

\[
	\begin{aligned}
		H_G(u_0,u_1)&=G(u_0,u_1,u_0a+u_1c,u_0b+u_1d)\\
		&=u_0^2+u_1^2+(u_0a+u_1c)^2+(u_0b+u_1d)^2\\
		&=(1+a^2+b^2)u_0^2+2(ac+bd)u_0u_1+(1+c^2+d^2)u_1^2.
	\end{aligned}
\]

\[
	F_1(Q)\cap W_{X,Y}=V(1+a^2+b^2,ac+bd,1+c^2+d^2)\subset \spec K[a,b,c,d].
\]

% p.195

\[
	\begin{aligned}
		a&=-\Pi_{YZ}/\Pi_{XY},&b&=-\Pi_{YW}/\Pi_{XY},\\
		c&=\Pi_{XZ}/\Pi_{XY},&d&=\Pi_{XW}/\Pi_{XY}.
	\end{aligned}
\]

\[
	ad-bc=\Pi_{ZW}/\Pi_{XY},
\]

\[
	\mathbb G=V(\Pi_{ZW}\Pi_{XY}+\Pi_{YZ}\Pi_{XW}-\Pi_{XZ}\Pi_{YW}).
\]

\[
	V(\Pi_{XY}^2+\Pi_{YZ}^2+\Pi_{YW}^2,\Pi_{YZ}\Pi_{XZ}+\Pi_{YW}\Pi_{XW},\Pi_{XY}^2+\Pi_{XZ}^2+\Pi_{XW}^2)
\]

\[
\begin{array}{l}{V\left(\Pi_{Y Z}^{2}-\Pi_{X W}^{2},\left(\Pi_{Y Z}+\Pi_{X W}\right)\left(\Pi_{Y W}+\Pi_{X Z}\right),\left(\Pi_{Y Z}+\Pi_{X W}\right)\left(\Pi_{Z W}-\Pi_{X Y}\right)\right.} \\ {\Pi_{Y W}^{2}-\Pi_{X Z}^{2},\left(\Pi_{Y W}-\Pi_{X Z}\right)\left(\Pi_{Y Z}-\Pi_{X W}\right),\left(\Pi_{Y W}-\Pi_{X Z}\right)\left(\Pi_{Z W}-\Pi_{X Y}\right)} \\ {\Pi_{Z W}^{2}-\Pi_{X Y}^{2},\left(\Pi_{Z W}+\Pi_{X Y}\right)\left(\Pi_{Y Z}-\Pi_{X W}\right),\left(\Pi_{Z W}+\Pi_{X Y}\right)\left(\Pi_{Y W}+\Pi_{X Z}\right)} \\ {\Pi_{X Y}^{2}+\Pi_{Y Z}^{2}+\Pi_{Y W}^{2}, \Pi_{X Y}^{2}+\Pi_{X Z}^{2}+\Pi_{X W}^{2} )}\end{array}
\]

\[
	\Lambda_{1}=V\left(\Pi_{Y Z}+\Pi_{X W}, \Pi_{Y W}-\Pi_{X Z}, \Pi_{Z W}+\Pi_{X Y}\right)
\]

\[
	\Lambda_{2}=V\left(\Pi_{Y Z}-\Pi_{X W}, \Pi_{Y W}+\Pi_{X Z}, \Pi_{Z W}-\Pi_{X Y}\right)
\]

\[
	F_{1}(Q)=\left(\Lambda_{1} \cup \Lambda_{2}\right) \cap \mathbb{G} \subset \mathbb{P}^{5}
\]

% p.196

\[
	\mathscr{Q}=V\left(t X^{2}+Y^{2}+Z^{2}+W^{2}\right) \subset \operatorname{Proj} K[t][X, Y, Z, W]=\mathbb{P}_{B}^{3}
\]

\[
\begin{aligned} H_{G}(a, x) &=G\left(u_{0}, u_{1}, u_{0} a+u_{1} c, u_{0} b+u_{1} d\right) \\ &=t u_{0}^{2}+u_{1}^{2}+\left(u_{0} a+u_{1} c\right)^{2}+\left(u_{0} b+u_{1} d\right)^{2} \\ &=\left(t+a^{2}+b^{2}\right) u_{0}^{2}+2(a c+b d) u_{0} u_{1}+\left(1+c^{2}+d^{2}\right) u_{1}^{2} \end{aligned}
\]

\[
	F_{1}(\mathscr{Q}) \cap W_{X, Y}=V\left(t+a^{2}+b^{2}, a c+b d, 1+c^{2}+d^{2}\right) \subset \spec K[t][a, b, c, d]
\]

% p.197

\[
	F_{1}\left(Q_{0}\right) \cap W_{X, Y}=V\left(a^{2}+b^{2}, a c+b d, 1+c^{2}+d^{2}\right) \subset \mathbb{A}_{K}^{4}
\]

\[
	\Lambda_{1}=V\left(\Pi_{Y Z}+\sqrt{\mu} \Pi_{X W}, \Pi_{Y W}-\sqrt{\mu} \Pi_{X Z}, \Pi_{Z W}+\sqrt{\mu} \Pi_{X Y}\right)
\]

\[
	\Lambda_{2}=V\left(\Pi_{Y Z}-\sqrt{\mu} \Pi_{X W}, \Pi_{Y W}+\sqrt{\mu} \Pi_{X Z}, \Pi_{Z W}-\sqrt{\mu} \Pi_{X Y}\right)
\]

\[
	F_{1}\left(Q_{\mu}\right)=\left(\Lambda_{1}(\mu) \cup \Lambda_{2}(\mu)\right) \cap \mathbb{G}_{K}(1,3) \subset \mathbb{P}_{K}^{5}
\]

% p.198

\[
	\Lambda=V\left(\Pi_{Y Z}, \Pi_{Y W}, \Pi_{Z W}\right)
\]

\[
	\mathscr{L}^{*} \subset \mathbb{P}_{B^{*}}^{5}=\operatorname{Proj} K\left[t, t^{-1}\right]\left[\Pi_{X Y}, \Pi_{X Z}, \Pi_{X W}, \Pi_{Y Z}, \Pi_{Y W}, \Pi_{Z W}\right]
\]

% p.199

\[
	\mathscr{Q}=V\left(s X^{2}+t Y^{2}+Z^{2}+W^{2}\right) \subset \operatorname{Proj} K[s, t][X, Y, Z, W]
\]

\[
	F_{1}(\mathscr{Q}) \cap W_{X, Y}=V\left(t+a^{2}+b^{2}, a c+b d, s+c^{2}+d^{2}\right) \subset \operatorname{Spec} K[s, t][a, b, c, d]
\]

\[
	F_{1}\left(Q_{0,0}\right) \cap W_{X, Y}=V\left(a^{2}+b^{2}, a c+b d, c^{2}+d^{2}\right) \subset \mathbb{A}_{K}^{4}
\]

\[
\begin{array}{l}{\Gamma_{1}=V(a+\sqrt{-1} b, c-\sqrt{-1} d)} \\ {\Gamma_{2}=V(a-\sqrt{-1} b, c+\sqrt{-1} d)}\end{array}
\]

\[
	\mathscr{Q}_{\alpha, \beta}=V\left(\alpha u X^{2}+\beta u Y^{2}+Z^{2}+W^{2}\right) \subset \operatorname{Proj} K[u][X, Y, Z, W]
\]

\[
\begin{aligned} F_{1}\left(\mathscr{Q}_{\alpha, \beta}\right) \cap W_{X, Y} &=V\left(\alpha u+a^{2}+b^{2}, a c+b d, \beta u+c^{2}+d^{2}\right) \\ & \subset \operatorname{Spec} K[\alpha, \beta][a, b, c, d] \end{aligned}
\]

\[
	\Phi(\gamma)=V(a-\gamma d, b+\gamma c)
\]

% p.200

\[
	P_{1}=[\sqrt{-1} \sqrt{\beta},-\sqrt{\alpha}, 0,0]
\]

\[
	\mathscr{Q}=V\left(t X^{2}+t Y^{2}+t Z^{2}+W^{2}\right) \subset \mathbb{P}_{B}^{3}
\]

% p.201

\[
\begin{aligned} Q_{1} &=V\left(X^{2}+Y^{2}-Z^{2}-W^{2}\right) \\ Q_{2} &=V\left(X^{2}+Y^{2}+Z^{2}-W^{2}\right) \\ Q_{3} &=V\left(X^{2}+Y^{2}+Z^{2}+W^{2}\right) \end{aligned}
\]

\[
\begin{aligned} \mathscr{Q}_{1} &=V\left(7 X^{2}+7 Y^{2}+Z^{2}+W^{2}\right) \\ \mathscr{Q}_{2} &=V\left(7 X^{2}+14 Y^{2}+Z^{2}+W^{2}\right) \\ \mathscr{Q}_{3} &=V\left(7 X^{2}+49 Y^{2}+Z^{2}+W^{2}\right) \end{aligned}
\]
