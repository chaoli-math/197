\section{Fano概形}\label{s:4.3}
\subsection{定义}\label{s:4.3.1}

在经典几何中,一种研究射影簇$X\subset \bbp_K^n$的方式是通过其与$\bbp_K^n$的线性子空间之间的关系。于是,对这样一个簇,就给出了一系列Grassmannian $\mathbb G_K(k,n)$的子簇。比如说,我们可以对$X\subset \bbp_K^n$给出$\mathbb G_K(k,n)$中的一族与$X$相交的线性空间;或者是$X$的切空间们;或者是$X$的割线们;或者$X$包含的线性子空间。所有这些子簇现在可以重新定义为对应$X\subset \bbp_S^n$的一个子概形的$\mathbb G_S(k,n)$的子概形,同样地,它们有着更丰富的结构,反应了$X$的几何。即使当我们从一个代数闭域$K$上的簇$X\subset \bbp_K^n$出发,此时对应于此的概形也可能是非约态的。

\nottran

% p.193

\[
	W=\spec R[\dots,x_{i,j},\dots]\cong \bba_S^{(k+1)(n+1)}
\]

\[
	H_G(u,x)=G\bigl(\sum u_ix_{0,i},\sum u_ix_{1,i},\dots,\sum u_ix_{k,i}\bigr)
\]

\[
	H_G(u,x)=\sum H_{G,J}(x)\cdot u^J.
\]

\[
	H_{G,c}(x)=G\bigl(\sum c_ix_{0,i},\sum c_ix_{1,i},\dots,\sum c_ix_{k,i}\bigr)
\]

% p.194

\subsection{二次曲面上的直线}\label{s:4.3.2}

\paragraph*{代数闭域上的光滑二次曲面上的直线}\addcontentsline{toc}{subsubsection}{代数闭域上的光滑二次曲面上的直线}

\[
	\begin{aligned}
		H_G(u_0,u_1)&=G(u_0,u_1,u_0a+u_1c,u_0b+u_1d)\\
		&=u_0^2+u_1^2+(u_0a+u_1c)^2+(u_0b+u_1d)^2\\
		&=(1+a^2+b^2)u_0^2+2(ac+bd)u_0u_1+(1+c^2+d^2)u_1^2.
	\end{aligned}
\]

\[
	F_1(Q)\cap W_{X,Y}=V(1+a^2+b^2,ac+bd,1+c^2+d^2)\subset \spec K[a,b,c,d].
\]

% p.195

\[
	\begin{aligned}
		a&=-\Pi_{YZ}/\Pi_{XY},&b&=-\Pi_{YW}/\Pi_{XY},\\
		c&=\Pi_{XZ}/\Pi_{XY},&d&=\Pi_{XW}/\Pi_{XY}.
	\end{aligned}
\]

\[
	ad-bc=\Pi_{ZW}/\Pi_{XY},
\]

\[
	\mathbb G=V(\Pi_{ZW}\Pi_{XY}+\Pi_{YZ}\Pi_{XW}-\Pi_{XZ}\Pi_{YW}).
\]

\[
	V(\Pi_{XY}^2+\Pi_{YZ}^2+\Pi_{YW}^2,\Pi_{YZ}\Pi_{XZ}+\Pi_{YW}\Pi_{XW},\Pi_{XY}^2+\Pi_{XZ}^2+\Pi_{XW}^2)
\]