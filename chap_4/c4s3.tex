\section{Fano概形}\label{s:4.3}
\subsection{定义}\label{s:4.3.1}

在经典几何中,一种研究射影簇$X\subset \bbp_K^n$的方式是通过其与$\bbp_K^n$的线性子空间之间的关系。于是,对这样一个簇,就给出了一系列Grassmannian $\mathbb G_K(k,n)$的子簇。比如说,我们可以对$X\subset \bbp_K^n$给出$\mathbb G_K(k,n)$中的一族与$X$相交的线性空间;或者是$X$的切空间们;或者是$X$的割线们;或者$X$包含的线性子空间。所有这些子簇现在可以重新定义为对应$X\subset \bbp_S^n$的一个子概形的$\mathbb G_S(k,n)$的子概形,同样地,它们有着更丰富的结构,反应了$X$的几何。即使当我们从一个代数闭域$K$上的簇$X\subset \bbp_K^n$出发,此时对应于此的概形也可能是非约态的。

\nottran

% p.193

\[
	W=\spec R[\dots,x_{i,j},\dots]\cong \bba_S^{(k+1)(n+1)}
\]

\[
	H_G(u,x)=G\bigl(\sum u_ix_{0,i},\sum u_ix_{1,i},\dots,\sum u_ix_{k,i}\bigr)
\]

\[
	H_G(u,x)=\sum H_{G,J}(x)\cdot u^J.
\]

\[
	H_{G,c}(x)=G\bigl(\sum c_ix_{0,i},\sum c_ix_{1,i},\dots,\sum c_ix_{k,i}\bigr)
\]

% p.194

\subsection{二次曲面上的直线}\label{s:4.3.2}

\paragraph*{代数闭域上的光滑二次曲面上的直线}\addcontentsline{toc}{subsubsection}{代数闭域上的光滑二次曲面上的直线}

\[
	\begin{aligned}
		H_G(u_0,u_1)&=G(u_0,u_1,u_0a+u_1c,u_0b+u_1d)\\
		&=u_0^2+u_1^2+(u_0a+u_1c)^2+(u_0b+u_1d)^2\\
		&=(1+a^2+b^2)u_0^2+2(ac+bd)u_0u_1+(1+c^2+d^2)u_1^2.
	\end{aligned}
\]

\[
	F_1(Q)\cap W_{X,Y}=V(1+a^2+b^2,ac+bd,1+c^2+d^2)\subset \spec K[a,b,c,d].
\]

% p.195

\[
	\begin{aligned}
		a&=-\Pi_{YZ}/\Pi_{XY},&b&=-\Pi_{YW}/\Pi_{XY},\\
		c&=\Pi_{XZ}/\Pi_{XY},&d&=\Pi_{XW}/\Pi_{XY}.
	\end{aligned}
\]

\[
	ad-bc=\Pi_{ZW}/\Pi_{XY},
\]

\[
	\mathbb G=V(\Pi_{ZW}\Pi_{XY}+\Pi_{YZ}\Pi_{XW}-\Pi_{XZ}\Pi_{YW}).
\]

\[
	V(\Pi_{XY}^2+\Pi_{YZ}^2+\Pi_{YW}^2,\Pi_{YZ}\Pi_{XZ}+\Pi_{YW}\Pi_{XW},\Pi_{XY}^2+\Pi_{XZ}^2+\Pi_{XW}^2)
\]

\[
\begin{array}{l}{V\left(\Pi_{Y Z}^{2}-\Pi_{X W}^{2},\left(\Pi_{Y Z}+\Pi_{X W}\right)\left(\Pi_{Y W}+\Pi_{X Z}\right),\left(\Pi_{Y Z}+\Pi_{X W}\right)\left(\Pi_{Z W}-\Pi_{X Y}\right)\right.} \\ {\Pi_{Y W}^{2}-\Pi_{X Z}^{2},\left(\Pi_{Y W}-\Pi_{X Z}\right)\left(\Pi_{Y Z}-\Pi_{X W}\right),\left(\Pi_{Y W}-\Pi_{X Z}\right)\left(\Pi_{Z W}-\Pi_{X Y}\right)} \\ {\Pi_{Z W}^{2}-\Pi_{X Y}^{2},\left(\Pi_{Z W}+\Pi_{X Y}\right)\left(\Pi_{Y Z}-\Pi_{X W}\right),\left(\Pi_{Z W}+\Pi_{X Y}\right)\left(\Pi_{Y W}+\Pi_{X Z}\right)} \\ {\Pi_{X Y}^{2}+\Pi_{Y Z}^{2}+\Pi_{Y W}^{2}, \Pi_{X Y}^{2}+\Pi_{X Z}^{2}+\Pi_{X W}^{2} )}\end{array}
\]

\[
	\Lambda_{1}=V\left(\Pi_{Y Z}+\Pi_{X W}, \Pi_{Y W}-\Pi_{X Z}, \Pi_{Z W}+\Pi_{X Y}\right)
\]

\[
	\Lambda_{2}=V\left(\Pi_{Y Z}-\Pi_{X W}, \Pi_{Y W}+\Pi_{X Z}, \Pi_{Z W}-\Pi_{X Y}\right)
\]

\[
	F_{1}(Q)=\left(\Lambda_{1} \cup \Lambda_{2}\right) \cap \mathbb{G} \subset \mathbb{P}^{5}
\]

% p.196

\[
	\mathscr{Q}=V\left(t X^{2}+Y^{2}+Z^{2}+W^{2}\right) \subset \operatorname{Proj} K[t][X, Y, Z, W]=\mathbb{P}_{B}^{3}
\]

\[
\begin{aligned} H_{G}(a, x) &=G\left(u_{0}, u_{1}, u_{0} a+u_{1} c, u_{0} b+u_{1} d\right) \\ &=t u_{0}^{2}+u_{1}^{2}+\left(u_{0} a+u_{1} c\right)^{2}+\left(u_{0} b+u_{1} d\right)^{2} \\ &=\left(t+a^{2}+b^{2}\right) u_{0}^{2}+2(a c+b d) u_{0} u_{1}+\left(1+c^{2}+d^{2}\right) u_{1}^{2} \end{aligned}
\]

\[
	F_{1}(\mathscr{Q}) \cap W_{X, Y}=V\left(t+a^{2}+b^{2}, a c+b d, 1+c^{2}+d^{2}\right) \subset \spec K[t][a, b, c, d]
\]

% p.197

\[
	F_{1}\left(Q_{0}\right) \cap W_{X, Y}=V\left(a^{2}+b^{2}, a c+b d, 1+c^{2}+d^{2}\right) \subset \mathbb{A}_{K}^{4}
\]

\[
	\Lambda_{1}=V\left(\Pi_{Y Z}+\sqrt{\mu} \Pi_{X W}, \Pi_{Y W}-\sqrt{\mu} \Pi_{X Z}, \Pi_{Z W}+\sqrt{\mu} \Pi_{X Y}\right)
\]

\[
	\Lambda_{2}=V\left(\Pi_{Y Z}-\sqrt{\mu} \Pi_{X W}, \Pi_{Y W}+\sqrt{\mu} \Pi_{X Z}, \Pi_{Z W}-\sqrt{\mu} \Pi_{X Y}\right)
\]

\[
	F_{1}\left(Q_{\mu}\right)=\left(\Lambda_{1}(\mu) \cup \Lambda_{2}(\mu)\right) \cap \mathbb{G}_{K}(1,3) \subset \mathbb{P}_{K}^{5}
\]

% p.198

\[
	\Lambda=V\left(\Pi_{Y Z}, \Pi_{Y W}, \Pi_{Z W}\right)
\]

\[
	\mathscr{L}^{*} \subset \mathbb{P}_{B^{*}}^{5}=\operatorname{Proj} K\left[t, t^{-1}\right]\left[\Pi_{X Y}, \Pi_{X Z}, \Pi_{X W}, \Pi_{Y Z}, \Pi_{Y W}, \Pi_{Z W}\right]
\]

% p.199

\[
	\mathscr{Q}=V\left(s X^{2}+t Y^{2}+Z^{2}+W^{2}\right) \subset \operatorname{Proj} K[s, t][X, Y, Z, W]
\]

\[
	F_{1}(\mathscr{Q}) \cap W_{X, Y}=V\left(t+a^{2}+b^{2}, a c+b d, s+c^{2}+d^{2}\right) \subset \operatorname{Spec} K[s, t][a, b, c, d]
\]

\[
	F_{1}\left(Q_{0,0}\right) \cap W_{X, Y}=V\left(a^{2}+b^{2}, a c+b d, c^{2}+d^{2}\right) \subset \mathbb{A}_{K}^{4}
\]

\[
\begin{array}{l}{\Gamma_{1}=V(a+\sqrt{-1} b, c-\sqrt{-1} d)} \\ {\Gamma_{2}=V(a-\sqrt{-1} b, c+\sqrt{-1} d)}\end{array}
\]

\[
	\mathscr{Q}_{\alpha, \beta}=V\left(\alpha u X^{2}+\beta u Y^{2}+Z^{2}+W^{2}\right) \subset \operatorname{Proj} K[u][X, Y, Z, W]
\]

\[
\begin{aligned} F_{1}\left(\mathscr{Q}_{\alpha, \beta}\right) \cap W_{X, Y} &=V\left(\alpha u+a^{2}+b^{2}, a c+b d, \beta u+c^{2}+d^{2}\right) \\ & \subset \operatorname{Spec} K[\alpha, \beta][a, b, c, d] \end{aligned}
\]

\[
	\Phi(\gamma)=V(a-\gamma d, b+\gamma c)
\]

% p.200

\[
	P_{1}=[\sqrt{-1} \sqrt{\beta},-\sqrt{\alpha}, 0,0]
\]

\[
	\mathscr{Q}=V\left(t X^{2}+t Y^{2}+t Z^{2}+W^{2}\right) \subset \mathbb{P}_{B}^{3}
\]

% p.201

\[
\begin{aligned} Q_{1} &=V\left(X^{2}+Y^{2}-Z^{2}-W^{2}\right) \\ Q_{2} &=V\left(X^{2}+Y^{2}+Z^{2}-W^{2}\right) \\ Q_{3} &=V\left(X^{2}+Y^{2}+Z^{2}+W^{2}\right) \end{aligned}
\]

\[
\begin{aligned} \mathscr{Q}_{1} &=V\left(7 X^{2}+7 Y^{2}+Z^{2}+W^{2}\right) \\ \mathscr{Q}_{2} &=V\left(7 X^{2}+14 Y^{2}+Z^{2}+W^{2}\right) \\ \mathscr{Q}_{3} &=V\left(7 X^{2}+49 Y^{2}+Z^{2}+W^{2}\right) \end{aligned}
\]