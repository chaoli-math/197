\section{Fano概形}\label{s:4.3}
\subsection{定义}\label{s:4.3.1}

在经典几何中,一种研究射影簇$X\subset \bbp_K^n$的方式是通过其与
$\bbp_K^n$的线性子空间之间的关系。于是,对这样一个簇,就给出了一系列
Grassmannian $\mathbb G_K(k,n)$的子簇。比如说,我们可以对
$X\subset \bbp_K^n$给出$\mathbb G_K(k,n)$中的一族与$X$相交的线性空间;
或者是$X$的切空间们;或者是$X$的割线们;或者$X$包含的线性子空间。
所有这些子簇现在可以重新定义为对应$X\subset \bbp_S^n$的一个子概形的
$\mathbb G_S(k,n)$的子概形,同样地,它们有着更丰富的结构,反应了$X$的
几何。但即使当我们从一个代数闭域$K$上的簇$X\subset \bbp_K^n$出发,此时
对应于此的概形也可能是非既约的。

在这节中,我们将定义和研究概形$F_k(X)\subset \mathbb G_S(k,n)$,其参数
化了包含在一个概形$X\subset \mathbb P_S^n$中的$k$-维线性空间,它被称为
$X$的第$k$-个\textit{Fano概形}。我们将特别地尝试展示一个非既约概形结构
如何和何时出现,以及其如何允许我们推广一些关于Fano簇的经典定理。
比方说,我们将看到,如果$K$是任意域而$X\subset \mathbb P_K^3$是任意的
并不为直线们扫过的三次曲面,则$X$上的直线的Fano概形在$K$上的次数则正好
为$27$,尽管,包含在$X$中的直线的集合仅当$X$非奇异时的大小是$27$,甚至
当$K$不是代数闭的时候,非奇异也无法保证。更一般地,我们将看到,在许多
情况中,簇的平坦族$\mathscr X\subset \P_B^n\to B$对应的Fano概形族本身
在$B$上是平坦的,因此我们可以更一般地去给出一些关于数量和次数的论断。

在这届中,我们将通过给出定义理想来定义Fano概形,相同的理想也给出了
Fano簇的经典定义,这里唯一的不同之处在于我们不用再取根式了,这会
扔掉一些信息。然而,我们在第 \ref{chap:6} 章中将看到,通过点函子,可以
更内蕴地定义Fano概形$F_k(X)$,且这个定义将反过来给我们一个它们的几何的
诸多方面的刻画(如切空间),这将更加直接地联系到概形$X$的几何。这些
描述即使在$X$和$F_k(X)$是簇的情况下依然很有用。

% p.193

令$S$为任意概形,令$X\subset \P_S^n$为$S$上的射影空间的任意子概形,
令$k<n$为任意的正整数。$X$的Fano概形$F_k(X)\subset \mathbb G=
\mathbb G_S(k,n)$是参数化了$\P_S^n$中位于$X$上的$k$-维线性子空间的概形。
(一如既往,词“参数化”有一个很明确的意思,我们将在下面的
第 \ref{s:6.2.2} 节中进一步讨论它。)我们首先在$S=\spec R$为仿射概形
时定义$F_k(X)$. 我们将用第 \ref{s:3.2.7} 节中将$\mathbb G$看成
仿射空间$W_I\cong \mathbb A_S^{(k+1)(n-k)}$的并的刻画来描述它。

回忆,在这个构造中,我们令
\[
	W=\spec R[\dots,x_{i,j},\dots]\cong \bba_S^{(k+1)(n+1)}
\]
(我们将其看作关联于$(k+1)\times (n+1)$矩阵的矢量空间的仿射空间),
对每个多重指标$I=(i_0,\dots,i_k)\subset \{0,1,\dots,n\}$,令
$W_I\cong \mathbb A_S^{(k+1)(n-k)}\subset W$为由理想
$(\dots,x_{i_\alpha,j_\beta}-\delta_{\alpha,\beta},\dots)$给出的
闭子概形(我们可以将其想做关联于第$I$个子矩阵为恒等矩阵的矩阵的
子空间的仿射空间)。现在,假设$G(Z_0,\dots,Z_n)\in I(X)$为任意$X$
理想中的齐次多项式。将其应用到一个$(k+1)\times (n+1)$矩阵的行的
一个一般的线性组合上,我们得到了一个多项式
\[
	H_G(u,x)=G\bigl(\sum u_ix_{0,i},\sum u_ix_{1,i},\dots,\sum u_ix_{k,i}\bigr),
\]
其我们可以将其写作一个以$u_0$, $\dots$, $u_k$为变量的
单项式$u^J=u_0^{j_0}u_1^{j_1}\cdots u_k^{j_k}$的线性组合:
\[
	H_G(u,x)=\sum H_{G,J}(x)\cdot u^J.
\]
系数多项式$H_{G,J}(x)$是变量$x_{i,j}$的多项式,限制到子概形
$W_I\cong \A_S^{(k+1)(n-k)}\subset W$上,它们也是那里的正则函数。
我们定义\emph{Fano概形}$F_k(X)$为$\mathbb G$的子概形,其在每个开集
$W_I$上由多项式$H_{G,J}(x)$生成的理想给出,其中$G$跑遍所有理想
$I(X)\subset R[Z_0,\dots,Z_n]$的元素,而$J$标记了变量为$u_0$, $\dots$, 
$u_k$的、次数为$d$的多项式。

或者,对$R$中元素的任意$(k+1)$-元组$c=(c_0,\dots,c_k)$,我们可以定义一个
多项式
\[
	H_{G,c}(x)=G\bigl(\sum c_ix_{0,i},\sum c_ix_{1,i},\dots,\sum c_ix_{k,i}\bigr)
\]
然后取Fano概形$F_k(X)$为$\mathbb G$的子概形,其在$W_I$中由
多项式$H_{G,c}(x)$生成的理想给出,其中$G$跑遍$I(X)\subset R[Z_0,\dots,Z_n]$,
而$c$跑遍$R^{n+1}$.

% p.194

为完成这个定义,我们必须检查一些事情:这些$W_I$的子概形在相交的地方
相容,且它们定义的子概形$F_k(X)\subset \mathbb G$并不依赖于坐标的选取
(若我们选择第二种生成$F_k(X)\cap W_I$的理想,则会简单些,但当然我们也
得证明两种方式产生了相同的理想)。最后,我们应当检查这个构造是自然的,
即如果$T\to S$为任意的态射,$X_T=X\times_S T\subset \mathbb P_T^n$,
则Fano概形$F_k(X_T)=F_k(X)\times_S T\subset \mathbb G_S(k,n)\times_S T
=\mathbb G_{T}(k,n)$. 这个最后的条件实际上保证了,给定一个任意
(可能非仿射)的基$S$上的射影概形$X\subset \P_S^n$,我们可以
通过限制到$S$的仿射开子概形上然后粘起来定义Fano
概形$F_k(X)\subset \mathbb G_S(k,n)$. 所有这些声明都可以直接从定义出发
验证,但是从第 \ref{s:6.2.2} 节中给出的Grassmannian和Fano概形的内蕴刻画
得到会更加容易。

\subsection{二次曲面上的直线}\label{s:4.3.2}

为展示Fano概形的定义,我们将考虑一个简单的例子:在一个任意代数闭域
$K$上的二次曲面$Q=V(X^2+Y^2+Z^2+W^2)\subset \P_K^3$上的直线们。
方便起见,我们假设$K$的特征不为$2$(若我们只考虑光滑二次曲面,
则特征$2$的情况也是一样的)。甚至,在这个例子中,我们将看到一些非常
有趣的现象,还有我们也将考虑在非代数闭域上的一些例子。

\paragraph*{代数闭域上的光滑二次曲面上的直线}
\addcontentsline{toc}{subsubsection}{代数闭域上的光滑二次曲面上的直线}
如上所建议的,我们将首先在一个开集$W_I\subset \mathbb G=
\mathbb G_K(1,3)$中写下$F_1(Q)$的方程,此例中的剩下情况由对称性给出。
比如,取$W_{X,Y}=W_{1,2}$为$\mathbb G$对应于不平行于直线$X=Y=0$的直线们
的子集,我们可以将其等同于仿射空间$\A_K^4=\spec K[a,b,c,d]$,其对应于
具有如下形式的矩阵的空间
\begin{equation}
	\begin{pmatrix}
		1&0&a&b\\
		0&1&c&d
	\end{pmatrix}.
\end{equation}
我们接着记多项式$G(X,Y,Z,W)=(X^2+Y^2+Z^2+W^2)$在此矩阵行的线性组合
$u_0(1,0,a,b)+u_1(0,1,c,d)$上的限制$H$为
\[
	\begin{aligned}
		H_G(u_0,u_1)&=G(u_0,u_1,u_0a+u_1c,u_0b+u_1d)\\
		&=u_0^2+u_1^2+(u_0a+u_1c)^2+(u_0b+u_1d)^2\\
		&=(1+a^2+b^2)u_0^2+2(ac+bd)u_0u_1+(1+c^2+d^2)u_1^2.
	\end{aligned}
\]
在$W_{X,Y}\cong \A_K^4$中的Fano概形$F_1(Q)$被定义为$H_G$的系数的零点集,
其中系数看作$u_0$和$u_1$的多项式,即
\[
	F_1(Q)\cap W_{X,Y}=V(1+a^2+b^2,ac+bd,1+c^2+d^2)\subset \spec K[a,b,c,d].
\]

% p.195

这些方程定义的$\A_K^4$的子概形并不难描述。它是既约的,有一个(不可约)
分支位于平面$a=d$, $b=c$上,另一个在平面$a=-d$, $b=c$上。每个分支都
通过投影同构于平面二次曲线$\spec K[c,d]/(c^2+d^2+1)\subset \A_K^2
=\spec K[c,d]$.

我们可以用这个以$\mathbb G\subset \P_K^5$上的齐次坐标来写下$F_1(Q)$
的方程。为此,首先回忆$\P_K^5$上的齐次坐标对应于一个$2\times 4$矩阵的
$2\times 2$子式,我们可以用$\Pi_{XY}$, $\Pi_{XZ}$, $\Pi_{XW}$, $\Pi_{YZ}$,
$\Pi_{YW}$和$\Pi_{ZW}$来标记它们。开集$W_{X,Y}\subset \mathbb G$为
$\mathbb G$和仿射开子集$\Pi_{XY}\neq 0$的交,而$W_{X,Y}\cong \A_K^4$
上面的坐标函数$a$, $b$, $c$和$d$为下面分式的限制
\[
	\begin{aligned}
		a&=-\Pi_{YZ}/\Pi_{XY},&b&=-\Pi_{YW}/\Pi_{XY},\\
		c&=\Pi_{XZ}/\Pi_{XY},&d&=\Pi_{XW}/\Pi_{XY}.
	\end{aligned}
\]
此外,
\[
	ad-bc=\Pi_{ZW}/\Pi_{XY},
\]
由此我们可以得到$\mathbb G\subset \P_K^5$的定义方程
\[
	\mathbb G=V(\Pi_{ZW}\Pi_{XY}+\Pi_{YZ}\Pi_{XW}-\Pi_{XZ}\Pi_{YW}).
\]

现在,从上面的$F_1(Q)\cap W_{X,Y}$的方程,我们可以看到Fano概形$F_1(Q)$
包含于
\[
	V(\Pi_{XY}^2+\Pi_{YZ}^2+\Pi_{YW}^2,\Pi_{YZ}\Pi_{XZ}+\Pi_{YW}\Pi_{XW},\Pi_{XY}^2+\Pi_{XZ}^2+\Pi_{XW}^2)
\]
中。对另五个$\P^5$的仿射开集做同样的事情将产生一个$F_1(Q)\subset \P^5$
的定义方程的完整集合。从方程的对称性,我们不难得到$F_1(Q)$具有如下表示
\[
\begin{array}{l}{V\left(\Pi_{Y Z}^{2}-\Pi_{X W}^{2},\left(\Pi_{Y Z}+\Pi_{X W}\right)\left(\Pi_{Y W}+\Pi_{X Z}\right),\left(\Pi_{Y Z}+\Pi_{X W}\right)\left(\Pi_{Z W}-\Pi_{X Y}\right)\right.} \\ {\Pi_{Y W}^{2}-\Pi_{X Z}^{2},\left(\Pi_{Y W}-\Pi_{X Z}\right)\left(\Pi_{Y Z}-\Pi_{X W}\right),\left(\Pi_{Y W}-\Pi_{X Z}\right)\left(\Pi_{Z W}-\Pi_{X Y}\right)} \\ {\Pi_{Z W}^{2}-\Pi_{X Y}^{2},\left(\Pi_{Z W}+\Pi_{X Y}\right)\left(\Pi_{Y Z}-\Pi_{X W}\right),\left(\Pi_{Z W}+\Pi_{X Y}\right)\left(\Pi_{Y W}+\Pi_{X Z}\right)} \\ {\Pi_{X Y}^{2}+\Pi_{Y Z}^{2}+\Pi_{Y W}^{2}, \Pi_{X Y}^{2}+\Pi_{X Z}^{2}+\Pi_{X W}^{2} )}.\end{array}
\]

若我们整理地更好些可能更容易理解,上面关于$F_1(Q)\cap W_{X,Y}$的描述
给了我们方向。令$\Gamma_1$和$\Gamma_2\cong \P_K^2\subset \P_K^5$
为两个由方程
\[
	\Lambda_{1}=V\left(\Pi_{Y Z}+\Pi_{X W}, \Pi_{Y W}-\Pi_{X Z}, \Pi_{Z W}+\Pi_{X Y}\right)
\]
和
\[
	\Lambda_{2}=V\left(\Pi_{Y Z}-\Pi_{X W}, \Pi_{Y W}+\Pi_{X Z}, \Pi_{Z W}-\Pi_{X Y}\right)
\]
定义的不交$2$-平面。则作为概形,我们有很简单的
\[
	F_{1}(Q)=\left(\Lambda_{1} \cup \Lambda_{2}\right) \cap \mathbb{G} \subset \mathbb{P}^{5}.
\]%
%
% p.196
%
每个平面$\Lambda_i$交$\mathbb G$于一个非奇异平面二次曲线$C_i$,于是
我们看到$F_1(Q)$即为两个二次曲线在互补平面中的并。(特别地,$F_1(Q)$
即为两个仿射二次曲线在上面的$F_1(Q)\cap W_{X,Y}$中的闭包。)
这对应于一个二次曲面的两条直纹的经典图像。\nottran

\inclugra{6.png}

\paragraph*{二次锥上的直线}
\addcontentsline{toc}{subsubsection}{二次锥上的直线}
接着,我们考虑一个随着二次曲面变化的其上的直线的Fano概形的变化,特别
当其退化到一个奇异二次曲面时。令我们的族的基为$B=\spec K[t]\cong
\A_K^1$,然后首先考虑二次曲面的族$\mathscr Q\subset \P_B^3$,其由
\[
	\mathscr{Q}=V\left(t X^{2}+Y^{2}+Z^{2}+W^{2}\right) \subset \operatorname{Proj} K[t][X, Y, Z, W]=\mathbb{P}_{B}^{3}
\]
给出。我们将$\mathscr Q$在点$(t-\mu)\in B=\A_K^1$上的纤维为$Q_\mu
\subset \P_K^3$. Fano概形$F_1(\mathscr Q)$同样是$\mathbb G_B(1,3)$的
子概形,其在点$(t-\mu)\in B=\A_K^1$上的纤维为在二次曲面$Q_\mu\subset 
\P_K^3$上的直线的Fano概形$F_1(Q_\mu)\subset \mathbb G_K(1,3)$.

同前,取$W_{X,Y}$为$\mathbb G_{B}(1,3)$对应于不平行于直线$X=Y=0$的
直线们的子集,并将其等同于关联于具有形式 \theequation 的矩阵的空间的
仿射空间$\A_B^4=\spec K[t][a,b,c,d]$.

记$H$为多项式$G(X,Y,Z,W)=(tX^2+Y^2+Z^2+W^2)$限制到行的线性组合上的
多项式
\[
\begin{aligned} 
	H_{G}(a, x) &=G\left(u_{0}, u_{1}, u_{0} a+u_{1} c, u_{0} b+u_{1} d\right) \\ &=t u_{0}^{2}+u_{1}^{2}+\left(u_{0} a+u_{1} c\right)^{2}+\left(u_{0} b+u_{1} d\right)^{2} \\ &=\left(t+a^{2}+b^{2}\right) u_{0}^{2}+2(a c+b d) u_{0} u_{1}+\left(1+c^{2}+d^{2}\right) u_{1}^{2}. 
\end{aligned}
\]
在$W_{X,Y}\cong \A^4$中的Fano概形$F_1(\mathscr Q)$为系数的零点集,即
\[
	F_{1}(\mathscr{Q}) \cap W_{X, Y}=V\left(t+a^{2}+b^{2}, a c+b d, 1+c^{2}+d^{2}\right) \subset \spec K[t][a, b, c, d].
\]

% p.197

对任意固定的非零标量$\mu\neq 0\in K$,在$(t-\mu)$处$F_1(\mathscr Q)
\cap W_{X,Y}$的纤维为$\A_K^4$的子概形,其同构于上面描述的子概形$F_1(Q)
\cap W_{X,Y}\subset \A_K^4$(必然如此,因为$Q_\mu$通过一个$\P_K^3$的
自同构射影等价,此自同构作用在$\mathbb G_K(1,3)$上保持$W_{X,Y}$不变)。
这是既约的,一个分支位于平面$a=\sqrt{\mu}d$, $b=-\sqrt{\mu}c$上,
另一个位于平面$a=-\sqrt{\mu}d$, $b=\sqrt{\mu}c$上。每个分支通过投影
同构于平面二次曲线$\spec K[c,d]/(c^2+d^2+1)\subset \A_K^2=\spec K[c,d]$.

现在考虑$F_1(\mathscr Q)\cap W_{X,Y}$在$(t)$上的纤维,即
二次锥$Q_0$的Fano概形$F_1(Q_0)$的开子集$F_1(Q_0)\cap W_{X,Y}$. 它的
方程为
\[
	F_{1}\left(Q_{0}\right) \cap W_{X, Y}=V\left(a^{2}+b^{2}, a c+b d, 1+c^{2}+d^{2}\right) \subset \mathbb{A}_{K}^{4}.
\]
并不难看到$F_1(Q_0)\cap W_{X,Y}$的支集为一个平面二次曲线,位于平面
$a=b=0$上,由方程$c^2+d^2=1$给出。但是$F_1(Q_0)$不是既约的!相反,
在每一点,切空间是二维的,由在平面$a=b=0$中的既约二次曲线$c^2+d^2+1=0$
的切线,和另一个并不位于此平面的向量一起张成:
\inclugra{7.png}

当我们考虑整个Fano概形$F_1(\mathscr Q)\subset \mathbb G_B(1,3)$和
其在$(t-\mu)\in B$处的纤维$F_1(Q_\mu)\subset \mathbb G_K(1,3)$时,
同样的图像照样成立。令$\Lambda_1(\mu)$和$\Lambda_2(\mu)\cong \P_K^2
\subset \P_K^5$为由下面方程
\[
	\Lambda_{1}=V\left(\Pi_{Y Z}+\sqrt{\mu} \Pi_{X W}, \Pi_{Y W}-\sqrt{\mu} \Pi_{X Z}, \Pi_{Z W}+\sqrt{\mu} \Pi_{X Y}\right)
\]
和
\[
	\Lambda_{2}=V\left(\Pi_{Y Z}-\sqrt{\mu} \Pi_{X W}, \Pi_{Y W}+\sqrt{\mu} \Pi_{X Z}, \Pi_{Z W}-\sqrt{\mu} \Pi_{X Y}\right)
\]
所定义的不交$2$-平面。接着,对$\mu\neq 0$,作为概形有
\[
	F_{1}\left(Q_{\mu}\right)=\left(\Lambda_{1}(\mu) \cup \Lambda_{2}(\mu)\right) \cap \mathbb{G}_{K}(1,3) \subset \mathbb{P}_{K}^{5}.
\]%
%
% p.198
%
同前,每个平面$\Lambda_i(\mu)$交$\mathbb G_K(1,3)$于一个非奇异平面
二次曲线$C_i(\mu)$. 但现在当$\mu$趋于$0$,两个平面曲线$\Lambda_i(\mu)$
具有相同的极限位置,平面
\[
	\Lambda=V\left(\Pi_{Y Z}, \Pi_{Y W}, \Pi_{Z W}\right).
\]
如下面的习题所展示的,概形$\Lambda(\mu)=\Lambda_1(\mu)\cup \Lambda_2(\mu)$
的平坦极限$\Lambda(0)$为支于平面$\Lambda$上的概形,但具有重数$2$.

\begin{exe}\label{exe:4.65}
	证明,概形$\Lambda(\mu)=\Lambda_1(\mu)\cup \Lambda_2(\mu)$
	和$\Lambda(0)$对$\mu\neq 0$构成了一个平坦族,即在
	$B^*=B\setminus \{0\}=\spec K[t,t^{-1}]$上平坦的概形
\[
	\mathscr{L}^{*} \subset \mathbb{P}_{B^{*}}^{5}=\operatorname{Proj} K\left[t, t^{-1}\right]\left[\Pi_{X Y}, \Pi_{X Z}, \Pi_{X W}, \Pi_{Y Z}, \Pi_{Y W}, \Pi_{Z W}\right],
\]
	其在$(t-\mu)\in B^*$处的纤维为$\Lambda(\mu)$. 找到$\mathscr L^*$在
	$\P_{B^*}^5$中的方程,找到$\mathscr L^*$的闭包在$\P_B^n$中的方程,
	也因此概形$\Lambda(\mu)$的极限$\Lambda(0)$的方程。最后,证明
	Fano概形$F_1(Q_\mu)$的极限就是$Q_0$的Fano概形$F_1(Q_0)$.
\end{exe}

\begin{exe}\label{exe:6.66}
	证明,$F_1(Q_0)\subset \P_K^5$并不包含于一个超平面中。
\end{exe}

\begin{exe}\label{exe:6.67}
	令$Q\subset \P_K^3$为一个非奇异二次曲线上的锥,$F_1(Q)\subset 
	\mathbb G_K(1,3)\subset \P_K^5$为其上直线的Fano概形。证明,
	$F_1(Q)$同构于一个二次曲面上的双重直线,即在第 \ref{s:3.3.4} 节
	中所描述的双重曲线$X_1$.
\end{exe}

利用第 \ref{s:6.2.3} 节中所给出的$F_1(Q)$的描述,我们可以更直接地看到
一个二次锥的Fano概形不是既约的。

如果我们令$L=K(t)$为我们族的基$B$的函数域,$Q_L\subset \P_L^3$为
$\mathscr Q$在我们基$B$的一般点$\spec L$上的纤维,$F_1(Q_L)\subset 
\mathbb G_L(1,3)$为$Q_L$的Fano概形,$F_1(Q_L)$将\underline{不是}\hspace{-0.8ex}
两个二次曲线的并。
如果我们将其放到$L$的二次扩张$L'=L(\sqrt \mu)$中,即取纤维积
$F_1(Q_L)\times_{\spec L}\spec L'$,我们得到的概形是一个$\spec \overline{L}$
上的两条二次曲线的并,但是$F_1(Q_L)$其本身是不可约的。这是个很好的例子,
展示了一个来自于纯几何语境的概形可以是既约但不是绝对不可约的。

\paragraph*{一个退化到两个平面的二次曲面}
\addcontentsline{toc}{subsubsection}{一个退化到两个平面的二次曲面}
现在考虑一个二次曲面族$\mathscr Q\to B$,其一般的对象是光滑的,而其可以
特化到一个二次曲面$Q_0$由两个平面的并构成。这个例子有趣的并不是Fano概形
$F_1(Q_0)$,毕竟$Q_0$是两个平面的并,上面的直线们构成的几何是非常简单的,
它有意思是这个族的几何。在我们主要的例子中,Fano概形$F_1(\mathscr Q)
\subset \mathbb G_B(1,3)$在$B$上并不平坦(且实际上,将其限制到开子集
$B^*=B\setminus \{0\}$上,它在$0$处并没有平坦极限);在另一个例子中,
一个平坦极限存在,但其依赖于特定的族而不仅依赖于$Q_0$.

% p.199

行易于言。首先,令$B=\spec K[s,t]\cong \A_K^2$,然后考虑由
\[
	\mathscr{Q}=V\left(s X^{2}+t Y^{2}+Z^{2}+W^{2}\right) \subset \operatorname{Proj} K[s, t][X, Y, Z, W]
\]
给出的族$\mathscr Q\subset \P_B^3\to B$,再令$Q_{\mu,\nu}\subset \P_K^3$
为$\mathscr Q$在点$(s-\mu,t-\nu)\in B$处的纤维。令$F_1(\mathscr Q)\subset 
\mathbb G_B(1,3)$为$\mathscr Q$的Fano概形。

即使还没写下方程,我们可以看到$F_1(\mathscr Q)$在$B$上并不平坦:
$F_1(\mathscr Q)$在$B$上的纤维$F_1(Q_{\mu,\nu})$都是一维的,除了
$F_1(Q_{0,0})$,其肉眼可见是二维的,支于两个平面的并。为看到更多,
我们写下方程。同上,我们从$F_1(\mathscr Q)$在$\mathbb G_B(1,3)$的开子集
$W_{X,Y}$中的方程开始,我们有
\[
	F_{1}(\mathscr{Q}) \cap W_{X, Y}=V\left(t+a^{2}+b^{2}, a c+b d, s+c^{2}+d^{2}\right) \subset \spec K[s, t][a, b, c, d].
\]
$F_1(\mathscr Q)$在原点$(s,t)\in B$处的纤维在$W_{X,Y}$中有
\[
	F_{1}\left(Q_{0,0}\right) \cap W_{X, Y}=V\left(a^{2}+b^{2}, a c+b d, c^{2}+d^{2}\right) \subset \mathbb{A}_{K}^{4}
\]
给出。我们同样可以将其描述为以下两个平面$\Gamma_1$和$\Gamma_2\subset \A_K^4$
的并:
\begin{align*}
	\Gamma_{1}&=V(a+\sqrt{-1} b, c-\sqrt{-1} d),\\
	\Gamma_{2}&=V(a-\sqrt{-1} b, c+\sqrt{-1} d).
\end{align*}

\begin{exe}\label{exe:6.68}
	证明,一个秩为$2$的二次曲面的Fano概形$F_1(Q_{0,0})$是既约的。
\end{exe}

现在让我们考虑这个双参数族的子族。首先,固定两个非零标量$\alpha$和
$\beta\in K$,然后考虑我们的族在穿过$B=\A_K^2$的原点的直线
$V(\beta s-\alpha t)$的直线,斜率为$\beta/\alpha$;即,由
\[
	\mathscr{Q}_{\alpha, \beta}=V\left(\alpha u X^{2}+\beta u Y^{2}+Z^{2}+W^{2}\right) \subset \operatorname{Proj} K[u][X, Y, Z, W]
\]
给出的族$\mathscr Q_{\alpha,\beta}$,基为$B'=\spec K[u]$.
同样的理由告诉我们,Fano概形$F_1(\mathscr Q_{\alpha,\beta})$在$B'$上并不平坦。
这里不同的是,在$B'$的原点的补上的Fano概形\emph{有}一个平坦极限。实际上,
$F_1(\mathscr Q_{\alpha,\beta})$是既约的,有$F_1(\mathscr Q_{0,0})$
这一分支,而若我们去掉那个分支,剩下的就是平坦的。为看到这点,
我们写下$F_1(\mathscr Q_{\alpha,\beta})$在$\mathbb G_B(1,3)$的开子集
$W_{X,Y}$中的方程:
\[
\begin{aligned} F_{1}\left(\mathscr{Q}_{\alpha, \beta}\right) \cap W_{X, Y} &=V\left(\alpha u+a^{2}+b^{2}, a c+b d, \beta u+c^{2}+d^{2}\right) \\ & \subset \spec K[\alpha, \beta][a, b, c, d]. \end{aligned}
\]
令$\Phi(\gamma)\subset \A_K^2$为右下
\[
	\Phi(\gamma)=V(a-\gamma d, b+\gamma c)
\]
给出的$2$-平面,且令$\Psi(\gamma)$为$\Phi(\gamma)$和$\Phi(-\gamma)$
的不交并。%
%
% p.200
%
从这些方程中,我们可以看到,对任意$\nu\neq 0\in K$,$F_1(\mathscr Q_{\alpha,\beta})$
在$(u-\mu)\in B'$上的纤维被包含于概形$\Psi(\sqrt{\alpha/beta})$中,
\emph{并不依赖于$\mu$},且在$\Psi(\sqrt{\alpha/\beta})$上由进一步的方程
$\beta u+c^2+d^2=0$切出。于是,Fano概形$F_1(Q_{\alpha u,\beta u})$
在$u$趋向于$0$时的平坦极限为$\Psi(\sqrt{\alpha/\beta})$与两个超平面
$V(c+\sqrt{-1}d)$和$V(c-\sqrt{-1}d)$的并$V(c^2+d^2)$的交。这是四条直线
的并。

为将此几何地表出,注意到这些直线其一是平面$H_1=V(Z+\sqrt{-1}W)$上穿过
点
\[
	P_{1}=[\sqrt{-1} \sqrt{\beta},-\sqrt{\alpha}, 0,0]
\]
的直线对应的Grassmannian $\mathbb G_K(1,3)$的子概形,
其二对应$H_1$上穿过点$P_2=[\sqrt{-1}\sqrt\beta,\sqrt\alpha,0,0]$的直线们,
其三对应$H_2=V(Z-\sqrt{-1}W)$上穿过点$P_1$的直线们,最后一个对应
$H_2$上穿过$P_2$的直线们。注意到这里的点$P_1$和$P_2$可以被刻画为
二次曲面$Q_0$的双重直线$M=H_1\cap H_2$与另一个束中的二次曲面
$Q_{\alpha \mu,\beta\mu}$的交。

\inclugra{8.png}

族$\mathscr Q_{\alpha,\beta}$的平坦极限随着斜率$\beta/\alpha$变化
而变化。特别地,它们的并在$F_1(Q_{0,0})$中稠密。这说明了,
整个族的Fano概形$F_1(\mathscr Q)\to B$是不可约的,也因此$F_1(\mathscr Q)$
在原点$(s,t)\in B$的补上的限制并没有一个平坦极限。

\begin{exe}\label{exe:6.69}
考虑趋于一个双重平面的二次曲面的单参族,其方程为
\[
	\mathscr{Q}=V\left(t X^{2}+t Y^{2}+t Z^{2}+W^{2}\right) \subset \mathbb{P}_{B}^{3}.
\]
Fano概形$F_1(\mathscr Q_t)$的平坦极限是什么?
\end{exe}

% p.201

\paragraph*{更多例子}
\addcontentsline{toc}{subsubsection}{更多例子}
为看到更多非代数闭域上的二次曲面上的直线的Fano概形的行为,我们考虑实
二次曲面的例子:

\begin{exe}\label{exe:4.70}
考虑在$\P_{\mathbb R}^3$中由方程
\[
\begin{aligned} Q_{1} &=V\left(X^{2}+Y^{2}-Z^{2}-W^{2}\right) \\ Q_{2} &=V\left(X^{2}+Y^{2}+Z^{2}-W^{2}\right) \\ Q_{3} &=V\left(X^{2}+Y^{2}+Z^{2}+W^{2}\right) \end{aligned}
\]
给出的二次曲面$Q_1$, $Q_2$, $Q_3$. 证明Fano概形$F_1(Q_1)$为两个$\P_{\mathbb R}^1$的并,
而$Q_2$上的直线的Fano概形是不可约但不是绝对不可约的。最后,证明,
Fano概形$F_1(Q_3)$是两个不同构于$\P_{\mathbb R}^1$的平面二次曲线的并。
\end{exe}

下面是一个函数域上的例子:

\begin{exe}\label{exe:4.71}
令$B=\P_K^9$为参数化了$\P_K^3$中二次曲面的射影空间,$L$为其函数域。
令$\mathscr Q_B\subset \P_B^3$为$B$上的万有二次曲面。令$Q_L\subset \P_L^3$
为$\mathscr Q$在$B$的一般点$\spec L$处的纤维,$F_L=F_1(Q_L)\subset 
\mathbb G_L(1,3)$为直线的Fano概形。描述$F_L$. 特别地,证明他与上面例子
中的行为不同,其中Fano概形同构于两个$\P_M^1$,$M$是$L$的一个四次扩张,
并不是任何的二次扩张。(实际上,存在一个$L$的二次扩张$L'$,在上面
$F_L$变得既约了,但是$F_L\times_L\spec L'$的分支是$\P_{L'}^1$的型
(型见第 \ref{s:4.4} 节),并不同构于$\P_{L'}^1$.)
\end{exe}

最后,是一个算数类比:

\begin{exe}\label{exe:4.72}
考虑下面给出的二次曲面$\mathscr Q_1$, $\mathscr Q_2$和$\mathscr Q_3\subset \P_\zz^3$,
\[
\begin{aligned} \mathscr{Q}_{1} &=V\left(7 X^{2}+7 Y^{2}+Z^{2}+W^{2}\right) \\ \mathscr{Q}_{2} &=V\left(7 X^{2}+14 Y^{2}+Z^{2}+W^{2}\right) \\ \mathscr{Q}_{3} &=V\left(7 X^{2}+49 Y^{2}+Z^{2}+W^{2}\right).\end{aligned}
\]
描述每个情况中的Fano概形$F_1(\mathscr Q_i)\subset \mathbb G_\zz(1,3)$.
特别地,描述在$\spec \zz$上优势%
\footnote{
	译者注:这里在$\spec \zz$上优势原文是`dominating $\spec \zz$',
	应该是指这个分支到$\spec \zz$的态射是优势的,即其在
	$\spec \zz$中的像是稠密的。
}%
的$F_1(\mathscr Q_i)$的分支,而它与
$\mathbb G_\zz(1,3)$的纤维$\mathbb G_{\zz/(7)}$交于点$(7)\in \spec \zz$.
\end{exe}

\subsection{三次曲面上的直线}

在下面的一系列习题中,我们将建立一些参数化了三次曲面$S\subset \P_K^3$
上的直线的Fano概形$F_1(S)$的有趣事实。首先,下面两个系统建立了
所有非奇异三次曲面包含相同数目的直线的事实(除了推出这是$27$)。

% p.202

\begin{exe}\label{exe:4.73}
令$K$是一个域,$S \subset \mathbb{P}_{K}^{3}$为一个非奇异三次曲面,
而$F=F_{1}(S) \subset \mathbb{G}_{K}(1,3)$为$S$上的直线的Fano概形。 
证明$F$是既约的。

\emph{提示}: 取$L \in F_{1}(S)$为由$X=Y=0$给出的直线,记定义$S$的三次
多项式为$X Q(Z, W)+Y P(Z, W)$模掉$(X, Y)^{2}$;证明,切空间
$T_{[L]} F_{1}(S)$有正维数的条件为$P$和$Q$有一个公共零点在$L$上。
(比较这个与第 \ref{s:4.2.3} 节中的讨论。)
\end{exe}

\begin{exe}\label{exe:4.74}
令$\mathbb{P}_{K}^{19}$为参数化 $\mathbb{P}_{K}^{3}$ 中三次曲面的
射影空间,$U \subset \mathbb{P}_{K}^{19}$ 为上面只有有限条直线的
曲面对应的开集(即除了锥面和柱面)。令
$\mathscr{S}_{U} \subset \mathbb{P}_{U}^{3}$为$U$上的射影$3$-空间
中的万有三次曲面,而$\mathscr{F}_{U}=$ $F_{1}\left(\mathscr{S}_{U}\right) \subset \mathbb{G}_{U}(1,3)$为其上直线的Fano概形。证明投影映射
$\pi: \mathscr{F}_{U} \rightarrow U$平坦。

\emph{提示}:使用如下事实,三次曲面$S \subset \mathbb{P}_{K}^{3}$上
直线的Fano概形$F_{1}(S)$是一个局部完全交,以及Proposition \ref{pro:2.32}.
\end{exe}

\nottran

% p.203

\nottran

\begin{exe}\label{exe:4.78}
取$S_0$为一个非奇异二次曲面$Q$和一个平面$H$的并,它们交于一个非奇异
二次曲线$C$. 令$\{P_1,\dots,P_6\}=C\cap S$为$C$上的束的基点。
证明,当$\lambda$趋于$0$时,Fano概形$F_1(S_\lambda)$的平坦极限是既约的,
次数为$27$,由$Q$上包含某个$P_i$的$12$条直线,与$H$上包含某两个
$P_i$的$15$条直线组成。
\end{exe}

\begin{exe}\label{exe:4.79}
现在取$S_0$为处于一般位置的三个平面$H_1$, $H_2$, $H_3$的并。 
再,当$\lambda$趋于$0$时,Fano概形$F_1(S_\lambda)$平坦极限是什么?
\end{exe}

\begin{exe}\label{exe:4.80}
同样的问题,但现在取$S_0$为一个非奇异平面三次曲线上的锥。
\end{exe}

最后,一个万有三次曲面上的直线的有趣例子。

\begin{exe}\label{exe:4.81}
令$B=\mathbb{P}_{K}^{19}$为参数化了$\mathbb{P}_{K}^{3}$中三次曲面的
射影空间,$L$为$B$的函数域,$\mathscr{S}_{B} \subset \mathbb{P}_{B}^{3}$
为$K$上的射影$3$-空间中的万有三次曲面,而$\mathscr{S}_{L}$
为$\mathscr{S}$在$B$的一般点$\spec L$处的纤维。令
$\mathscr{F}_{L}=F_{1}\left(\mathscr{S}_{L}\right) \subset \mathbb{G}_{L}(1,3)$ 
为其上直线的Fano概形。证明,$\mathscr{F}_{L}$由一个既约点构成,
其剩余类域是$L$的$27$次扩张。

\emph{提示}:这来自于如下事实,万有Fano簇$\mathscr{F}=\mathscr{F}_{1}(\mathscr{S}) \subset \mathbb{G}_{B}(1,3)=\mathbb{P}_{K}^{19} \times_{K} \mathbb{G}_{K}(1,3)$不可约,
这又来自于如下事实,到第二个因子的投影将$\mathscr F$表为了$\mathbb{G}_{K}(1,3)$
上的$\mathbb{P}_{K}^{19}$-丛。
\end{exe}

% p.204

\begin{exe}\label{exe:4.82}
现在考虑实数上的非奇异三次曲面$S \subset \mathbb{P}_{\mathbb{R}}^{3}$. 
正如我们已经看到的,$S_{\mathbb{C}}=S \times_{\spec \mathbb{R}}\spec \mathbb{C}$ 
的Fano概形$F_{1}\left(S_{\mathbb{C}}\right) \subset \mathbb{G}_{\mathbb{C}}(1,3)$
由$27$个既约点组成。于是,对于满足$a+2 b=27$的整数对$a$和$b$,
$F_{1}(S) \subset \mathbb{G}_{\mathbb{R}}(1,3)$将由$a$个剩余类域为
$\mathbb{R}$的既约点和$b$个剩余类域为$\mathbb{C}$的既约点组成。
证明$a$可以为$3$, $7$, $15$或$27$,而其他的值是不可能的。(见Segre [1942].)
\end{exe}