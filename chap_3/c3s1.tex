一旦我们理解了仿射概形,则射影概形\index{射影!概形}的理论并不会包含太多新奇的事物:在大多数情况下,它与射影簇的古典理论的差别完全类似于仿射概形和仿射簇的古典理论之间的差别。

我们首先将引入两个有限性条件,\textit{有限}以及\textit{有限型},然后定义和讨论\textit{分离}以及\textit{颇合}\footnote{译者注:将 proper morphism 译作“颇合态射”取自周健先生翻译的EGA.}态射\index{颇合!态射},它们分别对应于绝大多数几何中的Hausdorff性与紧性。也正是部分因为射影簇和概形有这些性质,它们才是经典代数几何以及概形理论的基本研究对象。

这章的下一部分将引入射影概形以及给出一些例子。就像在仿射概形中那样,有两种入手射影概形的法子:一种是定义出射影空间然后取它们的子概形,另一种是将所有的射影概形建立在同一个根基上,从分次代数入手。正如在仿射情况的处理,这里我们采用第二种方法。

在介绍了射影概形的基础定义以及它们的子概形后,我们将描述射影概形之间的态射,比起其仿射对应,它将更加微妙而难以捉摸(就像在簇范畴中那样)。我们将用一些射影概形的例子结束这节,其中最值得注意的是Grassmannian.

本章的最后一节将给出嵌入在射影空间中的射影概形的三个不变量,它们由David Hilbert所引入:Hilbert多项式、Hilbert函数以及自由分解(free resolution). 使用它们,我们有时能区分类似的概形,比如射影双重直线,以及我们能
揭示平坦性的一些新现象。在一个射影概形的不变量中,可以根据其Hilbert多项式定义的不变量的是它的度数;在这方面的联系,我们将讨论著名的B\'{e}zout定理。

\section{一些态射的性质}

\subsection{有限性条件}

\subsection{颇合性和分离性}