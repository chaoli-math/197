在我们理解了仿射概形后,射影概形\index{射影!概形}的理论并不会包含太多新奇的东西:在大多数情况下,它与射影簇的经典理论的差别完全类似于仿射概形和仿射簇的经典理论之间的差别。

我们首先将引入两个有限性条件,\textit{有限}以及\textit{有限型},然后定义和讨论\textit{分离}以及\textit{颇合}\footnote{译者注:将 proper morphism 译作“颇合态射”取自周健先生翻译的EGA.}态射\index{颇合!态射},它们分别对应于绝大多数几何中的Hausdorff性与紧性。而正是部分因为射影簇和概形有这些性质,它们才是经典代数几何以及概形理论的基本研究对象。

这章的下一部分将引入射影概形以及给出一些例子。如同仿射概形的情形,有两种入手射影概形的法子:一种是定义出射影空间然后取它们的子概形,另一种是将所有的射影概形建立在同一个根基上,从分次代数入手。正如处理仿射情况那般,这里我们采用第二种方法。

在介绍了射影概形的基础定义以及它们的子概形后,我们将描述射影概形之间的态射,相较于仿射对应,它将更加微妙而难以捉摸(就像在簇范畴中那样)。我们将用一些射影概形的例子结束这节,其中最值得注意的是Grassmannian.

本章的最后一节将给出嵌入在射影空间中的射影概形的三个不变量,它们由David Hilbert所引入:Hilbert多项式、Hilbert函数以及自由分解(free resolution). 使用它们,我们有时能区分类似的概形,比如射影双重直线,以及可以
揭示平坦性的一些新现象。在一个射影概形的不变量中,可以根据其Hilbert多项式定义的不变量的是它的度数;在这方面的联系,我们将讨论著名的B\'{e}zout定理。

\section{一些态射的性质}

在大多数有关概形的非平凡结论中,有两个有限性条件。它们有着相似的名字但非常不同的性质。第一个有限性条件,有限型,是一个在绝大多数几何背景中产生的态射都满足的直截了当的性质,他被引入常是为了排除无穷维的纤维或者“非几何”的概形,比如局部环的谱。而另一个有限性条件,有限,比较起来就是一个非常严格的条件:此时一个态射是颇合的以及它的所有纤维都是有限的(特别地,零维的)。

首先,我们称一个概形间的态射$\varphi:X\to Y$是\textit{有限型}的,如果对每一点$y\in Y$,都存在一个$y$的仿射开邻域$V=\spec B\subset Y$使得它的原像被一族仿射开集$U_i\cong \spec A_i$有限覆盖,即
\[
	\varphi^{-1}(V)=\bigcup_{i=1}^n U_i,
\]
并且在映射
\[
	\varphi_V^\#:B=\oo_Y(V)\to \oo_X (\varphi^{-1}V)\to \oo_X(U_i)=A_i
\]
下,每个$A_i$都是一个有限生成$B$\hyp 代数。因此,比如,任意$\mathbb{A}_K^n$或$\mathbb{P}_K^n$的子概形是在$K$上有限型的(意思是,结构态射$X\to \spec K$是有限型的),而一个正维数的局部$K$\hyp 代数的谱不是。

一个态射$\varphi:X\to Y$被称为有限的,如果对

\subsection{有限性条件}

\subsection{颇合性和分离性}