\section{分次环的\texorpdfstring{$\proj$}{Proj}}

\subsection{\texorpdfstring{$\proj S$}{Proj S}的构造}

到目前为止,非仿射概形最重要的例子是在一个仿射概形$\spec A$上的\textit{射影}概形,其中$A$是任意一个交换环。(为简单起见,我们常常说一个概形在$A$上射影,而不说在$\spec A$上射影。)这样的一个概形来自于一个分次$A$-代数,而构造过程非常类似于从它的齐次坐标环构造一个射影簇。我们可以从一个分次$\oo_B$-代数层开始,定义在任意的基概形$\spec B$上射影的概形,这个拓展有重要的应用。但对这个理论中的大部分情况而言,我们都可以将情况约化到$B$是一个仿射概形,故而在这里,我们对一般性的追求也仅止于此。

为了描述这个构造,我们从一个正分次$A$-代数开始,其中$A$作为这个代数的$0$次部分,即一个$A$代数$S$具有分次
\[
	S=\bigoplus_{\nu=0}^\infty S_\nu\quad \text{作为$A$-模}
\]
使得
\[
	S_\nu S_\mu \subset S_{\nu+\mu}\quad \text{以及} \quad S_0=A.
\]
$S$中的一个元素如果处于$S_\nu$中,则它被称为\textit{$\nu$次齐次}的。我们将从$S$定义一个$A$-概形$X=\proj S$. 在$A$上射影的概形就被定义为具有形式$\proj S$的概形,其中$S$是一个有限生成$A$-代数。代数$S$被称为$X$的\textit{齐次坐标环},尽管(类似于射影簇的齐次坐标环)实际上他并不由$X$所决定。

当$S$是$A$上的多项式环
\[
	S=A[\text{$x_0$, $\dots$, $x_r$}],
\]
具有如下分次:$A$中的元素是零次的,而每一个变量的分次都是$1$,则给出的概形$\proj S$被称为\textit{$A$上的射影$r$-空间},记作$\mathbb{P}_A^r$.(后面的习题将说明这个概形与第\ref{chap:1}章中定义的概形$\mathbb{P}_A^r$是相同的。)在$A=K$是一个域的例子中,概形

\subsection{\texorpdfstring{$\proj R$}{Proj R}的闭子概形}