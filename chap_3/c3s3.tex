\section{射影概形的不变量}\label{s:3.3}

在本节中,我们假设$K$是一个域,并且只在$K$-概形上工作,
除非明确提及相反的情况。

假设给定了射影空间中的一个概形,我们如何找到它的不变量?
最简单的想法是问:有多少独立的$d$-次形式在上面为零?
将对不同的$d$的答案放在一起,我们得到了以前被称为概形的假设
的东西(可能是因为当时人们对这些数给定的概形感兴趣)。 
现在,通常以等效形式的Hilbert函数来讨论这些信息。
我们将在这里讨论Hilbert函数方法的几种变体,
这些变体产生了一大类不变量。有些生成的不变量实际上
仅依赖于抽象概形,而不依赖于给定的射影嵌入,而其他的不变量则
依赖于嵌入; 所以我们将一路评注这些事情。我们采用的思路 % p.125
就是最初Hilbert [1890]使用的,而不是更被广为采用Samuel的
(比如可见Hartshorne [1977, Chapter I])。 Hilbert的方法
要求稍多一些的技巧但是能给出一个更强且更易理解的结果。

我们从定义基本的不变量开始。在这章的最后部分,我们将展示
一些简单的几何示例,说明不变量包含哪些信息。

\subsection{Hilbert函数与Hilbert多项式}\label{s:3.3.1}

首先,假设我们给定了一个闭子概形$X\subset \mathbb P^r_K$,
其由一个饱和理想$I=I(X)\subset S=K[x_0,\dots,x_r]$描述,
如Example \ref{exe:3.14}所定义的那样。假设,齐次多项式
$F_1$, $\dots$, $F_n$生成了$I$. 记$R=S/I(X)$是$X$的齐次
坐标环,再记$R_\nu$是$R$的$\nu$-次齐次分量。

这里基础的想法是对应$X\subset \mathbb P^r_K$给一个函数
\[
	H(X,\cdot):\mathbb N\to \mathbb N,
\]
它被称为$X$的\textit{Hilbert函数},由
\[
	H(X,\nu)=\dim_K R_\nu
\]
定义。更一般地,如果$M$是任意的有限生成分次$S$-模,我们
定义它的Hilbert函数为$H(M,\nu):=\dim_K M_\nu$. 下面是基本的
结论。

\begin{thm}[Hilbert]\label{thm:3.55}
存在唯一的$\nu$的多项式$P(X,\nu)$使得$H(X,\nu)=P(X,\nu)$对
所有足够大的$\nu$都成立。更一般地,对任意的有限生成分次$S$-模
$M$,存在一个唯一的多项式$P(M,\nu)$使得$H(M,\nu)=P(M,\nu)$对
所有足够大的$\nu$都成立。
\end{thm}

我们将在下面指出如何证明这一点(沿着Hilbert最初的证明
[1890]的思路)。

多项式$P(X,\nu)$称作$X$的\textit{Hilbert多项式}。就像在经典簇
中一样,它携带了概形$X$的基础信息。比方说,我们将看到它的次数
就是$X$的维数,当$X$是零维时,它的(常数)值就是$X$的度。
更一般地,我们定义$K$上的射影空间的任意$n$-维子概形$X$
的度为$n!$乘以$X$的Hilbert多项式的首项系数;这允许我们将经典的
簇的度推广到更大一类子概形$X\subset \mathbb P^r_K$上。

\subsection{平坦性 II:射影概形族的极限}\label{s:3.3.2}

Hilbert多项式的另一意义在于它给了我们平坦这个概念的几何解释。

% p.126

\begin{pro}\label{pro:3.56}
在约态连通基$B$上的射影空间的闭子概形族
$\mathscr X\subset \mathbb P_B^r$是平坦的当且仅当其所有纤维
具有相同的Hilbert多项式。
\end{pro}

一般情况的证明将把我们扯得太远,但是这个结论在
$B=\spec K[t]_{(t)}$时是简单的。

\begin{proof}[{当$B=\spec K\lbrack t\rbrack_{(t)}$时的证明}]
闭子概形$X\subset \mathbb P_K^r\times B$由
\[
	K[t]_{(t)}[x_0,\dots,x_r]
\]
中的理想$I$给出,且$I$对于$x_0$, $\dots$, $x_r$是齐次的。
因此齐次坐标环
\[
	R=K[t]_{(t)}[x_0,\dots,x_r]/I
\]
的每个分次部分都是一个$K[t]_{(t)}$-模。

我们已经知道,族$X\to B$是平坦的当且仅当每个局部环$\oo_{X,x}$
是$K[t]_{(t)}$-无挠的。这等价于说,如果我们让任意$x_i$是可逆的%
\footnote{译者注:这里意思应该是形式可逆,即考虑局部化。}%
,则$R$的挠子模将趋于零。于是,挠子模乘上理想
$(x_0,\dots,x_r)$的某个幂为零,而因此只与$R$的有限个分次部分
相交。但是如果$R_\nu$是$R$的一个分次部分,从$K[t]_{(t)}$是一个
主理想环以及$R_\nu$是一个有限生成$K[t]_{(t)}$-模,$R_\nu$是
无挠的当且仅当它是自由的。此外,$R_\nu$是自由的,如果他需要的
生成元个数,从Nakayama引理为
\[
	\dim_K R_\nu \otimes_{K[t]_{(t)}}K
\]
应该等于它的秩
\[
	\dim_{K(t)}R_\nu  \otimes_{K[t]_{(t)}}K(t),
\]
即,当且仅当$H(X_{(0)},\nu)$的值等于$H(X_{(t)},\nu)$的值,其中
$X_{(0)}$和$X_{(t)}$是族$X$在$B$的两个点$(0)$和$(t)$处的纤维。
(同时,Hilbert函数是常数当且仅当仿射锥族$\spec R$在$B$上是
平坦的。)
\end{proof}

这个命题告诉我们,射影概形的闭子概形的平坦族表现得比一般的
平坦族要好。比如,尽管一个仿射概形的非空子概形族的平坦极限
可能是空的,但这命题告诉我们,射影空间的子概形族的平坦极限
不会这样。这点连同单参闭子概形族平坦极限的存在和唯一性(
第 \ref{s:2.3.4} 节和第 \ref{s:2.3.4} 节),给出了一种证明
射影概形是逆紧的思路,使用所谓的“赋值判别法”。所有这些,
可见,比如,Hartshorne [1977, Chap. II].

当然,$H(X,\nu)$比$P(X,\nu)$包含了更多信息,似乎$P(X,\nu)$
作为一个只有有限个系数的多项式,比起整个Hilbert函数更容易% 
% p.127
操作。但实际上,使用二项式系数,Hilbert函数也有一个有限的
表达式。为看到这点,我们将引入一组更好的不变量,\textit{
$R$的自由分解的分次Betti数},使用它,Hilbert函数和Hilbert
多项式都可以被改写为更方便的形式。(Hilbert多项式比起
Hilbert函数真正的优点是它包含的信息依赖于稍少一些的 ---
下面我们会详细阐述 --- $X$的嵌入细节。)

\subsection{自由分解}\label{s:3.3.3}

我们将记$S(-b)$为具有次数为$b$的生成元的秩为$1$的分次自由模;
这里显然不幸的符号选取由更方便和容易记忆的公式
\[
	S(-b)_\nu =S_{\nu-b}
\]
所补偿。下面我们将用分次自由模来分解$R$,或实际上任意的分次
$S$-模,这里分次自由模是形如$S(-b)$的模的拷贝的直和。

假设$F_1$, $\dots$, $F_n$是$M$的齐次生成元的极小集,我们将记
$b_{0j}$为$F_j$的次数。定义一个满射
\[
	\varphi_0:E_0:=\bigoplus_{j=1}^n S(-b_{0j})\to M
\]
通过将$S(-b_{0j})$的生成元映射到$F_j\in M$. 令$M^{(1)}$为
$\varphi_0$的核。如果$M^{(0)}$非零,我们可以用$M^{(1)}$代替
$M$(它应被叫做$M^{(0)}$)重复操作;选一个$E_0$的极小齐次元组
$e^{(1)}_i$生成了$M^{(1)}$,$e^{(1)}_i$的次数为$b_{1i}$,我们
将一个生成元次数分别为$b_{1i}$的分次自由模映射到$M_1$,通过
\[
	\varphi_1:E_1:=\bigoplus_{j=1}^n S(-b_{1j})\to E_0
\]
将$E_1$的第$i$个生成元映射到$e_i^{(1)}$. 继续这样做下去,我们
得到了一个分解
\[
	E:\cdots \longrightarrow E_i 
	\xrightarrow{\,\,\varphi_i\,\,} E_{i-1}
	\longrightarrow \cdots  \xrightarrow{\,\,\varphi_1\,\,}
	E_0,
\]
其中
\[
	E_{i}=\bigoplus_j S(-b_{ij}).
\]
当然,这个过程在某个$\varphi_i$为单射时候就停止了,Hilbert的
重要发现即,如果$S$是多项式环,这总是会发生的。

% p.128

\begin{thm}[Hilbert合冲定理]
令$S=K[x_0,\dots,x_r]$. 在上面的任意极小自由分解中,对某个
$i\leq r+1$,即某个小于变量个数的数,$\varphi_i$是一个单射,
特别地,任意分次$S$-模有一个有限、分次、自由分解。
\end{thm}

这里我们将不证明这点,见Hilbert [1890],或更现代的陈述,Eisenbud
[1995, Section 1.10, Chap. 19]或Matsumura [1986, Theorem 19.5].
合冲定理让我们可以证明Theorem \ref{thm:3.55}.

\begin{proof}[{Theorem \ref{thm:3.55}的证明}]
模$S(-b)$的Hilbert方程是容易写出的。因为
\[
	S(-b)_\nu=S_{\nu-b}
\]
有一个基包含所有的次数为$\nu-b$的$r+1$变量首一多项式,所以
\[
	H(S(-b),\nu)=\binom{r+\nu-b}r,
\]
其中二项式系数当下面的数大于上面的时候为零。对$\nu\geq b-r$
时候,它与多项式
\[
	P(S(-b),\nu)=\frac{(r+\nu-b)(r+\nu-b-1)\cdots (\nu-b)}
	{r(r-1)\cdots 1}
\]
相同,所以我们看到$H(X,\nu)$对大$\nu$来说是一个多项式。

作为$S$-模,从$M$的一个有限、自由分解
\[
	E:0 \longrightarrow E_{r+1} 
	\xrightarrow{\varphi_{r+1}}E_{i}
	\longrightarrow \cdots  \longrightarrow E_1 
	\longrightarrow M \longrightarrow 0,
\]
其中
\[
	E_i=\bigoplus_j S(-b_{ij}).
\]
我们看到,$M$的Hilbert函数可以被写为
\[
	H(M,\nu)=\sum_{i=0}^r (-1)^i H(E_i,\nu)
	=\sum_{i=0}^r(-1)^i\sum_j H(S(-b_{ij}),\nu)
\]
的形式。因为我们已经说明了,每个$H(S(-b_{ij}),\nu)$
对大$\nu$是一个多项式,所以$H(M,\nu)$对大$\nu$也是多项式,
这就是我们需要的。Theorem \ref{thm:3.55}得证。
\end{proof}

显然,Hilbert函数和多项式是$X\subset \mathbb P_k^r$的不变量,
但\textit{分次Betti数}$b_{ij}$是不变量可能并不如此显然。这
来自于Nakayama引理,比如可见,Eisenbud [1995, Chap. 19] 或者
Matsumura [1986, Section 19] 上关于局部环上的极小自由分解的
讨论,它可以直接翻译到分次的情况。

因此,我们有了射影概形的三族逐步减弱的不变量:分次Betti数、
Hilbert函数以及Hilbert多项式。为方便读者,我们将列出关于它们
的一些事实,这些事实并不会在这里证明,同样也不会有必要的使用,
接着我们会给一些例子。

(1) 已经提到过,多项式$P(X,\nu)$的次数$d$是$X$的维数。

(2) 首项形如
\[
  \frac{\delta(X)}{d!}\nu^d,
\]
其中$\delta(X)$被称为$X$的\textit{度}。它可以等同于$X$与
$\mathbb P_K^r$的一个$r-d$维一般平面的相交子概形的长度。
(比如可见,Hartshorne [1977, Chapter I, 7.3 and 7.7].)
这来自于下面将会证明的观察(Proposition \ref{exe:3.59}),
$\mathbb P_K^n$中度为$\delta$的零维子概形的Hilbert多项式
为常多项式$\delta$,以及如果$Y$是$X$的一个一般的超平面截面,
则$Y$的Hilbert多项式为$X$的Hilbert多项式的一阶差分,即
\[
  P(Y,\nu)=P(X,\nu)-P(X,\nu-1).
\]

(3) 用第 \ref{s:3.2.5} 节的关于到射影空间的映射的刻画,
子概形$X\subset \mathbb P_K^r$的Hilbert多项式仅依赖于嵌入
$X\hookrightarrow \mathbb P_K^r$对应的可逆层$\mathscr L$,
而不依赖于特定的满射$\mathscr O_X^{r+1}\to \mathscr L$. 
实际上,对熟悉凝聚层的上同调的读者,对所有的$\nu$,
$P(X,\nu)$等于上同调群维数的交错和
\[
  \chi(\mathscr L^{\otimes nu})=\sum (-1)^i \dim_K H^i
  (X,\mathscr L^{\otimes \nu}).
\]
特别地,$P(X,0)=\chi(\mathscr O_X)=\sum (-1)^i \dim_K
H^i(\mathscr O_X)$是一个仅依赖于$X$而不依赖于嵌入!当$X$是
复数域上的非奇异曲线(即Riemann面),数
\[
  \dim_K H^1(\mathscr O_X)=g=1-P(X,0)
\]
是$X$的\textit{亏格},因而$1-P(X,0)$是任意一维概形的亏格的
正确概念。这被称为概形的\textit{算术亏格}。在$X$的维数$d$大于
$1$的情况,第一感会觉得一般情况是$H^1(\mathscr O_X)=0$对
$1< i < d$成立(以及这个上同调群在$i>d$恒为零),于是$X$的算术
亏格可类似定义为$1+(-1)^d P(X,0)$.

(4) $\mathbb P_K^r$中所有Hilbert多项式等于同一给定概形的簇的
全体,可以自然等同于一个射影概形的$K$-值点的集合,它被称为
对应于给定多项式的\textit{Hilbert 概形}. 比方说,任意Hilbert
多项式为$P(\nu)=\binom{\kappa+\nu}{k}$(即一个$k$-平面
的Hilbert多项式)的子概形$X\subset \mathbb P_K^r$实际上就是一个
$k$-平面,因此所有这样的子概形的Hilbert概形就是Grassmannian%
% p.130
$\mathbb G(k,r)=G(k+1,r+1)$. 然而,可以在几何上理解的
Hilbert概形的其他情况并不多见!我们将在最后一章的
第 \ref{s:6.2.2} \nottran 节回到这个构造。

\begin{exe}\label{exe:3.58}
令$A$是一个Noether环,而$\mathscr X$是一个$\mathbb P_A^n$的
闭子概形,可将其视作$\spec A$上的一个概形族。因为$\mathscr X$
在点$p\in \spec A$的纤维$X_p$是$\mathbb P_{\kappa(p)}^n$的一个
闭子概形,它有一个Hilbert函数$H(X_p,\nu)$. 证明函数$H(X_p,\nu)$,
看成$p$的函数,在$\spec A$的Zariski拓扑中是上半连续的,
即对任意$\nu$和数$m$,
\[
	\{p\in \spec A\;:\; H(X_p,\nu)\geq m\}
\]
是一个$\spec A$的闭子集。
\end{exe}

我们可将Hilbert 多项式的定义推广到任意不可约基$S$上的射影空间
的子概形$X\subset \mathbb P_S^r$上,通过定义$P(X,\nu)$为$X$
在$S$的一般点上的纤维的Hilbert多项式。这并没有引入任何新东西,
--- 由第 \ref{s:2.3.4} 中的一般平坦性定理以及 Proposition 
\ref{pro:3.56},或由 Exercise \ref{exe:3.58},$X$将在一个稠密
开子集$U\subset S_{\text{red}}$上平坦,于是$P(X,\nu)$就是$X_U$
在$U$上的纤维的寻常的Hilbert多项式 --- 但这是方便的术语。

(5) 在许多方面,由分次Betti数给出的不变量是最微妙的,
直到最近,除了Hilbert函数和多项式之外,
没有人知道它的几何意义。然而,今天我们知道,在一些情况下
(以及更多的猜想),它们如何反映了$X$内蕴几何的一些微妙性质。
更多信息比如可参见Green [1984]; Green and Lazarsfeld [1985].

\subsection{例子}\label{s:3.3.4}
\paragraph*{平面上的点}
\addcontentsline{toc}{subsubsection}{平面上的点}
对平面上的零维子概形,我们已然可从Hilbert多项式、Hilbert函数
和分次Betti数得到不同的信息。

首先,我们在上面说过了,子概形$X\subset \mathbb P_K^r$的
Hilbert多项式的次数等于$X$的维数;所以当$X$是零维的时候,
Hilbert多项式是一个常数。在点的情况中,我们可以容易地证明
它以及更多。

\begin{pro}\label{pro:3.59}
  一个度为$\delta$的$\mathbb P_K^r$的零维子概形的Hilbert
  函数对所有的$\nu$满足
  \[H(X,\nu)\leq \delta,\]
  且对大$\nu$上式取等号。因此,$P(X,\nu)\equiv \delta$.
\end{pro}

\begin{proof}
我们要证明,$K[x_0,\dots,x_r]$中在$X$上为零的$\nu$-次齐次形式%
\footnote{译者注:一个齐次多项式定义的函数被称为一个形式
(form),有时候干脆就把一个齐次多项式称为形式。这点依赖于
语境。}%
的集合的余维数,即$\codim I(X)_\nu$,小于或等于$\delta$,且对
大$\nu$是等式。理由是,多项式在一点上为零是其系数的线性条件,
于是在$X$为零对应$\delta$个线性条件;对大$\nu$,我们将证明
这些条件永远是线性无关的。

为明确之,我们转进到仿射开集的情况。改变坐标,我们可以假设$X$
包含于$x_r\neq 0$的仿射开集中,于是一个次数为$\nu$的形式
$F$属于$I(X)$当且仅当$F(x_0,\dots,x_r,1)$属于$X$在$x_r\neq 0$
的仿射开集的理想$J\subset K[x_0,\dots,x_{r-1}]$中。称$X$的长度
为$\delta$,就是说$J$在$K[x_0,\dots,x_{r-1}]$中的余维数为
$\delta$,因此,余维数小于等于$\delta$的多项式属于可以写成
$F(x_0,\dots,x_{r-1},1)$的$\nu$-次多项式$F$构成的空间中,而这
即是$K[x_0,\dots,x_{r-1}]$中次数小于等于$\nu$的多项式
构成的空间。这立刻说明$H(X,\nu)\leq \delta$对所有$\nu$成立,
且当$J$在次数小于等于$\nu$的多项式空间中的余维数为$\delta$时取
等号。但是,想要$J$在次数小于等于$\nu$的多项式空间中的余维数为
$\delta$,只要能从中找到一族$K[x_0,\dots,x_{r-1}]/J$的代表元
即可,这在所有足够大的$\nu$的情况总是成立的。
\end{proof}

如果$X\subset \mathbb P_K^r$非空,$I(X)$不包含任意零次元(我们
在域上工作!),所以$H(X,0)=1$. 因此,这个命题给出了一个
$P(X,0)\neq H(X,0)$的简单例子。

我们可以容易给出一族$\mathbb P_K^2$的子概形,它们有相同的
Hilbert多项式但不同的Hilbert函数。为构造这样的一个族
$\mathscr X\subset \mathbb P_K^2\times \spec K[t]$,比如,我们
可以取“常值”点$P$和$Q$,分别由$(x_2=x_1+x_0=0)$和
$(x_2=x_1-x_0=0)$给出,以及由$(x_1=x_2-tx_0)$给出的
跑动的点$R$,再令$\mathscr X$为$P$, $Q$, $R$在$\mathbb P_K^2
\times \spec K[t]$中的(不交)并。
\inclugra{3.png} % p.132
通过投影的第二个因子,我们将$\mathscr X$看成一个$\spec K[t]$
上的平坦族,其在一般点和每个闭点$(t-\lambda)$上的纤维
$X_{(0)}$和$X_\lambda$(分别作为$K(t)$和$K$上的概形)有着
Hilbert多项式$P(X_{(0)},\nu)=P(X_\lambda,\nu)\equiv 3$,但对
$\lambda =0$有Hilbert函数$H(X_{\lambda},1)=3$,而$H(X_0,1)=2$.

\begin{exe}\label{exe:3.60}
令$\tilde{X}_{\lambda}\subset \A_K^3$为上面的族$\mathscr X$的
纤维$X_\lambda$上的锥。证明,不存在平坦族$\tilde{\mathscr X}
\subset \A_K^3\times_K \spec K[t]$,其在每个点$(t-\lambda)$上
的纤维是$\tilde{\mathscr X}_\lambda$. (但在$\spec K[t]$的
补集上存在这样的一个族。)锥$\tilde{\mathscr X}_\lambda$在
$\lambda$趋向于$0$时候的平坦极限是上面?(见第 \ref{s:2.3.4}
节的例子。)
\end{exe}

\subsection{B\'{e}zout定理}\label{s:3.3.5}

\subsection{Hilbert级数}\label{s:3.3.6}
