\section{射影概形的不变量}\label{s:3.3}

在本节中,我们假设$K$是一个域,并且只在$K$-概形上工作,
除非明确提及相反的情况。

假设给定了射影空间中的一个概形,我们如何找到它的不变量?
最简单的想法是问:有多少独立的$d$-次型%
\footnote{译者注:一个$\nu$-次齐次多项式定义的函数被称为
一个$\nu$-次型(form),有时候干脆就把一个齐次多项式称为型。
这点依赖于语境。}%
在上面为零?把对不同的$d$的答案放在一起,
我们得到了以前被称为概形的假设的东西
(可能是因为当时人们对这些数给定的概形感兴趣)。 
现在,通常以等效形式的Hilbert函数来讨论这些信息。
我们将在这里讨论Hilbert函数方法的几种变体,
这些变体产生了一大类不变量。有些生成的不变量实际上
仅依赖于抽象概形,而不依赖于给定的射影嵌入,而其他的不变量则
依赖于嵌入; 所以我们将一路评注这些事情。我们采用的思路%
% p.125
就是最初Hilbert [1890]使用的,而不是更被广为采用Samuel的
(比如可见Hartshorne [1977, Chapter I])。 Hilbert的方法
要求稍多一些的技巧但是能给出一个更强且更易理解的结果。

我们从定义基本的不变量开始。在这章的最后部分,我们将展示
一些简单的几何示例,说明不变量包含哪些信息。

\subsection{Hilbert函数与Hilbert多项式}\label{s:3.3.1}

首先,假设我们给定了一个闭子概形$X\subset \mathbb P^r_K$,
其由一个饱和理想$I=I(X)\subset S=K[x_0,\dots,x_r]$描述,
如Example \ref{exe:3.14} 所定义的那样。假设,齐次多项式
$F_1$, $\dots$, $F_n$生成了$I$. 记$R=S/I(X)$是$X$的齐次
坐标环,再记$R_\nu$是$R$的$\nu$-次齐次分量。

这里基础的想法是对应$X\subset \mathbb P^r_K$给一个函数
\[
	H(X,\cdot):\mathbb N\to \mathbb N,
\]
它被称为$X$的\textit{Hilbert函数},由
\[
	H(X,\nu)=\dim_K R_\nu
\]
定义。更一般地,如果$M$是任意的有限生成分次$S$-模,我们
定义它的Hilbert函数为$H(M,\nu):=\dim_K M_\nu$. 下面是基本的
结论。

\begin{thm}[Hilbert]\label{thm:3.55}
存在唯一的$\nu$的多项式$P(X,\nu)$使得$H(X,\nu)=P(X,\nu)$对
所有足够大的$\nu$都成立。更一般地,对任意的有限生成分次$S$-模
$M$,存在一个唯一的多项式$P(M,\nu)$使得$H(M,\nu)=P(M,\nu)$对
所有足够大的$\nu$都成立。
\end{thm}

我们将在下面指出如何证明这一点(沿着Hilbert最初的证明
[1890]的思路)。

多项式$P(X,\nu)$称作$X$的\textit{Hilbert多项式}。就像在经典簇
中一样,它携带了概形$X$的基础信息。比方说,我们将看到它的次数
就是$X$的维数,当$X$是零维时,它的(常数)值就是$X$的次数。
更一般地,我们定义$K$上的射影空间的任意$n$-维子概形$X$
的次数为$n!$乘以$X$的Hilbert多项式的首项系数;这允许我们将经典的
簇的次数推广到更大一类子概形$X\subset \mathbb P^r_K$上。

\subsection{平坦性 II:射影概形族的极限}\label{s:3.3.2}

Hilbert多项式的另一意义在于它给了我们平坦这个概念的几何解释。

% p.126

\begin{pro}\label{pro:3.56}
在约态连通基$B$上的射影空间的闭子概形族
$\mathscr X\subset \mathbb P_B^r$是平坦的当且仅当其所有纤维
具有相同的Hilbert多项式。
\end{pro}

一般情况的证明将把我们扯得太远,但是这个结论在
$B=\spec K[t]_{(t)}$时是简单的。

\begin{proof}[{当$B=\spec K\lbrack t\rbrack_{(t)}$时的证明}]
闭子概形$X\subset \mathbb P_K^r\times B$由
\[
	K[t]_{(t)}[x_0,\dots,x_r]
\]
中的理想$I$给出,且$I$对于$x_0$, $\dots$, $x_r$是齐次的。
因此齐次坐标环
\[
	R=K[t]_{(t)}[x_0,\dots,x_r]/I
\]
的每个分次部分都是一个$K[t]_{(t)}$-模。

我们已经知道,族$X\to B$是平坦的当且仅当每个局部环$\oo_{X,x}$
是$K[t]_{(t)}$-无挠的。这等价于说,如果我们让任意$x_i$是可逆的%
\footnote{译者注:这里意思应该是形式可逆,即考虑局部化。}%
,则$R$的挠子模将趋于零。于是,挠子模乘上理想
$(x_0,\dots,x_r)$的某个幂为零,而因此只与$R$的有限个分次部分
相交。但是如果$R_\nu$是$R$的一个分次部分,从$K[t]_{(t)}$是一个
主理想环以及$R_\nu$是一个有限生成$K[t]_{(t)}$-模,$R_\nu$是
无挠的当且仅当它是自由的。此外,$R_\nu$是自由的,如果他需要的
生成元个数,从Nakayama引理为
\[
	\dim_K R_\nu \otimes_{K[t]_{(t)}}K
\]
应该等于它的秩
\[
	\dim_{K(t)}R_\nu  \otimes_{K[t]_{(t)}}K(t),
\]
即,当且仅当$H(X_{(0)},\nu)$的值等于$H(X_{(t)},\nu)$的值,其中
$X_{(0)}$和$X_{(t)}$是族$X$在$B$的两个点$(0)$和$(t)$处的纤维。
(同时,Hilbert函数是常数当且仅当仿射锥族$\spec R$在$B$上是
平坦的。)
\end{proof}

这个命题告诉我们,射影概形的闭子概形的平坦族表现得比一般的
平坦族要好。比如,尽管一个仿射概形的非空子概形族的平坦极限
可能是空的,但这命题告诉我们,射影空间的子概形族的平坦极限
不会这样。这点连同单参闭子概形族平坦极限的存在和唯一性(
第 \ref{s:2.3.4} 节和第 \ref{s:2.3.4} 节),给出了一种证明
射影概形是逆紧的思路,使用所谓的“赋值判别法”。所有这些,
可见,比如,Hartshorne [1977, Chap. II].

当然,$H(X,\nu)$比$P(X,\nu)$包含了更多信息,似乎$P(X,\nu)$
作为一个只有有限个系数的多项式,比起整个Hilbert函数更容易% 
% p.127
操作。但实际上,使用二项式系数,Hilbert函数也有一个有限的
表达式。为看到这点,我们将引入一组更好的不变量,\textit{
$R$的自由分解的分次Betti数},使用它,Hilbert函数和Hilbert
多项式都可以被改写为更方便的形式。(Hilbert多项式比起
Hilbert函数真正的优点是它包含的信息依赖于稍少一些$X$的嵌入
细节,这点下面我们会详细阐述。)

\subsection{自由分解}\label{s:3.3.3}

我们将记$S(-b)$为具有次数为$b$的生成元的秩为$1$的分次自由模;
这里显然不幸的符号选取由更方便和容易记忆的公式
\[
	S(-b)_\nu =S_{\nu-b}
\]
所补偿。下面我们将用分次自由模来分解$R$,或实际上任意的分次
$S$-模,这里分次自由模是形如$S(-b)$的模的拷贝的直和。

假设$F_1$, $\dots$, $F_n$是$M$的齐次生成元的极小集,我们将记
$b_{0j}$为$F_j$的次数。定义一个满射
\[
	\varphi_0:E_0:=\bigoplus_{j=1}^n S(-b_{0j})\to M
\]
通过将$S(-b_{0j})$的生成元映射到$F_j\in M$. 令$M^{(1)}$为
$\varphi_0$的核。如果$M^{(0)}$非零,我们可以用$M^{(1)}$代替
$M$(它应被叫做$M^{(0)}$)重复操作;选一个$E_0$的极小齐次元组
$e^{(1)}_i$生成了$M^{(1)}$,$e^{(1)}_i$的次数为$b_{1i}$,我们
将一个生成元次数分别为$b_{1i}$的分次自由模映射到$M_1$,通过
\[
	\varphi_1:E_1:=\bigoplus_{j=1}^n S(-b_{1j})\to E_0
\]
将$E_1$的第$i$个生成元映射到$e_i^{(1)}$. 继续这样做下去,我们
得到了一个分解
\[
	E:\cdots \longrightarrow E_i 
	\xrightarrow{\,\,\varphi_i\,\,} E_{i-1}
	\longrightarrow \cdots  \xrightarrow{\,\,\varphi_1\,\,}
	E_0,
\]
其中
\[
	E_{i}=\bigoplus_j S(-b_{ij}).
\]
当然,这个过程在某个$\varphi_i$为单射时候就停止了,Hilbert的
重要发现即,如果$S$是多项式环,这总是会发生的。

% p.128

\begin{thm}[Hilbert合冲定理]
令$S=K[x_0,\dots,x_r]$. 在上面的任意极小自由分解中,对某个
$i\leq r+1$,即某个小于变量个数的数,$\varphi_i$是一个单射,
特别地,任意分次$S$-模有一个有限、分次、自由分解。
\end{thm}

这里我们将不证明这点,见Hilbert [1890],或更现代的陈述,
Eisenbud [1995, Section 1.10, Chap. 19] 或 
Matsumura [1986, Theorem 19.5].
合冲定理让我们可以证明Theorem \ref{thm:3.55}.

\begin{proof}[{Theorem \ref{thm:3.55} 的证明}]
模$S(-b)$的Hilbert函数是容易写出的。因为
\[
	S(-b)_\nu=S_{\nu-b}
\]
有一个基包含所有的次数为$\nu-b$的$r+1$变量首一多项式,所以
\[
	H(S(-b),\nu)=\binom{r+\nu-b}r,
\]
其中二项式系数当下面的数大于上面的时候为零。对$\nu\geq b-r$
时候,它与多项式
\[
	P(S(-b),\nu)=\frac{(r+\nu-b)(r+\nu-b-1)\cdots (\nu-b)}
	{r(r-1)\cdots 1}
\]
相同,所以我们看到$H(X,\nu)$对大$\nu$来说是一个多项式。

作为$S$-模,从$M$的一个有限、自由分解
\[
	E:0 \longrightarrow E_{r+1} 
	\xrightarrow{\varphi_{r+1}}E_{i}
	\longrightarrow \cdots  \longrightarrow E_1 
	\longrightarrow M \longrightarrow 0,
\]
其中
\[
	E_i=\bigoplus_j S(-b_{ij}).
\]
我们看到,$M$的Hilbert函数可以被写为
\[
	H(M,\nu)=\sum_{i=0}^r (-1)^i H(E_i,\nu)
	=\sum_{i=0}^r(-1)^i\sum_j H(S(-b_{ij}),\nu)
\]
的形式。因为我们已经说明了,每个$H(S(-b_{ij}),\nu)$
对大$\nu$是一个多项式,所以$H(M,\nu)$对大$\nu$也是多项式,
这就是我们需要的。Theorem \ref{thm:3.55} 得证。
\end{proof}

显然,Hilbert函数和多项式是$X\subset \mathbb P_k^r$的不变量,
但\textit{分次Betti数}$b_{ij}$是不变量可能并不如此显然。这
来自于Nakayama引理,比如可见,Eisenbud [1995, Chap. 19] 或者
Matsumura [1986, Section 19] 上关于局部环上的极小自由分解的
讨论,它可以直接翻译到分次的情况。

因此,我们有了射影概形的三族逐步减弱的不变量:分次Betti数、
Hilbert函数以及Hilbert多项式。为方便读者,我们将列出关于它们
的一些事实,这些事实并不会在这里证明,同样也不会有必要的使用,
接着我们会给一些例子。

(1) 已经提到过,多项式$P(X,\nu)$的次数$d$是$X$的维数。

(2) 首项形如
\[
  \frac{\delta(X)}{d!}\nu^d,
\]
其中$\delta(X)$被称为$X$的\textit{次数}。它可以等同于$X$与
$\mathbb P_K^r$的一个$r-d$维一般的平面%
\footnote{译者注:“一般的平面”的意思
见第 \pageref{p:80} 页的译者注。}%
的相交子概形的长度。
(比如可见,Hartshorne [1977, Chapter I, 7.3 and 7.7].)
这来自于下面将会证明的观察(Proposition \ref{pro:3.59}),
$\mathbb P_K^n$中次数为$\delta$的零维子概形的Hilbert多项式
为常多项式$\delta$,以及如果$Y$是$X$的一个一般的超平面截面,
则$Y$的Hilbert多项式为$X$的Hilbert多项式的一阶差分,即
\[
  P(Y,\nu)=P(X,\nu)-P(X,\nu-1).
\]

(3) 用第 \ref{s:3.2.5} 节的关于到射影空间的映射的刻画,
子概形$X\subset \mathbb P_K^r$的Hilbert多项式仅依赖于嵌入
$X\hookrightarrow \mathbb P_K^r$对应的可逆层$\mathscr L$,
而不依赖于特定的满射$\mathscr O_X^{r+1}\to \mathscr L$. 
实际上,对熟悉凝聚层的上同调的读者,对所有的$\nu$,
$P(X,\nu)$等于上同调群维数的交错和
\[
  \chi(\mathscr L^{\otimes nu})=\sum (-1)^i \dim_K H^i
  (X,\mathscr L^{\otimes \nu}).
\]
特别地,$P(X,0)=\chi(\mathscr O_X)=\sum (-1)^i \dim_K
H^i(\mathscr O_X)$是一个仅依赖于$X$而不依赖于嵌入!当$X$是
复数域上的非奇异曲线(即Riemann面),数
\[
  \dim_K H^1(\mathscr O_X)=g=1-P(X,0)
\]
是$X$的\textit{亏格},因而$1-P(X,0)$是任意一维概形的亏格的
正确概念。这被称为概形的\textit{算术亏格}。在$X$的维数$d$大于
$1$的情况,第一感会觉得一般情况是$H^1(\mathscr O_X)=0$对
$1< i < d$成立(以及这个上同调群在$i>d$恒为零),于是$X$的算术
亏格可类似定义为$1+(-1)^d P(X,0)$.

(4) $\mathbb P_K^r$中所有Hilbert多项式等于同一给定概形的簇的
全体,可以自然等同于一个射影概形的$K$-值点的集合,它被称为
对应于给定多项式的\textit{Hilbert 概形}. 比方说,任意Hilbert
多项式为$P(\nu)=\binom{\kappa+\nu}{k}$(即一个$k$-平面
的Hilbert多项式)的子概形$X\subset \mathbb P_K^r$实际上就是一个
$k$-平面,因此所有这样的子概形的Hilbert概形就是Grassmannian %
% p.130
$\mathbb G(k,r)=G(k+1,r+1)$. 然而,可以在几何上理解的
Hilbert概形的其他情况并不多见!我们将在最后一章的
第 \ref{s:6.2.2} 和 \ref{s:6.2.2} 节回到这个构造。

\begin{exe}\label{exe:3.58}
令$A$是一个Noether环,而$\mathscr X$是一个$\mathbb P_A^n$的
闭子概形,可将其视作$\spec A$上的一个概形族。因为$\mathscr X$
在点$p\in \spec A$的纤维$X_p$是$\mathbb P_{\kappa(p)}^n$的一个
闭子概形,它有一个Hilbert函数$H(X_p,\nu)$. 证明函数$H(X_p,\nu)$,
看成$p$的函数,在$\spec A$的Zariski拓扑中是上半连续的,
即对任意$\nu$和数$m$,
\[
	\{p\in \spec A\;:\; H(X_p,\nu)\geq m\}
\]
是一个$\spec A$的闭子集。
\end{exe}

我们可将Hilbert多项式的定义推广到任意不可约基$S$上的射影空间
的子概形$X\subset \mathbb P_S^r$上,通过将$P(X,\nu)$定义为$X$
在$S$的一般点处的纤维的Hilbert多项式。这并没有引入任何新东西,
由第 \ref{s:2.3.4} 节中的一般平坦性定理以及 Proposition 
\ref{pro:3.56},或由 Exercise \ref{exe:3.58},$X$将在一个稠密
开子集$U\subset S_{\text{red}}$上平坦,于是$P(X,\nu)$就是$X_U$
在$U$上的纤维们的寻常的Hilbert多项式,但这是方便的术语。

(5) 在许多方面,由分次Betti数给出的不变量是最微妙的,
直到最近,除了Hilbert函数和多项式之外,
没有人知道它的几何意义。然而,今天我们知道,在一些情况下
(以及更多的猜想),它们如何反映了$X$内蕴几何的一些微妙性质。
更多信息比如可参见Green [1984]; Green and Lazarsfeld [1985].

\subsection{例子}\label{s:3.3.4}
\paragraph*{平面上的点}
\addcontentsline{toc}{subsubsection}{平面上的点}
对平面上的零维子概形,我们已然可从Hilbert多项式、Hilbert函数
和分次Betti数得到不同的信息。

首先,我们在上面说过了,子概形$X\subset \mathbb P_K^r$的
Hilbert多项式的次数等于$X$的维数;所以当$X$是零维的时候,
Hilbert多项式是一个常数。在点的情况中,我们可以容易地证明
它以及更多。

\begin{pro}\label{pro:3.59}
	一个次数为$\delta$的$\mathbb P_K^r$的零维子概形的Hilbert
	函数对所有的$\nu$满足
	\[H(X,\nu)\leq \delta,\]
	且对大$\nu$上式取等号。因此,$P(X,\nu)\equiv \delta$.
\end{pro}

\begin{proof}
我们要证明,$K[x_0,\dots,x_r]$中在$X$上为零的$\nu$-次型
的集合的余维数,即$\codim I(X)_\nu$,小于或等于$\delta$,且对
大$\nu$是等式。理由是,多项式在一点上为零是其系数的线性条件,
于是在$X$为零对应$\delta$个线性条件;对大$\nu$,我们将证明
这些条件永远是线性无关的。

为明确之,我们转进到仿射开集的情况。改变坐标,我们可以假设$X$
包含于$x_r\neq 0$的仿射开集中,于是一个$\nu$-次型
$F$属于$I(X)$当且仅当$F(x_0,\dots,x_r,1)$属于$X$在$x_r\neq 0$
的仿射开集的理想$J\subset K[x_0,\dots,x_{r-1}]$中。称$X$的长度
为$\delta$,就是说$J$在$K[x_0,\dots,x_{r-1}]$中的余维数为
$\delta$,因此,余维数小于等于$\delta$的多项式属于可以写成
$F(x_0,\dots,x_{r-1},1)$的$\nu$-次多项式$F$构成的空间中,而这
即是$K[x_0,\dots,x_{r-1}]$中次数小于等于$\nu$的多项式
构成的空间。这立刻说明$H(X,\nu)\leq \delta$对所有$\nu$成立,
且当$J$在次数小于等于$\nu$的多项式空间中的余维数为$\delta$时取
等号。但是,想要$J$在次数小于等于$\nu$的多项式空间中的余维数为
$\delta$,只要能从中找到一族$K[x_0,\dots,x_{r-1}]/J$的代表元
即可,这在所有足够大的$\nu$的情况总是成立的。
\end{proof}

如果$X\subset \mathbb P_K^r$非空,$I(X)$不包含任意零次元(我们
在域上工作!),所以$H(X,0)=1$. 因此,这个命题给出了一个
$P(X,0)\neq H(X,0)$的简单例子。

我们可以容易给出一族$\mathbb P_K^2$的子概形,它们有相同的
Hilbert多项式但不同的Hilbert函数。为构造这样的一个族
$\mathscr X\subset \mathbb P_K^2\times \spec K[t]$,比如,我们
可以取“常值”点$P$和$Q$,分别由$(x_2=x_1+x_0=0)$和
$(x_2=x_1-x_0=0)$给出,以及由$(x_1=x_2-tx_0)$给出的
跑动的点$R$,再令$\mathscr X$为$P$, $Q$, $R$在$\mathbb P_K^2
\times \spec K[t]$中的(不交)并。
\inclugra{3.png}% 
% p.132
通过投影的第二个因子,我们将$\mathscr X$看成一个$\spec K[t]$
上的平坦族,其在一般点和每个闭点$(t-\lambda)$上的纤维
$X_{(0)}$和$X_\lambda$(分别作为$K(t)$和$K$上的概形)有着
Hilbert多项式$P(X_{(0)},\nu)=P(X_\lambda,\nu)\equiv 3$,但对
$\lambda =0$有Hilbert函数$H(X_{\lambda},1)=3$,而$H(X_0,1)=2$.

\begin{exe}\label{exe:3.60}
令$\tilde{X}_{\lambda}\subset \A_K^3$为上面的族$\mathscr X$的
纤维$X_\lambda$上的锥。证明,不存在平坦族$\tilde{\mathscr X}
\subset \A_K^3\times_K \spec K[t]$,其在每个点$(t-\lambda)$上
的纤维是$\tilde{\mathscr X}_\lambda$. (但在$\spec K[t]$的
补集上存在这样的一个族。)锥$\tilde{\mathscr X}_\lambda$在
$\lambda$趋向于$0$时候的平坦极限是上面?(见第 \ref{s:2.3.4}
节的例子。)
\end{exe}

现在考虑当$X$是平面$\mathbb P_K^2$上四个不同点的情况。我们已经
知道$P(X,\nu)\equiv 4$. 下面将分别处理,所有点都在一条直线上
或者除了一点外全在一条直线上的情况。

(1) \textit{$X$处于一条直线上}。首先假设,点都处于方程为$l=0$
的直线$L$上。唯一包含$X$的直线是$L$,于是
\[
	H(X,1)=H(\mathbb P_K^2,1)-1=2.
\]
如果$q=0$是一个包含$X$的锥的方程,则$q$限制在$L$上是一个
二次型,在$X$的四点处为零,于是$q$必须在$L$上为零。因此,
$q=0$是$L$与另一条直线的并,而包含$X$的锥的方程的集合是
$l$乘以一个一次型构成的三维空间。这给出了
\[
	H(X,2)=H(\mathbb P_K^2,2)-3=3.
\]

然而你,从$\nu=3$开始,在四点为零给出了$\nu$-次型的四个独立
条件,于是$H(X,\nu)=4$. 为证明这点,只需对$X$的每个
三点集$X'$,找一个次数为$\nu$的曲线包含$X'$但不包含$X$的
第四个点。而这可以通过找$\nu$条直线,其中三条分别穿过$X'$的
一个点,而剩下的远离$X$,则包含这$\nu$条直线的曲线
即我们需要的。

\inclugra{4.png}
% p.133

为计算这个以及下个例子的极小自由分解,我们将使用一个Hilbert
的结果,他被Lindsay Burch推广到了局部情况。

\begin{thm}\label{thm:3.61}
如果$I$是一个零维子概形$X\subset \P_K^2$的齐次理想,则齐次
坐标环$S/I$的任意极小自由分解都具有形式
\[
	0\longrightarrow \sum_{j=1}^{n-1}S(-b_{2j})
	\xrightarrow{A}\sum_{j=1}^{n}S(-b_{1j})
	\longrightarrow S.
\]
此外,$I$的第$j$个生成元,即$S(-b_{1j})$在$S$中的像,
与矩阵$A$删去第$j$列的行列式只差一个非零标量。
\end{thm}

证明比如可见 Eisenbud [1995, Section 20.4].

下面的推论将用这个定理来计算理想$I(X)$的极小生成元。

\begin{coro}\label{coro:3.62}
如果$I$是一个零维子概形$X\subset \P_K^2$的齐次理想,且
如果$I$包含了一个$e$-次元素,则$I$可以由$e+1$个元素生成。
\end{coro}

\begin{proof}
如果$I$的极小生成元个数为$g$,那么$I$可以由一个矩阵$A$
的$(g-1)\times (g-1)$行列式所生成,其中$A$的元素都在$S$的分次
极大理想中,进而也都是正次型。所以,$I$中没有元素的次数小于
$g-1$,于是我们有$g\leq e+1$,此即所需。
\end{proof}

从定理,知道矩阵$A$的元素的次数等价于知道该情况中的分次
Betti数:$b_{ij}$就是$A$的子式的次数,而$b_{2j}$就是$b_{1j}$
加上$A$第$i$行第$j$列元素的次数。

将这点应用到手头上的例子,我们看到,因为$X$处于一条直线上,
$I(X)$可以由两个元素所生成,它们当然可被取作$L$和$I$中
一个不能被$L$整除的最小可能次数的型。

\inclugra{5.png}

已经说过,这个最小可能的次数为$4$,故我们可比如取$F$为一个
包含四条直线的四次方程,每一条穿过$X$的一个点。
% p.134
\inclugra{6.png}

因为$L$和$P$没有公共因子,我们看到,$S/I(X)$的极小自由分解
具有形式
\[
	0\longrightarrow S(-5)\longrightarrow S(-4)\oplus
	S(-1)\longrightarrow S,
\]
给出了Hilbert函数的表达式
\[
	H(X,\nu)=\binom{\nu+2}{2}-\binom{\nu+1}{2}
	-\binom{\nu-2}{2}+\binom{\nu-3}{2}.
\]

(2) \textit{$X$除了一点外都在一条直线上}。
接着,考虑四点中只有三点在一条直线$L$上的情况。现在,$I(X)$
中没有一次型了,所以$H(X,1)=3$.

\inclugra{7.png}

所有包含$L$上的三个点的二次型都必须,如之前的断言,包含$L$,
所以任意包含$X$的二次型是$L$和一条穿过第四个点的直线的并。
因为对应于穿过第四点的直线的一次型的空间是二维的,则包含$X$
的二次型的空间也是二维的,故我们有$H(X,2)=4$. 重复之前相同的
论断,我们可以证明$H(X,\nu)=4$对所有大$\nu$都成立,于是这里
是对所有$\nu\geq 2$的情况都成立。

% p. 135

对于分解,我们从上面的推论看到,$I(X)$至少需要三个生成元。但是
$I(X)$并不由其包含的两个独立二次型所生成,因为它们有一个
公共因子,于是,他的极小生成元集包含被二次型,还有一个其他的
生成元,它是不包含于两个二次型生成的理想中的最低可能次数型,
或者等价地,在不包含$L$的某个曲线上为零。不难看到,存在一个三次
曲线满足想要的性质,他可以被取作,比如,为三条直线的并,其中
每条直线穿过$L$上的点,而另一条穿过第四点。因为$I(X)$的极小生成元
集的次数分别为$2$, $2$, $3$,矩阵$A$必然是一个$2\times 3$矩阵,
其生成元的次数由下图给出(至多差一个行和列的重排)
\[
	\begin{pmatrix}
		1&2\\
		1&2\\
		0&1
	\end{pmatrix}.
\]
(当然,次数为$0$的元素必须就是$0$,因为所有的元素都必须包含于
极大分次理想中。)因此,极小自由分解具有形式
\[
	0\longrightarrow S(-3)\oplus S(-4)\xrightarrow{4}
	S(-2)\oplus S(-2)\oplus S(-3)\longrightarrow 0.
\]

(3) \textit{$X$的任意三点都不处于一条直线上}。最后,考虑$X$包含了
四个点,其任意三点都不在一条直线上。我们断言,$X$的Hilbert方程和
之前的情况一样:$H(X,1)=3$, $H(X,\nu)=4$对$\nu\geq 2$. 其中第一
点是显然的,因为$X$并不在任意直线上。对第二点,像之前一样,只需
注意到,存在二次型(于是更高次型)包含四点的任意子集而不包含
最后一点,这可以同之前一样用直线的并来构造。

现在来计算$S/I(X)$的自由分解。取由四点给出的四次型的两对对边,
这给出两个二次型$q_1$和$q_2$,它们在$X$的理想中没有公共因子。
\inclugra{8.png} % p.136
因为$q_1$和$q_2$是互素的,它们生成的$I$的自由分解具有形式
\[
	0\longrightarrow S(-4)\xrightarrow{A}
	S(-2)\oplus S(-2)\xrightarrow{\;\;B\;\;} S,
\]
其中
\[
	B=\begin{pmatrix}
		q_1&q_2
	\end{pmatrix}
	\quad
	A=\begin{pmatrix}
		-q_2\\
		q_2
	\end{pmatrix}.
\]

从这个分解计算$S/I$的Hilbert函数,我们看到他与$S/I(X)$的一样,
因为$I\subset I(X)$,我们必然有$I=I(X)$,即$I$由$q_1$, $q_2$
生成,而$X$对应于两个包含它的锥的交。

总之,我们看到所有三个例子从Hilbert多项式角度来看都是相同的,
前两个例子可由它们的Hilbert函数所区分,而后两个例子在Hilbert
函数上看是相同的,但是可由它们的分次Betti数区分。对长度为
$4$的子概形$X$,不难找到可以由概形的不变量所区分而不依赖于嵌入
的例子。比方说,概形$\spec K[x]/(x^4)$可以
被嵌入到$\P_K^2$中使得有着上面的任意一个Hilbert函数和Betti数
(比方说,就像由理想$(x_0,x_1^4)$, 
$(x_0x_2^2-x_1^3,x_0x_1,x_0^2)$和$(x_0x_2-x_1^2,x_0^2)$
分别定义的子概形,由$(x_0^2,x_1^2)$定义的子概形
将永远有着第三种情况中的分次Betti数和Hilbert函数)。

\begin{exe}\label{exe:3.63}
找到平面上所有长度为$3$的子概形的Hilbert多项式、Hilbert函数以及
分次Betti数。
\end{exe}

\paragraph*{例子:一般的以及在
\texorpdfstring{$\mathbb P_K^3$}{PK3}中的双重直线}
\addcontentsline{toc}{subsubsection}{例子:一般的以及在
\texorpdfstring{$\mathbb P_K^3$}{PK3}中的双重直线}
至今,我们关于射影概形的讨论是平行于簇理论的。现在,我们将考察
一类真正非经典的例子。

Exercise \ref{exe:2.35} 让你证明了,所有的仿射双重直线是等价的。
这对射影双重直线不再正确。这里有一些简单的例子。

令$K$是一个域。考虑分次环
\[
	S=K[u,v,x,y]/(x^2,xy,y^2,u^dx-v^dy)
\]
以及概形
\[
	X=X_d=\proj S.
\]

为看到$X$是一个双重直线,我们构造一个$X$的仿射开覆盖。元素$x$,
$y$在$S$中是幂零的,于是由$u$, $v$生成的理想的根是$S$的无关理想,
同时$X$由$X_u$, $X_v$所覆盖。从定义,我们看到
\[
	X_u=\spec (S[u^{-1}])_0.
\]%
% p.137
为分析环$(S[u^{-1}])_0$,我们注意到,这是
\[
	K[u,v,x,y][u^{-1}]_0=K[v',x',y']
\]
的分式环,其中
\[
	v'=\frac vu,\quad x'=\frac xu,\quad y'=\frac yu,
\]
而映射到$(S[u^{-1}])_0$的核由$(x')^2$, $x'y'$, $(y')^2$和
$x'-(v')^dy'$所生成(见Exercise \ref{exe:3.6}). 于是
\[
	(S[u^{-1}])_0\cong K[v',y']/(y')^2
\]
且$X_u$是一个仿射双重直线。由对称性,$X_v$也是,而这证明了$X$
是一个射影双重直线。具体地,
\[
	X_v\cong \spec (S[v^{-1}])_0
\]
以及
\begin{align*}
(S[v^{-1}])_0{}&\cong K[u'',x'',y'']/({x''}^2,x''y'',{y''}^2,
(u'')^dx''-y'')\\
{}&\cong K[u'',x'',y'']/({x''}^2),
\end{align*}
其中
\[
	u''=\frac uv=\frac 1{v'},\quad x''=\frac xv,\quad 
	y''=\frac yv.
\]

看到与仿射情况不同,不是所有的双重直线都同构的最简单方式是,
证明$X$的同构类依赖于整数$d$,其可以被想做给出了双重直线
绕着它的约态直线有多快。为证明它,我们将证明$X$的整体截面环
$\oo_X(X)$依赖于$d$. 为计算它,首先假设$\sigma\in \oo_X(X)$.
元素$\sigma$限制到$\oo_X(X_u)$,其同构于$K[v',y']/(y')^2$,
得到了一个元素
\[
	\sigma|_{X_u}=a(v')+b(v')y'
\]
以及类似地
\[
	\sigma|_{X_v}=f(u'')+g(u')x'',
\]
其中$a$, $b$, $f$和$g$都是唯一确定的$K$-系数多项式。但是,
在$X_u\cap X_v$上,我们有
\[
	u''=\frac 1{v'}
\]
以及
\[
	x''=\frac xv=\frac{x'u}v=(v')^d y'\frac uv=(v')^{d-1}y'.
\]
所以$f(1/v')=a(v')$,其只有当$f$和$a$都是常数多项式且$f=a$
才有可能。同样,$g(1/v')(v')^{d-1}=b(v')$,其只有当$g$和
$b$的次数都小于等于$d-1$(于是给出$g$和$b$的其中一个就确定了
另一个)。%
% p.138
反之,$\oo_X(X_u)$中任意形如$a+b(v')y'$的元素,其中
$a$是一个常数,而$b$是一个次数小于等于$d-1$的多项式,
可以唯一扩张为$\oo_X$的一个整体截面,所以$\oo_X(X)$的维数就是
$d+1$. 这证明了$X$的同构类依赖于$d$,如前所言。

实际上,我们在下面将看到整数$d$就是$X$的算术亏格的相反数,
算术亏格的定义见第 \ref{s:3.3.3} 节。事实证明,每个亏格为$-d$
的射影双重直线,其中$d\geq 0$,都同构于$X$.

同样,具有正算术亏格的双重直线也存在,最简单的例子是%
\textit{双重圆锥曲线} $\proj K[x,y,z]/(xy-z^2)^2$,它的
亏格是$3$,甚至在亏格大于等于$7$的时候,这样双重直线的连续族
也存在。这些对象在研究非奇异曲线的时候会自然出现:
当一个非奇异非椭圆曲线退化为超椭圆曲线,
这是一个经典簇理论中众所周知的现象,
一个典型的光滑曲线趋向于射影双重直线
(更多细节见Bayer以及Eisenbud [1995]和Fong [1993])。

\begin{exe}\label{exe:3.64}
上面的双重直线$X$的$\oo_X(X)$的环结构是什么?
\end{exe}

\begin{exe}\label{exe:3.65}
对双重直线
\[
	X=\proj K[u,v,x,y]/(x^2,xy,y^2,p(u,v)x+q(u,v)y)
\]
求$\oo_X(X)$,其中$p$, $q$是任意在$\P_K^1$中没有公共零点的
两个$d$-次齐次多项式。证明,这个双重直线同构于示例的双重直线
(因此并不依赖于$p$和$q$的选取)。
\end{exe}

为计算$X$的Hilbert多项式,观察对每个$d$,理想$I_d=(x^2,xy,y^2,
u^dx-v^dy)$包含理想
\[
	I=(x^2,xy,y^2).
\]
因为$S/I$是一个自由$K[u,v]$-模,生成元是$1$, $x$和$y$,我们看到
\[
	H(S/I,\nu)=H(S/(x,y),\nu)+2H(S/(x,y),\nu-1).
\]
再进一步,用这组基,我们容易看到,如果记$p=u^dx-v^dy$
为上面写的$I_d$的第四个生成元,于是对任意的齐次型
$q=q(x,y,u,v)$,我们有$qp\in I$当且仅当$q\in (x,y)$. 因此
\[
	H(S/I_d,\nu)=H(S/I,\nu)-H(S/(x,y),\nu-d-1).
\]
但是$P(S/(x,y),\nu)=\nu+1$. 将这些等式连立起来,我们有
\[
	P(X,\nu)=2\nu+d+1,
\]
于是Hilbert多项式,特别地,算术亏格
\[
	p_a(X)=1-P(X,0)=-d,
\]
对不同的$d$的双重直线是不同的。

这有一个对以下三个例子有用的习题。

% p.139

\begin{exe}\label{exe:3.66}
计算下面的$\P_K^3$的子概形的Hilbert多项式:
\begin{compactenum}[(a)]
\item 两条异面直线的并。
\item 两条相交直线的并。
\item 一个底空间是两条交于一个次数为$1$的嵌入点的直线的子概形,
且其并不处于这两条直线张成的平面上。同样,证明对双重直线$X_0$
上(活在上面的任意双重直线上)的任意点$p$,存在唯一的$\P_K^3$
的子概形包含$X_0$,在点$p$处有一个次数为$1$的嵌入点,再计算这个
子概形的Hilbert多项式。
\end{compactenum}
\end{exe}

考虑到这个习题,我们可以使用Hilbert多项式的概念来进一步阐明
一族趋于一对相交直线的斜线对的例子。
回忆在Exercise \ref{exe:2.25} 中,我们讨论了这样的一个族,以及
证明了其平坦极限是非约态的:它的支集是两个直线的并,但在它们的
交点处有一个嵌入点。同上面的习题所建议的,如果我们在$\P_K^3$
中完备化这些族,我们将看到,从Hilbert多项式观点来看,
这是必要的。

下面考虑$\P_K^3$中的一族斜线对,描述如下。首先,令
$L\subset \P_K^3$为常直线$x=y=0$,以及令$M\subset \P_K^3$为
直线$x=tv$, $y=tu$. 令$Y_t$为两个直线的并。接着我们可能会问
族$Y_t$的平坦极限,或者说,由$x=y=0$和$x=tv$, $y=tu$给出的
$\P_K^3\times \A_K^1$的子概形$\mathscr L$和$\mathscr M$的并
$\mathscr Y$在$\A_K^1$原点处的纤维$Y_0$. 当然,$Y_0$的支集
将是直线$L$,但是同样明显的是,它必然有一些非约态结构。
实际上,平面极限不是别的,就是上面的双重直线$X_1$.

\begin{exe}\label{exe:3.67}
验证平坦极限$Y_0$就是双重直线$X_1$. (通过比较Hilbert多项式,
只要证明在一个方向的包含就足够了。)
\end{exe}

一个关于上面构造的有趣技巧是考虑一族稍微不同的斜线对:我们
令$L$和上面一样,再令$M_t$是由$x=tv$, $y=-t^2u$给出的直线。
第一眼看上去,并$Y_t=L\cup M_t$的平坦极限是由$(x^2=y=0)$给出的
双重直线,他同构于上面的双重直线$X_0$,但这并不对,因为
它们的Hilbert多项式并不相同。下面的习题给出了真实的情况。

\begin{exe}\label{exe:3.68}
对上面的$L$和$M_t$,证明在$t\to 0$时,$L\cup M$的平坦极限是
双重直线$x^2=y=0$,其在点$[0,0,1,0]$处有一个次数为$1$的嵌入点。
\end{exe}

% p.140

\begin{exe}\label{exe:3.69}
令$L$, $M$和$N_t\subset \P_K^3$分别为直线$u=v=0$, $y=v=0$以及
$y+u=ty+(1-t)v=0$;令$Z_t$为它们在$\P_K^3$中的并。找到$Z_t$
和$Z_0$的Hilbert多项式。在上面讨论的意义下,概形$Z_t\subset 
\P_K^3$在$t\to 0$的极限是什么?
\end{exe}

\inclugra{9.png}

最后,存在上面三个例子的算术对应。比方说,考虑下面三族$\P_\zz^3$
的子概形(它们都在$\spec \zz$上平坦):
\begin{compactenum}[(a)]
\item 令$\mathscr L\subset \P_\zz^3$为常直线$x=y=0$,再令
$\mathscr M\subset \P_\zz^3$为直线$x=7v$, $y=-7u$;令$\mathscr U$
为这两个子概形的并。
\item 令$\mathscr L\subset \P_\zz^3$为常直线$x=y=0$,再令
$\mathscr M\subset \P_\zz^3$为直线$x=7v$, $y=-49u$;令$\mathscr U$
为这两个子概形的并。
\item 令$\mathscr L$, $\mathscr M$和$\mathscr N\subset \P_\zz^3$
分别为由$u=v=0$, $y=v=0$和$y+u=7y-6v=0$定义的子概形,令$\mathscr U$
为它们的并。
\end{compactenum}

\begin{exe}\label{exe:3.70}
对上面的每个子概形$\mathscr U\subset \P_\zz^3$,找到
$\mathscr U$在点$(7)\in\spec \zz$处的纤维。
将你的答案与前面三个习题中的答案进行比较。
\end{exe}

\subsection{B\texorpdfstring{\'{e}}{é}zout定理}
\label{s:3.3.5}

B\'ezout定理最经典的形式是在说,如果次数为$d$和$e$的
平面曲线$C$, $C'\subset \P_K^2$相交于有限点,其交点的个数
最多为$de$,如果两条曲线横截相交且$K$是代数闭的,则取等号。
这个重要的结果经历了一系列推广。特别是,概形的语言允许我们
给出一个既简单又比原来更通用的版本;尽管这个版本并不是最一般的,
但我们将专注于此。

在下面,我们将在域$K$上的概形上工作。和次数的讨论中一样,
我们可以陈述任意基$S$上的射影概形$X\subset \P_S^n$的
B\'ezout定理,但这将并不比域上的概形的B\'ezout定理携带更多
信息,只要将其应用到$X$在$S$的一般点处的纤维上。
此外,注意到我们并不假设$K$是代数闭的。我们将
在下面的Exercise \ref{exe:3.72} 到 \ref{exe:3.75} 
看到非代数闭域上的例子。

% p.141

B\'ezout定理的陈述非常简单。回忆,对域$K$上的射影空间
$\P_K^n=\proj K[X_0,\dots,X_n]$中的\textit{超曲面},
我们并不指任意的$\P_K^n$的$(n-1)$-维子概形,而是指
一个齐次多项式$F$的零点集$V(F)$. 特别地,纯维数
为$n-1$(比如可见Eisenbud [1995]);尽管它可能是非约态的
(如果$F$有重复的因子),但它将没有嵌入分支。
回忆,域$K$上的射影空间的子概形$X\subset \P_K^n$的次数
可以用其Hilbert多项式定义;特别地,如果$X$的维数为零,
则其次数就是整体截面空间$\oo_X(X)$作为$K$-矢量空间的维度。

\begin{thm}[完全相交情形的B\'ezout定理]
\label{thm:3.71}
假设$Z_1$, $\dots$, $Z_r\subset \P_K^n$是域$K$上的射影空间
中次数为$d_1$, $\dots$, $d_r$的超曲面,且交
$\Gamma=\bigcap Z_i$的维度为$n-r$,则
\[
	\deg(\Gamma)=\prod d_i.
\]
\end{thm}

因此,比如$D$和$E\subset \P_K^2$是次数为$d$和$e$的平面曲线,
且它们没有相同的分支,则交$\Gamma=D\cap E$的次数为$de$. 
作为直接的结果,我们可以从这里推出定理的经典形式
“$\deg(\Gamma)\leq de$”,以及等式成立当且仅当$\Gamma$是约态的
且$\Gamma$的每个点的剩余类域都是$K$.

更一般地,我们可以从Theorem \ref{thm:3.71} 中得到,代数闭域
上的对完全相交情形的经典的B\'ezout定理的等式命题的一般形式,
其中,我们将超曲面的次数的积$\prod d_i$表为
约态概形$\Gamma_{\text{red}}$的不可约分支$\Gamma_i$的次数的
线性组合,其系数被称为\textit{交$Z_1\cap \dots\cap Z_r$沿着
$\Gamma_i$的重数},来自于非约态结构。用这个形式,我们可以
进一步推广B\'ezout定理到射影空间中任意恰当的相交上去
(即,两个纯余维数分别为$k$和$l$的子概形$X$, 
$Y\subset \P_K^n$使得其交$X\cap Y$的余维数为$k+l$);但是,
为做这个,我们同样需要一般地定义两个概形的交沿着一个分支的
重数。这还有对完全相交情形的经典的B\'ezout定理的证明我们
以后再说,以方便读者去尝试一些例子。

\begin{exe}\label{exe:3.72}
令$C\subset \P_{\mathbb R}^2$为有
\[
	C=\proj \mathbb R[X,Y,Z]/(X^2+Y^2-Z^2)
	\subset \proj \mathbb R[X,Y,Z]
\]
给出的圆锥曲线,再令$L_1$, $L_2$和$L_3$分别为由$X$, $X-Z$和
$X-2Z$给出的直线。证明,概形$C\cap L_i$两两不同构,但是,
它们作为$\mathbb R$上的概形,次数都为$2$.
\end{exe}

% p.142

以下四个习题描述了非代数闭域上的B\'ezout定理会自然出现的情况:
概形乘积上的万有曲线的交,其中概形参数化了这些曲线。
这种情况经常发生,且作为“一般”交点的一个例子而富有意趣
(除了它作为B\'ezout定理的例子的价值之外)。

\begin{exe}\label{exe:3.73}
令$K$是一个域,再令
\[
	B=\A_K^{12}=\spec K[a,b,c,d,e,f,g,h,i,j,k,l].
\]
考虑有下面两个圆锥曲线$\mathscr C_i\subset \P_B^2$
\begin{align*}
	\mathscr C_1&=V(aX^2+bY^2+cZ^2+dXY+eXZ+fYZ)\\
	&\subset \proj\left(
	K[a,b,c,d,e,f,g,h,i,j,k,l][X,Y,Z]
	\right)=\P_B^2
\end{align*}
以及
\[
	\mathscr C_2=
	V(gX^2+hY^2+iZ^2+jXY+kXZ+lYZ)\subset \P_B^2.
\]
通过考察投影
\[
	\pi_2:\mathscr C_1\cap \mathscr C_2
	\subset \P_B^2=\P_K^2\times_{\spec K}B
	\longrightarrow \P_K^2,
\]
证明$\mathscr C_1\cap \mathscr C_2$是一个不可约$K$-概形。
\end{exe}

\begin{exe}~\label{exe:3.74}
\begin{compactenum}[(a)]
	\item $\mathscr C_1$, $\mathscr C_2$同上,证明交
	$\mathscr C_1\cap \mathscr C_2$一般是约态的,通过证明
	\[
		\pi_1:\mathcal C_1\cap \mathcal C_2\subset \P_B^2
		\longrightarrow B=\A_K^{12}
	\]
	有一个包含四个不同(因此是约态且$K$-有理的)点的纤维。
	\item 尽管(a)对下面习题的应用已然足够,
	从完全相交并不混合推出,$\mathscr C_1\cap \mathscr C_2$
	处处约态(比如可见Eisenbud [1995])。
	或者,通过直接的切空间计算来证明这是非奇异的。
\end{compactenum}
\end{exe}

\begin{exe}\label{exe:3.75}
令$L=K(a,b,c,d,e,f,g,h,i,j,k,l)$为$K$上的有$12$个变量的有理
函数域(即$B=\A_K^{12}$的函数域)。令
\[
	C_1=V(aX^2+bY^2+cZ^2+dXY+eXZ+fYZ)\subset \P_L^2
\]
以及
\[
	C_2=V(gX^2+hY^2+iZ^2+jXY+kXZ+lYZ)\subset \P_L^2;
\]
即,$C_1$和$C_2$为$\mathscr C_1$和$\mathscr C_2$在$\A_K^{12}$
的一般点处的纤维。从上两个习题,给出交$C_1\cap C_2$为一个奇异、
约态点$P$.
\end{exe}

% p.143

\begin{exe}\label{exe:3.76}
保持前面习题的记号,证明,如B\'ezout所料,交$C_1\cap C_2$作为
$L$上的概形,次数为$4$,即点$P=C_1\cap C_2$的剩余类域
$\kappa(P)$是一个$L$的四次扩张。

\textit{提示}:在$\P_L^2$的一个开子集上引入仿射坐标
$x=X/Z$和$y=Y/Z$,将$\kappa(P)$表为
\[
	\kappa(P)=L[x]/(R(x)),
\]
其中$R(x)$是$C_1$和$C_2$关于$x$的定义多项式的非齐次化的结式,
正如第 \ref{s:5.2} 节中所描述的。
\end{exe}

这个例子有非常有趣的一面,我们顺便提一下。一个此时我们可能会问
的问题是,扩张$L\subset \kappa(P)$的Galois正规化的Galois群?
为回答这个问题,至少对底域$K=\mathbb C$的情况,我们将引入
两个一般的圆锥曲线的四个交点的\textit{单值群}(
\textit{monodromy group})。简单地说,存在一个开集$U\subset B$,
其上投影$\varphi:\mathscr C_1\cap \mathscr C_2\to B$的纤维是
约态的;用经典拓扑的语言,映射$\varphi$限制在$U$的原像上是一个
拓扑覆叠空间。所以,对任一点$p\in U$,我们有一个基本群$\pi_1(U,p)$
在纤维$\varphi^{-1}(p)$中的点的单值作用(monodromy action):对
一条起于且终于$p$的曲线$\gamma:[0,1]\to U$,以及任一点
$q\in \varphi^{-1}(p)$,则存在$\gamma$到$\varphi^{-1}(U)$的
唯一提升$\tilde\gamma:[0,1]\to \varphi^{-1}(U)$,其满足
$\tilde\gamma(0)=q$,我们定义作用的结果为$\tilde\gamma$的终点。
没那么形式地,
设想我们允许两条圆锥曲线$C_1(t)$, $C_2(t)\subset \mathbb P_\cc^2$
随着一个实参数$t\in [0,1]$改变,并保持它们在任何时候都横截相交。
当$t$改变时,$C_1(t)\cap C_2(t)$的四个交点也随之变化,当
曲线回到它们原来的位置,即$C_i(0)=C_i(1)$,我们发现交集
$C_1(0)\cap C_2(0)=C_1(1)\cap C_2(1)$中的四个点各自并不回到
它们原来的位置,则置换这四点的群就被称为单值群。这就给出了我们
开始问题的答案,即$\kappa(P)$在$L$上的Galois normalization
的Galois群,就是这四个点的单值群,可以从几何刻画看到这个群就是
四个字母的对称群。

更一般地,在许多依赖于参数的计数问题(在这个例子里,是两个圆锥曲线
的交),一般的解为一个点$P$以及剩余类域$\kappa(P)$,其是参数化这个
问题的概形(这里是$B=\mathbb A_K^{12}$)的函数域$L$的有限扩张。
此时,我们可能会问,扩张$L\subset \kappa(P)$的Galois正规化的
Galois群是什么?一般来说,它和问题的单值群相同。一般的处理可见
Harris [1979].

% p.144

我们现在将给一个B\'ezourt定理的证明,以及讨论其可能的推广。我们将
用\textit{Koszul复形}来计算$\Gamma$的Hilbert多项式(特别是次数)
来证明它。Koszul复形在Eisenbud [1995, Chapter 17]中有完整的讨论,
我们这里将列出构造以及简单陈述一些我们需要的结果。

首先,我们引入超曲面$Z_i$的定义方程:记
\[
	Z_i=\proj K[X_0,\dots,X_n]/(F_i)\subset 
	\proj K[X_0,\dots,X_n],
\]
于是
\[
	\Gamma=\proj K[X_0,\dots,X_n]/(F_1,\dots,F_r).
\]
现在我们描述齐次坐标环$S_\Gamma$的一个分解如下。首先,对任意子集
\[
	I=\{i_1,i_2,\dots,i_k\}\subset \{1,2,\dots,r\},
\]
我们将记$|I|=k$为$I$的元素个数,
\[
	d_I=\sum_{\alpha=1}^k d_{i_\alpha}
\]
为对应多项式的次数之和。于是我们置
\[
	M_k=\bigoplus_{|I|=k}S(-d_I)
\]
其中$S=K[X_0,\dots,X_n]$如同往常为多项式环。当存在唯一的$I$
满足$|I|=0$,我们置$M_0=S$. 我们将用一族多项式$\{G_I\}$
来记$M_k$中的元素,其中$I$跑遍所有大小为$k$的多重指标,
由我们的定义,$\{G_I\}$是$d$-次齐次的,如果对每个$I$都有
$\deg(G_I)=d-d_I$.

我们现在定义一个复形
\[
	0\longrightarrow M_r\longrightarrow M_{r-1}
	 \longrightarrow \dots \longrightarrow M_2
	 \longrightarrow M_1\longrightarrow M_0=S.
\]
映射$\varphi_k:M_k\to M_{k-1}$可以通过置$\varphi_k(\{G_I\})$
等于多项式族$\{H_J\}$给出,其中
\[
	H_J=\sum_{\alpha\not\in J}\pm F_\alpha\cdot 
	G_{J\cup \{\alpha\}},
\]
而符号依赖于$J$的元素个数小于$\alpha$.

注意到$\varphi_1:M_1\to M_0=S$的像就是$\Gamma$的理想。
实际上,\textit{这个序列是坐标环$S_\Gamma$的一个自由分解}。
这是一个一般的现象:当有一族环$S$中的元素
$F_1$, $\dots$, $F_r$时,我们都能这样构造一个序列,其被称为
\textit{Koszul复形}。一个标准的定理告诉我们,
当族$F_1$, $\dots$, $F_r$十一个正规族时,
相应的Koszul复形是一个分解(比如可见Eisenbud [1995, Chapter 17])。
在这里,多项式$F_i$都是齐次的,关于$\Gamma$维数的假设连同
多项式环$S$是一个Cohen-Macaulay环告诉我们多项式$F_1$, $\dots$,
$F_r$构成一个$S$中的正规列(如同局部的情况),于是上面的序列
是一个分解。

% p.145

给定Koszul分解后,描述其Hilbert多项式$P(\Gamma,\nu)$是直接的。
如果我们记模$M_k$的Hilbert多项式为$H(M_k,\nu)$ (即,$H(M_k,\nu)$
为在$\nu$很大时其值正是$M_k$的$\nu$-次分次的维数的多项式),则
从Koszul复形的正和性,我们看到
\[
	P(\Gamma,\nu)=\sum (-1)^k P(M_k,\nu)
\]
仅仅依赖于数$d_i$而不依赖于那个多项式$F_i$. 方便起见,我们将记
这样一个完全相交的Hilbert多项式为$P_{d_1,\dots,d_r}(\nu)$.
(注意到,我们并不需要写下Koszul复形来看到给定多重指标的
完全相交都有着相同的Hilbert多项式,这直接来自于完全相交族的平坦性,
正如在Proposition \ref{pro:2.32} 中所陈述的一般)。

现在,简单地将上面的Kozusl复形中的直和项的贡献加起来,我们看到
\[
	P_{d_1,\dots,d_r}(\nu)
	=\sum_{I\subset \{1,\dots,r\}}(-1)^{|I|}
	\binom{n+\nu-d_I}{n},
\]
其中求和跑遍所有$\{1,\dots,r\}$的子集,包括空集和全集。

这在某种意义上是对$\Gamma$的Hilbert多项式的完全解答,
但仍然存在比如从其中读取$\Gamma$的次数之类的问题。
为此,我们对$r$归纳,联系起函数$P_{d_1,\dots,d_r}(\nu)$
和$P_{d_1,\dots,d_{r-1}}(\nu)$. 这是简单的:在上面的
$P_{d_1,\dots,d_r}(\nu)$的表达式中,我们简单分离出
哪些加项的$I$包含$r$,哪些不包含。对$r\not\in I$的那些项加起来
显然就等于$P_{d_1,\dots,d_{r-1}}(\nu)$;比较
$r\in I$的项与对应于$I\setminus \{r\}$的项,我们看到这些加起来
得到$P_{d_1,\dots,d_{r-1}}(\nu-d_r)$. 因此,
\[
	P_{d_1,\dots,d_r}(\nu)=P_{d_1,\dots,d_{r-1}}(\nu)-
	P_{d_1,\dots,d_{r-1}}(\nu-d_r).
\]

现在,因为
\[
	\nu^m-(\nu-\alpha)^m=m\alpha\nu^{m-1}+O(\nu^{m-2})
\]
(这里,从分析学家的记号,我们用$O(\nu^{m-2})$来记一些次数
至多为$m-2$的项的和)我们看到,如果$f(\nu)$为任意多项式,
写作
\[
	f(\nu)=c_m\nu^m+O(\nu^{m-1}),
\]
于是
\[
	f(\nu)-f(\nu-\alpha)=m\alpha c_m\nu^{m-1}+O(\nu^{m-2}).
\]
因为射影空间本身的Hilbert多项式为
\[
	P(\P_K^n,\nu)=\binom{\nu+n}{n}=\frac{1}{n!}\nu^n+
	O(\nu^{n-1}),
\]

% p.146

我们可以记
\begin{align*}
P_{d_1,\dots,d_r}(\nu)&=n(n-1)\cdots (n-r+1)d_1d_2\cdots
	d_r\frac 1{n!} \nu^{n-r}+O(\nu^{n-r-1})\\
&=\frac{d_1d_2\cdots d_r}{(n-r)!}\nu^{n-r}+O(\nu^{n-r-1}).
\end{align*}
因此,
\[
	\deg(\Gamma)=d_1d_2\cdots d_r,
\]
这就是我们所需要的。

我们也可以特化地避免上面证明中最后的计算:
通过使用一开始就建立的事实,
即给定多重次数的完全交集都具有相同的Hilbert多项式,
我们可以对每个偶对$(i,j)$选取一个一般线性型$L_{i,j}$,
其中$1\leq i\leq r$和$1\leq j\leq d_i$,再对每个$i$令
$Z_i=V(F_i)$,其中
\[
	F_i=\prod_{j=1}^{d_i}L_{i,j}.
\]
则交$\Gamma=\bigcap Z_i$为$\P_K^n$中
$\prod d_i$个约态线性子空间的并,因此有着次数$\prod d_i$.
所以,我们得到了,所有多重次数$(d_1,\dots,d_r)$的完全交集
的次数为$\prod d_i$.

\begin{exe}\label{exe:3.77}
另一种特化,考虑$Z_i\subset \P_K^n$为由
$F_i(X_0,\dots,X_n)=X_i^{d_i}$定义的子概形。
直接证明,交$\bigcap Z_i$的次数为$\prod d_i$. 
(提示:你可以将它约化到$r=n$的情况。)
\end{exe}

\paragraph*{交点的重数}
\addcontentsline{toc}{subsubsection}{交点的重数}
对完全相交情形的B\'ezout定理(Theorem \ref{thm:3.71})
给出了超曲面完全交集的次数,但是,在实际中,
我们往往对射影空间更一般的子簇或子概形感兴趣。因为我们
已经定义了射影开集的任意子概形的次数,所以很自然会问,
是否两个子概形$X$, $Y\subset \P_K^n$的交的次数就是两个
子概形次数的乘积呢,这里假设相交是恰当的,即有着期待的
余维数。一般来说,这是错的,但在我们对相交的概形的
奇性做出了一些假设后,这是对的:如果$X$和$Y$是$\P_K^n$
的\textit{局部完全相交子概形},或更一般的
\textit{Cohen-Macaulay子概形},我们有:

\begin{thm}[对Cohen-Macaulay子概形的B\'ezout定理]
\label{thm:3.78}
令$X$和$Y\subset \P_K^n$分别为$\P_K^n$中纯余维数
为$k$和$l$的Cohen-Macaulay子概形。如果交$X\cap Y$
余维数为$k+l$,则
\[
	\deg(X\cap Y)=\deg X\deg Y.
\]
\end{thm}

% p.147

\begin{exa}\label{exa:3.79}
正如我们指出的,在没有假设$X$和$Y$是Cohen-Macaulay时,
Theorem \ref{thm:3.78} 失效。考察一个这样的例子是有益的。
可能最简单的情况出现在 $\mathbb P_K^4=\proj K[Z_0,Z_1,Z_2,Z_3,Z_4]$
中:我们取$X=\Lambda_1\cup \Lambda_2$ 为两个
$2$-平面
\[
	\Lambda_1=V(Z_1,Z_2)\quad \text{和}\quad
	\Lambda_1=V(Z_3,Z_4)
\]
的并,而$Y$是$2$-平面
\[
	Y=V(Z_1-Z_3,Z_2-Z_4).
\]
我们已经在Exercise \ref{exe:1.43} 中及Lemma \ref{lem:2.30}后
讨论过这个例子。特别地,我们已知,
相交概形$X\cap Y$是平面$Y$的由原点的极大理想的平方
定义的子概形,所以其次数为$3$. (或者,因为$X$的射影切空间
都在$\mathbb P_K^4$中\nottran ,于是$X\cap Y$的Zariski切空间
是二维的,进而我们可以立刻看到$\deg(X\cap Y)\geq 3$.)
但$\deg X\deg Y=2\cdot 1=2$,于是Theorem \ref{thm:3.78} 并不成立。
\end{exa}

在这个例子中发生的事情并不是难以理解的。将$Y$表为两个
包含它的一般平面$H_1$, $H_2$的交,然后交$X\cap Y$表为
\[
	X\cap Y=X\cap(H_1\cap H_2)=(X\cap H_1)\cap H_2.
\]
第一次取交时,我们发现相交概形$X\cap H_1$在点
$(Z_1,Z_2,Z_3,Z_4)$处有一个嵌入点。而第二个平面$H_2$
经过这个点,进而取了进一步的相交。 

\vspace{1ex}

这个例子揭示了需要一种精确的方法,将重数归因于投影空间的子集的
交集的分支,并给出了一种方法。方法如下:假设给定了概形
$X$, $Y\subset \mathbb P_K^n$,纯维数分别为$k$和$l$,交于
纯维数为$k+l$的一个概形中。我们首先将情况约化到概形$Y$是一个
射影空间的线性子空间如下:取两个互补的$n$-维线性空间
$\Lambda_1$, $\Lambda_2\subset \mathbb P_K^{2n+1}$,以及两者
到$\mathbb P_K^n$的同构。(具体地,我们可以将$\mathbb P_K^{2n+1}$
的齐次坐标标记为$x_0$, $\dots$, $x_n$, $y_0$, $\dots$, $y_n$,
然后取线性空间分别为$x_0=\dots=x_n=0$和$y_0=\dots=y_n=0$给出
的线性空间。)记$X'$和$Y'$为$X$, $Y\subset \mathbb P_K^n$在这
两个嵌入下的像。令$J\subset \mathbb P_K^{2n+1}$为由$X$的
(用$x_i$写出的)方程和$Y$的(用$y_i$写出的)方程定义的子概形,
换句话说,这是$X'$上的顶点为$\Lambda_2$的锥和
$Y'$上的顶点为$\Lambda_1$的锥的交。$J$被称为$X'$和$Y'$的
\textit{join},集合论式地,他是连接$X'$和$Y'$中点的直线的并。
令$\Delta\subset \mathbb P_K^{2n+1}$为由方程
$x_0-y-0=\dots=x_n-y-n=0$定义的子概形。很清楚,概形$X\cap Y$
与概形$J\cap \Delta$同构,于是我们将定义$X\cap Y$沿着其
一个不可约分支$Z\subset X\cap Y$的重数定义为
$J$和$\Delta$沿着对应$J\cap \Delta$分支的相交的重数。
因此,我们已经将问题约化到定义$X$和$Y$沿着一个不可约分支
$Z\subset X\cap Y$的重数,其中$Y$是一个线性空间。

% p.148

我们现在处理这种情况,正如上面例子所建议的,记
$Y$为一族超平面的交$H_1\cap \dots \cap H_l$,然后与$X$
每次相交一个$H_i$. 在每一步后,我们去掉交的嵌入分支。
最后,我们得到了一个概形$W$包含于交$X\cap Y$,其中
次数满足B\'ezout定理:$\deg(W)=\deg(X)\deg(Y)$. 
为将其与经典语言联系起来,对交$X\cap Y$的每个不可约分支$Z$,
我们定义$X$和$Y$沿着$Z$的\textit{相交重数},记作
$\mu_Z(X\cdot Y)$,为$W$在$W$对应于分支$Z$的一般点上的局部环
的长度。我们于是有:

\begin{thm}[有多重交点的B\'ezout定理]
\label{thm:3.80}
	令$X$和$Y\subset \mathbb P_K^n$为$\mathbb P_K^n$
	中的纯余维数为$k$和$l$的的概形。如果交$X\cap Y$纯余维数为
	$k+l$,则
	\[
		\deg(X\cap Y)=\sum_Z\mu_Z(X\cdot Y)\deg Z_{\text{red}}.
	\]
\end{thm}

有很多其他方法来定义两个概形$X$, $Y\subset \mathbb P_K^n$沿着一个
分支$Z\subset X\cap Y$的重数$\mu_Z(X\cdot Y)$;在经典文献中,
充满了这种尝试。同时,也存在一种现代的方法,其用到了层
$\operatorname{Tor}(\mathscr O_X,\mathscr O_Y)$. 这些方法
中的据大多数在定义任意两个非奇异子概形的任意两个子概形$X$, $Y$的
相交重数时都适用,只要相交是逆紧的。

再进一步,B\'ezout定义还存在一个更一般的版本,
它对非奇异概形$T$的任意两个纯维数为$k$和$l$的子概形$X$和$Y$
适用,即使当交$X\cap Y$并没有纯维数$k+l$(或甚至
可能奇异概形$T$的子概形$X$, $Y$,其中有一个子概形局部是
$T$的完全相交子概形)。在这个情况中,可以将重数与交$X\cap Y$的
特定的子概形或子概形的等价类相关联,使得(当$T=\mathbb P_K^n$)
这些子概形的次数乘以相关重数加起来得到$\deg X\deg Y$. 
对此和进一步的改进,可见 Fulton [1984] 和 Vogel [1984].

\begin{exe}\label{exe:3.81}
	若在Example \ref{exa:3.79} 中取$X$为约态的会让读者觉得
	我们在作弊:证明同样的现象在我们取$X\subset \mathbb P_K^4$
	为在一个非奇异有理三次曲线$C\subset \mathbb P_K^3$上的锥
	时也会出现,其中$Y$依然是穿过顶点的$2$-平面。
\end{exe}

% p.149

\begin{exe}\label{exe:3.82}
	若将Theorem \ref{thm:3.78} 陈述中左手侧的$\deg(X\cap Y)$
	换成任何其他$X\cap Y$的不变量,Theorem \ref{thm:3.78}
	一般来说将不再成立。为看到这点,找一个例子,
	有概形$\Gamma\subset \mathbb P_K^n$和适当维数的
	子概形$X$, $Y$, $Z$, $W\subset \mathbb P_K^n$
	使得$X\cap Y=Z\cap W=\Gamma$,以及
	\[
		\deg X\deg Y=\deg \Gamma\neq \deg Z\deg W.
	\]
\end{exe}

\subsection{Hilbert级数}\label{s:3.3.6}

作为我们关于Hilbert方程、Hilbert多项式和自由分解的最后
注记,我们来谈谈一个子概形$X\subset \P_K^n$的
\textit{Hilbert级数},或更一般地,一个$\P_K^n$的坐标环
$S$上的分次模$M$的Hilbert级数。这是一个非常有用的轮子
来表达Hilbert多项式的信息;作为演示,我们可以以一种
更清楚的方式来写出一个完全交的Hilbert多项式。

模$M$的Hilbert级数$H_M(t)$是容易定义的:如果$P(M,\nu)$
为$M$的Hilbert方程,我们令$H_M(t)$为Laurent级数
\[
	H_M(t)=\sum_{\nu=-\infty}^\infty P(M,\nu)t^\nu.
\]
我们定义一个子概形$X\subset \P_K^n$的\textit{Hilbert级数}
为其坐标环$S_X=S/I(X)$的Hilbert级数。第一件可以注意到的
事是,射影空间本身的Hilbert级数很简单:我们有
\[
	H_{\P^n_K}(t)=H_S(t)=\frac{1}{(1-t)^{n+1}}.
\]
类似地,$S$任意的扭曲$S(d)$为
\[
	H_{S(d)}(t)=\frac{t^d}{(1-t)^{n+1}}.
\]

给任意一个分次$S$-模的正合列
\[
	0\longrightarrow M_r\longrightarrow M_{r-1}
	\longrightarrow \cdots \longrightarrow M_2
	\longrightarrow M_1 \longrightarrow M_0
	\longrightarrow 0,
\]
我们看到,它们的Hilbert级数满足关系
\[
	\sum_{k=0}^r(-1)^k H_{M_k}(t)=0.
\]
因此,如果我们有概形$X\subset \P_K^n$的一个自由分解
\[
	\cdots \longrightarrow \bigoplus_{i=1}^{k_2}
	S(-a_{2i})\longrightarrow \bigoplus_{i=1}^{k_1}
	S(-a_{1i})\longrightarrow S\longrightarrow S_X
	\longrightarrow 0,
\]
我们看到,Hilbert级数为
\[
	H_X(t)=\frac{\sum (-1)^i t^{a_{i,j}}}{(1-t)^{n+1}}.
\]%
% p.150
(这里我们采用约定$k_0=1$以及$a_{01}=0$)。从这里,我们
可以看到一个事实,\textit{任意射影空间的子概形的Hilbert
级数是一个$t$的有理函数}。

\begin{exe}\label{exe:3.83}
	证明,如果$X\subset \P_K^n$是一个$m$-维子概形,于是
	有理函数$H_X(t)$,约到最低项,具有分子
	\[
		\tilde{H}_X(t)=(1-t)^{m+1}H_X(t);
	\]
	特别地,这是一个$t$的多项式。证明,它在$1$的值为
	\[
		\tilde{H}_X(1)=\deg(X).
	\]
\end{exe}

现在,假设$X\subset \P_K^n$为$r$个次数为$d_1$, $\dots$, 
$d_r$的超曲面的完全交。从上面的Koszul分解,我们看到
Hilbert级数为
\[
	H_X(t)=\frac{\sum (-1)^{|I|}t^{|I|}}{(1-t)^{n+1}}.
\]
我们可以因式分解它,然后消去因子,写作
\[
	H_X(t)=\frac{\prod (1-t^{d_i})}{(1-t)^{n+1}}
	=\frac{\prod (1+t+\cdots+t^{d_{i}-1})}{(1-t)^{n-r+1}}.
\]
因此
\[
	\tilde{H}_X(t)=(1-t)^{\dim(X)+1}H_X(t)=
	\prod (1+t+\cdots+t^{d_{i}-1}).
\]
因为这个多项式在$t=1$的值为$\prod d_i$,我们就得到了
B\'ezout定理。
